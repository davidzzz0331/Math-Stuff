\documentclass{article}
\usepackage[utf8]{inputenc}
\usepackage{amsmath}
\usepackage{amsfonts}
\usepackage{amssymb}
\usepackage{tikz}
\usepackage{fullpage}
\usepackage{tikz-cd}
\usepackage{spectralsequences}
\usepackage{adjustbox}
\usepackage{xfrac}
\usepackage{tcolorbox}
\usepackage{xcolor}
\usepackage[backend=biber,style=alphabetic]{biblatex}
\usepackage{graphicx}
\graphicspath{ {D:/Chrome Downloads./} }
\addbibresource{HAK.bib}
\usepackage[parfill]{parskip}
\usepackage{amsthm}
\theoremstyle{definition}
\newtheorem{theorem}{Theorem}[section]
\theoremstyle{definition}
\newtheorem{definition}{Definition}[theorem]
\theoremstyle{definition}
\newtheorem{remark}{Remark}[theorem]
\theoremstyle{definition}
\newtheorem{proposition}{Proposition}[theorem]
\theoremstyle{definition}
\newtheorem{lemma}[theorem]{Lemma}
\theoremstyle{definition}
\newtheorem{corollary}{Corollary}[theorem]
\theoremstyle{definition}
\newtheorem{example}{Example}[theorem]
\title{Higher Algebraic K-Theory}
\author{David Zhu}

\begin{document}
\maketitle
$K$-Theory started with Grothendieck's $K_0$ and projective modules/vector bundles. In this note, we present the various constructions for higher $K$-groups.

\section{The Group Completion Theorem }
First, we recall Grothendieck's definition of $K_0$ for an abelian monoid $M$
\begin{tcolorbox}[colback=purple!5!white,colframe=purple!75!black]
    \begin{definition}
        Let $M$ be an abelian monoid. Then, the Grothendieck group $K_0(M)$ is an abelian group $K$ with the inclusion map $i: M\to K$ satisfying the following universal property: given an abelian group $A$ and a monoid morphism $f: M\to A$, we have the factorization
        \[\begin{tikzcd}
        M\arrow[d,"i"]\arrow[r," f"]&A\\
        K\arrow[ur,dashed,swap,"\exists! g"]
        \end{tikzcd}\]
    \end{definition}
    \end{tcolorbox}
It is an easy exercise to show that $K$ is unique up to isomorphism, which we will denote by $K_0(M)$. Explicitly, we can obtain $K_0(M)$ from the following "group completion" construction: 

\begin{tcolorbox}[colback=blue!5!white,colframe=blue!30!white]
\begin{proposition}
Given an abelian monoid $K_0(M)$ is the abelian group generated by $[m]$ for each $m\in M$, modulo the relation $[x+y]-[x]-[y]$.
\end{proposition}
\end{tcolorbox}


\begin{tcolorbox}[colback=yellow!5!white,colframe=yellow!30!white]
\begin{example}
($K_0$ of a ring) Given a ring $R$, $K_0(R)$ is defined to be Grothendieck group over the abelian monoid of the isomorphim class of finitely generated projective modules over $R$, with monoid operation given by direct sum. 
\end{example}
\end{tcolorbox}

If $M$ is a topological monoid, let $BM$ be its classifying space (viewing $M$ as a category with one object). Then, there is a natural map $M\to \Omega BM$, with $\pi_0(\Omega BM)=\pi_1(BM)$ an abelian group. When $\pi_0$ is a group, it can be shown that $M\to \Omega BM$ is a homotopy equivalence. Thus the map is referred to as the group completion of a toplogical monoid. When $\pi_0$ is not necessarily group, many can still be said: since $M$ is an $H$-space, $H_*(M)$ is naturally a ring, and $H_0(M)=\mathbb{Z}[\pi_(M)]$. Viewing $\pi_0(M)$ as a multiplicative subset of $H_0(M)$, the induced map $H_*(M)\to H_*(\Omega BM)$ sends $\pi_0(M)$ to units. Mcduff and Segal proved the following result:


\begin{tcolorbox}[colback=red!5!white,colframe=red!30!white]
\begin{theorem}
If $\pi:=\pi_0(M)$ is in the center of $H_*(M)$, then 
\[H_*(M)[\pi^{-1}]\cong H_*(\Omega BM)\]
\end{theorem}
\end{tcolorbox}
We will outline the idea of the proof following the original paper \cite{MS}, as it provides many insights to the motivate the $S^{-1}S$-construction.

The goal is to  find an intermediate space $M_{\infty}$ with presribed homology $H_*(M_{\infty})= H_*(M)[\pi^{-1}]$ and a homology equivalence $M_{\infty}\to \Omega BM$. The first step uses the Quillen's lemma, given in \cite{QGc}, regarding localization:

\begin{tcolorbox}
\begin{lemma}
Let $R$ be a ring (not necessarily commutative) and $S$ a multiplicative subset. Let $C$ be the category with objects elements of $S$ and morphisms from $s_1$ to $s_2$ an element $t\in S$ such that $s_1t=s_2$. Under the conditions that $C$ is filtered, there is a canonical $R$-module morphism 
\[u:\varinjlim_{C} R \to R[S^{-1}] \]
where the filtered colimit is defined on objects by the inclusion map, and on morphisms by right multiplication by $t$. If $S$ acts on $R$ by left multiplication bijectively, meaning
\begin{enumerate}
    \item (injective) Given $r\in R$ and $s\in S$ such that $sr=0$, then there exists $t\in S$ with $rt=0$.
    \item (bijective)Given $r\in R$ and $s\in S$, there exists $r'\in R$ and $t\in S$ such that $sr'=rt$
\end{enumerate}
Then $u$ is an isomorphism. 
\end{lemma}
\end{tcolorbox}
A simple example is colimit $\mathbb{Z}\xrightarrow{\times p}\mathbb{Z}\xrightarrow{\times p}...$ being isomorphic to $\mathbb{Z}[1/p]$ as $\mathbb{Z}$-modules. The hypothesis of the lemma is satisfied when $\pi$ is in the center of $H_*(M)$, and we can define $M_{\infty}$ to be the mapping telescope given by $M\xrightarrow{\times m}M\xrightarrow{\times m}...$, where $m$ is any arbitrary element in the component of $1\in \pi_0(M)$. Since homology commutes with filtered colimits, $M_{\infty}$ will have the prescribed homology. 

Given a topological group $M$, we may construct the universal bundle $EM\to BM$ by vieweing $M$ as a topological category. If $M$ acts on a space $X$, let $X_M$ denote the associated bundle to the universal bundle $X_M\to BM$ with fiber $X$. The construction still holds when $M$ is only a monoid, and instead of the homotopy equivalent between the fiber and homotopy fiber, Mcduff and Segal recovers the following proposition
\begin{tcolorbox}[colback=blue!5!white,colframe=blue!30!white]
    \begin{proposition}
        If $M$ is a topological monoid which acts on a space $X$, and for each $m\in M$ the map $x\mapsto mx$ from $X$ to itself is a homology equivalence, then the map $X_M\to BM$ is a homology fibration with fibre $X$, meaning the canonical map between the fiber and homotopy fiber induces isomorphism on homology.
    \end{proposition}
    \end{tcolorbox}



By construction, $M$ acts on $M_{\infty}$ (which also induces an homology equivalence), and $(M_{\infty})_M$ is also contratible since it will be a filtered colimit of the contractible $M_M=EM$. Thus, we have a map $(M_{\infty})_{M} \to BM$ with fiber $M_{\infty}$ and homotopy fiber $\Omega BM$. The theorem then follows from proposition $1.2.1$ and more colimit nonsense on components for the general case. 


We now define group completion for general homotopy commutative, homotopy associative $H$-spaces.

\begin{tcolorbox}[colback=purple!5!white,colframe=purple!75!black]
\begin{definition}
Let $X$ be a homotopy commutative, homotopy associative $H$-space. A \underline{\textbf{group completion}} of $X$ is an $H$-space map $f: X\to Y$ such that $f$ induces the group completion on $\pi:=\pi_0$, and the isomorphism
\[H_*(X)[\pi^{-1}]\cong H_*(Y)\] 

\end{definition}
\end{tcolorbox}


\section{The $S^{-1}S$-construction}
Given a symmetric monoidal category $S$, Quillen constructs the category $S^{-1}S$ such that $B(S^{-1}S)$ is the group completion of $BS$ in the sense of definition $1.2.1$. The group completion recovers the plus construction, and gives another more natural definition for higher $K$-theory. We shall note that the construction below is closely related to the group completion construction in chapter $1$.

A left action of a monoidal category $S$ on a category $X$ is a functor $+: S\times X\to X$ with natural isomorphisms $A+(B+F)\cong (A+B)+F$ and $0+F\cong F$, where $A,B\in S$ and $F\in X$, satisfying the associativity coherence conditions. An action is call invertible if each translation $F\mapsto A+F$ is a homotopy equivalence. 
\begin{tcolorbox}[colback=purple!5!white,colframe=purple!75!black]
\begin{definition}
If $S$ acts on $X$ the category $\langle S,X \rangle$ has the same objects as $X$. A morphisms is represented by an isomorphisms class of the tuple $(s,sx\xrightarrow{\phi}t)$, where $\phi$ is a morphism in $X$. An isomorphism of tuples is given by an isomorphism $s\cong s'$, which induces the commutative diagram
\[\begin{tikzcd}
s+x\arrow[rr,"\cong"]\arrow[dr]&& s'+x\arrow[dl]\\
&t
\end{tikzcd}\]
\end{definition}
\end{tcolorbox}
We let $S^{-1}S:= \langle S, S\times S \rangle$, where $S$ acts on both facts of $S\times S$ by the monoidal operation. We also define an action of $S$ on $S^{-1}S$ by $s+(s_1,s_2)=(s_1,s+s_2)$. Note that this action is invertible: the translation $(s_1,s_2)\mapsto (s_1,s+s_2)$ has homotopy inverse $(s_1,s_2)\mapsto (s+s_1,s_2)$, since there is the natural transformation $(s_1,s_2)\mapsto (s+s_1,s+s_2)$. Quillen then proves 

\begin{tcolorbox}[colback=red!5!white,colframe=red!30!white]
\begin{theorem}
If every map in $S$ is an isomorphism, and the translations are faithful in $S$, then the functor $S\to S^{-1}S$ given by $x\mapsto (0,x)$ induces isomorphism
\[H_*(S)[\pi^{-1}]\cong H_*(S^{-1}S)\]
In particular, $BS^{-1}S$ is the group completion of $BS$.
\end{theorem}
\end{tcolorbox}
\begin{proof}
    It is not hard to show that the projection functor $S^{-1}S\to \langle S,S \rangle$ onto the second factor is cofibered with fiber $S$. Then, there is a spectral sequence \cite{Grayson} 
    \[E^2_{p,q}=H_p(\langle S,S \rangle, H_q(S))\Rightarrow H_{p+q}(S^{-1}S)\]
    Localizating at $\pi_0(S)$ and noting the contractibility of $\langle S,S\rangle$ gives the desired degeneration. 
\end{proof}

We now show that the plus construction for $BGL(R)$ is a group completion.

\begin{tcolorbox}[colback=red!5!white,colframe=red!30!white]
\begin{theorem}
Let $S=\coprod GL_n(R)$ be the category of free $R$-modules. Then, $BS^{-1}S$ is the group completion of $BS=\coprod BGL_n(R)$, and 
\[BS^{-1}S\cong \mathbb{Z}\times BGL(R)^{+}\]
\end{theorem}
\end{tcolorbox}
\begin{proof}
    We will construct a map $f:BGL(R)\to Y_S$, where $Y_S$ is the connected component of $BS^{-1}S$ containing the basepoint, such that $f$ induces isomorphism on integral homology. The universal property of the plus contruction then implies a homotopy equivalence $BGL(R)^{+}\to Y_S$.


    
\end{proof}


\section{Exact Category}


\begin{tcolorbox}[colback=purple!5!white,colframe=purple!75!black]
\begin{definition}
An \underline{\textbf{exact category}} is an additive category $\mathcal{M}$ equipped with a class $\mathcal{E}$ of short exact sequences of the form
\[\begin{tikzcd}
    M'\arrow[r,"i"]&M\arrow[r,"j"]&M''
    \end{tikzcd}\]
where the first arrow $i$ is denoted an \underline{\textbf{admissible monomorphism}}, and the second arrow $j$ is denoted an \underline{\textbf{admissible epimorphism}}. In addition, the class $\mathcal{E}$ also satisfies the following properties: \\ 
\begin{enumerate}
    \item (closed under trivial extension) For any $M',M''$ in $\textrm{ob}(\mathcal{M}),$ the SES  \[
        \begin{tikzcd}
        M'\arrow[r,"{(id,0)}"]&M'\oplus M''\arrow[r,"pr_2"]&M''
        \end{tikzcd}
        \]
   is in $\mathcal{E}$.
   \item The class of admissible epimorphism is closed under composition. Dually for admissible monomorphisms. 
   \item (closed under base-change) If $M\to M''$ is an admissible epimorphism and given $N\to M''$, the pullback square exists
   \[\begin{tikzcd}
    N\times_{M''}M\arrow[r]\arrow[d,"p"]&M\arrow[d]\\
    N\arrow[r]&M''
   \end{tikzcd}\]
   and the morphism $p$ is an admissable epimorphism. Dually for admissible monomorphisms. 
   \item (admissible epimorphism is "epimorphism") Let $M\to M''$ be a map possessing a kernel in $\mathcal{M}$. If there exists a map $N\to M$ in $\mathcal{M}$ such that $N\to M\to M''$ is an admissible epimorphism, then $M\to M''$ is an admissible epimorphism. Dually for admissible monomorphisms. 
\end{enumerate}
\end{definition}
\end{tcolorbox}


This is where we want to do $K$-theory. The motivation for exact catgories is the following scenario: consider any additive category $\mathcal{M}$ embedded as a full subcategory of an abelian category $\mathcal{A}$. Suppose further that $\mathcal{M}$ is closed under taking extensions in $\mathcal{A}$, meaning if \[\begin{tikzcd}
0\arrow[r]&A\arrow[r,"f"]&B\arrow[r,"g"]&C\arrow[r]&0
\end{tikzcd}\]
is exact in $\mathcal{A}$ and $A,C$ is in $\mathcal{M}$, then $B$ is also in $\mathcal{M}$. Then $\mathcal{M}$ can be readily verified to be an exact category, with $\mathcal{E}$
being the class of sequences in $\mathcal{M}$ that is short exact in $\mathcal{A}$. The only non-trivial thing to check is axiom $3$, which is a standard theorem regarding pullbacks. We now have a wealth of examples in algebra/algebraic geometry:


\begin{tcolorbox}[colback=yellow!5!white,colframe=yellow!30!white]
\begin{example}
The category $\textbf{P(R)}$ of finitely generated projective modules over a commutative ring $R$ is exact by its embedding in $\textbf{RMod}$. Note that it is generally not abelian due to lacking of kernel/cokernels. For example $\textbf{P}(\mathbb{Z})$ is not abelian, since $\mathbb{Z}\xrightarrow{\times 2}\mathbb{Z}$ does not have a cokernel. 
\end{example}
\end{tcolorbox}

\begin{tcolorbox}[colback=yellow!5!white,colframe=yellow!30!white]
\begin{example}
The category $\textbf{VB}(X)$ over a paracompact space $X$ is exact by its embedding in the category of family of vector spaces over $X$. It is also generally not abelian due to lacking kernels. 
\end{example}
\end{tcolorbox}



\begin{tcolorbox}[colback=purple!5!white,colframe=purple!75!black]
\begin{definition}
An exact functor $F: M\to M'$ between exact categories is an additive functor preserving exact sequences.
\end{definition}
\end{tcolorbox}

\section{The $Q$-construction and Recovery of $K_0$}
To build $K$-theory on a exact category such as $\textbf{P}(\textbf{R})$, we have to go through an intermediate category, which is called the $Q$-construction. 
\begin{tcolorbox}[colback=purple!5!white,colframe=purple!75!black]
\begin{definition}
Given an exact category $\mathcal{M}$, let the category $Q\mathcal{M}$ have the same objects as $\mathcal{M}$, and morphisms from $M$ to $M'$ being isomorphisms classes of diagrams of the form 
\[M\twoheadleftarrow N\rightarrowtail M' \]
where $\twoheadleftarrow$ signifies an admissible epimorphism and $\rightarrowtail$ an admissible monomorphism in $\mathcal{M}$. An isomorphism between diagrams of the form is one that induces identity on both $M$ and $M'$. Composition of a morphisms is given by the pullback 
\[
\begin{tikzcd}
N\times_{M'}N'\arrow[r,rightarrowtail,"pr_2"]\arrow[d,twoheadrightarrow,"pr_1"]&N'\arrow[d,twoheadrightarrow,"j'"]\arrow[r,rightarrowtail,"i'"]&M''\\
N\arrow[d,twoheadrightarrow,"i"]\arrow[r,rightarrowtail,"i"]&M'\\
M
\end{tikzcd}
\]
\end{definition}
\end{tcolorbox}
It is clear that the composition is associative, and we have a well-defined category. Here are a few preliminary observations: 1. the classifying space $BQ\mathcal{M}$ is canonically a CW complex, and it is path-connected by the existence of a zero object, denoted $0$, in $\mathcal{M}$. 2. If $i:M'\rightarrowtail M$ is an admissible monomorphism, then it induces a morphism in $Q\mathcal{M}$ denoted by $i_!$ given by 
\[M'=M'\rightarrowtail M\] 
which will be referred to as \underline{\textbf{injective maps}}. Dually, If $j:M''\twoheadrightarrow M$ is an admissible epimorphism, then it induces a morphism in $Q\mathcal{M}$ denoted by $j^!$ given by 
\[M\twoheadleftarrow M''=M''\] 
which are called \underline{\textbf{surjective maps}}. Note the superscript/subscript follows the contravariant/covariance convention. Then, each morphism $u$ in $Q\mathcal{M}$ is the composition of $i_!\circ j^!$ for some $i$ and $j$ in $\mathcal{M}$ (check the pullback diagram). We will abuse notation onwards and use the same arrows corresponding to admissible monomorphism/epimorphism to denote their induced maps when clear.



As of now, the structure of the intermediate category $Q$ seems murky. We will motivate the definitions by proving it is the universal construction that recovers then well-accepted definition for $K_0$.



\begin{tcolorbox}[colback=purple!5!white,colframe=purple!75!black]
\begin{definition}
    given an exact category $\mathcal{M}$, $K_0(\mathcal{M})$ is defined to be the abelian group generated by $[C]$, one for each object $C$ in $\mathcal{M}$, subjected to the relations $[B]=[A]+[C]$ whenever there is an short exact sequence $A\to B\to C$. 
\end{definition}
\end{tcolorbox}
In some sense, $K_0$ "breaks up" short exact sequences, forcing them to split. We now examine how the $Q$-construction break up short exact sequences as well: recall the fact that every short exact sequence $A\rightarrowtail B\twoheadrightarrow C$ is equivalent to the bicartesian diagram
\[\begin{tikzcd}
A\arrow[r,rightarrowtail]\arrow[d,twoheadrightarrow]& B\arrow[d,twoheadrightarrow]\\
0\arrow[r,rightarrowtail]&C
\end{tikzcd}\]
On the other hand, we have the following proposition regarding bicartesian squares in $Q\mathcal{M}$, which occurs in definition for composition.

\begin{tcolorbox}[colback=blue!5!white,colframe=blue!30!white]
\begin{proposition}
    Given a bicartesian square in $\mathcal{M}$
    \[\begin{tikzcd}
    N\arrow[r,rightarrowtail,"i"]\arrow[d,twoheadrightarrow,"j"]& M'\arrow[d,twoheadrightarrow,"j'"]\\
    M\arrow[r,rightarrowtail,"i'"]&N'
    \end{tikzcd}\]
we have $i_!j^!=j'^!i'_!$ in $Q\mathcal{M}$
\end{proposition}
\end{tcolorbox}
The proof is simply tracing through the definitions. By the proposition, every short exact sequence $A\rightarrowtail B\twoheadrightarrow C$ in $\mathcal{M}$ leads to the equivalence between the morphisms $0\twoheadleftarrow A\rightarrowtail B$ and $0 \rightarrowtail C\twoheadleftarrow B $ in $Q\mathcal{M}$. We will see how this equivalence is exactly the splitting of exact sequences we desire in the proof of the following theorem:

\begin{tcolorbox}[colback=red!5!white,colframe=red!30!white]
\begin{theorem}
$\pi_1(BQ\mathcal{M},0)\cong K_0(\mathcal{M})$
\end{theorem}
\end{tcolorbox}

Before proving the theorem, we first introduce the following lemma regarding trees.

\begin{tcolorbox}
\begin{lemma}
    Suppose $T$ is a maximal tree in a small connected category $C$. Then, $\pi_1(BC)$ is the group generated by $[f]$, one for each morphism not in $T$, modulo the relations that \begin{enumerate}
        \item  $[t]=1$ for every $t\in T$, and $[Id_c]=1$ for each identity morphism.
        \item $[f]\cdot [g]=[f\circ g]$ for every composable $f,g$.
    \end{enumerate}
\end{lemma}
\end{tcolorbox}
\begin{proof}
  Let $X$ be the $1$ skeleton of $BC$, which is a connected graph with maximal tree $T$.  The proof for $\pi_1(X)$ being the free group generated by $X-T$ is given in Hatcher proposition $1A.2$. The lemma is then a direct application of Van-Kampen's Theorem. 
\end{proof}


\begin{proof}[Proof of Theorem $2.1$]
    We construct the isomorphism directly, which follows Weibel and slightly diverts from the original approach given by Quillen. Let $T$ be the set of injective morphisms of the form $0\rightarrowtail A$ in $Q\mathcal{M}$. Clearly, $T$ contains all vertices (objects) in $Q\mathcal{M}$ and is thus a maximal tree. 

    
    
    Let $B'\rightarrowtail B$ be an injective morphism. Note that its left composition with $0\rightarrowtail B'$ is the morphism induced by $0\rightarrowtail B'\rightarrowtail B$, which is in $T$. Thus by lemma $2.2$, all injective morphisms correspond to the identity element; given a surjective morphism $A\twoheadleftarrow A'$, note that its left composition with $0 \twoheadleftarrow A$ induces a surjective map $0\twoheadleftarrow A'$, so we have the relation $[A\twoheadleftarrow A']=[0\twoheadleftarrow A]^{-1}[0\twoheadleftarrow A']$. By the observation that every morphism in $Q\mathcal{M}$ factors as a surjective morphisms followed by a injective one, we note that $\pi_1(BQ\mathcal{M})$ is generated by classes of the form $[0\twoheadleftarrow A]$.

    By the observation following proposition $2.0.1$, each short exact sequence $A\rightarrowtail B\twoheadrightarrow C$ in $\mathcal{M}$ leads to the equivalence between the morphisms $0\twoheadleftarrow A\rightarrowtail B$ and $0 \rightarrowtail C\twoheadleftarrow B $ in $Q\mathcal{M}$, which in turn induces in the additivity relation in $\pi_1$
    \[[0\twoheadleftarrow C][0\twoheadleftarrow A]=[0\twoheadleftarrow B]=[0\twoheadleftarrow A][0\twoheadleftarrow C]\]

    The composition rule follows the additivity relation, and the map $[0\twoheadleftarrow A]\mapsto [A]$ is the desired isomorphism between $\pi_1(BQ\mathcal{M})$ and $K_0(\mathcal{M})$.
\end{proof}


We can now define $K$-theory for exact categories.


\begin{tcolorbox}[colback=purple!5!white,colframe=purple!75!black]
\begin{definition}
For a small exact category $\mathcal{M}$, let $K\mathcal{M}$ be the loop space $\Omega BQ\mathcal{M}$, and set
\[K_i(\mathcal{M}):=\pi_{i+1}(BQ\mathcal{M},0)=\pi_i(K\mathcal{M},0)\]
\end{definition}
\end{tcolorbox}
For skeletally small categories, we define its $K$-groups to be those of its skeleton. It is not hard to see that $K_i$ is a functor from the category of small exact categories and exact functors to $\textbf{Ab}$, noting that isomorphic functors induce isomorphism of $K$-groups by the following proposition

\begin{tcolorbox}[colback=blue!5!white,colframe=blue!30!white]
\begin{proposition}
A natural transformation $\theta: f\to g$ of functors from $C$ to $C'$ induces an homotopy $BC\times I\to BC'$ between $Bf$ and $Bg$. In particular, if a functor has a left/right adjoint, then it induces a homotopy equivalence between classifying spaces. 
\end{proposition}
\end{tcolorbox}
\begin{proof}
    The key is to realize the data of the triple $(f,g,\theta)$ is exactly a functor $C\times 1\to C$, where $1$ is the ordered set $\{0<1\}$ with classifying space the unit interval.
\end{proof}

With a bit more effort, one can show that $K_i$ commutes with filtered colimits and products. 





\printbibliography
\end{document}