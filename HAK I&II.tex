\documentclass{article}
\usepackage[utf8]{inputenc}
\usepackage{amsmath}
\usepackage{amsfonts}
\usepackage{amssymb}
\usepackage{tikz}
\usepackage{fullpage}
\usepackage{tikz-cd}
\usepackage{spectralsequences}
\usepackage{adjustbox}
\usepackage{xfrac}
\usepackage{tcolorbox}
\usepackage{xcolor}
\usepackage[backend=biber,style=alphabetic]{biblatex}
\usepackage{graphicx}
\graphicspath{ {D:/Chrome Downloads./} }
\addbibresource{HAK.bib}
\usepackage[parfill]{parskip}
\usepackage{amsthm}
\theoremstyle{definition}
\newtheorem{theorem}{Theorem}[section]
\theoremstyle{definition}
\newtheorem{definition}{Definition}[theorem]
\theoremstyle{definition}
\newtheorem{remark}{Remark}[theorem]
\theoremstyle{definition}
\newtheorem{proposition}{Proposition}[theorem]
\theoremstyle{definition}
\newtheorem{lemma}[theorem]{Lemma}
\theoremstyle{definition}
\newtheorem{corollary}{Corollary}[theorem]
\theoremstyle{definition}
\newtheorem{example}{Example}[theorem]
\tikzset{curve/.style={settings={#1},to path={(\tikztostart)
    .. controls ($(\tikztostart)!\pv{pos}!(\tikztotarget)!\pv{height}!270:(\tikztotarget)$)
    and ($(\tikztostart)!1-\pv{pos}!(\tikztotarget)!\pv{height}!270:(\tikztotarget)$)
    .. (\tikztotarget)\tikztonodes}},
    settings/.code={\tikzset{quiver/.cd,#1}
        \def\pv##1{\pgfkeysvalueof{/tikz/quiver/##1}}},
    quiver/.cd,pos/.initial=0.35,height/.initial=0}
\title{Higher Algebraic K-Theory}
\author{David Zhu}

\begin{document}
\maketitle
$K$-Theory started with Grothendieck's $K_0$ and projective modules/vector bundles. In this note, we present the various constructions for higher $K$-groups.
\section{A Little about $K_0$}
The first $K$ group, $K_0$, was constructed by Grothendieck in his generalization of the Riemann-Roch Theorem. The original purpose of the $K_0$ group gives a universal way to assign invariants to vector bundles over an algebraic variety that is compatible with exact sequences. A quick and dirty definition for the Grothendieck group for a commutative ring can be formulated in the following way:


\begin{tcolorbox}[colback=purple!5!white,colframe=purple!75!black]
\begin{definition}
    Given a commutative ring $R$, we define $K_0(R)$ to be the free group on the isomorphism classes of finitely generated projective $R$-modules modulo the relations $[P]+[Q]=[P\oplus Q]$.
\end{definition}
\end{tcolorbox}

In full generality, the Grothedieck group $K_0$ is actually the construction of the universal abelian group from an abelian monoid.


\begin{tcolorbox}[colback=purple!5!white,colframe=purple!75!black]
\begin{definition}
    The group completion of an abelian monoid $M$ is an abelian group $M^{-1}M$ together with a monoid morphism $M\xrightarrow{f}M^{-1}M$ that is universal in the sense that the following triangle commutes 
    \[
\begin{tikzcd}
M\arrow[rr,"f"]\arrow[dr,swap,"\alpha"]&&M^{-1}M\arrow[dl,dashrightarrow,"\exists !\alpha'"]\\
&A
\end{tikzcd}
\]
\end{definition}
\end{tcolorbox}



We may check that this definition of $K_0(R)$ actually gives us the group completion of the set of isomorphism classes of projective $R$-modules, which forms a abelian monoid. In fact, we can make $K_0(R)$ into a ring with the tensor product operation, provided that $R$ is commutative as then the tensor product distributes over the direct sum. Using this definition, we may compute an easy class of examples

\begin{tcolorbox}[colback=yellow!5!white,colframe=yellow!30!white]
\begin{example}
    If $R$ is a PID, then $K_0(R)=\mathbb{Z}$.
\end{example}
\end{tcolorbox}
It is a standard result from ring theory that finitely generated projective modules over a PID are actually free. Thus, the isomorphism classes are characterized by the rank of the free modules, which are natural numbers. The group completion of $\mathbb{N}$ is then $\mathbb{Z}$. \\


We can also construct an example where $K_0(R)$ is not a finitely generated abelian group. 

\begin{tcolorbox}[colback=yellow!5!white,colframe=yellow!30!white]
\begin{example}
    Let $M_k(F)$ denote the ring of $k\times k$ matrices with entries in a field $F$. Let $R$ be the direct limit of the inclusion 
    \[
    M_{1!}(F)\hookrightarrow M_{2!}(F)\hookrightarrow M_{3!}(F)...
    \]
    where for each inclusion $M_{n!}(F)\hookrightarrow M_{n+1!}(F)$, we view $M_{n+1!}(F)$ as the tensor product $M_{n!}(F)\otimes M_{n+1}(F)$, and the inclusion is given by $M\mapsto M\otimes \textbf{id}_{n+1}$. 
    It is possible to show that $R$ has a rank function $\rho$ such that $\rho: R\xrightarrow{g}[0,1]$ has image $\mathbb{Q}\cap [0,1]$. Then, it is not hard to see that $K_0(R)\cong \mathbb{Q}$.
\end{example}
\end{tcolorbox}


After the invention of the Grothendieck group, Atiyah and Hirzbruch were able to transport $K_0$ to topology: to each topological space $X$, they constructed a sequence of groups $K_n$ associated to the $n$-vector bundles on $X$, which forms a extraordinary cohomology theory. The discipline is now know as topological $K$-theory. However, it turned out that the generalization of higher dimensional $K$ groups were much more difficult on the algebraic side(for rings and varieties), and it was Quillen's plus construction that brought about the first satisfactory definition of higher algebraic $K$-theory.

\section{A Little about $K_1$}
Let $R$ be a associative ring with unit. We may define the inclusion $GL(n,R)\hookrightarrow GL(n+1,R)$ by turning a matrix $M\in GL(n,R)$ into a upper left block matrix with a $1$ on the $(n,n)$ entry. Then, we define $GL(R)$ to be the union of the infinite chain of inclusions of the general linear groups with entries in $R$.

\begin{tcolorbox}[colback=purple!5!white,colframe=purple!75!black]
\begin{definition}
    We define $GL(R)$ to be the direct limit of the inclusion
    \[
GL(1,R)\hookrightarrow GL(2,R)\hookrightarrow GL(3,R)\hookrightarrow GL(4,R)\hookrightarrow ...
\] in $\textbf{Grp}$.
\end{definition}
\end{tcolorbox}

We then can define $K_1(R)$:

\begin{tcolorbox}[colback=purple!5!white,colframe=purple!75!black]
\begin{definition}
    For any associated unital ring $R$, $K_1(R)$ is the abelian group $GL(R)/[GL(R),GL(R)]$, where $[GL(R),GL(R)]$ is the commutator subgroup of $GL(R)$. 
\end{definition}
\end{tcolorbox}
From basic group theory, we know that every homomorphism from $GL(R)$ to an abelian group factors through $K_1(R)$, so such universal properties matches that of $K_0$. Moreover, it is easier to see that $K_1$ is functorial, such that each ring map $R\xrightarrow{}S$ induces a group homomorphisms $GL(R)\xrightarrow{}GL(S)$, which desceneds to a morphisms $K_1(R)\xrightarrow{}K_1(S)$.

The development of $K_1$ was attributed to Bass and Schanuel, whose motivation came largely from the analogue in topological $K$-theory and Whitehead's work in simple homotopy type.

\section{The Plus Construction}
In general, Quillen's Plus Construction is a way to modify the fundamental group of a given topological space while not altering its homology. 

\begin{tcolorbox}[colback=purple!5!white,colframe=purple!75!black]
\begin{definition}
    If $X$ is a connected CW complex, with $P$ a perfect subgroup of its fundamental group, then an acyclic map $f:X\xrightarrow{}X^{+}$ is called a plus construction (relative to $P$) if the kernel of the induced map on fundamental groups is the normal subgroup generated by $P$. 
\end{definition}
\end{tcolorbox}
The acyclic condition on the homotopy fiber is equivalent to the condition that the map induces isomorphisms on all local coefficient systems, which we show in the proposition below. 

\begin{tcolorbox}[colback=blue!5!white,colframe=blue!30!white]
\begin{proposition}
    Let $X$ and $Y$ be connected CW complexes. Then, a map $f:X\xrightarrow{}Y$ is acyclic if and only if they induces isomorphism on homology with coefficients $M$ a $\pi_1(Y)$ module.
\end{proposition}
\end{tcolorbox}

\begin{proof}
We may turn $f:X\xrightarrow{}Y$ into its pathspace fibration first. Suppose $f$ is acyclic, then by definition its homotopy fiber $F_f$ is an acyclic space and has trivial reduced homology. So by universal coefficient theorem, we have $\Tilde{H}_i(X;M)=0$ and $H_0(X;M)=M$. Thus, we see that the $E_2$ page of the Leray-Serre spectral sequence of the fibration $F_f\hookrightarrow X\xrightarrow{}Y$ stabilizes and converges to the homology of $X$. In other words, we have
\[
H_i(X;M)\cong H_i(Y;M)
\]
for all $i$.
For the converse direction, we first assume that $\pi_1 (Y)=0$ and $H_*(X;\mathbb{Z})\cong H_*(Y;\mathbb{Z})$. Then by the comparison theroem for Serre spectral sequence associated to the fibrations $F_f\hookrightarrow{}X\xrightarrow{}Y$ and $*\hookrightarrow Y\xrightarrow{=}Y$, we have $\Tilde{H}_*(F_f;\mathbb{Z})=0$, so that $f$ is acyclic. For the general case, we may lift everything into the universal cover. Let $\Tilde{Y}$ be the universal cover for $Y$. Then, the pullback $f^*X:=X\times_Y \Tilde{Y}$ is a covering space for $X$. We also have the natural isomorphisms $H_*(X;\mathbb{Z}[\pi_1(Y)])\cong H_*(f^*X;\mathbb{Z})$ and $H_*(Y;\mathbb{Z}[\pi_1(Y)])\cong H_*(\Tilde{Y};\mathbb{Z})$. By naturality, the induced map $\title{f}:f^*X\xrightarrow{}\Tilde{Y}$ induces isomorphism on integral homology. Since $\pi_1(\Tilde{Y})$ is trivial, the homotopy fiber $F_{\Tilde{f}}$ is acyclic. Moreover, by path-lifting the homotopy fibers $F_f$ and $F_{\Tilde{f}}$ are homotopic. Thus, $F_f$ is also acyclic, and we are done. 
\end{proof}
We now introduce one possible model for the plus construction. First, we may show a simpler scenario where $H_1(X)=0$. 

\begin{tcolorbox}[colback=blue!5!white,colframe=blue!30!white]
\begin{proposition}
    Let $X$ be a connected CW complex with $H_1(X)=0$. Then, there exists a simply-connected CW complex $X^+$ and a map $X\xrightarrow{}X^+$ inducing isomorphisms of all homology groups.
\end{proposition}
\end{tcolorbox}
\begin{proof}
We may choose loops $\phi_i:S^1\xrightarrow{}X^1$ that generate $\pi_1(X)$ and use these to attach two cells $e_i^2$ to kill the fundamental group. Call the resulting space $X'$. Then, we have the homology LES of the pair $(X',X)$
\[
\begin{tikzcd}
0\arrow[r]&H_2(X)\arrow[r]&H_2(X')\arrow[r]&H_2(X',X)\arrow[r]&0
\end{tikzcd}
\]
Note that this short exact sequence splits, since $H_2(X'X)$ is free with basis $e_i^2$ by cellular homology. Since $X'$ is simply connected, Hurewicz theorem gives us $H_2(X')\cong \pi_2(X')$, so that we can write each basis $e_i^2$ of $H_2(X',X)$ by a map $\psi_i:S^2\xrightarrow{}X'$. We may assume that these maps are cellular by cellular approximation theorem. Then, we can attach three cells accordingly to form a simply-connected CW complex $X^+$, such that the inclusion $X\hookrightarrow X^+$ induces isomorphism on all homology groups. 
\end{proof}
Note that in the previous proposition, the condition $H_1(X)=0$ is equivalent to the fact that the the choice of the perfect normal subgroup is $\pi_1(X)$ itself, so that the abelianization is trivial. For the general case, we again use the trick to lift to covering spaces. 

\begin{tcolorbox}[colback=red!5!white,colframe=red!30!white]
\begin{theorem}
    Every connected CW complex $X$ has a plus constrcution $X\xrightarrow{}X^+$ relative to any choice of perfect subgroup $P$. Moreover, such construction is unique up to homotopy.
\end{theorem}
\end{tcolorbox}
\begin{proof}
Let $p:\Tilde{X}\xrightarrow{}X$ be the covering space corresponding to $P$, so that $\pi_1(\Tilde{X})=P$. From the previous proposition, we get an inclusion $f:\Tilde{X}\xrightarrow{}\Tilde{X}^+$ inducing isomorphisms on all homology groups. Then let $X^+$ be the disjoint union of $\Tilde{X}$ and the mapping cylinder $M_p$ by identifying the copies of $\Tilde{X}$. We then have the commutative diagram
\[
\begin{tikzcd}
\Tilde{X}\arrow[r,hookrightarrow]\arrow[d,swap,"p"]&\Tilde{x}^+\arrow[d,hookrightarrow]\\
X\cong M_p\arrow[r,hookrightarrow]&X^+
\end{tikzcd}
\]
By Van-Kampen theorem, the induced map $\pi_1(X)\xrightarrow{}\pi_1(X^+)$ is surjective with kernel generated by $P$. Moreover, $X^+/M_p$ is homeomorphic to $\Tilde{X}^+/\Tilde{X}$, so we have $H_*(X^+,M_p)\cong H_*(\Tilde{X}^+,\Tilde{X})=0$. By the LES of the pair $(X^+,X)$, we have that $X\hookrightarrow X^+$ induces isomorphisms on all homology groups. The uniqueness(up to homotopy) of the above construction is a direct result from obstruction theory: given an abelian space $Y$ such that $X\xrightarrow{} Y$ is a plus construction relative to the same perfect subgroup, we consider the lifting problem
\[
\begin{tikzcd}
X\arrow[d,hookrightarrow]\arrow[r]&Y\\
X^+\arrow[ur,dashrightarrow]
\end{tikzcd}
\]
By assumption, we have $\pi_1(Y)\cong \pi_1(X^+)=G$, and $H_n(Y;M)\cong H_n(X^;M)$ for any $G$-module. In particular, the higher homotopy groups of $Y$ are naturally $G$-modules, so we have the isomorphism $H_n(X;\pi_n(Y))\cong H_n(Y;\pi_n(Y))\cong H_n(X^;\pi_n(Y))$ for all $n$. An application of the naturality of universal coefficient theorem and the five lemma also gives us the isomorphism on cohomology $H^n(X;\pi_n(Y))\cong H^n(X^+;\pi_n(Y))$. By the LES, we know that $H^n(X,X^+;\pi_n(Y))$ should vanish, and the obstruction theorem gives us the unique dashed lift. In particular, the lift induces isomorphism on $\pi_1$ and all homology groups on universal covers, which must be a homotopy equivalence.(Hatcher Ex 4.2.12). Moreover, we also see that the plus construction is functorial from the $\textbf{Top}_*$ to $\textbf{HoTop}_*$.
\end{proof}

\section{The Higher $K$-Groups}
We are now ready to define the higher $K$-groups, first proposed by Quillen. First, we shall consider the space $BGL(R)$, which is the classifying space for the group $GL(R)$, defined in section $2$. Using the plus construction with respect to the commutator subgroup of $\pi_1(BGL(R))=GL(R)$, we have a map $BGL(R)\xrightarrow{}BGL(R)^+$ that satisfies the two following properties:
\begin{enumerate}
    \item $\pi_1(BGL(R)^+)\cong K_1(R)$
    \item $H_*(BGL(R);M)\cong H_*(BGL(R)^+;M)$ for every $K_1(R)$-module $M$
\end{enumerate}
Quillen's originally defined $K_n(R)$ to be the homotopy groups $\pi_n(BGL(R)^+)$. However by the path-connectedness of the plus space, we cannot recover the original definition of $K_0(R)$, as we saw it can be not even finitely generated in example $1.2$. The better solution was proposed by Quillen as the new $Q$-construction, where he moved most of the construction to abelian categories. However, there is a way to sidestep this and give a definition for $K_n(R)$ for all $n\geq 0$

\begin{tcolorbox}[colback=purple!5!white,colframe=purple!75!black]
\begin{definition}
    Let $K(R)$ be the product space $K_0(R)\times BGL(R)^+$, where $K_0(R)$ is given the discrete topology. We then define $K_n(R):=\pi_n(K(R))$ for all $n\geq 0$.
\end{definition}
\end{tcolorbox}
We see from definition that $K_0(R)\cong \pi_0(K(R))$, and $\pi_n(K(R))$ agrees with the definition of $K_n(R)$ above. We are happy to see that $K_n$ are functorial as it is the composition of multiple functors.

\begin{tcolorbox}[colback=green!5!white,colframe=green!30!white]
\begin{corollary}
    Each $K_n$ is a functor from $\textbf{Ring}$ to $\textbf{Ab}$.
\end{corollary}
\end{tcolorbox}

The important calculation that Quillen gave was the calculation of the algebraic $K$-theory of finite fields, which we list as an unproved theorem below:

\begin{tcolorbox}[colback=red!5!white,colframe=red!30!white]
\begin{theorem}
    (Quillen) Let BU be the classifying space of the direct limit of $U(n)$. There is a fibration
    \[
    BGL(F_q)^+\xrightarrow{}BU\xrightarrow{\psi^q-1}BU
    \]
    where $\psi^q-1$ is the Adams operation. The LES of homotopy groups then gives the isomorphism
    \[
    K_n(F_q)=\pi_n(BGL(F_q))^+\cong
    \begin{cases}
    \mathbb{Z}/(q^n-1)& n=2i-1\\
    0& n \textup{ even}
    \end{cases}
    \]
\end{theorem}
\end{tcolorbox}
\section{The Group Completion Theorem }
First, we recall Grothendieck's definition of $K_0$ for an abelian monoid $M$
\begin{tcolorbox}[colback=purple!5!white,colframe=purple!75!black]
    \begin{definition}
        Let $M$ be an abelian monoid. Then, the Grothendieck group $K_0(M)$ is an abelian group $K$ with the inclusion map $i: M\to K$ satisfying the following universal property: given an abelian group $A$ and a monoid morphism $f: M\to A$, we have the factorization
        \[\begin{tikzcd}
        M\arrow[d,"i"]\arrow[r," f"]&A\\
        K\arrow[ur,dashed,swap,"\exists! g"]
        \end{tikzcd}\]
    \end{definition}
    \end{tcolorbox}
It is an easy exercise to show that $K$ is unique up to isomorphism, which we will denote by $K_0(M)$. Explicitly, we can obtain $K_0(M)$ from the following "group completion" construction: 

\begin{tcolorbox}[colback=blue!5!white,colframe=blue!30!white]
\begin{proposition}
Given an abelian monoid $K_0(M)$ is the abelian group generated by $[m]$ for each $m\in M$, modulo the relation $[x+y]-[x]-[y]$.
\end{proposition}
\end{tcolorbox}


\begin{tcolorbox}[colback=yellow!5!white,colframe=yellow!30!white]
\begin{example}
($K_0$ of a ring) Given a ring $R$, $K_0(R)$ is defined to be Grothendieck group over the abelian monoid of the isomorphim class of finitely generated projective modules over $R$, with monoid operation given by direct sum. 
\end{example}
\end{tcolorbox}

If $M$ is a topological monoid, let $BM$ be its classifying space (viewing $M$ as a category with one object). Then, there is a natural map $M\to \Omega BM$, with $\pi_0(\Omega BM)=\pi_1(BM)$ an abelian group. When $\pi_0$ is a group, it can be shown that $M\to \Omega BM$ is a homotopy equivalence. Thus the map is referred to as the group completion of a toplogical monoid. When $\pi_0$ is not necessarily group, many can still be said: since $M$ is an $H$-space, $H_*(M)$ is naturally a ring, and $H_0(M)=\mathbb{Z}[\pi_(M)]$. Viewing $\pi_0(M)$ as a multiplicative subset of $H_0(M)$, the induced map $H_*(M)\to H_*(\Omega BM)$ sends $\pi_0(M)$ to units. Mcduff and Segal proved the following result:


\begin{tcolorbox}[colback=red!5!white,colframe=red!30!white]
\begin{theorem}
If $\pi:=\pi_0(M)$ is in the center of $H_*(M)$, then 
\[H_*(M)[\pi^{-1}]\cong H_*(\Omega BM)\]
\end{theorem}
\end{tcolorbox}
We will outline the idea of the proof following the original paper \cite{MS}, as it provides many insights to the motivate the $S^{-1}S$-construction.

The goal is to  find an intermediate space $M_{\infty}$ with presribed homology $H_*(M_{\infty})= H_*(M)[\pi^{-1}]$ and a homology equivalence $M_{\infty}\to \Omega BM$. The first step uses the Quillen's lemma, given in \cite{QGc}, regarding localization:

\begin{tcolorbox}
\begin{lemma}
Let $R$ be a ring (not necessarily commutative) and $S$ a multiplicative subset. Let $C$ be the category with objects elements of $S$ and morphisms from $s_1$ to $s_2$ an element $t\in S$ such that $s_1t=s_2$. Under the conditions that $C$ is filtered, there is a canonical $R$-module morphism 
\[u:\varinjlim_{C} R \to R[S^{-1}] \]
where the filtered colimit is defined on objects by the inclusion map, and on morphisms by right multiplication by $t$. If $S$ acts on $R$ by left multiplication bijectively, meaning
\begin{enumerate}
    \item (injective) Given $r\in R$ and $s\in S$ such that $sr=0$, then there exists $t\in S$ with $rt=0$.
    \item (bijective)Given $r\in R$ and $s\in S$, there exists $r'\in R$ and $t\in S$ such that $sr'=rt$
\end{enumerate}
Then $u$ is an isomorphism. 
\end{lemma}
\end{tcolorbox}
A simple example is colimit $\mathbb{Z}\xrightarrow{\times p}\mathbb{Z}\xrightarrow{\times p}...$ being isomorphic to $\mathbb{Z}[1/p]$ as $\mathbb{Z}$-modules. The hypothesis of the lemma is satisfied when $\pi$ is in the center of $H_*(M)$, and we can define $M_{\infty}$ to be the mapping telescope given by $M\xrightarrow{\times m}M\xrightarrow{\times m}...$, where $m$ is any arbitrary element in the component of $1\in \pi_0(M)$. Since homology commutes with filtered colimits, $M_{\infty}$ will have the prescribed homology. 

Given a topological group $M$, we may construct the universal bundle $EM\to BM$ by vieweing $M$ as a topological category. If $M$ acts on a space $X$, let $X_M$ denote the associated bundle to the universal bundle $X_M\to BM$ with fiber $X$. The construction still holds when $M$ is only a monoid, and instead of the homotopy equivalent between the fiber and homotopy fiber, Mcduff and Segal recovers the following proposition
\begin{tcolorbox}[colback=blue!5!white,colframe=blue!30!white]
    \begin{proposition}
        If $M$ is a topological monoid which acts on a space $X$, and for each $m\in M$ the map $x\mapsto mx$ from $X$ to itself is a homology equivalence, then the map $X_M\to BM$ is a homology fibration with fibre $X$, meaning the canonical map between the fiber and homotopy fiber induces isomorphism on homology.
    \end{proposition}
    \end{tcolorbox}



By construction, $M$ acts on $M_{\infty}$ (which also induces an homology equivalence), and $(M_{\infty})_M$ is also contratible since it will be a filtered colimit of the contractible $M_M=EM$. Thus, we have a map $(M_{\infty})_{M} \to BM$ with fiber $M_{\infty}$ and homotopy fiber $\Omega BM$. The theorem then follows from proposition $1.2.1$ and more colimit nonsense on components for the general case. 


We now define group completion for general homotopy commutative, homotopy associative $H$-spaces.

\begin{tcolorbox}[colback=purple!5!white,colframe=purple!75!black]
\begin{definition}
Let $X$ be a homotopy commutative, homotopy associative $H$-space. A \underline{\textbf{group completion}} of $X$ is an $H$-space map $f: X\to Y$ such that $f$ induces the group completion on $\pi:=\pi_0$, and the isomorphism
\[H_*(X)[\pi^{-1}]\cong H_*(Y)\] 

\end{definition}
\end{tcolorbox}


\section{The $S^{-1}S$-construction}
Given a symmetric monoidal category $S$, Quillen constructs the category $S^{-1}S$ such that $B(S^{-1}S)$ is the group completion of $BS$ in the sense of definition $1.2.1$. The group completion recovers the plus construction, and gives another more natural definition for higher $K$-theory. We shall note that the construction below is closely related to the group completion construction in chapter $1$.

A left action of a monoidal category $S$ on a category $X$ is a functor $+: S\times X\to X$ with natural isomorphisms $A+(B+F)\cong (A+B)+F$ and $0+F\cong F$, where $A,B\in S$ and $F\in X$, satisfying the associativity coherence conditions. An action is call invertible if each translation $F\mapsto A+F$ is a homotopy equivalence. 
\begin{tcolorbox}[colback=purple!5!white,colframe=purple!75!black]
\begin{definition}
If $S$ acts on $X$ the category $\langle S,X \rangle$ has the same objects as $X$. A morphisms is represented by an isomorphisms class of the tuple $(s,sx\xrightarrow{\phi}t)$, where $\phi$ is a morphism in $X$. An isomorphism of tuples is given by an isomorphism $s\cong s'$, which induces the commutative diagram
\[\begin{tikzcd}
s+x\arrow[rr,"\cong"]\arrow[dr]&& s'+x\arrow[dl]\\
&t
\end{tikzcd}\]
\end{definition}
\end{tcolorbox}
We let $S^{-1}S:= \langle S, S\times S \rangle$, where $S$ acts on both facts of $S\times S$ by the monoidal operation. We also define an action of $S$ on $S^{-1}S$ by $s+(s_1,s_2)=(s_1,s+s_2)$. Note that this action is invertible: the translation $(s_1,s_2)\mapsto (s_1,s+s_2)$ has homotopy inverse $(s_1,s_2)\mapsto (s+s_1,s_2)$, since there is the natural transformation $(s_1,s_2)\mapsto (s+s_1,s+s_2)$. Quillen then proves 

\begin{tcolorbox}[colback=red!5!white,colframe=red!30!white]
\begin{theorem}
If every map in $S$ is an isomorphism, and the translations are faithful in $S$, then the functor $S\to S^{-1}S$ given by $x\mapsto (0,x)$ induces isomorphism
\[H_*(S)[\pi^{-1}]\cong H_*(S^{-1}S)\]
In particular, $BS^{-1}S$ is the group completion of $BS$.
\end{theorem}
\end{tcolorbox}
\begin{proof}
    It is not hard to show that the projection functor $S^{-1}S\to \langle S,S \rangle$ onto the second factor is cofibered with fiber $S$. Then, there is a spectral sequence \cite{Grayson} 
    \[E^2_{p,q}=H_p(\langle S,S \rangle, H_q(S))\Rightarrow H_{p+q}(S^{-1}S)\]
    Localizating at $\pi_0(S)$ and noting the contractibility of $\langle S,S\rangle$ gives the desired degeneration. 
\end{proof}

We now show that the plus construction for $BGL(R)$ is a group completion.

\begin{tcolorbox}[colback=red!5!white,colframe=red!30!white]
\begin{theorem}
Let $S=\coprod GL_n(R)$ be the category of free $R$-modules. Then, $BS^{-1}S$ is the group completion of $BS=\coprod BGL_n(R)$, and 
\[BS^{-1}S\cong \mathbb{Z}\times BGL(R)^{+}\]
\end{theorem}
\end{tcolorbox}
\begin{proof}
    We will construct a map $f:BGL(R)\to Y_S$, where $Y_S$ is the connected component of $BS^{-1}S$ containing the basepoint, such that $f$ induces isomorphism on integral homology. The universal property of the plus contruction then implies a homotopy equivalence $BGL(R)^{+}\to Y_S$.

    Pick the distinguished component $e\in \pi_0(BS)$ that represents $R$, so that the other components are represented by $e^n$ for some $n$. By the previous theorem, $H_*(BS^{-1}S)$ is $H_*(BS)$ localized at $\{e^n\}$. More explcitly, it is also the colimit of the map $H_*(BS)\to H_*(BS)$ induced by the translation $\oplus R:S\to S$. We now have an algebraic fact due to Quillen in his previous work \cite{QGc}

    
    \begin{tcolorbox}
    \begin{lemma}[Quillen]
    Suppose a unital ring $R$ is graded with respect to a multiplicative subset $S\subset R$, i.e 
    \[R=\oplus_{s\in S}R_s\]
    with $R_sR_{s'}\subset R_{ss'}$. Let $\overline{S}$ be the group completion of $S$. Then, $RS^{-1}$ is graded with respect to $\overline{S}$, and 
    \[\overline{R}_e\cong \varinjlim_{s}R_s\]
    where $\overline{R}_e$ represents the graded component that corresponds to $e\in \overline{S}$.
    \end{lemma}
    \end{tcolorbox}
In our case, we have $H_*(Y_S)\cong \varinjlim H_*(BGL_n(R))\cong H_*(BGL(R))$, as desired. 
\end{proof}

To transport the result to the category of finite generated projective modules, we need the a cofinality theorem: 


\begin{tcolorbox}[colback=purple!5!white,colframe=purple!75!black]
\begin{definition}
A monoidal functor $f: S\to T$ is called cofinal if for every object $t\in T$, there exists some $s\in \textrm{ob}S$ and $t'\in \textrm{ob} T$ such that $f(s)=t\otimes t'$.
\end{definition}
\end{tcolorbox}
Recall that the category of finitely generated free modules and the category of finitely generated projective modules and are both symmetric monoidal under the direct sum, and the inclusion functor is monoidal by a characterization of projective module being a direct summand of a free module. 



\begin{tcolorbox}[colback=red!5!white,colframe=red!30!white]
\begin{theorem}
Suppose $f: S\to T$ is cofinal, and $\textrm{Aut}_S(s)\cong \textrm{Aut}_T(f(s))$ for all $s\in \textrm{ob}S$, then the basepoint components of $BS^{-1}S$ and $BT^{-1}T$ are homotopy equivalent.  
\end{theorem}
\end{tcolorbox}
\begin{proof}
    Let $Y_S$ and $Y_T$ denote the respective basepoint components. Then 
    \[H_*(Y_S)\cong \varinjlim_{s\in S}H_*(B \textrm{Aut}_S(s))\cong \varinjlim_{s\in S}H_*(B \textrm{Aut}_T(f(s)))\cong \varinjlim_{t\in T}H_*(B \textrm{Aut}_T(t))\cong H_*(Y_T)\]
    where the third isomorphism is an easy exercise on general cofinal inductive limits.
\end{proof}
We have two immediate applications:
\begin{tcolorbox}[colback=green!5!white,colframe=green!30!white]
\begin{corollary}
Let $S$ be the category of finitely generated projective modules with morphisms isomorphisms. Then for $n\geq 1$, we have the isomorphism 
\[BS^{-1}S\cong K_0(R)\times BGL(R)^+\]
\end{corollary}
\end{tcolorbox}

\begin{tcolorbox}[colback=green!5!white,colframe=green!30!white]
\begin{remark}
A cofinal functor in general does not induce isomorphism on $K_0$, since the $K_0$ of the category of free $R$ modules must be $\mathbb{Z}$.
\end{remark}
\end{tcolorbox}


\section{Exact Category}


\begin{tcolorbox}[colback=purple!5!white,colframe=purple!75!black]
\begin{definition}
An \underline{\textbf{exact category}} is an additive category $\mathcal{M}$ equipped with a class $\mathcal{E}$ of short exact sequences of the form
\[\begin{tikzcd}
    M'\arrow[r,"i"]&M\arrow[r,"j"]&M''
    \end{tikzcd}\]
where the first arrow $i$ is denoted an \underline{\textbf{admissible monomorphism}}, and the second arrow $j$ is denoted an \underline{\textbf{admissible epimorphism}}. In addition, the class $\mathcal{E}$ also satisfies the following properties: \\ 
\begin{enumerate}
    \item (closed under trivial extension) For any $M',M''$ in $\textrm{ob}(\mathcal{M}),$ the SES  \[
        \begin{tikzcd}
        M'\arrow[r,"{(id,0)}"]&M'\oplus M''\arrow[r,"pr_2"]&M''
        \end{tikzcd}
        \]
   is in $\mathcal{E}$.
   \item The class of admissible epimorphism is closed under composition. Dually for admissible monomorphisms. 
   \item (closed under base-change) If $M\to M''$ is an admissible epimorphism and given $N\to M''$, the pullback square exists
   \[\begin{tikzcd}
    N\times_{M''}M\arrow[r]\arrow[d,"p"]&M\arrow[d]\\
    N\arrow[r]&M''
   \end{tikzcd}\]
   and the morphism $p$ is an admissable epimorphism. Dually for admissible monomorphisms. 
   \item (admissible epimorphism is "epimorphism") Let $M\to M''$ be a map possessing a kernel in $\mathcal{M}$. If there exists a map $N\to M$ in $\mathcal{M}$ such that $N\to M\to M''$ is an admissible epimorphism, then $M\to M''$ is an admissible epimorphism. Dually for admissible monomorphisms. 
\end{enumerate}
\end{definition}
\end{tcolorbox}


This is where we want to do $K$-theory. The motivation for exact catgories is the following scenario: consider any additive category $\mathcal{M}$ embedded as a full subcategory of an abelian category $\mathcal{A}$. Suppose further that $\mathcal{M}$ is closed under taking extensions in $\mathcal{A}$, meaning if \[\begin{tikzcd}
0\arrow[r]&A\arrow[r,"f"]&B\arrow[r,"g"]&C\arrow[r]&0
\end{tikzcd}\]
is exact in $\mathcal{A}$ and $A,C$ is in $\mathcal{M}$, then $B$ is also in $\mathcal{M}$. Then $\mathcal{M}$ can be readily verified to be an exact category, with $\mathcal{E}$
being the class of sequences in $\mathcal{M}$ that is short exact in $\mathcal{A}$. The only non-trivial thing to check is axiom $3$, which is a standard theorem regarding pullbacks. We now have a wealth of examples in algebra/algebraic geometry:


\begin{tcolorbox}[colback=yellow!5!white,colframe=yellow!30!white]
\begin{example}
The category $\textbf{P(R)}$ of finitely generated projective modules over a commutative ring $R$ is exact by its embedding in $\textbf{RMod}$. Note that it is generally not abelian due to lacking of kernel/cokernels. For example $\textbf{P}(\mathbb{Z})$ is not abelian, since $\mathbb{Z}\xrightarrow{\times 2}\mathbb{Z}$ does not have a cokernel. 
\end{example}
\end{tcolorbox}

\begin{tcolorbox}[colback=yellow!5!white,colframe=yellow!30!white]
\begin{example}
The category $\textbf{VB}(X)$ over a paracompact space $X$ is exact by its embedding in the category of family of vector spaces over $X$. It is also generally not abelian due to lacking kernels. 
\end{example}
\end{tcolorbox}



\begin{tcolorbox}[colback=purple!5!white,colframe=purple!75!black]
\begin{definition}
An exact functor $F: M\to M'$ between exact categories is an additive functor preserving exact sequences.
\end{definition}
\end{tcolorbox}

\section{The $Q$-construction and Recovery of $K_0$}
To build $K$-theory on a exact category such as $\textbf{P}(\textbf{R})$, we have to go through an intermediate category, which is called the $Q$-construction. 
\begin{tcolorbox}[colback=purple!5!white,colframe=purple!75!black]
\begin{definition}
Given an exact category $\mathcal{M}$, let the category $Q\mathcal{M}$ have the same objects as $\mathcal{M}$, and morphisms from $M$ to $M'$ being isomorphisms classes of diagrams of the form 
\[M\twoheadleftarrow N\rightarrowtail M' \]
where $\twoheadleftarrow$ signifies an admissible epimorphism and $\rightarrowtail$ an admissible monomorphism in $\mathcal{M}$. An isomorphism between diagrams of the form is one that induces identity on both $M$ and $M'$. Composition of a morphisms is given by the pullback 
\[
\begin{tikzcd}
N\times_{M'}N'\arrow[r,rightarrowtail,"pr_2"]\arrow[d,twoheadrightarrow,"pr_1"]&N'\arrow[d,twoheadrightarrow,"j'"]\arrow[r,rightarrowtail,"i'"]&M''\\
N\arrow[d,twoheadrightarrow,"i"]\arrow[r,rightarrowtail,"i"]&M'\\
M
\end{tikzcd}
\]
\end{definition}
\end{tcolorbox}
It is clear that the composition is associative, and we have a well-defined category. Here are a few preliminary observations: 1. the classifying space $BQ\mathcal{M}$ is canonically a CW complex, and it is path-connected by the existence of a zero object, denoted $0$, in $\mathcal{M}$. 2. If $i:M'\rightarrowtail M$ is an admissible monomorphism, then it induces a morphism in $Q\mathcal{M}$ denoted by $i_!$ given by 
\[M'=M'\rightarrowtail M\] 
which will be referred to as \underline{\textbf{injective maps}}. Dually, If $j:M''\twoheadrightarrow M$ is an admissible epimorphism, then it induces a morphism in $Q\mathcal{M}$ denoted by $j^!$ given by 
\[M\twoheadleftarrow M''=M''\] 
which are called \underline{\textbf{surjective maps}}. Note the superscript/subscript follows the contravariant/covariance convention. Then, each morphism $u$ in $Q\mathcal{M}$ is the composition of $i_!\circ j^!$ for some $i$ and $j$ in $\mathcal{M}$ (check the pullback diagram). We will abuse notation onwards and use the same arrows corresponding to admissible monomorphism/epimorphism to denote their induced maps when clear.



As of now, the structure of the intermediate category $Q$ seems murky. We will motivate the definitions by proving it is the universal construction that recovers then well-accepted definition for $K_0$.



\begin{tcolorbox}[colback=purple!5!white,colframe=purple!75!black]
\begin{definition}
    given an exact category $\mathcal{M}$, $K_0(\mathcal{M})$ is defined to be the abelian group generated by $[C]$, one for each object $C$ in $\mathcal{M}$, subjected to the relations $[B]=[A]+[C]$ whenever there is an short exact sequence $A\to B\to C$. 
\end{definition}
\end{tcolorbox}
In some sense, $K_0$ "breaks up" short exact sequences, forcing them to split. We now examine how the $Q$-construction break up short exact sequences as well: recall the fact that every short exact sequence $A\rightarrowtail B\twoheadrightarrow C$ is equivalent to the bicartesian diagram
\[\begin{tikzcd}
A\arrow[r,rightarrowtail]\arrow[d,twoheadrightarrow]& B\arrow[d,twoheadrightarrow]\\
0\arrow[r,rightarrowtail]&C
\end{tikzcd}\]
On the other hand, we have the following proposition regarding bicartesian squares in $Q\mathcal{M}$, which occurs in definition for composition.

\begin{tcolorbox}[colback=blue!5!white,colframe=blue!30!white]
\begin{proposition}
    Given a bicartesian square in $\mathcal{M}$
    \[\begin{tikzcd}
    N\arrow[r,rightarrowtail,"i"]\arrow[d,twoheadrightarrow,"j"]& M'\arrow[d,twoheadrightarrow,"j'"]\\
    M\arrow[r,rightarrowtail,"i'"]&N'
    \end{tikzcd}\]
we have $i_!j^!=j'^!i'_!$ in $Q\mathcal{M}$
\end{proposition}
\end{tcolorbox}
The proof is simply tracing through the definitions. By the proposition, every short exact sequence $A\rightarrowtail B\twoheadrightarrow C$ in $\mathcal{M}$ leads to the equivalence between the morphisms $0\twoheadleftarrow A\rightarrowtail B$ and $0 \rightarrowtail C\twoheadleftarrow B $ in $Q\mathcal{M}$. We will see how this equivalence is exactly the splitting of exact sequences we desire in the proof of the following theorem:

\begin{tcolorbox}[colback=red!5!white,colframe=red!30!white]
\begin{theorem}
$\pi_1(BQ\mathcal{M},0)\cong K_0(\mathcal{M})$
\end{theorem}
\end{tcolorbox}

Before proving the theorem, we first introduce the following lemma regarding trees.

\begin{tcolorbox}
\begin{lemma}
    Suppose $T$ is a maximal tree in a small connected category $C$. Then, $\pi_1(BC)$ is the group generated by $[f]$, one for each morphism not in $T$, modulo the relations that \begin{enumerate}
        \item  $[t]=1$ for every $t\in T$, and $[Id_c]=1$ for each identity morphism.
        \item $[f]\cdot [g]=[f\circ g]$ for every composable $f,g$.
    \end{enumerate}
\end{lemma}
\end{tcolorbox}
\begin{proof}
  Let $X$ be the $1$ skeleton of $BC$, which is a connected graph with maximal tree $T$.  The proof for $\pi_1(X)$ being the free group generated by $X-T$ is given in Hatcher proposition $1A.2$. The lemma is then a direct application of Van-Kampen's Theorem. 
\end{proof}


\begin{proof}[Proof of Theorem $2.1$]
    We construct the isomorphism directly, which follows Weibel and slightly diverts from the original approach given by Quillen. Let $T$ be the set of injective morphisms of the form $0\rightarrowtail A$ in $Q\mathcal{M}$. Clearly, $T$ contains all vertices (objects) in $Q\mathcal{M}$ and is thus a maximal tree. 

    
    
    Let $B'\rightarrowtail B$ be an injective morphism. Note that its left composition with $0\rightarrowtail B'$ is the morphism induced by $0\rightarrowtail B'\rightarrowtail B$, which is in $T$. Thus by lemma $2.2$, all injective morphisms correspond to the identity element; given a surjective morphism $A\twoheadleftarrow A'$, note that its left composition with $0 \twoheadleftarrow A$ induces a surjective map $0\twoheadleftarrow A'$, so we have the relation $[A\twoheadleftarrow A']=[0\twoheadleftarrow A]^{-1}[0\twoheadleftarrow A']$. By the observation that every morphism in $Q\mathcal{M}$ factors as a surjective morphisms followed by a injective one, we note that $\pi_1(BQ\mathcal{M})$ is generated by classes of the form $[0\twoheadleftarrow A]$.

    By the observation following proposition $2.0.1$, each short exact sequence $A\rightarrowtail B\twoheadrightarrow C$ in $\mathcal{M}$ leads to the equivalence between the morphisms $0\twoheadleftarrow A\rightarrowtail B$ and $0 \rightarrowtail C\twoheadleftarrow B $ in $Q\mathcal{M}$, which in turn induces in the additivity relation in $\pi_1$
    \[[0\twoheadleftarrow C][0\twoheadleftarrow A]=[0\twoheadleftarrow B]=[0\twoheadleftarrow A][0\twoheadleftarrow C]\]

    The composition rule follows the additivity relation, and the map $[0\twoheadleftarrow A]\mapsto [A]$ is the desired isomorphism between $\pi_1(BQ\mathcal{M})$ and $K_0(\mathcal{M})$.
\end{proof}


We can now define $K$-theory for exact categories.


\begin{tcolorbox}[colback=purple!5!white,colframe=purple!75!black]
\begin{definition}
For a small exact category $\mathcal{M}$, let $K\mathcal{M}$ be the loop space $\Omega BQ\mathcal{M}$, and set
\[K_i(\mathcal{M}):=\pi_{i+1}(BQ\mathcal{M},0)=\pi_i(K\mathcal{M},0)\]
\end{definition}
\end{tcolorbox}
For skeletally small categories, we define its $K$-groups to be those of its skeleton. It is not hard to see that $K_i$ is a functor from the category of small exact categories and exact functors to $\textbf{Ab}$, noting that isomorphic functors induce isomorphism of $K$-groups by the following proposition

\begin{tcolorbox}[colback=blue!5!white,colframe=blue!30!white]
\begin{proposition}
A natural transformation $\theta: f\to g$ of functors from $C$ to $C'$ induces an homotopy $BC\times I\to BC'$ between $Bf$ and $Bg$. In particular, if a functor has a left/right adjoint, then it induces a homotopy equivalence between classifying spaces. 
\end{proposition}
\end{tcolorbox}
\begin{proof}
    The key is to realize the data of the triple $(f,g,\theta)$ is exactly a functor $C\times 1\to C$, where $1$ is the ordered set $\{0<1\}$ with classifying space the unit interval.
\end{proof}

With a bit more effort, one can show that $K_i$ commutes with filtered colimits and products. 

\section{The + = Q Theorem}
Concretely, let us consider the additive category of finitely generated projective modules. It is a symmetric monoidal category with the direct sum operation, and we have $S^{-1}S$ construction for the higher $K$-theory; on the other hand, it is also an exact by example $7.0.1$, and we have the $Q$-construction. The goal of this chapter is to demonstrate the equivalence. By Corollary $6.4.1$, the $S^{-1}S$ construction is also equivalent to the plus construction, which give us the "+=Q" theorem.

\begin{tcolorbox}[colback=red!5!white,colframe=red!30!white]
\begin{theorem}
If $\mathcal{A}$ is a split exact category and $S=iso A$, then $\Omega BQA\cong B(S^{-1}S)$. Thus, $K_n(\mathcal{A})\cong K_n(S)$ for all $n\geq 0$. 
\end{theorem}
\end{tcolorbox}


\begin{tcolorbox}[colback=green!5!white,colframe=green!30!white]
\begin{corollary}[+ = Q Theorem]
$\Omega BQ \textbf{P}(R)\cong K_0(R)\times BGL(R)^+$
\end{corollary}
\end{tcolorbox}

The strategy is to construct a homotopy fibration sequence that relates the two constructions: 


\begin{tcolorbox}[colback=purple!5!white,colframe=purple!75!black]
\begin{definition}
Given an exact category $\mathcal{A}$ and an object $C$. Let $\mathcal{E}_C$ be the category whose objects are exact sequences 
\[A\rightarrowtail B\twoheadrightarrow C\]
with morphisms equivalence classes of diagrams 
\[\begin{tikzcd}
	A & B & C \\
	{A'} & {B'} & {C'}
	\arrow[from=1-1, to=1-2]
	\arrow["\cong", from=2-1, to=1-1]
	\arrow[from=1-2, to=1-3]
	\arrow["{\cong }", from=1-2, to=2-2]
	\arrow[equal, from=1-3, to=2-3]
	\arrow[from=2-1, to=2-2]
	\arrow[from=2-2, to=2-3]
\end{tikzcd}\]
\end{definition}
\end{tcolorbox}

\begin{tcolorbox}[colback=yellow!5!white,colframe=yellow!30!white]
\begin{example}
Let $0$ be the zero object of $A$. Note that the category $\mathcal{E}_0$ is homotopy equivalent to the category $S=iso \mathcal{A}$: consider the functor $S\to \mathcal{E}_0$ that sends an object $C$ to the exact sequence $A\xrightarrow{\textrm{Id}}A\twoheadrightarrow 0$. The functor is fully faithful and essentially surjective, thus an equivalence of categories. It is a standard lemma that any adjunction induce a homotopy equivalence. 
\end{example}
\end{tcolorbox}
We want to construct a fibered category over $Q \mathcal{A}$ with fibers $\mathcal{E}_C$. 

\begin{tcolorbox}[colback=purple!5!white,colframe=purple!75!black]
\begin{definition}
Given exact category $\mathcal{A}$, we define the extension category $\mathcal{E}\mathcal{A}$ as follows: the objects of $\mathcal{E}\mathcal{A}$ are the admissible exact sequence in $\mathcal{A}$; a morphism of admissible exact sequence from $A'\rightarrowtail B'\twoheadrightarrow C'$ to $A\rightarrowtail B\twoheadrightarrow C$ is an equivalence class of diagrams of the form 
\[\begin{tikzcd}
	{A'} & {B'} & {C'} \\
	A & {B'} & {C''} \\
	A & B & C
	\arrow[tail, from=1-1, to=1-2]
	\arrow[two heads, from=1-2, to=1-3]
	\arrow[tail, from=2-1, to=1-1]
	\arrow[tail, from=2-1, to=2-2]
	\arrow[equal, from=2-2, to=1-2]
	\arrow[two heads, from=2-2, to=2-3]
	\arrow[tail, from=2-2, to=3-2]
	\arrow[two heads, from=2-3, to=1-3]
	\arrow[tail, from=2-3, to=3-3]
	\arrow[equal, from=3-1, to=2-1]
	\arrow[tail, from=3-1, to=3-2]
	\arrow[two heads, from=3-2, to=3-3]
\end{tikzcd}\]
where two diagrams are considered isomorphic if there is an isomorphism of diagrams which is the identity at all vertices except possibly for the $C''$ vertex. 
\end{definition}
\end{tcolorbox}
It is easy to see that the rightmost column is a morphism $C\to C'$ in $Q \mathcal{A}$, and the projection functor $\pi: \mathcal{E}\mathcal{A}\to Q \mathcal{A}$ by $(A\rightarrowtail B\twoheadrightarrow C)\mapsto C$ has fiber category $\mathcal{E}_C$. We may equip $\mathcal{E}\mathcal{A}$ with an $S$-action, defined by $ D\cdot (A\rightarrowtail B\twoheadrightarrow C):= A\oplus D\rightarrowtail B\oplus D \twoheadrightarrow C$. The action lifts to an action on each fiber $\mathcal{E}_C$.

\begin{tcolorbox}[colback=blue!5!white,colframe=blue!30!white]
\begin{proposition}
$\mathcal{E}_C$ is symmetric monoidal, thus realizes to an $H$-space.
\end{proposition}
\end{tcolorbox}
\begin{proof}
    Given $E_i=A_i\rightarrowtail B_i\twoheadrightarrow C$, the tensor product $E_1\otimes E_2$ is defined to be 
    \[A_1\oplus A_2\rightarrowtail (B_1\times_C B_2) \twoheadrightarrow C \]
    and the unit can be checked to be $e: 0\rightarrowtail C\twoheadrightarrow C$. 
\end{proof}


\begin{tcolorbox}[colback=blue!5!white,colframe=blue!30!white]
\begin{proposition}
$M:=\langle S, \mathcal{E}_C \rangle$ is contractible.
\end{proposition}
\end{tcolorbox}
\begin{proof}
    First we show that $M$ is a connected $H$-space. Given $E_i=A_i\rightarrowtail B_i\twoheadrightarrow C$, since $\mathcal{A}$ is split exact, a splitting induces an isomorphism $A_2\cdot E_1\cong E_1\otimes E_2$, which determinjes morphisms $E_1\to E_1\otimes E_2$ and $E_2\to E_1\otimes E_2$ in $M$, and we have connectedness. 

    Note that there is a natural transformation between the identity functor and multiplication by $2$ functor given by the diagonal map, i.e 
    \[\begin{tikzcd}
        {E:} & A & B & C \\
        \\
        {E\otimes E:} & {A\oplus A} & {B\times_C B} & C
        \arrow[from=1-1, to=3-1]
        \arrow[tail, from=1-2, to=1-3]
        \arrow[from=1-2, to=3-2]
        \arrow[two heads, from=1-3, to=1-4]
        \arrow[from=1-3, to=3-3]
        \arrow[equal, from=1-4, to=3-4]
        \arrow[tail, from=3-2, to=3-3]
        \arrow[two heads, from=3-3, to=3-4]
    \end{tikzcd}\]

It is a standard fact that the homotopy classes of maps $[M,M]$ is a group, and by the previous observation satisfies $x^2=x$, which implies it is trivial and is contractible. 
\end{proof}


\begin{tcolorbox}[colback=blue!5!white,colframe=blue!30!white]
\begin{proposition}
We have a canonical fibration $S^{-1}S\to S^{-1}\mathcal{E}_C\to \langle S, \mathcal{E}_C\rangle$. 
\end{proposition}
\end{tcolorbox}
By the contracitiblity of $M$, the fibration in proposition $9.1.3$ implies $S^{-1}S\cong S^{-1}\mathcal{E}_C$. 


\begin{tcolorbox}[colback=red!5!white,colframe=red!30!white]
\begin{theorem}
If $\mathcal{A}$ is split exact category and $S= iso \mathcal{A}$, then we have the homotopy fibration sequence 
\[S^{-1}S\to S^{-1}\mathcal{E} \mathcal{A}\to QA\]
\end{theorem}
\end{tcolorbox}

To show this, we recall Quillen's Theorem $B$ in \cite{Quillen}


\begin{tcolorbox}[colback=red!5!white,colframe=red!30!white]
\begin{theorem}[Quillen's Theorem B]
Suppose $f: C\to C'$ is a prefibered and that for every arrow $u: Y\to Y'$ the base chaneg functor $u^*: f^{-1}(Y')\to f^{-1}(Y)$ is a homotopy equivalence. Then, for any $Y\in C'$, the category $f^{-1}(Y)$ is a homotopy equivalent to the homotopy fiber of $f$ over $Y$. In particular, we have the homotopy fibration
\[f^{-1}(Y)\to C\to C'\]
\end{theorem}
\end{tcolorbox}
\begin{proof}[Proof of theorem $9.2$]
    It suffice to show the base-change functor associated to the prefibered functor $S^{-1}\mathcal{E}\mathcal{A}\to Q \mathcal{A}$ are homotopy equivalences. Moreover, it is enough to consider those associated to injective and surjective morphisms of $Q \mathcal{A}$ of the form $0\rightarrowtail C$ and $0\twoheadleftarrow C'$. By example $9.0.1$, we may identify $\mathcal{E}_0$ with $S$, so a base change map $j^*: \mathcal{E}_0\to \mathcal{E}_C$ is a homotopy equivalence by proposition $9.1.3$. The case is similar for the other direction, and we are done.
\end{proof}

Finally, we only have to show that $S^{-1}\mathcal{E}\mathcal{A}$ is contractible, which together with Theorem $9.2$ implies Theorem $9.1$. 


\begin{tcolorbox}[colback=blue!5!white,colframe=blue!30!white]
\begin{proposition}
$S^{-1}\mathcal{E}\mathcal{A}$ is contractible.
\end{proposition}
\end{tcolorbox}
\begin{proof}
    If $X$ is a category, we define its subdivision category $\textrm{sub}(X)$ whose objects are arrows of $X$, and arrows from $f$ to  $g$ is a pair of arrows $h,k$ such that $kfh=g$. It is not hard to verify that $\textrm{Sub}(X)\to X$ is a homotopy equivalence. 

    If $X$ is the subcategory of $Q \mathcal{A}$ of injective arrows, then $\mathcal{E}\mathcal{A}$ is equivalent to $\textrm{Sub} (X)$. Note that $X$ has initial object $0$, and $\mathcal{E}\mathcal{A}$ is contractible. Since $S$ acts invertibly, so we know that $\mathcal{E} \mathcal{A}$ and $S^{-1}\mathcal{E} \mathcal{A}$ are homotopy equivalent, so we finish.
\end{proof}


\section{Waldhausen's $wS_{\cdot}$ construction}
For his investigation of the $s$-cobordism theorem, Waldhausen introduced an algebraic $K$-theory for spaces, which is now called the $wS_{\cdot}$ construction. 





\begin{tcolorbox}[colback=purple!5!white,colframe=purple!75!black]
\begin{definition}
A \underline{\textbf{category with cofibrations}} is any category $\mathcal{C}$, equipped with a distinguished class of morphisms called the \underline{\textbf{cofibrations}}, denoted $\underline{\textbf{co}(\mathcal{C})}$, that satisfy the following axioms:
\begin{enumerate}
    \item Every isomorphism in $\mathcal{C}$ is a cofibrations.
    \item $\mathcal{C}$ has a dinstinguished $0$ object, and the map $0\rightarrowtail A$ in $\mathcal{C}$ is a cofibrations for every $A$ in $\mathcal{C}$. 
    \item  The pushout along any cofibration exists, and the cofibrations are stable under pushout. Specifically, if $A\rightarrowtail B$ is a cofibration and $A\to C$ is any morphism in $\mathcal{C}$, then we have the pushout square 
    \[\begin{tikzcd}
        A & B \\
        C & {B\cup_AC}
        \arrow[tail, from=1-1, to=1-2]
        \arrow[from=1-1, to=2-1]
        \arrow[from=1-2, to=2-2]
        \arrow[tail, from=2-1, to=2-2]
    \end{tikzcd}\]
\end{enumerate}
\end{definition}
\end{tcolorbox}


\begin{tcolorbox}[colback=blue!5!white,colframe=blue!30!white]
\begin{proposition}
Finite coproducts exist in a catgeory with cofibrations; cokernels of cofibrations exist in a catgeory with cofibrations.
\end{proposition}
\end{tcolorbox}
\begin{proof}
    A finite coproduct $A\coprod B$ is the pushout $A\cup_0B$, and given a cofibration $A\rightarrowtail B$, the cokernel is the pushout
    \[\begin{tikzcd}
        A & B \\
        0 & {B/A}
        \arrow[tail, from=1-1, to=1-2]
        \arrow[from=1-1, to=2-1]
        \arrow[from=1-2, to=2-2]
        \arrow[tail, from=2-1, to=2-2]
    \end{tikzcd}\]
\end{proof}


\begin{tcolorbox}[colback=purple!5!white,colframe=purple!75!black]
\begin{definition}[Waldhausen Category]
A \underline{\textbf{Waldhausen Category}} $\mathcal{C}$ is a category with cofibrations, together with another distinguished class of morphisms called \underline{\textbf{weak equivalences}}, denoted by $w(\mathcal{C})$, that satisfies the additional axioms:
\begin{enumerate}
    \item Every isomorphism is a weak equivalence.
    \item Weak equivalences are closed under composition. 
    \item (Glueing axioms) For every diagram of the form
    \[\begin{tikzcd}
        C & A & B \\
        {C'} & {A'} & {B'}
        \arrow["\sim", from=1-1, to=2-1]
        \arrow[from=1-2, to=1-1]
        \arrow[tail, from=1-2, to=1-3]
        \arrow["\sim", from=1-2, to=2-2]
        \arrow["\sim"', from=1-3, to=2-3]
        \arrow[from=2-2, to=2-1]
        \arrow[tail, from=2-2, to=2-3]
    \end{tikzcd}\]
    where the vertical maps are weak equivalences, the induced map 
    \[B\cup_A C\to B'\cup_AC'\]
is also a weak equivalence. 

\end{enumerate}
\end{definition}
\end{tcolorbox}



\begin{tcolorbox}[colback=yellow!5!white,colframe=yellow!30!white]
    \begin{example}
    Any exact category is naturally a Waldhausen category by setting the admissible monic to be the cofibrations and isomorphisms being the weak equivalences. 
    \end{example}
    \end{tcolorbox}

Exact categories are algebraic in nature, and Waldhausen categories apply to topological situations
\begin{tcolorbox}[colback=yellow!5!white,colframe=yellow!30!white]
\begin{example}
Let $\mathcal{C}$ be the category of based CW complexes with countably many cells, with morphism cellular maps. Then, $\mathcal{C}$ is a Waldhausen category by letting the cofibrations be cellular inclusions(which is a cofibration in the classical sense), and weak equivalences to be weak homotopy equivalences. The pushouts corresponds to the usual glueing construction. 
\end{example}
\end{tcolorbox}


\begin{tcolorbox}[colback=purple!5!white,colframe=purple!75!black]
\begin{definition}[$S_{\cdot}\mathcal{C}$]
Let $\mathcal{C}$ be a category with cofibrations. Then, define $S_n \mathcal{C}$ to be the category with objects a cofibration sequence of the form
\[0=A_0\rightarrowtail A_1\rightarrowtail A_2\rightarrowtail ...\rightarrowtail A_n\]
with a choice of subquotient $A_{ij}:=A_j/A_i$ for $j\geq i$ such that there is a diagram 
\[\begin{tikzcd}
	&&&& {A_{n-1,n}} \\
	&&&& {...} \\
	&& {A_{23}} & {...} & {A_{2n}} \\
	& {A_{12}} & {A_{13}} & {...} & {A_{1n}} \\
	{A_1} & {A_2} & {A_3} & {...} & {A_n}
	\arrow[from=3-3, to=3-4]
	\arrow[from=3-4, to=3-5]
	\arrow[tail, from=4-2, to=4-3]
	\arrow[from=4-3, to=3-3]
	\arrow[tail, from=4-3, to=4-4]
	\arrow[from=4-4, to=4-5]
	\arrow[from=4-5, to=3-5]
	\arrow[tail, from=5-1, to=5-2]
	\arrow[two heads, from=5-2, to=4-2]
	\arrow[tail, from=5-2, to=5-3]
	\arrow[two heads, from=5-3, to=4-3]
	\arrow[tail, from=5-3, to=5-4]
	\arrow[tail, from=5-4, to=5-5]
	\arrow[two heads, from=5-5, to=4-5]
\end{tikzcd}\]
A morphism of in $S_n \mathcal{C}$ is a morphism of diagrams.
\end{definition}
\end{tcolorbox}
For notational simplicity, we will denote $A_i:=A_{0i}$ 


\begin{tcolorbox}[colback=red!5!white,colframe=red!30!white]
\begin{theorem}
    $S_{\cdot} \mathcal{C}$ defines a simplicial Waldhausen category. 
\end{theorem}
\end{tcolorbox}
\begin{proof}
    First, we verify that each $S_n \mathcal{C}$ has the structure of a Waldhausen category: a map of sequeneces $A_{\cdot}\to B_{\cdot}$ is a cofibration when for $0\leq i<j< k\leq n$ we have a map of cofibration sequences 
    \[\begin{tikzcd}
        {A_{ij}} & {A_{ik}} & {A_{jk}} \\
        {B_{ij}} & {B_{ik}} & {B_{jk}}
        \arrow[tail, from=1-1, to=1-2]
        \arrow[from=1-1, to=2-1]
        \arrow[two heads, from=1-2, to=1-3]
        \arrow[from=1-2, to=2-2]
        \arrow[from=1-3, to=2-3]
        \arrow[tail, from=2-1, to=2-2]
        \arrow[two heads, from=2-2, to=2-3]
    \end{tikzcd}\]

and the outer vertical maps $A_{ij}\to B_{ij}$ and $A_{jk}\to B_{jk}$ are cofibrations, and the induced morphism $$B_{ij}\cup_{A_{ij}}A_{ik}\to B_{ik}$$ is a cofibration. A morphism $A_{\cdot}\to B_{\cdot}$ is a weak equivalence if every induced morphism $A_{ij}\to B_{ij}$ is a weak equilence in $\mathcal{C}$(equivalently, having $A_i\to B_i$ being weak equivalences suffice)




\end{proof}











\section{The Fundamental Theorems of algebraic $K$-theory}
















\printbibliography
\end{document}