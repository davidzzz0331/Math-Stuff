\documentclass{article}
\usepackage[utf8]{inputenc}
\usepackage{amsmath}
\usepackage{amsfonts}
\usepackage{amssymb}
\usepackage{tikz}
\usepackage{fullpage}
\usepackage{tikz-cd}
\usepackage{spectralsequences}
\usepackage{adjustbox}
\usepackage[backend=biber, style=alphabetic]{biblatex}
\usepackage{xfrac}
\usepackage{tcolorbox}
\usepackage{xcolor}
\usepackage{graphicx}
\graphicspath{ {D:/Chrome Downloads./} }
\usepackage[parfill]{parskip}
\usepackage{amsthm}
\addbibresource{sample.bib}
\usetikzlibrary{calc}
\theoremstyle{definition}
\newtheorem{theorem}{Theorem}[section]
\theoremstyle{definition}
\newtheorem{definition}{Definition}[theorem]
\theoremstyle{definition}
\newtheorem{remark}{Remark}[theorem]
\theoremstyle{definition}
\newtheorem{proposition}{Proposition}[theorem]
\theoremstyle{definition}
\newtheorem{lemma}[theorem]{Lemma}
\theoremstyle{definition}
\newtheorem{corollary}{Corollary}[theorem]
\theoremstyle{definition}
\newtheorem{example}{Example}[theorem]
\tikzset{curve/.style={settings={#1},to path={(\tikztostart)
    .. controls ($(\tikztostart)!\pv{pos}!(\tikztotarget)!\pv{height}!270:(\tikztotarget)$)
    and ($(\tikztostart)!1-\pv{pos}!(\tikztotarget)!\pv{height}!270:(\tikztotarget)$)
    .. (\tikztotarget)\tikztonodes}},
    settings/.code={\tikzset{quiver/.cd,#1}
        \def\pv##1{\pgfkeysvalueof{/tikz/quiver/##1}}},
    quiver/.cd,pos/.initial=0.35,height/.initial=0}
\title{\'Etale Homotopy Theory and Adams Conjecture}
\author{David Zhu}

\begin{document}
\maketitle



\begin{tcolorbox}[colback=purple!5!white,colframe=purple!75!black]
\begin{definition}[\u Cech Nerve ]
Let $X$ be a finite CW complex, and $\mathcal{U}:=\{U_i: i\in I\}$ be an open cover of $X$. Then, we may define a simplicial set call the \underline{\textbf{\u Cech Nerve}} $N \mathcal{U}$ as follows: we have the assignment on objects $[n]\mapsto \{\textrm{functions from }[n] \textrm{ to } I: \cap^n_{i=1}U_{f(i)}\neq \emptyset \}$. The face maps and degeneracy maps are defined by deleting and inserting appropriate indices. 
\end{definition}
\end{tcolorbox}

Alternatively, we can think of a covering $\mathcal{U}$ as follows: suppose given a covering $X=\cup_{i\in I}U_i$; let $\mathcal{U}=\coprod_{i\in i} U_i$, and the covering is the obvious map $\mathcal{U}\to X$. Note that we have 
\[U_i\cap U_j=U_i\times_X U_j\]
so the $n$-fold fiber product $U\times_X...\times_X U$ is the disjoint union of $n$-fold intersections of opens in the cover. Then, the $n$th simplices of the \u Cech nerve is $\pi_0(  \underbrace{U\times_X...\times_X U }_{n-fold})$. The face maps are projections, and the degeneracy maps are various diagonal embeddings.


\begin{tcolorbox}[colback=red!5!white,colframe=red!30!white]
\begin{theorem}
If the covering $\mathcal{U}$ satisfies the property that arbitrary intersections of opens in the cover is either empty or contractible, then th realization $|N \mathcal{U}|$ is weakly equivalent to $X$.
\end{theorem}
\end{tcolorbox}


\section{Adam's conjecture}


\begin{tcolorbox}[colback=purple!5!white,colframe=purple!75!black]
\begin{definition}
Let $X$ be compact Hausdorff and let $KU(X)$ be Grothendieck group of complex vector bundles over $X$, and let $\mathcal{SF}(X)$ be the Grothendieck group of sphere bundles over $X$ modulo fiber homotopy equivalence. 
\end{definition}
\end{tcolorbox}


\begin{tcolorbox}[colback=red!5!white,colframe=red!30!white]
\begin{theorem}
The stable sphere bundles over $X$ is classified by the the groups of self-homotopy equivalences of $S^n$, which we denote by $G(n):=\textrm{Equiv}(S^n,S^n)$. 
\end{theorem}
\end{tcolorbox}


\begin{tcolorbox}[colback=blue!5!white,colframe=blue!30!white]
\begin{proposition}
The complex $J$-homomorphism $J: KU(X)\to \mathcal{SF}(X)$ is induced by a map between classifying spaces, which we also denote 
\[J: BU\to BG:=\varinjlim_{n} BG(n)\]
\end{proposition}
\end{tcolorbox}


\begin{tcolorbox}[colback=purple!5!white,colframe=purple!75!black]
\begin{definition}
The Adams' operation $\psi^k: KU(X)\to KU(X)$ is induced by a map of classifying spaces
\[\psi^k: BU\to BU\]
\end{definition}
\end{tcolorbox}


\begin{tcolorbox}[colback=red!5!white,colframe=red!30!white]
\begin{theorem}[The Adams Conjecture]
The composite 
\[BU\xrightarrow{\psi^k-1}BU\xrightarrow{J}BG\]
is nullhomotopic up to multiplication by some $k^n$. 
\end{theorem}
\end{tcolorbox}


\begin{tcolorbox}[colback=blue!5!white,colframe=blue!30!white]
\begin{proposition}
The composite $J\circ i: BU(n)\to BU\to BG$, classifyies a sphere bundle over $BU(n)$, and is fiber homotopy equivalent to the fibration 
\[BU(n-1)\to BU(n)\]
\end{proposition}
\end{tcolorbox}

Fact: the composition $J\circ i: BU(n)\to BG$ classifies the sphere bundle associated to the canonical bundle over $BU(n)$, and it is fiber homotopy equivalent to the fibration 
\[BU(n-1)\to BU(n)\]

We can dream of a proof here: if we have unstable adams operations $\psi^k: BU(n)\to BU(n)$, which are homotopy equivalences, with a pullback diagram
\[\begin{tikzcd}
	{BU(n-1)} & {BU(n-1)} \\
	{BU(n)} & {BU(n)}
	\arrow["{\psi^k}", from=1-1, to=1-2]
	\arrow["i", from=1-1, to=2-1]
	\arrow["i", from=1-2, to=2-2]
	\arrow["{\psi^k}"', from=2-1, to=2-2]
\end{tikzcd}\]
Then, the bundle $J\circ i$ is fiber homotopy equivalent $J\circ \psi^k$. However, unstable Adams operation does not exist on $BU(n)$. However, Sullivan proves that 
\[\begin{tikzcd}
	{\widehat{BU(n-1)}_p} & {\widehat{BU(n-1)}_p} \\
	{\widehat{BU(n)}_p} & {\widehat{BU(n)}_p}
	\arrow["{\psi^k}", from=1-1, to=1-2]
	\arrow["i", from=1-1, to=2-1]
	\arrow["i", from=1-2, to=2-2]
	\arrow["{\psi^k}"', from=2-1, to=2-2]
\end{tikzcd}\]
where $\psi^k$ is an unstable Adams operation on the profinite completion. 







\section{Algebraic Side}



\section*{Idea of proof}

\underline{\textbf{Step 1}}: Sullivan proves that the stable fiber homotopy types injects into profinite stable homotopy types. In particular, we have the isomorphism on classifying space level 
\[\textrm{Stable profinite theory}: \ \widehat{B}_{\infty}\cong B_{SG}\times K(\widehat{\mathbb{Z}^*},1)\]
\[\textrm{Stable theory}: \ BG\cong B_{SG}\times K(\mathbb{Z}/2,1)\]
where $B_{SG}=\varinjlim_n B_{SG(n)}$, and $B_{SG(n)}$ is the classifying space for the component of the identity map in $G(n)$. (Alternatively, it is also the universal cover of $BG(n)$). Thus, the classifying space for the stable theory is a direct factor of the stable profinite theory, and it suffice to formulate and prove the Adams conjecture in the profinite setting. 

\underline{\textbf{Step 2}}: We identify the classical Adams operation in the following way: the classical Adams operation 
\[K(X)\xrightarrow{\psi^k}K(X)\]
naturally desends to maps on the profinite completion, which factors as 
\[\prod_p \widehat{K(X)}_p\xrightarrow{\widehat{\psi}_p} \prod_p \widehat{K(X)}_p\]
and $\widehat{\psi^k}_p: \widehat{K(X)}_p\to \widehat{K(X)}_p$ is an isomorphism iff $k$ is prime to $p$. If $k$ is divisible by $p$, we redefine $\widehat{\psi^k}_p$ to be the identity map. After the redefinition, we obtain 
\[\widehat{K(X)}\xrightarrow{\psi^k}\widehat{K(x)}\]
which we call the "isomorphic" part of the Adams operation. 


\begin{tcolorbox}[colback=green!5!white,colframe=green!30!white]
\begin{remark}
Before the redefinition, in the case where $k|p$, we note that$ widehat{\psi^k}_p$ is topologically nilpotent. 
\end{remark}
\end{tcolorbox}

Following this, Sullivan observed that this isomorphic part of the Adams operation is compatible with the natural action of $Gal(\overline{\mathbb{Q}}|\mathbb{Q})$ in the category of profinite homotopy type and maps coming from the algebraic varieties defiend over $\mathbb{Q}$. In particular, there iare homomorphisms
\[Gal(\mathbb{Q}|\mathbb{Q})\to \widehat{\mathbb{Z}}^*\to \textrm{Aut}(\widehat{K(X)})\]
by letting $G$ act on the roots of unity. Moreover, for each $\psi^k$, we note that $k$ defines an element $(k)\in \widehat{\mathbb{Z}}^*$ by giving the automorphism 
\[(k)x=\begin{cases}
	k\cdot x & \textrm{if } x\in \widehat{\mathbb{Z}_p}, (k,p)=1\\
	x & \textrm{if } x\in  \widehat{\mathbb{Z}_p}, (k,p)\neq 1
\end{cases}\]
Clearly this is compatible with the Adams operation on profinite $K$ theory. Thus, we have identified the profinite Adams operation with the action of the profinite group $\widetilde{\mathbb{Z}}^*$. 

\underline{\textbf{Step 3}}: There is a natural action of the absolute Galois group $Gal(\overline{\mathbb{Q}}|\mathbb{Q})$ on the profinite classifying space $\widehat{BU}$, and the Adams operation is compactible with such action through the abelianization map.
\[Gal(\overline{\mathbb{Q}|\mathbb{Q}})\to \widehat{\mathbb{Z}}^*\]

The aboslute Galois group action is how the etale homotopy type theory factors in. Note that the absolute galois group acts algebraically on $\mathbb{C}^n$ and $\mathbb{CP}^n$, but with classical topology this action is wildly discontinuous. However, for every algebraic variety $V$, we may constrtcut an inverse system of nerves $N_{\alpha}$, with natural maps $V\to \{N_{\alpha}\}$ giving 
\[\widehat{\pi_1(V)}\cong \varprojlim_{\alpha}\pi_1N_{\alpha} \textrm{   and   } H^i(V; M)\cong \varinjlim H^i(N_{\alpha; M})\]
 for all finite coefficient $M$.


Sullivan proves that the the above isomorphism imply the profinite completion of $V$ can be constructed from the nerves 
\[\widehat{V}\cong \varprojlim_{\alpha} N_{\alpha}\]
in the sense of compact functors (with the extra assumption that $\pi_i(N_{\alpha})$ is finite. )


Since each $N_{\alpha}$ is constructed using the algebraic structure of $V$, and each automorphism $\sigma\in Gal(\overline{Q}|\mathbb{Q})$ determines a simplicial automorpihism of $N_{\alpha}$, and thus the profinite homotopy type of any complex algebraic variety defined over $\mathbb{Q}$.

Recall that the classifying space $BU(n)$ is constructed as the direct limit of complex grassmannians $\varinjlim_{k}Gr_n(k)$. Via Pl\"{u}cker embeddings, the complex grassmannians are naturally affine complex varieties embedded in projective space. Moreover, the defining polynomials also have coefficients in $\mathbb{Q}$.(Example here?)


By naturality and splitting principal, understanding the action of $Gal(\mathbb{C}|\mathbb{Q})$ on the profinite complex $K$-theory reduces to understanding the action on $\cup_n \widehat{\mathbb{CP}^n}\cong K(\widehat{\mathbb{Z}},2)$, which can be checked to be the composition 
\[Gal(\mathbb{C}|\mathbb{Q})\to \widehat{\mathbb{Z}}^*\to K(\widehat{Z},2)\]
and thus agrees with the isomorphic part of he Adams operations discussed above. 

\section{The Adams Conjecture-Proof}





\end{document}