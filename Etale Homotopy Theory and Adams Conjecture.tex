\documentclass{article}
\usepackage[utf8]{inputenc}
\usepackage{amsmath}
\usepackage{amsfonts}
\usepackage{amssymb}
\usepackage{tikz}
\usepackage{fullpage}
\usepackage{tikz-cd}
\usepackage{spectralsequences}
\usepackage{adjustbox}
\usepackage[backend=biber, style=alphabetic]{biblatex}
\usepackage{xfrac}
\usepackage{tcolorbox}
\usepackage{xcolor}
\usepackage{graphicx}
\graphicspath{ {D:/Chrome Downloads./} }
\usepackage[parfill]{parskip}
\usepackage{amsthm}
\addbibresource{sample.bib}
\usetikzlibrary{calc}
\theoremstyle{definition}
\newtheorem{theorem}{Theorem}[section]
\theoremstyle{definition}
\newtheorem{definition}{Definition}[theorem]
\theoremstyle{definition}
\newtheorem{remark}{Remark}[theorem]
\theoremstyle{definition}
\newtheorem{proposition}{Proposition}[theorem]
\theoremstyle{definition}
\newtheorem{lemma}[theorem]{Lemma}
\theoremstyle{definition}
\newtheorem{corollary}{Corollary}[theorem]
\theoremstyle{definition}
\newtheorem{example}{Example}[theorem]
\tikzset{curve/.style={settings={#1},to path={(\tikztostart)
    .. controls ($(\tikztostart)!\pv{pos}!(\tikztotarget)!\pv{height}!270:(\tikztotarget)$)
    and ($(\tikztostart)!1-\pv{pos}!(\tikztotarget)!\pv{height}!270:(\tikztotarget)$)
    .. (\tikztotarget)\tikztonodes}},
    settings/.code={\tikzset{quiver/.cd,#1}
        \def\pv##1{\pgfkeysvalueof{/tikz/quiver/##1}}},
    quiver/.cd,pos/.initial=0.35,height/.initial=0}
\title{\'Etale Homotopy Theory and Adams Conjecture}
\author{David Zhu}

\begin{document}
\maketitle

We only prove the Adams conjecture for complex vector bundles, but the same strategy also applies for the real case, albeit needing some modifications. 

\section{The Adams Conjecture}
\begin{tcolorbox}[colback=purple!5!white,colframe=purple!75!black]
	\begin{definition}
	Let $X$ be compact Hausdorff and let $KU(X)$ be Grothendieck group of complex vector bundles over $X$, and let $\mathcal{SF}(X)$ be the Grothendieck group of stable sphere bundles over $X$ modulo fiber homotopy equivalence. The \underline{\textbf{complex $J$-homomorphism}} can be viewed as the homomorphism
	\[J: KU(X)\to \mathcal{SF}(X)\]
	by sending a complex vector bundle to its fiberwise one-point compactification. 
	\end{definition}
	\end{tcolorbox}

	
	\begin{tcolorbox}[colback=red!5!white,colframe=red!30!white]
		\begin{theorem}
		The stable sphere bundles over $X$ is classified by the the groups of self-homotopy equivalences of $S^n$, which we denote by $G(n):=\textrm{Equiv}(S^n,S^n)$. 
		\end{theorem}
		\end{tcolorbox}
		

	\begin{tcolorbox}[colback=blue!5!white,colframe=blue!30!white]
		\begin{proposition}
		The complex $J$-homomorphism $J: KU(X)\to \mathcal{SF}(X)$ is induced by a map between respective classifying spaces, which we also denote 
		\[J: BU\to BG:=\varinjlim_{n} BG(n)\]
		\end{proposition}
		\end{tcolorbox}


	
	
	
	
	\begin{tcolorbox}[colback=purple!5!white,colframe=purple!75!black]
	\begin{definition}
	\underline{\textbf{The $k$-th Adams Operation}} $\psi^k: KU(X)\to KU(X)$ is defined to be a ring homomorphism that is natural in $X$, and satisfies 
	\[\psi^k(L)=L^{\otimes k}\]
	where $L$ is any line bundle over $X$. 	
	\end{definition}
	\end{tcolorbox}
	Note that $\psi^k$ is unique by the splitting principal. 
	
	\begin{tcolorbox}[colback=red!5!white,colframe=red!30!white]
	\begin{theorem}[\underline{\textbf{The Adams Conjecture}}]
	The composite 
	\[BU\xrightarrow{\psi^k-1}BU\xrightarrow{J}BG\]
	is nullhomotopic up to multiplication by some $k^n$. 
	\end{theorem}
	\end{tcolorbox}
	
	
	\begin{tcolorbox}[colback=green!5!white,colframe=green!30!white]
	\begin{remark}
	Historically, Adams realized Whithead's $J$-homomorphism $J: \pi_i(SO)\to \pi^s_i$ can be used to understand the stable homotopy groups of spheres, since $\pi_i(SO)$ is known by Bott periodicity. The Adams conjecture became one of the steps in understanding the image of the $J$ homomorphism. \\

	Multiple proofs of the Adams conjecture were completed around 1970. Notably, Quillen's (second) proof using Brauer lifting and by computing the cohomology of $BGL(\mathbb{F}_q)$ led to his later construction of higher algebraic $K$-theory. In this talk, we present Sullivan's proof using profinite completion and \'etale homotopy theory, which turns the Adams conjecture into a case of Galois symmetry of algebraic varieties.


	\end{remark}
	\end{tcolorbox}



\section{Preview of Proof}

We are trying to show that the map
\[BU\xrightarrow{id} BU\xrightarrow{J}BG\]
and 
\[BU\xrightarrow{\psi^k} BU\xrightarrow{J}BG\]
are homotopic modulo $k$. The first map classies the spherical bundle associated to the tautological bundle $\gamma$ over $BU$, and the second map is the spherical bundle associated to the pullback $\psi^{k*}\gamma$, so it suffices to show that these sphere bundles are fiber homotopy equivalent. Sullivan noted that 


\begin{tcolorbox}[colback=red!5!white,colframe=red!30!white]
\begin{theorem}[Inertia Lemma]
A filtered automorphism $\varphi: BU\to BU$, meaning a automorphism coming from the limit of automorphisms $\varphi_n:BU(n)\to BU(n)$, induces a fiber homotopy equivalence $\gamma\sim \varphi^* \gamma$. 
\end{theorem}
\end{tcolorbox}
However, the classical Adams operation cannot descend to compatible self-homotopy equivalences on $BU(n)$. However, we can do this following Sullivan's idea of turning the classical theory into the "profinite theory".

\section{Profinite Completion}
We recall some facts about profinte completion in algebra.


\begin{tcolorbox}[colback=purple!5!white,colframe=purple!75!black]
\begin{definition}
A \underline{\textbf{profinite group}} is the projective limit of finite discrete groups.
\end{definition}
\end{tcolorbox}


\begin{tcolorbox}[colback=yellow!5!white,colframe=yellow!30!white]
\begin{example}
The Galois group of an infinite field extension $K|F$ is the profinite limit of the Galois groups $Gal(L|F)$, where $L$ ranges from all finite Galois extensions of $F$. \\

Let $\mathbb{Q}^{ab}$ denote the maximal abelian extension of $\mathbb{Q}$. From class field theory, one know that any abelian extension of $\mathbb{Q}$ is contained in some cyclotomic extension (meaning joining some root of unity). Thus, 
\[Gal(\mathbb{Q}^{ab}|\mathbb{Q})=\varprojlim Gal(\mathbb{Q}(\zeta_{n})|\mathbb{Q})=\varprojlim_n (\mathbb{Z}/n)^{\times}\]
\end{example}
\end{tcolorbox}
So here is Sullivan's idea of profinite completion of spaces, inspired by Artin-Mazur's 
\'etale homotopy theory. We start by the following example:


\begin{tcolorbox}[colback=yellow!5!white,colframe=yellow!30!white]
\begin{example}
Fix a space $F$ with finite homotopy groups. For every CW complex $Y$, the set $[Y,F]$ carries a compact topology since we have 
\[[Y,F]=\varprojlim_{\alpha}[Y_{\alpha}, F]\]
where $Y_{\alpha}$ ranges from all finite subcomplexes of $Y$ and as $[Y_{\alpha}, F]$ is finite by obstruction theory. \\


Now fix a CW complex $X$. Let $X/\mathcal{F}$ denote the category whose objects are maps $X\to F$ for some $F$ with finite homotopy groups, and morphisms diagrams of the form 
\[\begin{tikzcd}
	& X \\
	{F'} && F
	\arrow[from=1-2, to=2-1]
	\arrow[from=1-2, to=2-3]
	\arrow[from=2-1, to=2-3]
\end{tikzcd}\]
Artin-Mazur showed that this category is equivalent to a small filtering category. 
\end{example}
\end{tcolorbox}


\begin{tcolorbox}[colback=red!5!white,colframe=red!30!white]
\begin{theorem}
The functor $Y\mapsto \varprojlim_{X/\mathcal{F}}[Y,F]$ satisfies the hypothesis of Brown Representability, therefore is represented by some CW complex, which we call $\widehat{X}$, the \underline{\textbf{profinite completion of $X$}}. Note that there is a natural map $X\to \widehat{X}$ given by the structure maps $X\to F$ in $X/\mathcal{F}$.
\end{theorem}
\end{tcolorbox}
Sullivan also proved the following equivalence:

\begin{tcolorbox}[colback=red!5!white,colframe=red!30!white]
\begin{theorem}
For a CW complex $X$, the profinite completion $X\to \widehat{X}$, and the inverse system of spaces ${F}$ in $X/\mathcal{F}$, the following hold:
\begin{enumerate}
	\item $\widehat{X}=\varprojlim_{X/\mathcal{F}}F$
	\item $\widehat{\pi_1(X)}\cong \varprojlim_{X/\mathcal{F}}\pi_1(F)$.
	\item $H^*(X;M)\cong \varprojlim_{X/\mathcal{F}}H^*(F;M)$ for all finite coefficient $M$.
\end{enumerate}
The converse is also true: let ${F_{\alpha}}$ be any inverse system of spaces with finite homotopy groups, together with maps $X\to F_{\alpha}$. If The system satisfies 2 and $3$ listed above, then $\widehat{X}\cong \varprojlim_{\alpha} F_{\alpha}$.
\end{theorem}
\end{tcolorbox}

If the reader if familiar, criteria 2 and 3 corresponds to the comparison theorems of \'etale fundamental group and  \'etale cohomology respectively. The general construction of this is motivated by the theory of \'etale homotopy type, which tried to capture the theory of \'etale fundamental group and cohomology into one unifying ordinary homotopy type. 

\section{\'Etale Homotopy Theory}
Since we are only dealing with smooth complex varieties, we follow Sullivan's approach to this topic and avoid much of Artin-Mazur's technical machinery. In analogy of the classical complex geometry, etale maps are more or less local diffeomorphism, such that they satisfiy the hypothesis of the inverse function theorem; an etale covering is then closely related to a topological covering spaces by Riemann existence theorem.

We focus on smooth complex varieties.
\begin{tcolorbox}[colback=purple!5!white,colframe=purple!75!black]
\begin{definition}
A morphism between schemes $f: X\to Y$ is \underline{\textbf{\'etale}} if it is flat and unramified 
\end{definition}
\end{tcolorbox}
More explicitly, a morphism is etale if for given a point $x\in X$, there exists an affine neighborhood $U=Spec(R)$ containing $X$ and an affine neighborhood $V=Spec(S)$ containing $f(x)$ such that $F(U)\subseteq V$, such that the induced ring map $S\to R$ is flat and unramified. 


\begin{tcolorbox}[colback=yellow!5!white,colframe=yellow!30!white]
\begin{example}
Suppose we have smooth complex varieties. Then, any ring morphism between finite $k$-algebras
\end{example}
\end{tcolorbox}








\section{The Case of $\mathbb{CP}^1$}

\section{Proof}





















\begin{tcolorbox}[colback=purple!5!white,colframe=purple!75!black]
\begin{definition}[\u Cech Nerve ]
Let $X$ be a finite CW complex, and $\mathcal{U}:=\{U_i: i\in I\}$ be an open cover of $X$. Then, we may define a simplicial set call the \underline{\textbf{\u Cech Nerve}} $N \mathcal{U}$ as follows: we have the assignment on objects $[n]\mapsto \{\textrm{functions from }[n] \textrm{ to } I: \cap^n_{i=1}U_{f(i)}\neq \emptyset \}$. The face maps and degeneracy maps are defined by deleting and inserting appropriate indices. 
\end{definition}
\end{tcolorbox}

Alternatively, we can think of a covering $\mathcal{U}$ as follows: suppose given a covering $X=\cup_{i\in I}U_i$; let $\mathcal{U}=\coprod_{i\in i} U_i$, and the covering is the obvious map $\mathcal{U}\to X$. Note that we have 
\[U_i\cap U_j=U_i\times_X U_j\]
so the $n$-fold fiber product $U\times_X...\times_X U$ is the disjoint union of $n$-fold intersections of opens in the cover. Then, the $n$th simplices of the \u Cech nerve is $\pi_0(  \underbrace{U\times_X...\times_X U }_{n-fold})$. The face maps are projections, and the degeneracy maps are various diagonal embeddings.


\begin{tcolorbox}[colback=red!5!white,colframe=red!30!white]
\begin{theorem}
If the covering $\mathcal{U}$ satisfies the property that arbitrary intersections of opens in the cover is either empty or contractible, then th realization $|N \mathcal{U}|$ is weakly equivalent to $X$.
\end{theorem}
\end{tcolorbox}


\section{Adam's conjecture}




We can dream of a proof here: if we have unstable adams operations $\psi^k: BU(n)\to BU(n)$, which are homotopy equivalences, with a pullback diagram
\[\begin{tikzcd}
	{BU(n-1)} & {BU(n-1)} \\
	{BU(n)} & {BU(n)}
	\arrow["{\psi^k}", from=1-1, to=1-2]
	\arrow["i", from=1-1, to=2-1]
	\arrow["i", from=1-2, to=2-2]
	\arrow["{\psi^k}"', from=2-1, to=2-2]
\end{tikzcd}\]
Then, the bundle $J\circ i$ is fiber homotopy equivalent $J\circ \psi^k$. However, unstable Adams operation does not exist on $BU(n)$. However, Sullivan proves that 
\[\begin{tikzcd}
	{\widehat{BU(n-1)}_p} & {\widehat{BU(n-1)}_p} \\
	{\widehat{BU(n)}_p} & {\widehat{BU(n)}_p}
	\arrow["{\psi^k}", from=1-1, to=1-2]
	\arrow["i", from=1-1, to=2-1]
	\arrow["i", from=1-2, to=2-2]
	\arrow["{\psi^k}"', from=2-1, to=2-2]
\end{tikzcd}\]
where $\psi^k$ is an unstable Adams operation on the profinite completion. 







\section{Algebraic Side}



\section*{Idea of proof}

\underline{\textbf{Step 1}}: Sullivan proves that the stable fiber homotopy types injects into profinite stable homotopy types. In particular, we have the isomorphism on classifying space level 
\[\textrm{Stable profinite theory}: \ \widehat{B}_{\infty}\cong B_{SG}\times K(\widehat{\mathbb{Z}^*},1)\]
\[\textrm{Stable theory}: \ BG\cong B_{SG}\times K(\mathbb{Z}/2,1)\]
where $B_{SG}=\varinjlim_n B_{SG(n)}$, and $B_{SG(n)}$ is the classifying space for the component of the identity map in $G(n)$. (Alternatively, it is also the universal cover of $BG(n)$). Thus, the classifying space for the stable theory is a direct factor of the stable profinite theory, and it suffice to formulate and prove the Adams conjecture in the profinite setting. 

\underline{\textbf{Step 2}}: We identify the classical Adams operation in the following way: the classical Adams operation 
\[K(X)\xrightarrow{\psi^k}K(X)\]
naturally desends to maps on the profinite completion, which factors as 
\[\prod_p \widehat{K(X)}_p\xrightarrow{\widehat{\psi}_p} \prod_p \widehat{K(X)}_p\]
and $\widehat{\psi^k}_p: \widehat{K(X)}_p\to \widehat{K(X)}_p$ is an isomorphism iff $k$ is prime to $p$. If $k$ is divisible by $p$, we redefine $\widehat{\psi^k}_p$ to be the identity map. After the redefinition, we obtain 
\[\widehat{K(X)}\xrightarrow{\psi^k}\widehat{K(x)}\]
which we call the "isomorphic" part of the Adams operation. 


\begin{tcolorbox}[colback=green!5!white,colframe=green!30!white]
\begin{remark}
Before the redefinition, in the case where $k|p$, we note that$ widehat{\psi^k}_p$ is topologically nilpotent. 
\end{remark}
\end{tcolorbox}

Following this, Sullivan observed that this isomorphic part of the Adams operation is compatible with the natural action of $Gal(\overline{\mathbb{Q}}|\mathbb{Q})$ in the category of profinite homotopy type and maps coming from the algebraic varieties defiend over $\mathbb{Q}$. In particular, there iare homomorphisms
\[Gal(\mathbb{Q}|\mathbb{Q})\to \widehat{\mathbb{Z}}^*\to \textrm{Aut}(\widehat{K(X)})\]
by letting $G$ act on the roots of unity. Moreover, for each $\psi^k$, we note that $k$ defines an element $(k)\in \widehat{\mathbb{Z}}^*$ by giving the automorphism 
\[(k)x=\begin{cases}
	k\cdot x & \textrm{if } x\in \widehat{\mathbb{Z}_p}, (k,p)=1\\
	x & \textrm{if } x\in  \widehat{\mathbb{Z}_p}, (k,p)\neq 1
\end{cases}\]
Clearly this is compatible with the Adams operation on profinite $K$ theory. Thus, we have identified the profinite Adams operation with the action of the profinite group $\widetilde{\mathbb{Z}}^*$. 

\underline{\textbf{Step 3}}: There is a natural action of the absolute Galois group $Gal(\overline{\mathbb{Q}}|\mathbb{Q})$ on the profinite classifying space $\widehat{BU}$, and the Adams operation is compactible with such action through the abelianization map.
\[Gal(\overline{\mathbb{Q}|\mathbb{Q}})\to \widehat{\mathbb{Z}}^*\]

The aboslute Galois group action is how the etale homotopy type theory factors in. Note that the absolute galois group acts algebraically on $\mathbb{C}^n$ and $\mathbb{CP}^n$, but with classical topology this action is wildly discontinuous. However, for every algebraic variety $V$, we may constrtcut an inverse system of nerves $N_{\alpha}$, with natural maps $V\to \{N_{\alpha}\}$ giving 
\[\widehat{\pi_1(V)}\cong \varprojlim_{\alpha}\pi_1N_{\alpha} \textrm{   and   } H^i(V; M)\cong \varinjlim H^i(N_{\alpha; M})\]
 for all finite coefficient $M$.


Sullivan proves that the the above isomorphism imply the profinite completion of $V$ can be constructed from the nerves 
\[\widehat{V}\cong \varprojlim_{\alpha} N_{\alpha}\]
in the sense of compact functors (with the extra assumption that $\pi_i(N_{\alpha})$ is finite. )


Since each $N_{\alpha}$ is constructed using the algebraic structure of $V$, and each automorphism $\sigma\in Gal(\overline{Q}|\mathbb{Q})$ determines a simplicial automorpihism of $N_{\alpha}$, and thus the profinite homotopy type of any complex algebraic variety defined over $\mathbb{Q}$.

Recall that the classifying space $BU(n)$ is constructed as the direct limit of complex grassmannians $\varinjlim_{k}Gr_n(k)$. Via Pl\"{u}cker embeddings, the complex grassmannians are naturally affine complex varieties embedded in projective space. Moreover, the defining polynomials also have coefficients in $\mathbb{Q}$.(Example here?)


By naturality and splitting principal, understanding the action of $Gal(\mathbb{C}|\mathbb{Q})$ on the profinite complex $K$-theory reduces to understanding the action on $\cup_n \widehat{\mathbb{CP}^n}\cong K(\widehat{\mathbb{Z}},2)$, which can be checked to be the composition 
\[Gal(\mathbb{C}|\mathbb{Q})\xrightarrow{\chi} \widehat{\mathbb{Z}}^*\to K(\widehat{Z},2)\]
and thus agrees with the isomorphic part of he Adams operations discussed above. 

\section{The Adams Conjecture-Proof}
Again, we note that by choosing $\sigma\in Gal(\overline{\mathbb{Q}}|\mathbb{Q})$ such that $\chi(\sigma)=k^{-1}$, we have the commutative square

\[\begin{tikzcd}
	{\widehat{BU(n)}} & {\widehat{BU}} \\
	{\widehat{BU(n)}} & {\widehat{BU}}
	\arrow[from=1-1, to=1-2]
	\arrow["\sigma", from=1-1, to=2-1]
	\arrow["{\psi^k}"', from=1-2, to=2-2]
	\arrow[from=2-1, to=2-2]
\end{tikzcd}\]

\underline{\textbf{Final Step}}: We note that the inclusion map $\widehat{BU(n)}\to \widehat{BU}$ is the tautological spherical fibration. The pullback of the fibration is also the bundle classfied by the map $\psi^k\circ i$. In other words, we have the homotopy cartesian square 
\[\begin{tikzcd}
	{\widehat{BU(n-1)}} & {\widehat{BU(n-1)}} \\
	{\widehat{BU(n)}} & {\widehat{BU(n)}}
	\arrow["i", from=1-1, to=2-1]
	\arrow["{\psi^k}"', from=1-2, to=1-1]
	\arrow["i"', from=1-2, to=2-2]
	\arrow["{\psi^k}"', from=2-2, to=2-1]
\end{tikzcd}\]
where we are pulling back along a homotopy equivalence, so the two fibrations are fiber homotopically equivalent, and we finish. 




\end{document}