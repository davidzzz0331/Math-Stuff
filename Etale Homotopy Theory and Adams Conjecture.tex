\documentclass{article}
\usepackage[utf8]{inputenc}
\usepackage{amsmath}
\usepackage{amsfonts}
\usepackage{amssymb}
\usepackage{tikz}
\usepackage{fullpage}
\usepackage{tikz-cd}
\usepackage{spectralsequences}
\usepackage{adjustbox}
\usepackage[backend=biber, style=alphabetic]{biblatex}
\usepackage{xfrac}
\usepackage{tcolorbox}
\usepackage{xcolor}
\usepackage{graphicx}
\graphicspath{ {D:/Chrome Downloads./} }
\usepackage[parfill]{parskip}
\usepackage{amsthm}
\addbibresource{sample.bib}
\usetikzlibrary{calc}
\theoremstyle{definition}
\newtheorem{theorem}{Theorem}[section]
\theoremstyle{definition}
\newtheorem{definition}{Definition}[theorem]
\theoremstyle{definition}
\newtheorem{remark}{Remark}[theorem]
\theoremstyle{definition}
\newtheorem{proposition}{Proposition}[theorem]
\theoremstyle{definition}
\newtheorem{lemma}[theorem]{Lemma}
\theoremstyle{definition}
\newtheorem{corollary}{Corollary}[theorem]
\theoremstyle{definition}
\newtheorem{example}{Example}[theorem]
\tikzset{curve/.style={settings={#1},to path={(\tikztostart)
    .. controls ($(\tikztostart)!\pv{pos}!(\tikztotarget)!\pv{height}!270:(\tikztotarget)$)
    and ($(\tikztostart)!1-\pv{pos}!(\tikztotarget)!\pv{height}!270:(\tikztotarget)$)
    .. (\tikztotarget)\tikztonodes}},
    settings/.code={\tikzset{quiver/.cd,#1}
        \def\pv##1{\pgfkeysvalueof{/tikz/quiver/##1}}},
    quiver/.cd,pos/.initial=0.35,height/.initial=0}
\title{\'Etale Homotopy Theory and Adams Conjecture}
\author{David Zhu}

\begin{document}
\maketitle



\begin{tcolorbox}[colback=purple!5!white,colframe=purple!75!black]
\begin{definition}[\u Cech Nerve ]
Let $X$ be a finite CW complex, and $\mathcal{U}:=\{U_i: i\in I\}$ be an open cover of $X$. Then, we may define a simplicial set call the \underline{\textbf{\u Cech Nerve}} $N \mathcal{U}$ as follows: we have the assignment on objects $[n]\mapsto \{\textrm{functions from }[n] \textrm{ to } I: \cap^n_{i=1}U_{f(i)}\neq \emptyset \}$. The face maps and degeneracy maps are defined by deleting and inserting appropriate indices. 
\end{definition}
\end{tcolorbox}

Alternatively, we can think of a covering $\mathcal{U}$ as follows: suppose given a covering $X=\cup_{i\in I}U_i$; let $\mathcal{U}=\coprod_{i\in i} U_i$, and the covering is the obvious map $\mathcal{U}\to X$. Note that we have 
\[U_i\cap U_j=U_i\times_X U_j\]
so the $n$-fold fiber product $U\times_X...\times_X U$ is the disjoint union of $n$-fold intersections of opens in the cover. Then, the $n$th simplices of the \u Cech nerve is $\pi_0(  \underbrace{U\times_X...\times_X U }_{n-fold})$. The face maps are projections, and the degeneracy maps are various diagonal embeddings.


\begin{tcolorbox}[colback=red!5!white,colframe=red!30!white]
\begin{theorem}
If the covering $\mathcal{U}$ satisfies the property that arbitrary intersections of opens in the cover is either empty or contractible, then th realization $|N \mathcal{U}|$ is weakly equivalent to $X$.
\end{theorem}
\end{tcolorbox}


\section{Adam's conjecture}


\begin{tcolorbox}[colback=purple!5!white,colframe=purple!75!black]
\begin{definition}
Let $X$ be compact Hausdorff and let $KU(X)$ be Grothendieck group of complex vector bundles over $X$, and let $\mathcal{SF}(X)$ be the Grothendieck group of sphere bundles over $X$ modulo fiber homotopy equivalence. 
\end{definition}
\end{tcolorbox}


\begin{tcolorbox}[colback=red!5!white,colframe=red!30!white]
\begin{theorem}
The stable sphere bundles over $X$ is classified by the the groups of self-homotopy equivalences of $S^n$, which we denote by $G(n):=\textrm{Equiv}(S^n,S^n)$. 
\end{theorem}
\end{tcolorbox}


\begin{tcolorbox}[colback=blue!5!white,colframe=blue!30!white]
\begin{proposition}
The complex $J$-homomorphism $J: KU(X)\to \mathcal{SF}(X)$ is induced by a map between classifying spaces, which we also denote 
\[J: BU\to BG:=\varinjlim_{n} BG(n)\]
\end{proposition}
\end{tcolorbox}


\begin{tcolorbox}[colback=purple!5!white,colframe=purple!75!black]
\begin{definition}
The Adams' operation $\psi^k: KU(X)\to KU(X)$ is induced by a map of classifying spaces
\[\psi^k: BU\to BU\]
\end{definition}
\end{tcolorbox}


\begin{tcolorbox}[colback=red!5!white,colframe=red!30!white]
\begin{theorem}[The Adams Conjecture]
The composite 
\[BU\xrightarrow{\psi^k-1}BU\xrightarrow{J}BG\]
is nullhomotopic up to multiplication by some $k^n$. 
\end{theorem}
\end{tcolorbox}


\begin{tcolorbox}[colback=blue!5!white,colframe=blue!30!white]
\begin{proposition}
The composite $J\circ i: BU(n)\to BU\to BG$, classifyies a sphere bundle over $BU(n)$, and is fiber homotopy equivalent to the fibration 
\[BU(n-1)\to BU(n)\]
\end{proposition}
\end{tcolorbox}




\section{Algebraic Side}

Recall that the classifying space $BU(n)$ is constructed as the direct limit of complex grassmannians $\varinjlim_{k}Gr_n(k)$. Via Pl\"{u}cker embeddings, the complex grassmannians are naturally affine complex varieties embedded in projective space. Moreover, the defining polynomials also have coefficients in $\mathbb{Q}$.(Example here?)

Thus, an automorphism $Gal(\mathbb{C}|\mathbb{Q})$ gives rise to an automorphism of these varieties, but they are wildly discontinuous in the classical topology. However, the Cech/etale story tells us that 



\end{document}