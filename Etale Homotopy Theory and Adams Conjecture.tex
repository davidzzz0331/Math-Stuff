\documentclass{article}
\usepackage[utf8]{inputenc}
\usepackage{amsmath}
\usepackage{amsfonts}
\usepackage{amssymb}
\usepackage{tikz}
\usepackage{fullpage}
\usepackage{tikz-cd}
\usepackage{spectralsequences}
\usepackage{adjustbox}
\usepackage[backend=biber, style=alphabetic]{biblatex}
\usepackage{xfrac}
\usepackage{tcolorbox}
\usepackage{xcolor}
\usepackage{graphicx}
\graphicspath{ {D:/Chrome Downloads./} }
\usepackage[parfill]{parskip}
\usepackage{amsthm}
\addbibresource{Etale Talk.bib}
\usetikzlibrary{calc}
\theoremstyle{definition}
\newtheorem{theorem}{Theorem}[section]
\theoremstyle{definition}
\newtheorem{definition}{Definition}[theorem]
\theoremstyle{definition}
\newtheorem{remark}{Remark}[theorem]
\theoremstyle{definition}
\newtheorem{proposition}{Proposition}[theorem]
\theoremstyle{definition}
\newtheorem{lemma}[theorem]{Lemma}
\theoremstyle{definition}
\newtheorem{corollary}{Corollary}[theorem]
\theoremstyle{definition}
\newtheorem{example}{Example}[theorem]
\tikzset{curve/.style={settings={#1},to path={(\tikztostart)
    .. controls ($(\tikztostart)!\pv{pos}!(\tikztotarget)!\pv{height}!270:(\tikztotarget)$)
    and ($(\tikztostart)!1-\pv{pos}!(\tikztotarget)!\pv{height}!270:(\tikztotarget)$)
    .. (\tikztotarget)\tikztonodes}},
    settings/.code={\tikzset{quiver/.cd,#1}
        \def\pv##1{\pgfkeysvalueof{/tikz/quiver/##1}}},
    quiver/.cd,pos/.initial=0.35,height/.initial=0}
\title{\'Etale Homotopy Theory and the Adams Conjecture}
\author{David Zhu}

\begin{document}
\maketitle

For simplicity, we only present the proof of the Adams conjecture for complex vector bundles, with real case following similarly but with modifications.

\section{The Adams Conjecture}
\begin{tcolorbox}[colback=purple!5!white,colframe=purple!75!black]
	\begin{definition}
	Let $X$ be compact Hausdorff and let $KU(X)$ be Grothendieck group of complex vector bundles over $X$, and let $\mathcal{SF}(X)$ be the Grothendieck group of stable sphere bundles over $X$. The \underline{\textbf{$J$-homomorphism}} can be viewed as the homomorphism
	\[J: KU(X)\to \mathcal{SF}(X)\]
	by sending a complex vector bundle to its fiberwise one-point compactification. 
	\end{definition}
	\end{tcolorbox}

	
	\begin{tcolorbox}[colback=red!5!white,colframe=red!30!white]
		\begin{theorem}
		The stable sphere bundles over $X$ is classified by the the groups of self-homotopy equivalences of $S^n$, which we denote by $G(n):=\textrm{Equiv}(S^n,S^n)$. 
		\end{theorem}
		\end{tcolorbox}
		

	\begin{tcolorbox}[colback=blue!5!white,colframe=blue!30!white]
		\begin{proposition}
		The complex $J$-homomorphism $J: KU(X)\to \mathcal{SF}(X)$ is induced by a map between respective classifying spaces, which we also denote 
		\[J: BU\to BG:=\varinjlim_{n} BG(n)\]
		\end{proposition}
		\end{tcolorbox}


	
	
	
	
	\begin{tcolorbox}[colback=purple!5!white,colframe=purple!75!black]
	\begin{definition}
	\underline{\textbf{The $k$-th Adams Operation}} $\psi^k: KU(X)\to KU(X)$ is defined to be a ring homomorphism that is natural in $X$, and satisfies 
	\[\psi^k(L)=L^{\otimes k}\]
	where $L$ is any line bundle over $X$. 	
	\end{definition}
	\end{tcolorbox}
	Note that $\psi^k$ is unique by the splitting principal. Adams constructed such operation of $K$-theory using exterior powers and Newton polynomials.
	
	\begin{tcolorbox}[colback=red!5!white,colframe=red!30!white]
	\begin{theorem}[\underline{\textbf{The Adams Conjecture}}]
	The composite 
	\[BU\xrightarrow{\psi^k-1}BU\xrightarrow{J}BG\]
	is nullhomotopic up to multiplication by some $k^n$. 
	\end{theorem}
	\end{tcolorbox}
	
	
	\begin{tcolorbox}[colback=green!5!white,colframe=green!30!white]
	\begin{remark}
	Historically, Adams realized Whithead's $J$-homomorphism $J: \pi_i(SO)\to \pi^s_i$ can be used to understand the stable homotopy groups of spheres, since $\pi_i(SO)$ is known by Bott periodicity. The Adams conjecture then serves a central role in the identification of the image of the J-homomorphism.
	\\

	Multiple proofs of the Adams conjecture were completed around 1970. Notably, Quillen's (second) proof using Brauer lifting and by computing the cohomology of $BGL(\mathbb{F}_q)$ led to his later construction of higher algebraic $K$-theory. In this talk, we present Sullivan's proof using profinite completion and \'etale homotopy theory, which turns the Adams conjecture into a beautiful case of Galois symmetry of algebraic varieties.


	\end{remark}
	\end{tcolorbox}



\section{Preview of Proof}

We are trying to show that the map
\[BU\xrightarrow{id} BU\xrightarrow{J}BG\]
and 
\[BU\xrightarrow{\psi^k} BU\xrightarrow{J}BG\]
are homotopic modulo $k$. The first map classifies the spherical bundle associated to the tautological bundle $\gamma$ over $BU$, and the second map is the spherical bundle associated to the pullback $\psi^{k*}\gamma$, so it suffices to show that these sphere bundles are fiber homotopy equivalent. To make a further reduction, the tautological bundle over $BU$ are constructed as the limit of the tautological bundles over $BU(n)$, so it suffices to prove that the "unstable version" of the Adams conjecture is true for all $n$. 

It turns out that the sphere bundle associated to the tautological bundle over $BU(n)$ is fiber homotopy equivalent to $BU(n-1)$. Thus, we will be done if we have unstable Adams operations $\psi^k: BU(n)\to BU(n)$, which are homotopy equivalences, with a pullback diagram
\[\begin{tikzcd}
	{BU(n-1)} & {BU(n-1)} \\
	{BU(n)} & {BU(n)}
	\arrow["{\psi^k}", from=1-1, to=1-2]
	\arrow["i", from=1-1, to=2-1]
	\arrow["i", from=1-2, to=2-2]
	\arrow["{\psi^k}"', from=2-1, to=2-2]
\end{tikzcd}\]
It follows that the bundle $J\circ i$ is fiber homotopy equivalent $J\circ \psi^k$. However, such unstable Adams operation does not exist on $BU(n)$.



Sullivan's idea is to produce an unstable Adams operation on the pro-$p$ completion $BU(n)$, where one naturally has the diagram
\[\begin{tikzcd}
	{\widehat{BU(n-1)}_p} & {\widehat{BU(n-1)}_p} \\
	{\widehat{BU(n)}_p} & {\widehat{BU(n)}_p}
	\arrow["{\psi^k}", from=1-1, to=1-2]
	\arrow["i", from=1-1, to=2-1]
	\arrow["i", from=1-2, to=2-2]
	\arrow["{\psi^k}"', from=2-1, to=2-2]
\end{tikzcd}\]
And then he showed that this implies the classical Adams conjecture. 



\section{Profinite Completion}
We recall some facts about profinte completion in algebra.  
\begin{tcolorbox}[colback=purple!5!white,colframe=purple!75!black]
\begin{definition}
A \underline{\textbf{profinite group}} is a topological group that is the projective limit of finite discrete groups.
\end{definition}
\end{tcolorbox}


\begin{tcolorbox}[colback=yellow!5!white,colframe=yellow!30!white]
\begin{example}
The Galois group of an infinite field extension $K|F$ is the profinite limit of the Galois groups $Gal(L|F)$, where $L$ ranges from all finite Galois extensions of $F$. \\

Let $\mathbb{Q}^{ab}$ denote the maximal abelian extension of $\mathbb{Q}$. From class field theory, one know that any abelian extension of $\mathbb{Q}$ is contained in some cyclotomic extension (meaning joining some root of unity). Thus, 
\[Gal(\mathbb{Q}^{ab}|\mathbb{Q})=\varprojlim Gal(\mathbb{Q}(\zeta_{n})|\mathbb{Q})=\varprojlim_n (\mathbb{Z}/n)^{\times}\]
\end{example}
\end{tcolorbox}
So here is Sullivan's idea of profinite completion of spaces, inspired by Artin-Mazur's 
\'etale homotopy theory. We start by the following example, and everything we consider will be connected CW complexes:


\begin{tcolorbox}[colback=yellow!5!white,colframe=yellow!30!white]
\begin{example}
Fix a space $F$ with finite homotopy groups. For every CW complex $Y$, the set $[Y,F]$ carries a compact topology since we have 
\[[Y,F]=\varprojlim_{\alpha}[Y_{\alpha}, F]\]
where $Y_{\alpha}$ ranges from all finite subcomplexes of $Y$ and as $[Y_{\alpha}, F]$ is finite by obstruction theory.
\end{example}
\end{tcolorbox}
Now fix a CW complex $X$. Let $X/\mathcal{F}$ denote the category whose objects are maps $X\to F$ for some $F$ with finite homotopy groups, and morphisms diagrams of the form 
\[\begin{tikzcd}
	& X \\
	{F'} && F
	\arrow[from=1-2, to=2-1]
	\arrow[from=1-2, to=2-3]
	\arrow[from=2-1, to=2-3]
\end{tikzcd}\]
Artin-Mazur showed that this category is equivalent to a small filtering category. Then, one may consider the functor $\widehat{X}$ from the homotopy category of finite CW complexes to compact Hausdorff spaces, defined by $\widehat{X}(Y)=\varprojlim_{X/\mathcal{F}}[Y,F]$. Sullivan then developed the theory of \underline{\textbf{compact Brownian functors}} and showed that this functor is represented by a CW complex, which we also denote by $\widehat{X}$. 


\begin{tcolorbox}[colback=green!5!white,colframe=green!30!white]
\begin{remark}
Note that $\widehat{X}$ is not the same as the naive inverse limit $\varprojlim_{X/\mathcal{F}}F$. Recall that given an inverse system of CW complexes $X:=\varprojlim_n X_n$ the Milnor exact sequence tells us that $\pi_i(X)$ differs from $\varprojlim_n \pi_i(X_n)$ by a $\textrm{lim}^1$ term. In contrast, we have $\pi_n(\widehat{X})=\varprojlim_n \pi_n(X_n)$ for all $n$ by construction.
\end{remark}
\end{tcolorbox}




\begin{tcolorbox}[colback=purple!5!white,colframe=purple!75!black]
\begin{definition}
The \underline{\textbf{profinite completion}} of a connected CW complex $X$ is the space $\widehat{X}$ constructed above, with the structure morphism $X\to \widehat{X}$ induced from the limit of the morphisms $X\to F$.
\end{definition}
\end{tcolorbox}

\begin{tcolorbox}[colback=red!5!white,colframe=red!30!white]
\begin{theorem}
For a CW complex $X$, the profinite completion $X\to \widehat{X}$, and the inverse system of spaces ${F}$ in $X/\mathcal{F}$, the following hold:
\begin{enumerate}
	\item $\widehat{\pi_1(X)}\cong \varprojlim_{X/\mathcal{F}}\pi_1(F)$.
	\item $H^*(X;M)\cong \varinjlim_{X/\mathcal{F}}H^*(F;M)$ for all finite coefficient $M$.
\end{enumerate}
The converse is also true: let ${F_{\alpha}}$ be any inverse system of spaces with finite homotopy groups, together with maps $X\to F_{\alpha}$. If the system satisfies the criteria listed above, then $\widehat{X}\cong \varprojlim_{\alpha} F_{\alpha}$, where the limit means the compact Brownian functor construction.
\end{theorem}
\end{tcolorbox}

If the reader if familiar, the two criteria in Theorem 3.1 correspond to the comparison theorems of \'etale fundamental group and  \'etale cohomology respectively. The general construction of this is motivated by the theory of \'etale homotopy type, which tried to capture the theory of \'etale fundamental group and cohomology into one projective system of ordinary homotopy types.

By replacing $X/\mathcal{F}$ with the category whose objects are maps $X\to F$, with $F$ with finite $p$-groups, we may also form the pro-$p$ completion of $X$, which we will denote by $X\to \widehat{X}_p$. Sullivan then showed

\begin{tcolorbox}[colback=red!5!white,colframe=red!30!white]
\begin{theorem}
If $X$ is nilpotent of finite type, then we have an isomorphism 
\[\widehat{X}\to \prod_p \widehat{X}_p\]
\end{theorem}
\end{tcolorbox}
which is a nice analog of the fact $\widehat{\mathbb{Z}}=\prod_p \mathbb{Z}_p$.








Sullivan then introduced profinite $K$-theory, defined by 
\[\widehat{K}(X)=\varprojlim_n K(X)\otimes \mathbb{Z}/n\]
and showed that it is classified by $\widehat{BU}$, which is also built from the colimit of $\widehat{BU}(n)$. Moreover, Sullivan developed the theory of profinite/ p-completed spherical fibrations, with the main theorem being:

\begin{tcolorbox}[colback=red!5!white,colframe=red!30!white]
\begin{theorem}
The stable fiber homotopy types injects into profinite stable homotopy types. In particular, we have the isomorphism on classifying space level 
	\[\textrm{Stable profinite theory}: \ \widehat{B}_{\infty}\cong B_{SG}\times K(\widehat{\mathbb{Z}^*},1)\]
	\[\textrm{Stable theory}: \ BG\cong B_{SG}\times K(\mathbb{Z}/2,1)\]
	where $B_{SG}=\varinjlim_n B_{SG(n)}$, and $B_{SG(n)}$ is the classifying space for the component of the identity map in $G(n)$. . Thus, the classifying space for the stable theory is a direct factor of the stable profinite theory.
\end{theorem}
\end{tcolorbox}
A direct consequence is that it suffice to formulate and prove the Adams conjecture in the profinite setting, and by the splitting theorem 3.2 above, every pro-p completed setting. We recall the diagram 
\[\begin{tikzcd}
	{\widehat{BU(n-1)}_p} & {\widehat{BU(n-1)}_p} \\
	{\widehat{BU(n)}_p} & {\widehat{BU(n)}_p}
	\arrow["{\psi^k}", from=1-1, to=1-2]
	\arrow["i", from=1-1, to=2-1]
	\arrow["i", from=1-2, to=2-2]
	\arrow["{\psi^k}"', from=2-1, to=2-2]
\end{tikzcd}\]
so having a compatible Adams operation that is a homotopy equivalence of $\widehat{BU(n)}_p$ is what we need for the proof. For this, we recognize $BU(n)$ as a complex variety, and turn to \'etale homotopy theory.



\section{\'Etale Homotopy Theory}
Suppose we have a smooth complex variety $X$. Its complex points has a complex analytic topology, which we denote $X^{an}$. The Zariski topology is much more coarse than the analytic topology, so how can we extract classical algebraic invariants such as homotopy groups and cohomology groups of $X^{an}$ from algebraic data of $X$? On the other hand, the homotopy type given by the analytic topology is not an algebraic invariant: Serre gave an example where different embeddings of a curve produced different $\pi_1$ in the analytic topology. Grothendieck's idea is to consider the \'etale topology, which will give us more information.


\subsection{\'Etale morphisms}
The motivation is to have an analog of covering space theory in algebraic geometry, and the correct notion for that is a finite \'etale morphism.


\begin{tcolorbox}[colback=purple!5!white,colframe=purple!75!black]
\begin{definition}
A morphism of schemes $f: X\to Y$ is \underline{\textbf{\'etale}} if it is flat and unramified.
\end{definition}
\end{tcolorbox}


\begin{tcolorbox}[colback=blue!5!white,colframe=blue!30!white]
\begin{proposition}
Suppose $Y$ is a connected variety/scheme. A finite \'etale morphism $\varphi: X\to Y$ is open, closed, and thus surjective. Moreover, for each point $y\in Y$, there is an \'etale neighborhood $(U,u)\to (Y,y)$ such that $X\times_Y U$ is a disjoint union of open subvarieties/subschemes, each mapping isomorphically down to $U$.
\end{proposition}
\end{tcolorbox}
From this we see that finite \'etale morphisms are indeed natural analog of a finite covering spaces. However, there is no direct algebraic analog for the "universal cover," when it is often an infinite sheeted covering.
\begin{tcolorbox}[colback=yellow!5!white,colframe=yellow!30!white]
\begin{example}
The universal cover of $\mathbb{C}-\{0\}$ is given by 
\[\mathbb{C}\xrightarrow{exp} \mathbb{C}-\{0\}\]
which is a $\mathbb{Z}$-sheeted cover.  
\end{example}
\end{tcolorbox}


\begin{tcolorbox}[colback=yellow!5!white,colframe=yellow!30!white]
\begin{example}
There is no algebraic analog for the universal covering space as in example $4.1.1$: any morphism 
\[\mathbb{A}^1_{\mathbb{C}}\to \mathbb{A}^1_{\mathbb{C}}-\{0\}\]
must be a constant map, hence not even \'etale. The finite \'etale covers of $\mathbb{A}^1_{\mathbb{C}}-\{0\}$ are of the form 
\[\mathbb{A}^1_{\mathbb{C}}-\{0\}\xrightarrow{z\mapsto z^n}\mathbb{A}^1_{\mathbb{C}}-\{0\}\]
for some $n$. 
\end{example}
\end{tcolorbox}
It is not hard to check by hand that these morphisms are \'etale. The fact that these are essentially the only ones is given by Riemann existence theorem.

\begin{tcolorbox}[colback=green!5!white,colframe=green!30!white]
\begin{remark}
The lack of universal cover illustrates one of the reason that we need to considering pro-objects so often.
\end{remark}
\end{tcolorbox}

\subsection{\'Etale Homotopy Type}
Recall in topology, we have the Leray nerve theorem, which says given a "good" open covering of $X$ (good meaning each finite intersection of opens sets in the cover is either empty of contractible), then the associated nerve complex is weak homotopy equivalent to $X$.


\begin{tcolorbox}[colback=yellow!5!white,colframe=yellow!30!white]
\begin{example}
Let $\{U_1,..,U_n\}$ be a Zariski covering of an irreducible variety $X$. Then, the associated nerve complex is $\Delta^{n-1}$.
\end{example}
\end{tcolorbox}
This is because any intersection of non-empty open set in the Zariski topology is non-empty. Grothendieck's idea is then to replace Zariski open sets with \'etale opens.

\begin{tcolorbox}[colback=purple!5!white,colframe=purple!75!black]
\begin{definition}
An \underline{\textbf{\'etale covering}} of a scheme $V$ consists of a collection of \'etale morphisms $\{\varphi_i: U_i\to V\}$ such that $V=\cup_i\varphi_i(U_i)$.
\end{definition}
\end{tcolorbox}
An \'etale morphism is open; moreover, an \'etale morphism between smooth affine varieties over an algebraically closed field is precisely a regular map that induces an isomorphism of Zariski tangent spaces. Thus, one can think of \'etale morphism as a generalization for local diffeomorphisms.

The idea of \'etale homotopy type is consider all \'etale covers over a (nice) variety/scheme $X$, and take the Cech-like nerve (Artin-Mazur approach or Lubkin approach) of each \'etale cover and form an inverse system of the nerves. One then can take the homotopy groups and cohomology group of the inverse system of nerves, which produces a "topological" invariant for the scheme. For schemes that are not so nice, this method has to be modified (see remark 4.3.1 below).


We now explain more in detail Sullivan's slight variation of Lubkin's approach: suppose we have an \'etale covering $\mathcal{U}_{\alpha}:=\{\varphi_i: U_i\to V\}$ of a smooth complex variety $V$. Then, consider $\mathcal{U}_{\alpha}$ as a full category of schemes over $V$, so that morphism $U_i\to U_j$ commutes with the structure morphism to $V$. Let $N_{\alpha}$ denote the realization of $\mathcal{U}_{\alpha}$, and we get an inverse system of complexes $\{N_{\alpha}\}$ indexed by \'etale covers of $V$ and filtered by refinement. The inverse system of nerves is the \underline{\textbf{\'etale homotopy type}} of $V$.

The work of Artin-Mazur yield the following comparison theorem:

\begin{tcolorbox}[colback=red!5!white,colframe=red!30!white]
\begin{theorem}[Comparison Theorem]
If $V$ is normal, then the homotopy groups of the \'etale nerves are finite. Moreover,  
\[\widehat{\pi_1(V^{an})}\cong \varprojlim \pi_1{N_{\alpha}}\]
and 
\[H^*(V^{an};M)\cong \varinjlim H^*(N_{\alpha};M)\]
for finite and profinite coefficients.
\end{theorem}
\end{tcolorbox}
so indeed the \'etale homotopy type $\varprojlim N_{\alpha}$ does give the profinite completion of $V^{an}$, through a purely algebraic method. 



\begin{tcolorbox}[colback=green!5!white,colframe=green!30!white]
\begin{remark}
For more general schemes, the category of etale covers (which is filtered) is not fine enough to compute the correct invariants (does not recover \'etale cohomology). Artin and Mazur uses Verdier method of "hypercoverings", and to make the hypercoverings filtered, one has to move to the homotopy category of hypercovers. They also completely remained in the land of pro-objects, vis-a-vis Sullivan's approach to finding an actual CW complex to represent the projective system.
\end{remark}
\end{tcolorbox}

\section{The Case of $\mathbb{CP}^1$}
We now discuss the main example to the construction above. Let $\mathcal{U}_n$ denote the following \'etale cover of $\mathbb{P}^1_{\mathbb{C}}$: take $V_0=\mathbb{A}^1$ and $V_1=\mathbb{A}^1$ be the standard affine opens of $\mathbb{P}^1$ and $V_{01}:=V_1\cap V_2\cong \mathbb{A}^1-\{0\}$. Then, we may \'etale cover $\mathbb{P}^1$ by 
$$U_0=\mathbb{A}_1\xrightarrow{\cong }V_0$$
$$U_1=\mathbb{A}_1\xrightarrow{\cong }V_0$$
$$U_2=\mathbb{A}_1-\{0\}\xrightarrow{z\mapsto z^n }V_{01}$$

The category $\mathcal{U}_n$ has the following morphisms: there is a unique morphism from $U_2$ to $U_0$ and to $U_1$, and $U_2$ has automorphisms group $\mathbb{Z}/n$. We may represent  $\mathcal{U}_n$ as 
\[\begin{tikzcd}
	\bullet && {\mathbb{Z}/n} && \bullet
	\arrow[from=1-3, to=1-1]
	\arrow[from=1-3, to=1-5]
\end{tikzcd}\]
and the nerve of the category is $\Sigma K(\mathbb{Z}/n,1)$. One may see this as attaching two cones over $K(\mathbb{Z}/n,1)$. 


We may consider the inverse system of nerves $\{N\mathcal{U}_{n}\}=\{\Sigma K(\mathbb{Z}/n,1)\}$ partially ordered by divisiblility of $n$. The system is homotopically cofinal in the inverse system of all nerves of etale coverings, so it represents the \'etale homotopy type of $\mathbb{P}^1$. Let $\widehat{X}=\varprojlim_n \Sigma K(\mathbb{Z}/n,1)$ as compact brownian functors. One easily see that $\pi_1(\widehat{X})=0=\pi_1(\mathbb{CP}^1)^{\wedge}$, and 
\[H^*(\widehat{X};M)=\varinjlim_n H^*(\Sigma K(\mathbb{Z}/n,1);M)=H^*(\mathbb{CP}^1;M)\]
for all finite coefficient system $M$. Thus, we have constructed the profintie completion of $\mathbb{CP}^1$ purely from the \'etale coverings of $\mathbb{P}^1$.

\section{Galois symmetry}
Note that $BU(n)$ is the inductive limit of grassmannians, which are naturally complex varieties through Plucker embedding. In fact, they are varieties over $\mathbb{Q}$, and there is a natural absolute Galois group $Gal(\overline{\mathbb{Q}}|\mathbb{Q})$ action on the grassmannian varieties. Note that this action is algebraic, and highly discontinuous in the analytic topology. Thus naively, the action does not induce an homomorphism of singular cohomology.



The good new is the action also induces an automorphism of the \'etale covers, as we have the pullback
\[\begin{tikzcd}
	{\sigma^*U} & U \\
	V & V
	\arrow[from=1-1, to=1-2]
	\arrow[from=1-1, to=2-1]
	\arrow[from=1-2, to=2-2]
	\arrow["\sigma", from=2-1, to=2-2]
\end{tikzcd}\]
Thus, the aboslute Galois group acts on the \'etale nerves and thus the profinite/\'etale homotopy type of the variety, and in turn the colimits of varieties such as $BU$. Thus, there is a Galois action on the profinite $K$-theory. We will detect the action through cohomology. It is easy to verify for the example 
\[\mathbb{A}^1-\{0\}\xrightarrow{z\mapsto z^n}\mathbb{A}^1-\{0\}\]
which has covering automoprhism $\mathbb{Z}/n$, the aboslute Galois group acts by fixing the objects, and on the morphisms by moving the $n$th root of unity around. Thus, there is a visible action on each nerve $\Sigma K(\mathbb{Z}/n,1)$, and induces an action on cohomology
\[H^2(\widehat{\mathbb{P}}^1;\widehat{\mathbb{Z}})=\varinjlim H^2(\Sigma K(\mathbb{Z}/n,1);\widehat{\mathbb{Z}})\]



\begin{tcolorbox}[colback=red!5!white,colframe=red!30!white]
\begin{theorem}
The absolute Galois $Gal(\overline{\mathbb{Q}}|\mathbb{Q})$ acts via the cyclotomic character through $\widehat{\mathbb{Z}}^*$ on $H^2(\widehat{\mathbb{P}}^1;\widehat{\mathbb{Z}})$, and the action is given by sending the generator $x\to ax$, where $a\in \widehat{\mathbb{Z}}^*$.
\end{theorem}
\end{tcolorbox}


\begin{tcolorbox}[colback=green!5!white,colframe=green!30!white]
\begin{corollary}
	The absolute Galois group $Gal(\overline{\mathbb{Q}}|\mathbb{Q})$ acts via the cyclotomic character through $\widehat{\mathbb{Z}}^*$ on $H^*(\widehat{BU}(n);\mathbb{Z}_p)=\mathbb{Z}_p[c_1,...,c_n]$, and the action is given by map on the ith chern classs $c_i\to a^ic_i$, where $a\in \widehat{\mathbb{Z}}^*$.
\end{corollary}
\end{tcolorbox}
The first Chern class classifies isomorphism class of line bundles. So for any $k$, we may choose a class $\sigma\in Gal(\overline{\mathbb{Q}}|\mathbb{Q})$ such that $\chi(\sigma)=k\in \widehat{\mathbb{Z}}^*$. It follows from the profinite theory of characteristic classes, the homotopy equivalence 
\[\widehat{BU(n)}_p\xrightarrow{\sigma} \widehat{BU(n)}_p\]
sends lines bundles to their $k$th power if $p\nmid k$. Moreover, the aboslute Galois action is compactible with the filtration of $\widehat{BU}$, and the proof is complete.

Lastly, we present the reformulation of Adams conjecture under Sullivan's approach

\begin{tcolorbox}[colback=red!5!white,colframe=red!30!white]
\begin{theorem}
	The stable fiber homotopy type of elements in profinite K-theory is constant on the orbits of the Galois group.
\end{theorem}
\end{tcolorbox}




\end{document}