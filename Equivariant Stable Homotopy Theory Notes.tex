\documentclass{article}
\usepackage[utf8]{inputenc}
\usepackage{amsmath}
\usepackage{amsfonts}
\usepackage{amssymb}
\usepackage{tikz}
\usepackage{fullpage}
\usepackage{tikz-cd}
\usepackage{spectralsequences}
\usepackage{adjustbox}
\usepackage[backend=biber, style=alphabetic]{biblatex}
\usepackage{xfrac}
\usepackage{tcolorbox}
\usepackage{xcolor}
\usepackage{graphicx}
\graphicspath{ {D:/Chrome Downloads./} }
\usepackage[parfill]{parskip}
\usepackage{amsthm}
\usetikzlibrary{calc}
\addbibresource{EHT.bib}
\theoremstyle{definition}
\newtheorem{theorem}{Theorem}[section]
\theoremstyle{definition}
\newtheorem{definition}{Definition}[theorem]
\theoremstyle{definition}
\newtheorem{remark}{Remark}[theorem]
\theoremstyle{definition}
\newtheorem{proposition}{Proposition}[theorem]
\theoremstyle{definition}
\newtheorem{lemma}[theorem]{Lemma}
\theoremstyle{definition}
\newtheorem{corollary}{Corollary}[theorem]
\theoremstyle{definition}
\newtheorem{example}{Example}[theorem]
\tikzset{curve/.style={settings={#1},to path={(\tikztostart)
    .. controls ($(\tikztostart)!\pv{pos}!(\tikztotarget)!\pv{height}!270:(\tikztotarget)$)
    and ($(\tikztostart)!1-\pv{pos}!(\tikztotarget)!\pv{height}!270:(\tikztotarget)$)
    .. (\tikztotarget)\tikztonodes}},
    settings/.code={\tikzset{quiver/.cd,#1}
        \def\pv##1{\pgfkeysvalueof{/tikz/quiver/##1}}},
    quiver/.cd,pos/.initial=0.35,height/.initial=0}
\title{Equivariant Stable Homotopy Notes}
\author{David Zhu}

\begin{document}
\maketitle
For the entire note, we will assume a group $G$ to be a compact Lie group, and subgroups $H\subset G$ are always closed. 

\section{Unstable Equivariant Homotopy Theory}
\subsection{G-CW Complexes}
Fix a compact Lie group $G$ acting on a space $X$. Similar to $CW$-complexes, we want to deconstruct $X$ into cells, but this time with the addditional data of the $G$-action along with each cell. The idea is that cells are of the form of a product $G/H\times D^{n}$, where $G$ acts trivially on $D^n$, and $G/H$ "represents" the orbits of $D^n$. To make this work, $H$ must be the isotropy group of $D^n$. 


\begin{tcolorbox}[colback=purple!5!white,colframe=purple!75!black]
\begin{definition}
A \underline{\textbf{G-CW complex}} is the sequential colimit of spaces $X_n$, where $X_{n+1}$ is a pushout: 
\[\begin{tikzcd}
\coprod G/H\times S^n\arrow[r]\arrow[d]&X_n\arrow[d]\\
\coprod G/H\times D^{n+1}\arrow[r]&X_{n+1}
\end{tikzcd}\]
We will denote  $G/H\times D^{n}$ as an \underline{\textbf{n-cell}}. 
\end{definition}
\end{tcolorbox}

\begin{tcolorbox}[colback=green!5!white,colframe=green!30!white]
\begin{remark}
Note that the topological dimension of an $n$-cell in a $G$-CW complex might be greater than $n$. For example, a $0$-cell $S^1/e\times *$ is one dimensional. 
\end{remark}
\end{tcolorbox}

\begin{tcolorbox}[colback=yellow!5!white,colframe=yellow!30!white]
\begin{example}
Let $G=C_2$ acting on $S^2$ by rotation by $\pi$ along the Z-axis. It has a $G$-CW structure given by the following cells: $2$ zero-cells $C_2/C_2\times *$, which are the poles corresponding to the fixed points of the $C_2$ action. $1$ one-cell $C_2/e\times D^1$, which are the two great circles joining the poles; $1$ two-cell $C_2/C_2\times D^2$, which are the two hemispheres.  
\end{example}
\end{tcolorbox}


\begin{tcolorbox}[colback=yellow!5!white,colframe=yellow!30!white]
\begin{example}
    Let $G=C_2$ acting on $S^2$ by the antipodal map. It has a $G$-CW structure given by the following cells: $1$ zero-cells $C_2/e\times *$, which are the poles; $1$ one-cell $C_2/e\times D^1$, which are the two great circles joining the poles; $1$ two-cell $C_2/C_2\times D^2$, which are the two hemispheres. 
\end{example}
\end{tcolorbox}


\begin{tcolorbox}[colback=purple!5!white,colframe=purple!75!black]
\begin{definition}
Let $H$ be a subgroup of $G$. Define $\pi_n^H(X):=\pi_n(X^H)$. A map $f: X\to Y$ of $G$-spaces is a \underline{\textbf{weak equivalence}} if for all subgroups $H\subset G$,
\[f_*:\pi_n^H(X)\to \pi_n^H(Y)\]
is an isomorphism. 
\end{definition}
\end{tcolorbox}

Let $\textbf{GTop}$ be the category of $G$-spaces and $G$-maps. There is a cofibrantly-generated model structure that we can put on $\textbf{GTop}$:

\begin{tcolorbox}[colback=red!5!white,colframe=red!30!white]
\begin{theorem}
    There is a cofibrantly-generated model structure on $\textbf{GTop}$, given by 
    \begin{enumerate}
        \item A $G$-map $f:X\to Y$ is a fibration iff for all $H\subset G$, $f^H: X^H\to Y^H$ is a fibration.
        \item A $G$-map $f:X\to Y$ is a weak equivalence iff for all $H\subset G$, $f^H: X^H\to Y^H$ is a weak equivalence. 
    \end{enumerate}
\end{theorem}
\end{tcolorbox}
An immediate consequence of the model category structure is the equivariant Whitehead's Theorem

\begin{tcolorbox}[colback=green!5!white,colframe=green!30!white]
\begin{corollary}
Let $f: X\to Y$ be a weak equivalence of cofibrant-fibrant objects in a model category. Then, $f$ is a homotopy equivalence. In particular, every object in $\textbf{GTop}$ is fibrant, and $G$-CW complexes are  cofibrant. 
\end{corollary}
\end{tcolorbox}

\subsection{Elmendorf's Theorem}
From the model structure given in Theorem $1.1$, we have a vague sense of the following "equivalence":
\[\textrm{G-Homotopy Type of } X\Leftrightarrow \{ \textrm{ordinary homotopy type of } X^H:H\subset G \}\]
And Elmendorf's Theorem will make the equivalence precise. We start by introducing the orbit category:

\begin{tcolorbox}[colback=purple!5!white,colframe=purple!75!black]
\begin{definition}
The \underline{\textbf{orbit category}} $\mathcal{O}_G$ is the full subcategory of $\textrm{GTop}$ on the objects $\{G/H: H\subset G\}$.
\end{definition}
\end{tcolorbox}
The following lemma will make the structure of $\mathcal{O}_G$ clearer.

\begin{tcolorbox}
\begin{lemma}
$\textrm{Map}^G(G/H,G/K)\cong (G/K)^H$
\end{lemma}
\end{tcolorbox}
\begin{proof}
    Note that there exists a $G$-equivariant maps $\varphi: G/H\to G/K$, determined by $\varphi(H)=gK$ iff $gHg^{-1}\subseteq K$ iff $h(gK)=gK$ for all $h\in H$. 
\end{proof}

Let $\textrm{Fun}(\mathcal{O}_G^op,\textrm{Top})$ be the functor category. We have the following fact on the model structure on functor categories:

\begin{tcolorbox}[colback=red!5!white,colframe=red!30!white]
\begin{theorem}
Let $\mathcal{D}$ be a model category and $\mathcal{C}$ be a cofibrantly generated model category. Then, $\textrm{Fun}(\mathcal{C},\mathcal{D})$ admits a model structure. 
\end{theorem}
\end{tcolorbox}

It is useful to know that the weak equivalences in $\textrm{Fun}(\mathcal{O}^{op}_G,\textrm{Top})$ is given pointwise: a natural transformation $\eta: \mathcal{F}\to \mathcal{G}$ is a weak equivalence iff $\eta_{G/H}:\mathcal{F}(G/H)\to \mathcal{G}(G/H) $ is a weak equivalence. 



\begin{tcolorbox}[colback=purple!5!white,colframe=purple!75!black]
\begin{definition}
There is a functor $\psi: \textrm{GTop}\to \textrm{Fun}(\mathcal{O}^{op}_G,\textrm{Top})$ given by 
\[X\to (G/H\mapsto X^H)\]
\end{definition}
\end{tcolorbox}
It is easy to check the functoriality. Note that if we restrict $\psi$ to $\mathcal{O}_G$, the functor is just the Yoneda embedding: $\textrm{Map}^G(G/H,G/K)\cong (G/K)^H$.

\begin{tcolorbox}[colback=blue!5!white,colframe=blue!30!white]
    \begin{proposition}
    There is a funcor $\theta: \textrm{Fun}(\mathcal{O}^{op}_G,\textrm{Top})\to \textrm{GTop}$ given by $X\mapsto X(G/e)$, where $X(G/e)$ is equipped with the following $G$-action: note that every $g\in G$ defines an $G$-map $G/e\to G/e$, which we denote by $R_g$.
    \[g\cdot x=X(R_g)(x)\] 
    \end{proposition}
    \end{tcolorbox}
    
    It is easy to check that $(\theta,\psi)$ is an adjoint pair. In fact, more can be said:
    
    \begin{tcolorbox}[colback=red!5!white,colframe=red!30!white]
        \begin{theorem}
        (Elmendorf's Theorem) $\textrm{Fun}(\mathcal{O}^{op}_G,\textrm{Top})$ and $\textrm{GTop}$ have the same homotopy category. 
        \end{theorem}
        \end{tcolorbox}
    The original proof due to Elmendorf constructs the equivalence explicitly using the Bar construction to obtain a homotopy inverse to the embedding $\psi$. The theorem can now be put into a more modern framework:

    \begin{tcolorbox}[colback=red!5!white,colframe=red!30!white]
    \begin{theorem}
    $(\theta,\psi)$ is an Quillen equivalence. $\psi$ is an equivalence of $(\infty,1)$ categories. 
    \end{theorem}
    \end{tcolorbox}

\subsection{Bredon Cohomology}  
The goal is to construct a cohomology theory satisfying the Eilenberg-Steenrod axioms under the equivariant setting. 
 \begin{tcolorbox}[colback=purple!5!white,colframe=purple!75!black]
 \begin{definition}
 (Equivariant reduced generalized cohomology) Let $G\textrm{CW}_*$ be the category of pointed $G$-CW complexes with equivariant maps. Then, a generalized cohomology theory on $G\textrm{CW}_*$ is a sequence of contravariant functors 
 \[\tilde{H}^n:=G\textrm{CW}_*\to \textbf{Ab} \]
 satisfying the following:
 \begin{enumerate}
    \item if $f,g$ are equivariantly homotopic, then $\tilde{H}^n(f)=\tilde{H}^n(g)$.
    \item There exists a sequence of natural isomorphisms
   \[\tilde{H}^n(X)\to \tilde{H}^{n+1}(S^1\wedge X)\]
   where $G$ acts trivially on $S^1$ in the smash. 
    \item The sequence 
    \[\tilde{H}^n(X/A)\to \tilde{H}^n(X)\to \tilde{H}^n(A)\]
    is exact.
 \end{enumerate}
 \end{definition}
 \end{tcolorbox}

\begin{tcolorbox}[colback=green!5!white,colframe=green!30!white]
\begin{remark}
The above axioms is built upon pointed "single" spaces. It is in fact equivalent to the usual theory built upon pairs, and is justified in 
\end{remark}
\end{tcolorbox}
For a non-equivariant reduced generalized cohomology theory $\tilde{h}^*$, the Atiyah-Hirzebruch spectral sequence tells us that knowing $\tilde{h}^*(pt)$ basically determines the cohomology theory on CW complexes. Heuristically, the cohomology is determined by the building blocks, which are contractible open cells. However, in the equivariant setting, the building blocks are more complicated: the building blocks have become orbits of the form $G/H$. We are lead to the following definitions:

\begin{tcolorbox}[colback=purple!5!white,colframe=purple!75!black]
\begin{definition}
A \underline{\textbf{coefficient system}} is a contravariant functor $\mathcal{F}: \mathcal{O}_G\to \textrm{Ab}$.
\end{definition}
\end{tcolorbox}
Recall that if a reduced cohomology theory is called \underline{\textbf{ordinary}} if it satisfies the dimension axiom, i.e the zeroth reduced cohomology group (a.k.a the coeffcient system) is trivial on a point. Our goal now is to construct such a theory in the equivariant setting, which is called Bredon cohomology. 

Note that by the general theory of abelian categories, the functor category of coefficient systems $\mathcal{CS}:=\textrm{Fun}(\mathcal{O}_G, \textrm{Ab})$ is abelian. It is now possible to define Bredon cohomology on a $G$-CW complex by explicitly defining the cochain complexes on cells, for example see . However, we may package the cochains into the following form 

\begin{tcolorbox}[colback=purple!5!white,colframe=purple!75!black]
\begin{definition}
For each $n$, we may define a coefficient system $C_n(-)$, given by 
\[G/H\mapsto H_n((X^H)_{n},(X^H)_{n-1}; \mathbb{Z})=C_n^{CW}(X^H)\]
The differential of the CW chain complex induces a chain complex of coefficient systems $C_{\cdot}(-)$. 
\end{definition}
\end{tcolorbox}
It is easy to check that $Hom_{\mathcal{CS}}(C_{\cdot}(-),M)$ is a cochain complex whose differentials is induced by those of $C_{\cdot}()$. 
\begin{tcolorbox}[colback=purple!5!white,colframe=purple!75!black]
\begin{definition}
The \underline{\textbf{Bredon cohomology}} of $X$ with coefficients in a system $M$ is defined by 
\[H_G^n(X;M):=H^n(Hom_{\mathcal{CS}}(C_{\cdot}(X),M))\]
\end{definition}
\end{tcolorbox}

proof of the axioms and computation of examples. 

\subsection{Classical Application}
homotopy fixed points and actual fixed points: coarse vs fine
Here are two questions we may ask: given a finite $p$-group $G$ acting on $X$, can we recover the cohomology of $X^G$ based on the cohomology of $X$, while somehow incoporating the $G$-action? The second point of interest is the Sullivan conjecture: in  The study of \'etale homotopy theory allows one to determines algebraically the profinite completion of a variety given only its \'etale homotopy type. The same situation occurs as before: given a variety $V$ over the complex numbers, there is a natural $\mathbb{Z}/2$-action on the variety, with $V^G=V(\mathbb{R})$. How much can we say about $V(\mathbb{R})$ if we are given the \'etale homotopy type of $V$ and the $G$-action? IN this case, the fixed points is not a homotopy invariant, so we should consider the homotopy fixed points instead, and the sullivan conjecture states that 


\begin{tcolorbox}[colback=red!5!white,colframe=red!30!white]
\begin{theorem}
(Sullivan Conjecture) $V(\mathbb{C})^{\mathbb{Z}/2}\to V(\mathbb{C})^{h \mathbb{Z}/2}$ becomes an isomorphism after $2$-adic completion. More generally, let $G$ be a finite $p$-group, then $X^G\to X^{hG}$ becomes an equivalence after $p$-adic completion. 
\end{theorem}
\end{tcolorbox}




One possible application is a quick proof of a theorem by Smith: 


\begin{tcolorbox}[colback=red!5!white,colframe=red!30!white]
\begin{theorem}
    Let $G$ be a finite $p$-group and $X$ be a finite $G$-CW complex such that (the underlying topological space of) $X$ is an $p$-cohomology sphere.16 Then, $X^G$ is either empty or an $p$-cohomology sphere of smaller dimension.
\end{theorem}
\end{tcolorbox}











\section{Spectra and the Stable Category}
\subsection{Motivations}
The first motivation is Brown Reresentability Theorem, which is about represeting reduced cohomology theories 

\begin{tcolorbox}[colback=red!5!white,colframe=red!30!white]
\begin{theorem}[Brown Representability Theorem]
Let $\tilde{h}^*$ be a reduced cohomology theory on pointed CW-complexes. Then, for each $n\in \mathbb{Z}$, there exists a connected pointed CW complex $K_n$ such that 
\[\tilde{h}^n(X)\cong [X,K_n]\]
for all $n$. Moreover, the $K_n$ are determined up to homotopy equivalence. 
\end{theorem}
\end{tcolorbox}
In fact, there are more structure to the set $\{K_n\}$: using the suspension axiom and loop-suspension adjunction, we see that there must be an isomorphism 
\[[X,K_n]\cong \tilde{h}^n(X)\cong \tilde{h}^{n+1}(\Sigma X)\cong [\Sigma X,K_n]\cong [X,\Omega K_n]\]
Taking $X$ to be $K_n$, we note that the indetitiy map from the LHS corresponds uniquely to a map $\alpha_n: K_n\to \Omega K_n$, which we will call the \underline{\textbf{structure map.}}. By naturality and taking $A=S^k$, the structure map induces a weak equivalence 
\[K_n\cong  \Omega K_n\]
. This motivates the definintion for $\Omega$-Spectra

\begin{tcolorbox}[colback=purple!5!white,colframe=purple!75!black]
\begin{definition}
A $\underline{\textbf{spectrum}}$ is a sequence of pointed topological spaces $\{X_n\}$ with $\underline{\textbf{structure maps}}$.
\[\Sigma X_n\to X_{n+1}\]
. An \underline{\textbf{$\Omega$ spectrum}} is a spectrum whose adjoint structure maps
\[X_n\to \Omega X_{n+1}\]
are weak equivalences. 
\end{definition}
\end{tcolorbox}
So we see that a reduced cohomology theory corresponds to an $\Omega$-spectrum. In fact, it is not hard to show that an $\Omega$-spectrum defines a reduced cohomology theory as well. We can then present Brown Representability Theorem in the following way:
\begin{tcolorbox}[colback=red!5!white,colframe=red!30!white]
\begin{theorem}[Brown Representability Theorem]
    Every reduced cohomology theory on the category of basepointed CW complexes  has the form $\tilde{h}^n(X) = [X,K_n]$ for some $\Omega$-spectrum $\{K_n\}$
\end{theorem}
\end{tcolorbox}

\begin{tcolorbox}[colback=yellow!5!white,colframe=yellow!30!white]
\begin{example}[Eilenberg-Maclane Spectrum]
The $\Omega$-spectrum that represents reduced ordinary cohomology with coefficients in $\mathbb{Z}$ is given by the $\Omega$-spectra that is the Eilenberg-Maclane spaces $HG:=\{K(G,n)\}$.
\end{example}
\end{tcolorbox}

\begin{tcolorbox}[colback=yellow!5!white,colframe=yellow!30!white]
\begin{example}
    The $\Omega$-spectrum that represents reduced complex topological $K$-theory is given by $\{KU_n\}$, where 
    \[  KU_n=
    \begin{cases}
        BU\times \mathbb{Z}& $n$-\textrm{even}\\
        \Omega BU & $n$-\textrm{odd}
    \end{cases}\]
  In particular, Bott-periodicity shows that the structure maps are weak-equivalences.
\end{example}
\end{tcolorbox}

Here is a natural question: how do we design a category of spectra, such that an "equivalence" if spectra will give equivalent reduced cohomology theories?

The second motivation is stable homotopy groups and stable maps: 


\begin{tcolorbox}[colback=purple!5!white,colframe=purple!75!black]
\begin{definition}
Let $X$ and $Y$ be pointed CW-complexes. The set of $\underline{\textbf{stable homotopy classes of maps}}$ from $X$ to $Y$ is defined to be 
\[[X,Y]^s:= \varinjlim_{k}[\Sigma^k X, \Sigma^k Y] \]
\end{definition}
\end{tcolorbox}
One form of the Freudenthal suspension theorem says that if $Y$ is $n$-connected and $X$ has dimension less than $2n+1$, then the suspension map $[X,Y]\to [\Sigma X,\Sigma Y]$ is bijective. In this case, we see that the colimit actual stablizes after at a finite stage. 

\begin{tcolorbox}[colback=purple!5!white,colframe=purple!75!black]
\begin{definition}
For a pointed CW-complex, the $n$-th \underline{\textbf{stable homotopy group}} is defined to be 
\[\pi_n^{st}(X):=\varinjlim_{k}\pi_{n+k}(\Sigma^k X)=\varinjlim_{k}[\Sigma^k S^n, \Sigma^k X]=[S^n,X]^s\]
\end{definition}
\end{tcolorbox}
By definition, we see that the stable homotopy group should be the homotopy group in some "stable category". Moreover, stable homotopy groups actually defines a reduced homology theory: the two axioms that are not trivial are the LES and the wedge axiom. The key point is that Blaker's Massey Theorem plus the LES of homotopy groups of a pair will give us the LES of stable homotopy groups of a pair; the wedge axiom follows from that $\Sigma^i X\wedge \Sigma^i Y$ is the $2i-1$ skeleton of  $\Sigma^i X\times \Sigma^i Y$. Generalizing this, we have 

\begin{tcolorbox}[colback=red!5!white,colframe=red!30!white]
\begin{theorem}
    Let $K$ be a CW-complex. The sequence $h_i(X) = \pi^s_i(X \wedge K)$ forms a reduced homology theory on the category of basepointed CW-complexes and basepoint-preserving maps.
\end{theorem}
\end{tcolorbox}

We can also define the homotopy groups of a spectrum as a generalization:

\begin{tcolorbox}[colback=purple!5!white,colframe=purple!75!black]
\begin{definition}[Homotopy groups of a spectrum]
Suppose $K=\{K_i\}$ is a spectrum. Then, 
\[\pi_n(K):=\varinjlim \pi_{n+i}(K_i)\]
where the inductive limit is induced by the suspension structure maps. 
\end{definition}
\end{tcolorbox}

\begin{tcolorbox}[colback=yellow!5!white,colframe=yellow!30!white]
\begin{example}
Given a topological space $X$, we have the associated \underline{\textbf{suspension spectrum}} $\Sigma^{\infty}X=\{\Sigma^kX\}$. Then, the stable homotopy groups of $X$ is given by the homotopy groups of its associated suspension spectrum. 
\end{example}
\end{tcolorbox}

\begin{tcolorbox}[colback=purple!5!white,colframe=purple!75!black]
\begin{definition}
Given a spectrum $K=\{K_n\}$ and a CW complex $X$, there is a associated smash spectrum $K\wedge X$, where $(K\wedge X)_n=K_n\wedge X$. The structure maps is given by 
\[\Sigma (K\wedge X)_n =\Sigma K_n\wedge X\to K_{n+1}\wedge X=\Sigma (K\wedge X)_{n+1}\]
\end{definition}
\end{tcolorbox}

\begin{tcolorbox}[colback=blue!5!white,colframe=blue!30!white]
\begin{proposition}
Let $K$ be a spectrum. Given a CW complex $X$, the sequence
\[h_n:=\pi_n(X\wedge K)\]
is a reduced homology theory.
\end{proposition}
\end{tcolorbox}
The proof mostly follows from the same tactics in proving the stable homotopy groups being a reduced homology theory.
\begin{tcolorbox}[colback=yellow!5!white,colframe=yellow!30!white]
\begin{example}[Singular homology]
If $K=HG$ is the Eilenberg-Maclane spectrum, then $h_i(X):=\pi_i(X\wedge K)$ is isomorphic to singular homology with coefficients in $G$. We only have to check the dimension axiom 
\[h_i(S^0):=\varinjlim_n \pi_{n+i}(K(G,n))\]
which is trivial except $i=0$. 
\end{example}
\end{tcolorbox}


Thus, we have seen that the singular homology is recovered from a spectrum; moreover, a spectrum gives a reduced homology theory by proposition $2.3.1$. 

\subsection{The Desired Stable Homotopy Category}
Building upon the motivations in the previous section, we want to build a category $\mathcal{S}$ that captures the stable phenomena and cohomology theories/homology theories. In particular, we want it to satisfy the following properties (not necessarily axioms):
\begin{enumerate}
    \item There is a functor $\Sigma^{\infty}: \textrm{HoTop}_*\to \textrm{Sp}$, together with an adjoint $\Omega^{\infty}: \mathcal{S}\to \textrm{HoTop}_*$. \\
    \textcolor{blue}{The motivation for this is the suspension spectrum}.
    \item Let $A,B$ be CW complexes, where $A$ has finite dimension. Then, there is a natural isomorphism
    \[[\Sigma^{\infty}A, \Sigma^{\infty}B]\cong [A,B]^s\]
    \textcolor{blue}{The motivation for this is to capture stable morphisms}.
    \item $\mathcal{S}$ has a closed symmetric monoidal structure, with $\wedge$ as the tensor product, and $F(-,-)$ as internal hom. $\Sigma^{\infty}S^0$ is the identity object.
    \item The morphisms of $\mathcal{S}$ has the structure of graded abelian groups, with bilinear composition. We denote that graded components as $[-,-]_*$. Moreover, given a reduced cohomology theory $E^*$ there exists an object $K$ in $\textrm{Sp}$ such that 
    \[E^*(A)\cong [\Sigma^{\infty}A, K]_{-*}\]
    \textcolor{blue}{The motivation for this is to represent cohomology theories}.\\
    \item For every object $K$ in $\mathcal{S}$, one defines a reduced homology theory via 
    \[E_n(A):=\pi(K\wedge A)\]
    \textcolor{blue}{The motivation for this is to deterime homology theories}.
    \\
    \item $\mathcal{S}$ has a closed symmetric monoidal structure, with $\wedge$ as the tensor product and $F(-,-)$ as internal hom. The object $\Sigma^{\infty}S^0$ is the identity object.\\
    \textcolor{blue}{The motivation for this is that smashing with the suspension funcor induces isomorphism of stable homotopy groups, so it should induce an equivalence of spectra. The tensor-hom adjunction is going to help set up duality statements similar to Spanier-Whitehead duality and Poincar\'e duality.}.
\end{enumerate}
In particular, property $6$ was the central one in the sense that the entire theory began with the work of Spanier and Whitehead with their search for a category where one has the desired dualities in homotopy theory. It is also the structure that seems to be the hardest to obtain.  



\subsection{Attempts}

\subsubsection{The Spanier-Whitehead Category}
The first attempt to construct such a category is the Spanier-Whitehead category, where the objects are finite CW complexes, and the morphisms are stable maps. The catgeory was originally set up to perform Spanier-Whitehead duality (chronologically it was ), but the objects "almost" represents cohomology theories. We use the word almost because since the objects are only finite complexes, we do not have arbitrary products and coproduts, and the wedge axiom do not hold. 
\subsubsection{An Attempt for \textrm{Sp}}
By Brown representability, it should be natural to consider a category with objects spectra, and the natural choice for morphisms between spectra is a collection of level-wise maps that is compatible with the structure maps. A natural candidate for homotopies in this category is a map $H: X\wedge I_{+}\to Y$ However, it turns out there are not enough maps in this category, such that non-homotopy equivalent spectra might represent isomorphic cohomology theories. Adams then in his 74 paper resolves the issue by consider equivalences classes of "cofinal spectra," which turns out to be quite a complicated a construction. 

\subsubsection{\textrm{Ho(Sp)} and Structured Spectra}
The first construction in line with modern treatment of the subject is Bousfield and Friedlander's construction, which puts a model structure on the naive category of spectra, and considers its homotopy category. This category turns out satisfies almost all the properties that people wanted. (We say "almost" since Lewis in 91 showed that it is impossible to put a symmetric monoidal structure on the category of spectra that also satifies a list of desirable axioms). However, people still searched for a point-set model that does so without going through the homotopy category.  People later constructed more structured spectra, such as orthogonal spectra and symmetric spectra, whose additional structure then provided a fix to such a problem. 

\subsection{Orthogonal Spectra}

\begin{tcolorbox}[colback=purple!5!white,colframe=purple!75!black]
\begin{definition}
An \underline{\textbf{orthogonal spectrum}} consists of the following data:
\begin{enumerate}
    \item A sequence of pointed topological spaces $X_n$
    \item A basepoint-preserving left $O(n)$-action on $X_n$.
    \item Structure maps $\sigma_n:X_n\wedge S^1\to X_{n+1}$
    \item The interated structure maps
    \[X_n\wedge S^m\to X_{n+m}\]
    define by the composition 
    \[X_{n}\wedge S^m=(X_n\wedge S^1)\wedge S^{m-1}\xrightarrow{\sigma_n\wedge Id_{S^{m-1}}}X_{n+1}\wedge S^{m-1}\to...\to X_{n+m}\]
    is $O(n)\times O(m)$-equivariant. Specifically, $O(m)$-acts on $S^m$ canonically since $S^m$ is the one-point compactification of $\mathbb{R}^m$; $O(n)\times O(m)$, through orthogonal sum, is identified as a subgroup of $O(n+m)$ and acts by restriction on $X_{n+m}$.
\end{enumerate}
\end{definition}
\end{tcolorbox}


\begin{tcolorbox}[colback=purple!5!white,colframe=purple!75!black]
\begin{definition}
A morphism of orthogonal spectra $f: X\to Y$ is a sequences of maps $f_n: X_n\to Y_n$ that are $O(n)$-equivariant, and compatible with the structure maps such that 
\[\begin{tikzcd}
X_n \wedge S^1 \arrow[r,"f_n\wedge S^1"]\arrow[d,"\sigma_n"]& Y_n\wedge S^1\arrow[d,"\sigma_n"]\\
X_{n+1}\arrow[r,"f_{n+1}"]& Y_{n+1}
\end{tikzcd}\]
\end{definition}
\end{tcolorbox}
Our goal now is to show that there exists a tensor product structure. We now recall a few constructions: 

\begin{tcolorbox}[colback=purple!5!white,colframe=purple!75!black]
\begin{definition}
Given subgroup $H\subset G$ and a pointed $H$-space $X$, the \underline{\textbf{balanced smash product}} $G_+\wedge_{H} X$ is the quotient space of $G_+\wedge H$ by the relation $gh\wedge x\sim g\wedge hx$
\end{definition}
\end{tcolorbox}
Note that $G_+\wedge_{H} X$ is canonically a $G$-space by the action on the $G_+$-coordinate. One can think of this as the base-change from $H$ to $G$. 

\begin{tcolorbox}[colback=blue!5!white,colframe=blue!30!white]
\begin{proposition}
The funcor $\textrm{Htop}_*\to \textrm{GTop}_*$ given by the balanced smash product is left adjoint to the forgetful functor. Specifically, we have the natural isomorphism
\[Hom_H(X,Y)\cong Hom_G(G\wedge_H X, Y)\]
\end{proposition}
\end{tcolorbox}
Recall the universal property of the tensor product in $\textbf{RMod}$ is given by the universal object from which a $R$-balanced map from the product factors through. In our situation, the ring $R$ is more or less the sphere spectrum, and in order to define the tensor product, we should first give an appropriate definition of a balanced map.


\begin{tcolorbox}[colback=purple!5!white,colframe=purple!75!black]
\begin{definition}
    Given orthogonal spectra $X,Y, Z$, a \underline{\textbf{bimorphism}} from the pair $(X,Y)$ to $Z$ is given by a collection of $O(p)\times O(q)$-equivariant maps 
    \[b_{p,q}: X_p\wedge Y_q\to Z_{p+q}\] 
    subjected to the following commutative diagram
    \[\begin{tikzcd}
        {X_p\wedge Y_q\wedge S^1} && {X_{p}\wedge Y_{q+1}} \\
        {Z_{p+q}\wedge S^1} & {} & {Z_{p+q+1}}
        \arrow["{X_p\wedge \sigma_{Y}}", from=1-1, to=1-3]
        \arrow["{b_{p,q}\wedge S^1}"', from=1-1, to=2-1]
        \arrow["{b_{p,q+1}}", from=1-3, to=2-3]
        \arrow["{\sigma_{Z}}"', from=2-1, to=2-3]
    \end{tikzcd}\]
    and 
    \[\begin{tikzcd}
        {X_p\wedge Y_q\wedge S^1} && {X_{p}\wedge Y_{q+1}} && {Z_{p+q+1}} \\
        {X_p\wedge S^1\wedge Y_q} & {} & {X_{p+1}\wedge Y_{q}} && {Z_{p+1+q}}
        \arrow["{X_p\wedge \sigma_{Y}}", from=1-1, to=1-3]
        \arrow["{X_p\wedge twist}"', from=1-1, to=2-1]
        \arrow["{b_{p,q+1}}", from=1-3, to=1-5]
        \arrow["{\sigma_X\wedge Y_q}"', from=2-1, to=2-3]
        \arrow["{b_{p+1,q}}"', from=2-3, to=2-5]
        \arrow["{Id\times \chi_{1,q}}"', from=2-5, to=1-5]
    \end{tikzcd}\]
\end{definition}
\end{tcolorbox}
The map $\chi_{p,q}$ correspond to the element in $O(p+q)$ that swaps the first $q$ coordinates with the remaining $q$-coordinates. In analogy to \textrm{RMod}, the first diagram says that $(x\otimes y)\cdot r=x\otimes (y\cdot r)$, and the second diagram says that $x\otimes (y\cdot r)=(x\cdot r)\otimes y$. 

Recall that the tensor product of two $R$-modules $A,B$ is the coequalizer of the diagram 
\[\alpha_1,\alpha_2: A\otimes R\otimes B\to A\otimes B\]
where the tensor is in $\textbf{Ab}$ and the two maps are given by the action on $A$ and $B$, respectively. We shall recreate this in our category $\textbf{Sp}$:


\begin{tcolorbox}[colback=purple!5!white,colframe=purple!75!black]
\begin{definition}
Given orthogonal spectra $X,Y$, define $(X\wedge Y)_n$ to be the coequalizer of the maps
\[\alpha_X,\alpha_Y: \bigvee_{p+1+q=n}O(n)_+\wedge_{O(p)\times Id\times O(q)}X_p\wedge S^1\wedge Y_q\to  \bigvee_{p+q=n}O(n)_+\wedge_{O(p)\times O(q)}X_p\wedge Y_q\]
in the category of $O(n)$-spaces, where $\alpha_X$ is induced by the maps
\[X_p\wedge S^1\wedge Y_q\xrightarrow{\sigma_X\wedge Y_q} X_{p+1}\wedge Y_q\] on each summand, and $\alpha_Y$ is induced by the maps 
\[X_p\wedge S^1\wedge Y_q\xrightarrow{X_p\wedge twist}X_p\wedge Y_p\wedge S^1\to X_{p}\wedge Y_{q+1}\xrightarrow{X_p\wedge \chi_{q,1}} X_{p}\wedge Y_{1+q}\]
\end{definition}
\end{tcolorbox}
Note that the coequalizer in $GTop$ is created by the coequalizer in the underlying $Top$, so in our case, $(X\wedge Y)_n$ is a quotient space of 
$\bigvee_{p+q=n}O(n)_+\wedge_{O(p)\times O(n)}X_p\wedge Y_q$. Then, the maps 
\[O(n)_+\wedge_{O(p)\times O(q)}X_p\wedge Y_q\wedge S^1\to O(n+1)_+\wedge_{O(p)\times O(q)}X_p\wedge Y_q\]
given by $Id\wedge \sigma_Y$ induces a morphism $\varphi_n:(X\wedge Y)_n\wedge S^1\to (X\wedge Y)_{n+1}$. It follows that 

\begin{tcolorbox}[colback=blue!5!white,colframe=blue!30!white]
\begin{proposition}
The collection $(X\wedge Y)_n$, with structure maps $\varphi_n$ forms an orthogonal spectra. 
\end{proposition}
\end{tcolorbox}

As remarked before, the smash product is equipped with a universal bimorphism $i: (X,Y)\to X\wedge Y$ given by 
\[X_p\wedge Y_q\xrightarrow{x\wedge y\mapsto e\wedge x\wedge y} \bigvee_{p+q=n}O(n)_+\wedge_{O(p)\times O(q)}X_p\wedge Y_q\xrightarrow{projection} (X\wedge Y)_{p+q}\]

\subsection{The Equivariant Case}
One way to motivate the correct definition of equivariant stable homotopy category is to represent the equivariant cohomology theories. In particular, we want a form of the suspension isomorphism
\[H^*(X)\cong H^{*+V}(\Sigma^VX)\]
In the non-equivariant case, we only take $V$ to be trivial representations, which means we are smashing with a sphere with trivial $G$-action. However, it turns out that the cohomology would carry more useful data if we want to keep track all possible representations $V$. 


\begin{tcolorbox}[colback=purple!5!white,colframe=purple!75!black]
\begin{definition}
A \underline{\textbf{complete $G$-universe}} $U$ is an infinite dimensional real inner product space with a $G$-action, such that it is a direct sum of countably many copies of each irreducible representations of $G$. 
\end{definition}
\end{tcolorbox}


\begin{tcolorbox}[colback=green!5!white,colframe=green!30!white]
\begin{remark}
If we drop the adjective "complete" from the definition, we would only require that the universe containing some subset of $G$-representations and the trivial representation instead of all possible representations. However, if we want dualizable orbit, a convenient assumption is requiring $U$ to be complete.  
\end{remark}
\end{tcolorbox}


\begin{tcolorbox}[colback=purple!5!white,colframe=purple!75!black]
\begin{definition}
A $G$-\underline{\textbf{prespectrum}} $X$ is a collection of pointed $G$-spaces $X(V)$, one for each finite dimensional subspace $V\subset U$ of a given $G$ universe. The collection is equipped with structure maps 
\[\Sigma^WX(V)\cong X(V\oplus W)\]
The structure maps must be associative. A $G$-prespectrum is a \underline{\textbf{G-Spectrum}}(or $\Omega-G$-spectrum) if the adjoint structure maps 
\[X(V)\to \Omega^WX(V\oplus W)\] 
are homeomorphisms.
\end{definition}
\end{tcolorbox}
This definition enlarges the usual $\mathbb{Z}$-graded notion of spectra. Sometimes in literature this representation graded spectrum is referred to as the genuine $G$-spectra. 


\begin{tcolorbox}[colback=purple!5!white,colframe=purple!75!black]
\begin{definition}
The \underline{\textbf{homotopy groups}} of $G$-prespectrum are 
\[\pi_q^H(X):=\textrm{colim}_{V\subset U}\pi_q^H\Omega^VX(V)\]
\[\pi_{-q}^H(X):=\textrm{colim}_{\mathbb{R}^m\subset V}\pi_q^H\Omega^{V-\mathbb{R}^m}X(V)\]
where $V-\mathbb{R}^m$ denotes the orthogonal complement. 

A \underline{\textbf{weak-equivalence}} between $G$-spectra is a $\pi_*$ equivanlence. 
\end{definition}
\end{tcolorbox}


\begin{tcolorbox}[colback=red!5!white,colframe=red!30!white]
\begin{theorem}
There is a model structure on $G$-prespectra where the weak equivalences are the stable equivalences. 
\end{theorem}
\end{tcolorbox}














\section{$RO(G)$-graded cohomology Theories}
\subsection{Mackey Functors}
Mackey Functors will serve the role of coefficient system in $RO(G)$-graded coefficient theories. In particular, the take into account of the transfer maps, which is not present in the orbit category $\mathcal{O}_G$. We first give a functorial definition for Mackey functors in terms of the Burnside category, and then present an alternative definition with axioms instead, which hopefully further elucidates the structure of a Macket functor. For the following discussions, $G$ will be a finite discrete group.


\begin{tcolorbox}[colback=purple!5!white,colframe=purple!75!black]
\begin{definition}
Let $C$ be a category with finite limits and colimits. Then, $\underline{\textbf{Span}(C)}$ is the category with the same objects as $C$, and morphisms equivalence classes of spans. Composition of morphisms is given by pullback of Spans.\\

Given two morphisms in $\textbf{Span}(C)(X,Y)$ $X\leftarrow Z\rightarrow Y$ and $X\leftarrow Z'\rightarrow Y$, the disjoint union operation $X\leftarrow Z\coprod Z'\rightarrow Y$ endows the homsets a monoid structure. Let $\underline{\textbf{Span}^+(C)}$ denote the preadditive completion of $\textbf{Span}(C)$. 
\end{definition}
\end{tcolorbox}



\begin{tcolorbox}[colback=purple!5!white,colframe=purple!75!black]
\begin{definition}
Let $G$ be a finite group. Then, the \underline{\textbf{Burnside Category}} is defined to be $\mathcal{A}(G):=\textbf{Span}^+(\textrm{GSet}^{\textrm{Fin}})$.
\end{definition}
\end{tcolorbox}
Recall in the orbit category, we have a morphism $G/H\to G/K$ iff $H$ is subconjugate to $K$. The Burnside category can be viewed as formally adding the "wrong way maps" to each morphisms in the orbit category. 

\begin{tcolorbox}[colback=purple!5!white,colframe=purple!75!black]
\begin{definition}
A $\underline{\textbf{Mackey Functor}}$ is an additive functor $F: \mathcal{A}(G)^{op}\to \textbf{Ab}$. 
\end{definition}
\end{tcolorbox}
And Mackey functors are coefficients systems with wrong way maps (transfers) added. 



we can then define a functor $i_*:\mathcal{O}_G\to \mathcal{A}(G)$ by sending a morphism $f:G/H\to G/K$ to the span $G/H\xleftarrow{Id}G/H\xrightarrow{f}G/K$; similarly, we have a functor $i^*:\mathcal{O}^{op}_G\to \mathcal{A}(G)$ by sending a morphism $f:G/H\to G/K$ to the span $G/K\xleftarrow{f}G/H\xrightarrow{Id}G/H$. Given a Mackey functor $\underline{M}: \mathcal{A}(G)\to \textrm{Ab}$, precomposition with $i_*$ gives a coefficients system, which we will denote $M^*$, and precomposition with $i^*$ is a covariant funcor $\mathcal{O}_G\to \textrm{Ab}$, which we will denote as $M_*$. By construction, $M_*$ and $M_*$ agree on objects, and they determine the value of the Mackey functor on objects: a Mackey functor is additive, its data is determined by the disjoint orbits. We would also like to know how the morphisms commute: if we have the pullback square of orbits
\[\begin{tikzcd}
	W & Y \\
	X & Z
	\arrow["\beta", from=1-1, to=1-2]
	\arrow["\alpha"', from=1-1, to=2-1]
	\arrow["\delta", from=1-2, to=2-2]
	\arrow["\gamma"', from=2-1, to=2-2]
\end{tikzcd}\]
then $M_*(\beta)M^*(\alpha)=M^*(\delta)M_*(\gamma)$, as illustrated by the following composition diagram. 

\[\begin{tikzcd}
	&& W \\
	& X && Y \\
	X && Z & {} & Y
	\arrow["\alpha", from=1-3, to=2-2]
	\arrow["\beta"', from=1-3, to=2-4]
	\arrow["\lrcorner"{anchor=center, pos=0.125, rotate=-45}, draw=none, from=1-3, to=3-3]
	\arrow["Id"', from=2-2, to=3-1]
	\arrow["\gamma", from=2-2, to=3-3]
	\arrow["\delta"', from=2-4, to=3-3]
	\arrow["Id", from=2-4, to=3-5]
\end{tikzcd}\]

From the above discussion, we have another equivalent definition for Mackey functors, which is sometimes more useful 


\begin{tcolorbox}[colback=purple!5!white,colframe=purple!75!black]
    \begin{definition}
    A \underline{\textbf{Mackey Functor}} is a pair of functors $M^*: \textrm{GSet}^{\textrm{Fin}}\to \textrm{Ab}$ and $M_*: \textrm{GSet}^{\textrm{Fin}}\to \textrm{Ab}$ such that the following axioms hold:
    \begin{enumerate}
        \item $M^*$ and $M_*$ agree on objects;
        \item For every pullback square of orbits
        \[\begin{tikzcd}
            W & Y \\
            X & Z
            \arrow["\beta", from=1-1, to=1-2]
            \arrow["\alpha"', from=1-1, to=2-1]
            \arrow["\delta", from=1-2, to=2-2]
            \arrow["\gamma"', from=2-1, to=2-2]
        \end{tikzcd}\]
        then $M_*(\beta)M^*(\alpha)=M^*(\delta)M_*(\gamma)$;
        \item Both functors maps disjoint unions to direct sums. 
    \end{enumerate}
    \end{definition}
    \end{tcolorbox}
    


Before describing concrete examples, we want to clarify some structure inherent in the burnside categories and Mackey Functors: suppose we have a $G$-Mackey functor $\underline{M}$; since it is additive, its data is completely determined by its value on orbits, and the restriction/transfer morphisms induced by morphisms between orbits. Further, recall that a $G$-map between orbits $G/H\to G/K$ given by $H\mapsto gK$ must satisfy $gHg^{-1}\subseteq K$. Thus, the map can be decomposed into 
\[G/H\xrightarrow{c_{g}} G/gHg^{-1}\xrightarrow{\pi} G/K\] 
where $c_{x}: G/H\to G/gHg^{-1}$ is the conjugation map defined by $H\mapsto g(gHg^{-1})$, and $\pi$ is the canonical projection that sends $gHg^{-1}\mapsto K$. herefore, the data of a Mackey functor is completely determined by the conjugation morphisms and the canonical projection. We will see how these morphisms commute with each other.

Let $c_g^*:=M^*(c_g)$ and $c^g_*:=M_*(c_g)$. 

We have the following pullback square 
\[\begin{tikzcd}
	{G/H} & {G/H} \\
	{G/H} & {G/gHg^{-1}}
	\arrow["Id", from=1-1, to=1-2]
	\arrow["Id"', from=1-1, to=2-1]
	\arrow["\lrcorner"{anchor=center, pos=0.125}, draw=none, from=1-1, to=2-2]
	\arrow["{c_g}", from=1-2, to=2-2]
	\arrow["{c_g}"', from=2-1, to=2-2]
\end{tikzcd}\]
So in fact we have $c_g^*=(c^g_*)^{-1}$. 

Given a canonical projection $\pi: G/H\to G/K$, denote $\textrm{Res}_H^K:= M^*(\pi)$ and  $\textrm{Tr}_H^K:= M_*(\pi)$. 
 Now consider the following commutative diagram:
\[\begin{tikzcd}
	{G/H} & {G/G} \\
	{G/gHg^{-1}} & {G/G}
	\arrow["\pi", from=1-1, to=1-2]
	\arrow["{c_g}"', from=1-1, to=2-1]
	\arrow["Id", from=1-2, to=2-2]
	\arrow["\pi"', from=2-1, to=2-2]
\end{tikzcd}\]
which implies $\textrm{Tr}_H^G\circ c_*^g=\textrm{Tr}_H^G$ and $\textrm{Res}_H^G(x)=c_g^*\circ\textrm{Res}_H^G(x)$ for all $g$. Moreover, consider the pullback diagram


\[\begin{tikzcd}
	G/H\times G/K & G/K \\
	G/H & G/G\cong *
	\arrow["\beta", from=1-1, to=1-2]
	\arrow["\alpha"', from=1-1, to=2-1]
	\arrow["\delta", from=1-2, to=2-2]
	\arrow["\gamma"', from=2-1, to=2-2]
\end{tikzcd}\]

By the $G$-isomorphism $G/H\times G/K\cong G\times_H G/K$, it is easy to see the the class $(1,xK)_H$ has stablizer $H\cap xKx^{-1}$, and the orbits are indexed by a class in the double coset $H\backslash G/K$. Thus, $$G/H\times G/K\cong \coprod_{H\backslash G/K}G/H\cap xKx^{-1}$$. 
We may use this decomposition to compute the composition of tranfers and restrictions: the composition $\textrm{Res}_K^G\circ \textrm{Tr}_H^G: M(G/H)\to M(G/K)$ is given by the value of the Mackey functor on the diagram 
\[\begin{tikzcd}
	&& {\coprod_{H\backslash G/K} G/H\cap xKx^{-1}} \\
	& {G/H} && {G/K} \\
	{G/H} && {G/G} && {G/K}
	\arrow[from=1-3, to=2-2]
	\arrow[from=1-3, to=2-4]
	\arrow["\lrcorner"{anchor=center, pos=0.125, rotate=-45}, draw=none, from=1-3, to=3-3]
	\arrow[from=2-2, to=3-1]
	\arrow[from=2-2, to=3-3]
	\arrow[from=2-4, to=3-3]
	\arrow[from=2-4, to=3-5]
\end{tikzcd}\]
so by the additivity, it is the direct sum of the value of the morphisms
\[\begin{tikzcd}
	& {G/H\cap xKx^{-1}} \\
	{G/H} && {G/K}
	\arrow[from=1-2, to=2-1]
	\arrow[from=1-2, to=2-3]
\end{tikzcd}\]
which is the composition $\textrm{Tr}_{x^{-1}Hx\cap K}^{K}\circ c_{x}\circ \textrm{Res}_{H\cap xKx^{-1}}^{H}$. We now have the double coset formula


\begin{tcolorbox}[colback=red!5!white,colframe=red!30!white]
\begin{theorem}[Double Coset Formula]
    \[\textrm{Res}_K^G\circ \textrm{Tr}_H^G=\oplus_{x\in H\backslash G/K }\textrm{Tr}_{x^{-1}Hx\cap K}^{K}\circ c_{x}\circ \textrm{Res}_{H\cap xKx^{-1}}^{H}\]
\end{theorem}
\end{tcolorbox}

By the above discussion, we have a very concrete alternative definition for Mackey functors


\begin{tcolorbox}[colback=purple!5!white,colframe=purple!75!black]
\begin{definition}
A Mackey functor $\underline{M}: \textrm{GSet}^{\textrm{Fin}}\to \textrm{Ab}$ consists of the following data:  
\begin{enumerate}
    \item An abelian group $\underline{M}(G/H)$ for every subgroup $H\leq G$;
    \item For every subgroup $H\leq K\leq G$, a restriction map $\textrm{Res}_H^K: \underline{M}(G/K)\to \underline{M}(G/H)$ and a transfer map $\textrm{Tr}_H^K: \underline{M}(G/H)\to \underline{M}(G/K)$.
    \item For every subgroup $H\leq G$ and $g\in G$, a conjugation homomorphism $c_g: \underline{M}(G/H)\to \underline{M}(G/H)$.
    

\end{enumerate}
The three types of morphisms must also satisfy the following relations:
\begin{enumerate}
    \item $\textrm{Tr}_H^H$ and $\textrm{Res}_H^H$ are both identity maps for all subgroups $H\leq G$.
    \item $\textrm{Res}_H^K\circ \textrm{Res}_K^L=\textrm{Res}_H^L$ and 
     $\textrm{Tr}_K^L\circ \textrm{Tr}_H^K=\textrm{Tr}_H^L$ for $K\leq H\leq L$.
    \item $c_g\circ c_h=c_{gh}$ for all $g,h\in G$.
    \item $\textrm{Res}_{gHg^{-1}}^{gKg^{-1}}\circ c_g=c_g\circ \textrm{Res}^{K}_{H}$ and $\textrm{Tr}_{gHg^{-1}}^{gKg^{-1}}\circ c_g=c_g\circ \textrm{Tr}^{K}_{H}$ for $H\leq K$.
    \item   For subgroups $H\leq K\leq L$
      \[\textrm{Res}_K^L\circ \textrm{Tr}_H^L=\oplus_{x\in H\backslash L/K }\textrm{Tr}_{x^{-1}Hx\cap K}^{K}\circ c_{x}\circ \textrm{Res}_{H\cap xKx^{-1}}^{H}\]
\end{enumerate}
\end{definition}
\end{tcolorbox}
For any group $G$, we can now give an example of a constant Mackey functor:
\begin{tcolorbox}[colback=yellow!5!white,colframe=yellow!30!white]
    \begin{example}[$\underline{\textbf{Constant Mackey functor}}$]
    For every abelian group $A$, we have the constant Mackey functor $\underline{A}$, which assigns every orbit the abelian group $A$, and restriction maps the identity morphism. One may check that this forces the transfer maps $\textrm{Tr}_H^K: \mathbb{Z}\to \mathbb{Z}$ to be multiplication by $|K/H|$, and the conjugation maps to be the identity. \\


    In the case where $G=C_2$ and $A=\mathbb{Z}$, the Mackey functor is described by the following diagram:
    \[\begin{tikzcd}
        {\mathbb{Z}} \\
        {\mathbb{Z}}
        \arrow["{\times 2}",swap, curve={height=12pt}, from=2-1, to=1-1]
        \arrow["Id"', curve={height=12pt}, from=1-1, to=2-1]
        \arrow["Id"', from=2-1, to=2-1, loop, in=300, out=240, distance=5mm]
    \end{tikzcd}\]
    In the case where $G=C_2$ and $A=\mathbb{Z}/2$, the Mackey functor is described by the following diagram:
    \[\begin{tikzcd}
        {\mathbb{Z}/2} \\
        {\mathbb{Z}/2}
        \arrow["{\times 0}",swap, curve={height=12pt}, from=2-1, to=1-1]
        \arrow["Id"', curve={height=12pt}, from=1-1, to=2-1]
        \arrow["Id"', from=2-1, to=2-1, loop, in=300, out=240, distance=5mm]
    \end{tikzcd}\]



\end{example}
\end{tcolorbox}

    
\begin{tcolorbox}[colback=green!5!white,colframe=green!30!white]
    \begin{remark}
    Note that the "constant" Mackey functor $\underline{\mathbb{Z}}$ breaks up to a constant coefficient system $\mathbb{Z}$, but the covariant part is $\textbf{NOT}$ the constant functor. We will run into this problem later. 
    \end{remark}
\end{tcolorbox}



Definintion $3.1.1$ seems bulky, but when we are dealing with $G=C_p$ Mackey functors, the axioms can be reduced to the following: 

\begin{tcolorbox}[colback=blue!5!white,colframe=blue!30!white]
\begin{proposition}
A $C_p$-Mackey functor $\underline{M}$ consists of the following data: 
\[\begin{tikzcd}
	{\underline{M}(C_p/C_p)} \\
	\\
	{\underline{M}(C_p/e)}
	\arrow["{\textrm{Res}^{C_p}_{e}}",swap, bend right=50, from=1-1, to=3-1]
	\arrow["{\textrm{Tr}^{C_p}_{e}}",swap, bend right=50, from=3-1, to=1-1]
	\arrow["{c_g}", from=3-1, to=3-1, loop, in=305, out=235, distance=10mm]
\end{tikzcd}\]
where we ignore the some of all of the obvious identity morphisms. The morphisms satifies the following:
\begin{enumerate}
    \item $c_g\circ \textrm{Res}^{C_p}_{e}=\textrm{Res}^{C_p}_{e}$.
    \item ${\textrm{Tr}^{C_p}_{e}}\circ c_g={\textrm{Tr}^{C_p}_{e}}$
    \item $\textrm{Res}^{C_p}_{e}\circ \textrm{Tr}^{C_p}_{e}=\sum_{\gamma\in C_p}c_{\gamma}$
\end{enumerate}

\end{proposition}
\end{tcolorbox}
Here are a few more examples of Mackey functors:


\begin{tcolorbox}[colback=yellow!5!white,colframe=yellow!30!white]
\begin{example}[$\underline{\textbf{Burnside Mackey Functor}}$]
The Burnside Mackey Functor $A_G$ assigns every orbit $G/H$ the Grothendieck group of the symmetric monoidal category of finite $H$-sets under coproduct. The transfer and restriction maps come from induction and the forgetful functor, respectively, between $H$-Sets and $K$-Sets. \\

When $G=C_p$, the Burnside Mackey Functor $A_{C_p}$ is given by the following: there are two subgroups $e$ and $C_p$; the corresponding Grothendieck groups are $A_{C_p}(e)=\mathbb{Z}$, generated by the singleton with trivial action, and $A_{C_p}(C_p)=\mathbb{Z}\oplus \mathbb{Z}$, generated by the singleton with trivial action and a $C_p$ orbit. The induced $G$ set $C_p\times_e *$ is the $C_p$ orbit, and forgetting the $G$ action on the $C_p$-orbit gives us a disjoint union of $p$ singletons with trivial action. Diagramatically, we have the following  
\[\begin{tikzcd}
	{\mathbb{Z}} \\
	{\mathbb{Z}\oplus \mathbb{Z}}
	\arrow["{(0,Id)}"', curve={height=12pt}, from=1-1, to=2-1]
	\arrow["{(Id,p)}"', curve={height=12pt}, from=2-1, to=1-1]
	\arrow["Id"', from=2-1, to=2-1, loop, in=300, out=240, distance=5mm]
\end{tikzcd}\]

\end{example}
\end{tcolorbox}


\begin{tcolorbox}[colback=yellow!5!white,colframe=yellow!30!white]
\begin{example}
The homotopy groups of a genuine $G$-spectra $X$, $\pi^*_n(X)$, is naturally a Mackey functor: we have the assignment $\pi_n^*(X)(G/H):=\pi^H_n(X)$; by the Wirthmuller isomorphism and the fact that orbits are self dual, we have 
\[\pi_n^H(X)\cong [S^n\wedge G/H_+, X]\cong [S^n,X\wedge G/H_+]\]
since we have availabilities on both variances, the tranfers and restrictions maps are easy to define. 
\end{example}
\end{tcolorbox}


\begin{tcolorbox}[colback=purple!5!white,colframe=purple!75!black]
\begin{definition}(RO(G))

\end{definition}
\end{tcolorbox}


\begin{tcolorbox}[colback=red!5!white,colframe=red!30!white]
\begin{theorem}
(Eilenberg-Maclane Spectra) Given a Mackey functor $\underline{M}$, there exists a Eilenberg Maclane Spectra $HM$ that represents the ordinary $RO(G)$-graded cohomology theories on $Sp^G$. 
\end{theorem}
\end{tcolorbox}


\subsection{The $RO(C_2)$ cohomology of a point}


\begin{tcolorbox}[colback=blue!5!white,colframe=blue!30!white]
\begin{proposition}
The representation ring $RO(C_2)\cong \mathbb{Z}[\sigma]/(\sigma^2-1)$. 
\end{proposition}
\end{tcolorbox}
Thus as abelian group, $RO(C_2)\cong \mathbb{Z}^2$, generated by the trivial representation and the sign representation. We now calculate the ordinary $RO(C_2)$-cohomology of a point, $H^{p,q}(*;\underline{\mathbb{Z}})$ where $p$ is the dimension of the trivial representation
and $q$ is the dimension of the sign representation. 

First, we recall the important fact that the $RO(C_2)$-cohomology at a trivial representation reduces to the Bredon cohomology with constant coefficient system induced from the Mackey functor. Our game plan is then to first abuse the suspension isomorphism and reduce to Bredon cohomology. We now break the calculation to two parts:

If $q<0$, then by the suspension isomorphism we have $$H^{p,q}(*;\underline{\mathbb{Z}}):= \tilde{H}^{p,q}(S^0;\underline{\mathbb{Z}})\cong \tilde{H}^{p,0}(S^{-q\sigma};\underline{\mathbb{Z}})\cong H^{p}_{\textrm{Bredon}}(S^{-q\sigma};\underline{\mathbb{Z}})$$
We note that the $C_2$ action on the representation sphere $S^{-q\sigma}$ is the antipodal action, which is free, so we can use the following proposition


\begin{tcolorbox}[colback=blue!5!white,colframe=blue!30!white]
\begin{proposition}
If the $G$-action is free, then 
\[H^*_{\textrm{Bredon}}(X; M)\cong H^*_{\textrm{Sing}}(X/G; M(e))\]
\end{proposition}
\end{tcolorbox}








\end{document}