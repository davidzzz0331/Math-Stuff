\documentclass{article}
\usepackage[utf8]{inputenc}
\usepackage{amsmath}
\usepackage{amsfonts}
\usepackage{amssymb}
\usepackage{tikz}
\usepackage{fullpage}
\usepackage{tikz-cd}
\usepackage{spectralsequences}
\usepackage{adjustbox}
\usepackage[backend=biber, style=alphabetic]{biblatex}
\usepackage{xfrac}
\usepackage{tcolorbox}
\usepackage{xcolor}
\usepackage{graphicx}
\graphicspath{ {D:/Chrome Downloads./} }
\usepackage[parfill]{parskip}
\usepackage{amsthm}
\addbibresource{EHT.bib}
\theoremstyle{definition}
\newtheorem{theorem}{Theorem}[section]
\theoremstyle{definition}
\newtheorem{definition}{Definition}[theorem]
\theoremstyle{definition}
\newtheorem{remark}{Remark}[theorem]
\theoremstyle{definition}
\newtheorem{proposition}{Proposition}[theorem]
\theoremstyle{definition}
\newtheorem{lemma}[theorem]{Lemma}
\theoremstyle{definition}
\newtheorem{corollary}{Corollary}[theorem]
\theoremstyle{definition}
\newtheorem{example}{Example}[theorem]
\title{Equivariant Stable Homotopy Notes}
\author{David Zhu}

\begin{document}
\maketitle
For the entire note, we will assume a group $G$ to be a compact Lie group, and subgroups $H\subset G$ are always closed. 

\section{Unstable Equivariant Homotopy Theory}
\subsection{G-CW Complexes}
Fix a compact Lie group $G$ acting on a space $X$. Similar to $CW$-complexes, we want to deconstruct $X$ into cells, but this time with the addditional data of the $G$-action along with each cell. The idea is that cells are of the form of a product $G/H\times D^{n}$, where $G$ acts trivially on $D^n$, and $G/H$ "represents" the orbits of $D^n$. To make this work, $H$ must be the isotropy group of $D^n$. 


\begin{tcolorbox}[colback=purple!5!white,colframe=purple!75!black]
\begin{definition}
A \underline{\textbf{G-CW complex}} is the sequential colimit of spaces $X_n$, where $X_{n+1}$ is a pushout: 
\[\begin{tikzcd}
\coprod G/H\times S^n\arrow[r]\arrow[d]&X_n\arrow[d]\\
\coprod G/H\times D^{n+1}\arrow[r]&X_{n+1}
\end{tikzcd}\]
We will denote  $G/H\times D^{n}$ as an \underline{\textbf{n-cell}}. 
\end{definition}
\end{tcolorbox}

\begin{tcolorbox}[colback=green!5!white,colframe=green!30!white]
\begin{remark}
Note that the topological dimension of an $n$-cell in a $G$-CW complex might be greater than $n$. For example, a $0$-cell $S^1/e\times *$ is one dimensional. 
\end{remark}
\end{tcolorbox}

\begin{tcolorbox}[colback=yellow!5!white,colframe=yellow!30!white]
\begin{example}
Let $G=C_2$ acting on $S^2$ by rotation by $\pi$ along the Z-axis. It has a $G$-CW structure given by the following cells: $2$ zero-cells $C_2/C_2\times *$, which are the poles corresponding to the fixed points of the $C_2$ action. $1$ one-cell $C_2/e\times D^1$, which are the two great circles joining the poles; $1$ two-cell $C_2/C_2\times D^2$, which are the two hemispheres.  
\end{example}
\end{tcolorbox}


\begin{tcolorbox}[colback=yellow!5!white,colframe=yellow!30!white]
\begin{example}
    Let $G=C_2$ acting on $S^2$ by the antipodal map. It has a $G$-CW structure given by the following cells: $1$ zero-cells $C_2/e\times *$, which are the poles; $1$ one-cell $C_2/e\times D^1$, which are the two great circles joining the poles; $1$ two-cell $C_2/C_2\times D^2$, which are the two hemispheres. 
\end{example}
\end{tcolorbox}


\begin{tcolorbox}[colback=purple!5!white,colframe=purple!75!black]
\begin{definition}
Let $H$ be a subgroup of $G$. Define $\pi_n^H(X):=\pi_n(X^H)$. A map $f: X\to Y$ of $G$-spaces is a \underline{\textbf{weak equivalence}} if for all subgroups $H\subset G$,
\[f_*:\pi_n^H(X)\to \pi_n^H(Y)\]
is an isomorphism. 
\end{definition}
\end{tcolorbox}

Let $\textbf{GTop}$ be the category of $G$-spaces and $G$-maps. There is a cofibrantly-generated model structure that we can put on $\textbf{GTop}$:

\begin{tcolorbox}[colback=red!5!white,colframe=red!30!white]
\begin{theorem}
    There is a cofibrantly-generated model structure on $\textbf{GTop}$, given by 
    \begin{enumerate}
        \item A $G$-map $f:X\to Y$ is a fibration iff for all $H\subset G$, $f^H: X^H\to Y^H$ is a fibration.
        \item A $G$-map $f:X\to Y$ is a weak equivalence iff for all $H\subset G$, $f^H: X^H\to Y^H$ is a weak equivalence. 
    \end{enumerate}
\end{theorem}
\end{tcolorbox}
An immediate consequence of the model category structure is the equivariant Whitehead's Theorem

\begin{tcolorbox}[colback=green!5!white,colframe=green!30!white]
\begin{corollary}
Let $f: X\to Y$ be a weak equivalence of cofibrant-fibrant objects in a model category. Then, $f$ is a homotopy equivalence. In particular, every object in $\textbf{GTop}$ is fibrant, and $G$-CW complexes are  cofibrant. 
\end{corollary}
\end{tcolorbox}

\subsection{Elmendorf's Theorem}
From the model structure given in Theorem $1.1$, we have a vague sense of the following "equivalence":
\[\textrm{G-Homotopy Type of } X\Leftrightarrow \{ \textrm{ordinary homotopy type of } X^H:H\subset G \}\]
And Elmendorf's Theorem will make the equivalence precise. We start by introducing the orbit category:

\begin{tcolorbox}[colback=purple!5!white,colframe=purple!75!black]
\begin{definition}
The \underline{\textbf{orbit category}} $\mathcal{O}_G$ is the full subcategory of $\textrm{GTop}$ on the objects $\{G/H: H\subset G\}$.
\end{definition}
\end{tcolorbox}
The following lemma will make the structure of $\mathcal{O}_G$ clearer.

\begin{tcolorbox}
\begin{lemma}
$\textrm{Map}^G(G/H,G/K)\cong (G/K)^H$
\end{lemma}
\end{tcolorbox}
\begin{proof}
    Note that there exists a $G$-equivariant maps $\varphi: G/H\to G/K$, determined by $\varphi(H)=gK$ iff $gHg^{-1}\subseteq K$ iff $h(gK)=gK$ for all $h\in H$. 
\end{proof}

Let $\textrm{Fun}(\mathcal{O}_G^op,\textrm{Top})$ be the functor category. We have the following fact on the model structure on functor categories:

\begin{tcolorbox}[colback=red!5!white,colframe=red!30!white]
\begin{theorem}
Let $\mathcal{D}$ be a model category and $\mathcal{C}$ be a cofibrantly generated model category. Then, $\textrm{Fun}(\mathcal{C},\mathcal{D})$ admits a model structure. 
\end{theorem}
\end{tcolorbox}

It is useful to know that the weak equivalences in $\textrm{Fun}(\mathcal{O}^{op}_G,\textrm{Top})$ is given pointwise: a natural transformation $\eta: \mathcal{F}\to \mathcal{G}$ is a weak equivalence iff $\eta_{G/H}:\mathcal{F}(G/H)\to \mathcal{G}(G/H) $ is a weak equivalence. 



\begin{tcolorbox}[colback=purple!5!white,colframe=purple!75!black]
\begin{definition}
There is a functor $\psi: \textrm{GTop}\to \textrm{Fun}(\mathcal{O}^{op}_G,\textrm{Top})$ given by 
\[X\to (G/H\mapsto X^H)\]
\end{definition}
\end{tcolorbox}
It is easy to check the functoriality. Note that if we restrict $\psi$ to $\mathcal{O}_G$, the functor is just the Yoneda embedding: $\textrm{Map}^G(G/H,G/K)\cong (G/K)^H$.

\begin{tcolorbox}[colback=blue!5!white,colframe=blue!30!white]
    \begin{proposition}
    There is a funcor $\theta: \textrm{Fun}(\mathcal{O}^{op}_G,\textrm{Top})\to \textrm{GTop}$ given by $X\mapsto X(G/e)$, where $X(G/e)$ is equipped with the following $G$-action: note that every $g\in G$ defines an $G$-map $G/e\to G/e$, which we denote by $R_g$.
    \[g\cdot x=X(R_g)(x)\] 
    \end{proposition}
    \end{tcolorbox}
    
    It is easy to check that $(\theta,\psi)$ is an adjoint pair. In fact, more can be said:
    
    \begin{tcolorbox}[colback=red!5!white,colframe=red!30!white]
        \begin{theorem}
        (Elmendorf's Theorem) $\textrm{Fun}(\mathcal{O}^{op}_G,\textrm{Top})$ and $\textrm{GTop}$ have the same homotopy category. 
        \end{theorem}
        \end{tcolorbox}
    The original proof due to Elmendorf constructs the equivalence explicitly using the Bar construction to obtain a homotopy inverse to the embedding $\psi$. The theorem can now be put into a more modern framework:

    \begin{tcolorbox}[colback=red!5!white,colframe=red!30!white]
    \begin{theorem}
    $(\theta,\psi)$ is an Quillen equivalence. $\psi$ is an equivalence of $(\infty,1)$ categories. 
    \end{theorem}
    \end{tcolorbox}

\subsection{Bredon Cohomology}  
The goal is to construct a cohomology theory satisfying the Eilenberg-Steenrod axioms under the equivariant setting. 
 \begin{tcolorbox}[colback=purple!5!white,colframe=purple!75!black]
 \begin{definition}
 (Equivariant reduced generalized cohomology) Let $G\textrm{CW}_*$ be the category of pointed $G$-CW complexes with equivariant maps. Then, a generalized cohomology theory on $G\textrm{CW}_*$ is a sequence of contravariant functors 
 \[\tilde{H}^n:=G\textrm{CW}_*\to \textbf{Ab} \]
 satisfying the following:
 \begin{enumerate}
    \item if $f,g$ are equivariantly homotopic, then $\tilde{H}^n(f)=\tilde{H}^n(g)$.
    \item There exists a sequence of natural isomorphisms
   \[\tilde{H}^n(X)\to \tilde{H}^{n+1}(S^1\wedge X)\]
   where $G$ acts trivially on $S^1$ in the smash. 
    \item The sequence 
    \[\tilde{H}^n(X/A)\to \tilde{H}^n(X)\to \tilde{H}^n(A)\]
    is exact.
 \end{enumerate}
 \end{definition}
 \end{tcolorbox}

\begin{tcolorbox}[colback=green!5!white,colframe=green!30!white]
\begin{remark}
The above axioms is built upon pointed "single" spaces. It is in fact equivalent to the usual theory built upon pairs, and is justified in 
\end{remark}
\end{tcolorbox}
For a non-equivariant reduced generalized cohomology theory $\tilde{h}^*$, the Atiyah-Hirzebruch spectral sequence tells us that knowing $\tilde{h}^*(pt)$ basically determines the cohomology theory on CW complexes. Heuristically, the cohomology is determined by the building blocks, which are contractible open cells. However, in the equivariant setting, the building blocks are more complicated: the building blocks have become orbits of the form $G/H$. We are lead to the following definitions:

\begin{tcolorbox}[colback=purple!5!white,colframe=purple!75!black]
\begin{definition}
A \underline{\textbf{coefficient system}} is a contravariant functor $\mathcal{F}: \mathcal{O}_G\to \textrm{Ab}$.
\end{definition}
\end{tcolorbox}
Recall that if a reduced cohomology theory is called \underline{\textbf{ordinary}} if it satisfies the dimension axiom, i.e the zeroth reduced cohomology group (a.k.a the coeffcient system) is trivial on a point. Our goal now is to construct such a theory in the equivariant setting, which is called Bredon cohomology. 

Note that by the general theory of abelian categories, the functor category of coefficient systems $\mathcal{CS}:=\textrm{Fun}(\mathcal{O}_G, \textrm{Ab})$ is abelian. It is now possible to define Bredon cohomology on a $G$-CW complex by explicitly defining the cochain complexes on cells, for example see . However, we may package the cochains into the following form 

\begin{tcolorbox}[colback=purple!5!white,colframe=purple!75!black]
\begin{definition}
For each $n$, we may define a coefficient system $C_n(-)$, given by 
\[G/H\mapsto H_n((X^H)_{n},(X^H)_{n-1}; \mathbb{Z})=C_n^{CW}(X^H)\]
The differential of the CW chain complex induces a chain complex of coefficient systems $C_{\cdot}(-)$. 
\end{definition}
\end{tcolorbox}
It is easy to check that $Hom_{\mathcal{CS}}(C_{\cdot}(-),M)$ is a cochain complex whose differentials is induced by those of $C_{\cdot}()$. 
\begin{tcolorbox}[colback=purple!5!white,colframe=purple!75!black]
\begin{definition}
The \underline{\textbf{Bredon cohomology}} of $X$ with coefficients in a system $M$ is defined by 
\[H_G^n(X;M):=H^n(Hom_{\mathcal{CS}}(C_{\cdot}(X),M))\]
\end{definition}
\end{tcolorbox}

proof of the axioms and computation of examples. 



\section{Spectra and the Stable Category}









\end{document}