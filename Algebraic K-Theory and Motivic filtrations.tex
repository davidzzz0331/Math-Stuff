\documentclass{article}
\usepackage[utf8]{inputenc}
\usepackage{amsmath}
\usepackage{amsfonts}
\usepackage{amssymb}
\usepackage{tikz}
\usepackage{fullpage}
\usepackage{tikz-cd}
\usepackage{spectralsequences}
\usepackage{adjustbox}
\usepackage{xfrac}
\usepackage{tcolorbox}
\usepackage{xcolor}
\usepackage{graphicx}
\graphicspath{ {D:/Chrome Downloads./} }
\usepackage[parfill]{parskip}
\usepackage{amsthm}
\theoremstyle{definition}
\newtheorem{theorem}{Theorem}[section]
\theoremstyle{definition}
\newtheorem{definition}{Definition}[section]
\theoremstyle{definition}
\newtheorem{problem}{problem}[section]
\theoremstyle{definition}
\newtheorem{proposition}{Proposition}[section]
\theoremstyle{definition}
\newtheorem{lemma}[theorem]{Lemma}
\theoremstyle{definition}
\newtheorem{corollary}{Corollary}[theorem]
\theoremstyle{definition}
\newtheorem{example}{Example}[section]
\title{Algebraic K-Theory and Motivic Filtrations}
\author{David Zhu}

\begin{document}
\maketitle

This is a colloquim talk given by Mathew Morrow on March 27. 

Algebraic K-theory is basically a cohomology theory for rings. We associate a given ring with a collection of abelian groups. The goal is to understand these groups and understand their relations with other invariants. In principal, they see information about many areas/subjects in mathematics. 

\section{$K_0$}


\begin{tcolorbox}[colback=purple!5!white,colframe=purple!75!black]
\begin{definition}
$K_0(R)$ is the free abelian group generated by the finitely generated projective modules over $R$ under the relations of direct sums.
\end{definition}
\end{tcolorbox}


\begin{tcolorbox}[colback=yellow!5!white,colframe=yellow!30!white]
\begin{example}
Over a field $F$, $K_0(F)=\mathbb{Z}$ by dimension. 
\end{example}
\end{tcolorbox}


\begin{tcolorbox}[colback=yellow!5!white,colframe=yellow!30!white]
\begin{example}
$D=$ ring of integers of number field. Any ideal of $D$ is projective. Class group of the number field is embedded in $K_0(D)$.
\end{example}
\end{tcolorbox}

\begin{tcolorbox}[colback=yellow!5!white,colframe=yellow!30!white]
\begin{example}
X compact Hausdorff space, and $C(X)$ be the set of continuous functions. The section of any locally trivial vector bundles of a projective $C(X)$-module. This is topological $K$-theory $K_0(C(X))$.
\end{example}
\end{tcolorbox}




\section{$K_1$}

\begin{tcolorbox}[colback=purple!5!white,colframe=purple!75!black]
\begin{definition}
$GL_{\infty}(R)/[GL_{\infty}(R),GL_{\infty}(R)]$, where $GL_{\infty}(R)$ iscolimit of $GL_n(R)$.
\end{definition}
\end{tcolorbox}


\begin{tcolorbox}
\begin{lemma}
$M,M'\in GL_{\infty}(R)$ iff they are equivalent up to row and column operations. 
\end{lemma}
\end{tcolorbox}


\begin{tcolorbox}[colback=yellow!5!white,colframe=yellow!30!white]
\begin{example}
If $F$ is a field, then we have Gaussian elimination, and reduce the invariant to the determinant. $K_1(F)=F^{\times}$
\end{example}
\end{tcolorbox}


\begin{tcolorbox}[colback=yellow!5!white,colframe=yellow!30!white]
\begin{example}
$D$ integers of number field: if it is euclidean domains have analog of Gaussian elimination, which gives us $k_1(D)=D^{\times}$. If not, the result still holds but proof is more involved by Bass-Milnor-Serre. 
\end{example}
\end{tcolorbox}


\begin{tcolorbox}[colback=yellow!5!white,colframe=yellow!30!white]
\begin{example}
X is a smooth compact manifold of dimension $\geq 5$. Consider the $G:= \mathbb{Z}\pi_1(X)$ as the group ring over the fundamental group. Then it is the $s-$ cobordism theorem $K_1(G)/(\pm 1\oplus \pi_1)$ classifified $h-$ cobordism of $X$. 
\end{example}
\end{tcolorbox}

In the 60s, more relations between $k_0$ and $K_1$ are discovered. 


\begin{tcolorbox}[colback=red!5!white,colframe=red!30!white]
\begin{theorem}
(Fundamental theorem of $K_0$, Bass)$K_0(R)$ is the cokernel of the map 
$K_1(R[t]\oplus K_1(R[t ^{-1}]))\to K_1(R[t,t ^{-1}])$
\end{theorem}
\end{tcolorbox}
The slogan is $K_1$ is a refinement of $K_0$. Thus, it is natural to continue to look for higher $K_n$.

\section{$K_n$}

Quillen's idea of to derive the construction of $K_1$ is as follows: we start with $R\to GL_{\infty}(R)$. Consider its classifying space $BGL_{\infty}(R)$ with $\pi_1(BGL_{\infty}(R))=GL_{\infty}(R)$. Then, we have to modify the spaces using the plus construction to kill the maximal perfect subgroup (which turns out to be the commutator subgroups) of $\pi_1$, which makes the $K_1$ abelian. 

\begin{tcolorbox}[colback=red!5!white,colframe=red!30!white]
\begin{theorem}
(Quillen) $K_n(R):=\pi_n(BGL_{\infty}(R)^+)$
\end{theorem}
\end{tcolorbox}

\section{Techniques to Understand $K$-theory(for algebrac/arithmetic geometry)}
The idea is to replace $K$-theory with something easier to understand: Motivic cohomology, cylic homology and a common ground, motivic filtrations. 

For Motivic cohomology, $R$ is a smooth algebra over a field. $K(R)$ admits a filtration/stratification/ \ decomposition with buiding bricks motivic cohomoogy groups $H^i_{mot}(R,\mathbb{Z}(j))$. Encode Blcok-Kato conjecture $H_{Gal}(F,\mu_t^{\otimes n})\cong K^M_n(F)$. The decomposition of $K(R)$ is the first example of a "motivic" filtration. This only works for smooth algebras over a field. 


\begin{tcolorbox}[colback=purple!5!white,colframe=purple!75!black]
\begin{definition}
$R$ a fring, its Hochschild homology groups are the homology of the complex
\[
    R^{\otimes n}\to R^{\otimes n-1};\  a\otimes b\mapsto ab-ba
\]
Refinements: Cylic hmoology.
\end{definition}
\end{tcolorbox}
The above is an analog of De Rham cohomology, and it works well with rings of characteristic $0$. A better approach is Crystalline/prismatic cohomology proposed by Nikolas-Scholze. We get Topological cyclic cohomology $TC_n(R)$. 


\begin{tcolorbox}[colback=red!5!white,colframe=red!30!white]
\begin{theorem}
There exists delogaritmn maps $K_n(R)\to TC_n(R)\to HH_n(R)=\Omega^1(R)$, where the last equality holds when the ring is commutative. 
\end{theorem}
\end{tcolorbox}


\begin{tcolorbox}[colback=red!5!white,colframe=red!30!white]
\begin{theorem}
The failure of dlog to be an somorphism has remarkably nice properties. Example: $R_1\to R_2$ nilpotent surjection implies $3$ out of $4$ properties: $K(R_1),K(R_2), TC(R_1), TC(R_2)$, which means we get a MV sequence relating all four cohomologies. 
\end{theorem}
\end{tcolorbox}
The question is: for which ring $R$ is $K$-theory an homotopy invariant: $K_n(R[t])=K_n(R)$ for all $n$?


\begin{tcolorbox}[colback=red!5!white,colframe=red!30!white]
\begin{theorem}
(Fundamental Theorem, Quillen) $R$ is regular Noetherian implies it is an invariant..

\end{theorem}
\end{tcolorbox}

\begin{tcolorbox}[colback=red!5!white,colframe=red!30!white]
\begin{theorem}
(Cortinas)$R$ commutative $C^*$-algebras, probe $R=C(x)$ by blowing up affine algebraic varieties. 
\end{theorem}
\end{tcolorbox}
 
 \begin{tcolorbox}[colback=red!5!white,colframe=red!30!white]
 \begin{theorem}
 (Mathhew-Antieu-M) $R$ perfectoid. Tilt, blow-up, homological properties of valuation rings.
 \end{theorem}
 \end{tcolorbox}

 The conjecture is for non-regular Noetherian rings, it is not invariant. Classical motivic cohomology: geometric in flavor, smooth algebra over a field; cyclic homology: very homological algebra. What is the common ground?

 
 \begin{tcolorbox}[colback=red!5!white,colframe=red!30!white]
 \begin{theorem}
 (Bhatt-Scholze) $R$ ring with mild singularities. Then, the cyclic homologies of $R$ admit motivic filtrations with building blocks simpler/more geometric cohomologies(De Rham, crystalline, etale, prismatic).
 \end{theorem}
 \end{tcolorbox}

\begin{tcolorbox}[colback=yellow!5!white,colframe=yellow!30!white]
\begin{example}
$K(\mathbb{Z}/p^\mathbb{Z})\to TC(\mathbb{Z}/p^m \mathbb{Z})$ with building blocks syntomic cohomology, which leads to new computations of $K(\mathbb{Z}/p \mathbb{Z})$ by Antieau. 
\end{example}
\end{tcolorbox}


\begin{tcolorbox}[colback=red!5!white,colframe=red!30!white]
\begin{theorem}
The $K$-theory of any ring containing a field admits a motivic filtration with building blocks given by a new theory of motivic cohomology that recovers motivic cohomology when $R$ is smooth.  
\end{theorem}
\end{tcolorbox}
The idea of proof is approximate $K$-theory orthogonally by TC and KH. Equip the latter two thoeries with motivic filtration. Finally glue. A corollary is intersection theory on singular algebraic varieties. Hot current questions: which invariants admmit motivic filtrations, and in what generality?


\end{document}