\documentclass{article}
\usepackage[utf8]{inputenc}
\usepackage{amsmath}
\usepackage{amsfonts}
\usepackage{amssymb}
\usepackage{tikz}
\usepackage{fullpage}
\usepackage{tikz-cd}
\usepackage{spectralsequences}
\usepackage{adjustbox}
\usepackage{xfrac}
\usepackage{tcolorbox}
\usepackage{xcolor}
\usepackage{graphicx}
\graphicspath{ {D:/Chrome Downloads./} }
\usepackage[parfill]{parskip}
\usepackage{amsthm}
\theoremstyle{definition}
\newtheorem{theorem}{Theorem}[section]
\theoremstyle{definition}
\newtheorem{problem}{problem}[section]
\theoremstyle{definition}
\newtheorem{proposition}{Proposition}[section]
\theoremstyle{definition}
\newtheorem{lemma}[theorem]{Lemma}
\theoremstyle{definition}
\newtheorem{definition}{Definition}[section]
\theoremstyle{definition}
\newtheorem{corollary}{Corollary}[theorem]
\theoremstyle{definition}
\newtheorem{example}{Example}[section]
\title{How to Compute $\pi_4(S^3)$}
\author{David Zhu}

\begin{document}
\maketitle


In this exposition, we compute the first not-easily-computable homotopy group of spheres, $\pi_4(S^3)$. First we recall the definition for higher homotopy groups

\begin{tcolorbox}[colback=purple!5!white,colframe=purple!75!black]
\begin{definition}
For $n>0$, the  $n$th homotopy group of a pointed topological space $(X,x_0)$, denoted by $\pi_n(X,x_0)$, is the group of homotopy classes of maps from $(I^n,\partial I^n)\to (X,x_0)$. Equivalently, it is also the group of homotopy classes of maps from $(S^n,s_0)\to (X,x_0)$. 
\end{definition}
\end{tcolorbox}
Note that when $n=0$, the homotopy classes of maps no longer form a group, but it is still well-defined as a set. Based on this definition, one might be tempted to think that $\pi_n(S^k)$ is trivial when $n>k$. However, the Hopf fibration $S^1\hookrightarrow S^3\to S^2$ gives us a non-trivial element of $\pi_3(S^2)$. It is generally very hard to compute higher homotopy groups, even for spheres except for a certain number of cases. We now introduce some tools and preliminary results.

Without introducing any new tools, we can do at least one case: recall that a covering space $(\tilde{X},\tilde{x}_0)\to (X,x_0)$ satisfies the homotopy lifting property. In particular, the induced map $\pi_n(\tilde{X},\tilde{x}_0)\to \pi_n(X,x_0)$ is injective. On the other hand, the lifting criterion states that every map $S^n\to X$ can be lifted to $\Tilde{X}$ as $S^n$ is simply-connected for $n\geq 2$. Therefore, we have

\begin{tcolorbox}[colback=red!5!white,colframe=red!30!white]
\begin{theorem}
A covering space projection $\tilde{X}\to X$ induces isomorphism on nth homotopy groups for $n\geq 2$.
\end{theorem}
\end{tcolorbox}

\begin{tcolorbox}[colback=green!5!white,colframe=green!30!white]
\begin{corollary}
    $\pi_n(S^1)=0$ for $n>1$
\end{corollary}
\end{tcolorbox}
\begin{proof}
    Take $\mathbb{R}$ to be the universal cover for $S^1$, which has trivial homotopy groups since it is contractible.
\end{proof}

\section{Prelimary Results}

\begin{tcolorbox}[colback=red!5!white,colframe=red!30!white]
\begin{theorem}
(Cellular Approximation Theorem) Any map $f: X\to Y$ of CW-complexes is homotopic a cellular map, i.e the image of the $n$-skeleton of $X$ is contained in the $n$-skeleton of $Y$.
\end{theorem}
\end{tcolorbox}

\begin{tcolorbox}[colback=green!5!white,colframe=green!30!white]
\begin{corollary}
$\pi_n(S^k)=0$ for $k>n$.
\end{corollary}
\end{tcolorbox}
\begin{proof}
    If the image of the map $\phi: S^n\to S^k$ the image misses a point $s_0\in S^k$, then $S^k-\{s_0\}$ is homotopy equivalent to $\mathbb{R}^{k}$, and everything in $\mathbb{R}^{k}$ is contractible. Equip $S^n$ with the CW structure of $2$ $k$-cell in each dimension $k$. Then, every map $\phi: S^n\to S^k$ is homotopic to a cellular map that is not surjective.
\end{proof}

The next result is very important to our discussion. 

\begin{tcolorbox}[colback=red!5!white,colframe=red!30!white]
\begin{theorem}
(Hurewicz Theorem) A space $X$ is called \underline{\textbf{n-connected}} if $\pi_k(X)=0$ for all $0\leq k\leq n$. For $n\geq 2$, if $X$ is $n$-connected, then $\pi_n(X)\cong \tilde{H}_n(X)$.
\end{theorem}
\end{tcolorbox}
An immediate corollary of this result is that we can compute $\pi_n(S^n)$, which is generated by the degree map, as one might expect.
\begin{tcolorbox}[colback=green!5!white,colframe=green!30!white]
\begin{corollary}
$\pi_n(S^n)\cong \mathbb{Z}$
\end{corollary}
\end{tcolorbox}
\begin{proof}
    Combine the fact that $S^n$ is $n-1$-connected by Corollary $1.1.1$ and the fact that $H_n(S^n)=\mathbb{Z}$.
\end{proof}

Recall that a cofibration is a map $A\hookrightarrow X$ satisfying the homotopy extension property. Cofibration plays well with homology/cohomology as it gives us a long exact sequence, and we can then extract homological information from one space from the other. The dual notion is a fibration, which satisfies the homotopy lifiting property.

\begin{tcolorbox}[colback=purple!5!white,colframe=purple!75!black]
\begin{definition}
A map $E\to B$ is said to satisfy the \underline{\textbf{homotopy lifting property}} (HLP) with respect to a space $X$ if the following diagram commutes
\[
\begin{tikzcd}
X\times \{0\}\arrow[r,"\tilde{H}_0"]\arrow[d,"i"]&E\arrow[d]\\
X\times [0,1]\arrow[r,"H"]\arrow[ur,dashed,"\tilde{H}"]&B
\end{tikzcd}    
\]
In other words, given a homotopy in $H_t: X\to B$ and a initial lift $\tilde{H}_0: X\to E$, we can lift the homotopy entirely.
\end{definition}
\end{tcolorbox}

\begin{tcolorbox}[colback=purple!5!white,colframe=purple!75!black]
\begin{definition}
A map $p:E\to B$ satisfying the HLP with respect to arbitrary $X$ is called a (Hurewicz)  \underline{\textbf{fibration}}. A map satisfying the HLP with respect to CW-complexes is called a \underline{\textbf{Serre fibration}}. Assume $B$ is path-connected and based at $b_0$, the \underline{\textbf{fiber}} of the fibration is $F=p^{-1}(b_0)\subseteq E$.
\end{definition}
\end{tcolorbox}
Note that covering maps are fibrations with discrete fibers. In practice, Serre fibrations is good enough to give us most of the desired properties tools. It is fun to know that pathological examples exists (even in CGWH) such that a Serre Fibration is not a fibration. 

From now on we assume the base-space is path-connected.
\begin{tcolorbox}[colback=red!5!white,colframe=red!30!white]
\begin{theorem}
Given a Serre fibration $p:E\to B$ with fiber $F$ and a choice $x_0\in F$, we have the following LES:
\[
\begin{tikzcd}
...\arrow[r]&\pi_n(F,x_0)\arrow[r]&\pi_n(E,x_0)\arrow[r]&\pi_n(B,b_0)\arrow[r]&\pi_{n-1}(F,x_0)\arrow[r]&...
\end{tikzcd}
\]
\end{theorem}
\end{tcolorbox}
From this theorem, we can already calculated a not so obvious homotopy group

\begin{tcolorbox}[colback=green!5!white,colframe=green!30!white]
\begin{corollary}
$\pi_3(S^2)\cong \mathbb{Z}$
\end{corollary}
\end{tcolorbox}
\begin{proof}
    We have the exact sequence from the hopf fibration
    \[
    \begin{tikzcd}
    \pi_3(S^1)=0\arrow[r]&\pi_3(S^3)\cong \mathbb{Z}\arrow[r]&\pi_3(S^2)\arrow[r]&\pi_2(S^1)=0
    \end{tikzcd}
    \]
    where the triviality of the two groups on the end follows from Corollary $0.1.1$; $\pi_3(S^3)\cong \mathbb{Z}$ follows from Corollary $1.2.1$
\end{proof}


\end{document}