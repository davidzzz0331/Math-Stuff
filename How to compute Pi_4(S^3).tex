\documentclass{article}
\usepackage[utf8]{inputenc}
\usepackage{amsmath}
\usepackage{amsfonts}
\usepackage{amssymb}
\usepackage{tikz}
\usepackage{fullpage}
\usepackage{tikz-cd}
\usepackage{spectralsequences}
\usepackage{adjustbox}
\usepackage{xfrac}
\usepackage{tcolorbox}
\usepackage{xcolor}
\usepackage{graphicx}
\graphicspath{ {D:/Chrome Downloads./} }
\usepackage[parfill]{parskip}
\usepackage{amsthm}
\theoremstyle{definition}
\newtheorem{theorem}{Theorem}[section]
\theoremstyle{definition}
\newtheorem{problem}{problem}[section]
\theoremstyle{definition}
\newtheorem{proposition}{Proposition}[section]
\theoremstyle{definition}
\newtheorem{lemma}[theorem]{Lemma}
\theoremstyle{definition}
\newtheorem{definition}{Definition}[section]
\theoremstyle{definition}
\newtheorem{corollary}{Corollary}[theorem]
\theoremstyle{definition}
\newtheorem{example}{Example}[section]
\title{How to Compute $\pi_4(S^3)$}
\author{David Zhu}

\begin{document}
\maketitle


In this exposition, we compute the first not-easily-computable homotopy group of spheres, $\pi_4(S^3)$. First we recall the definition for higher homotopy groups

\begin{tcolorbox}[colback=purple!5!white,colframe=purple!75!black]
\begin{definition}
For $n>0$, the  $n$th homotopy group of a pointed topological space $(X,x_0)$, denoted by $\pi_n(X,x_0)$, is the group of homotopy classes of maps from $(I^n,\partial I^n)\to (X,x_0)$. Equivalently, it is also the group of homotopy classes of maps from $(S^n,s_0)\to (X,x_0)$. 
\end{definition}
\end{tcolorbox}
Note that when $n=0$, the homotopy classes of maps no longer form a group, but it is still well-defined as a set. Based on this definition, one might be tempted to think that $\pi_n(S^k)$ is trivial when $n>k$. However, the Hopf fibration $S^1\hookrightarrow S^3\to S^2$ gives us a non-trivial element of $\pi_3(S^2)$. It is generally very hard to compute higher homotopy groups, even for spheres except for a certain number of cases. We now introduce some tools and preliminary results.

Without introducing any new tools, we can do at least one case: recall that a covering space $(\tilde{X},\tilde{x}_0)\to (X,x_0)$ satisfies the homotopy lifting property. In particular, the induced map $\pi_n(\tilde{X},\tilde{x}_0)\to \pi_n(X,x_0)$ is injective. On the other hand, the lifting criterion states that every map $S^n\to X$ can be lifted to $\Tilde{X}$ as $S^n$ is simply-connected for $n\geq 2$. Therefore, we have

\begin{tcolorbox}[colback=red!5!white,colframe=red!30!white]
\begin{theorem}
A covering space projection $\tilde{X}\to X$ induces isomorphism on nth homotopy groups for $n\geq 2$.
\end{theorem}
\end{tcolorbox}

\begin{tcolorbox}[colback=green!5!white,colframe=green!30!white]
\begin{corollary}
    $\pi_n(S^1)=0$ for $n>1$
\end{corollary}
\end{tcolorbox}
\begin{proof}
    Take $\mathbb{R}$ to be the universal cover for $S^1$, which has trivial homotopy groups since it is contractible.
\end{proof}

\section{Prelimary Results}

\begin{tcolorbox}[colback=red!5!white,colframe=red!30!white]
\begin{theorem}
(Cellular Approximation Theorem) Any map $f: X\to Y$ of CW-complexes is homotopic a cellular map, i.e the image of the $n$-skeleton of $X$ is contained in the $n$-skeleton of $Y$.
\end{theorem}
\end{tcolorbox}

\begin{tcolorbox}[colback=green!5!white,colframe=green!30!white]
\begin{corollary}
$\pi_n(S^k)=0$ for $k>n$.
\end{corollary}
\end{tcolorbox}
\begin{proof}
    If the image of the map $\phi: S^n\to S^k$ the image misses a point $s_0\in S^k$, then $S^k-\{s_0\}$ is homotopy equivalent to $\mathbb{R}^{k}$, and everything in $\mathbb{R}^{k}$ is contractible. Equip $S^n$ with the CW structure of $2$ $k$-cell in each dimension $k$. Then, every map $\phi: S^n\to S^k$ is homotopic to a cellular map that is not surjective.
\end{proof}

The next result is very important to our discussion. 

\begin{tcolorbox}[colback=red!5!white,colframe=red!30!white]
\begin{theorem}
(Hurewicz Theorem) A space $X$ is called \underline{\textbf{n-connected}} if $\pi_k(X)=0$ for all $0\leq k\leq n$. For $n\geq 2$, if $X$ is $n$-connected, then $\pi_n(X)\cong \tilde{H}_n(X)$.
\end{theorem}
\end{tcolorbox}
An immediate corollary of this result is that we can compute $\pi_n(S^n)$, which is generated by the degree map, as one might expect.
\begin{tcolorbox}[colback=green!5!white,colframe=green!30!white]
\begin{corollary}
$\pi_n(S^n)\cong \mathbb{Z}$
\end{corollary}
\end{tcolorbox}
\begin{proof}
    Combine the fact that $S^n$ is $n-1$-connected by Corollary $1.1.1$ and the fact that $H_n(S^n)=\mathbb{Z}$.
\end{proof}

Recall that a cofibration is a map $A\hookrightarrow X$ satisfying the homotopy extension property. Cofibration plays well with homology/cohomology as it gives us a long exact sequence, and we can then extract homological information from one space from the other. The dual notion is a fibration, which satisfies the homotopy lifiting property.

\begin{tcolorbox}[colback=purple!5!white,colframe=purple!75!black]
\begin{definition}
A map $E\to B$ is said to satisfy the \underline{\textbf{homotopy lifting property}} (HLP) with respect to a space $X$ if the following diagram commutes
\[
\begin{tikzcd}
X\times \{0\}\arrow[r,"\tilde{H}_0"]\arrow[d,"i"]&E\arrow[d]\\
X\times [0,1]\arrow[r,"H"]\arrow[ur,dashed,"\tilde{H}"]&B
\end{tikzcd}    
\]
In other words, given a homotopy in $H_t: X\to B$ and a initial lift $\tilde{H}_0: X\to E$, we can lift the homotopy entirely.
\end{definition}
\end{tcolorbox}

\begin{tcolorbox}[colback=purple!5!white,colframe=purple!75!black]
\begin{definition}
A map $p:E\to B$ satisfying the HLP with respect to arbitrary $X$ is called a (Hurewicz)  \underline{\textbf{fibration}}. A map satisfying the HLP with respect to CW-complexes is called a \underline{\textbf{Serre fibration}}. Assume $B$ is path-connected and based at $b_0$, the \underline{\textbf{fiber}} of the fibration is $F=p^{-1}(b_0)\subseteq E$.
\end{definition}
\end{tcolorbox}
Note that covering maps are fibrations with discrete fibers. In practice, Serre fibrations is good enough to give us most of the desired properties tools. It is fun to know that pathological examples exists (even in CGWH) such that a Serre Fibration is not a fibration. 

From now on we assume the base-space is path-connected.
\begin{tcolorbox}[colback=red!5!white,colframe=red!30!white]
\begin{theorem}
Given a Serre fibration $p:E\to B$ with fiber $F$ and a choice $x_0\in F$, we have the following LES:
\[
\begin{tikzcd}
...\arrow[r]&\pi_n(F,x_0)\arrow[r]&\pi_n(E,x_0)\arrow[r]&\pi_n(B,b_0)\arrow[r]&\pi_{n-1}(F,x_0)\arrow[r]&...
\end{tikzcd}
\]
\end{theorem}
\end{tcolorbox}
From this theorem, we can already calculated a not so obvious homotopy group
\begin{tcolorbox}[colback=green!5!white,colframe=green!30!white]
\begin{corollary}
$\pi_3(S^2)\cong \mathbb{Z}$
\end{corollary}
\end{tcolorbox}
\begin{proof}
    We have the exact sequence from the hopf fibration
    \[
    \begin{tikzcd}
    \pi_3(S^1)=0\arrow[r]&\pi_3(S^3)\cong \mathbb{Z}\arrow[r]&\pi_3(S^2)\arrow[r]&\pi_2(S^1)=0
    \end{tikzcd}
    \]
    where the triviality of the two groups on the end follows from Corollary $0.1.1$; $\pi_3(S^3)\cong \mathbb{Z}$ follows from Corollary $1.2.1$
\end{proof}
 

\begin{tcolorbox}[colback=red!5!white,colframe=red!30!white]
\begin{theorem}
(Fiber replacement)Every map $f: X\to Y$ can be turned into a fibration in the following sense: there exists a space $E_f$ in 
\end{theorem}
\end{tcolorbox}

\begin{tcolorbox}[colback=red!5!white,colframe=red!30!white]
\begin{theorem}
(Puppe Sequence) Given a fibration $F\to E\to B$, we have the following sequence where any two consecutive maps form a fibration
\[
\begin{tikzcd}
    ...\arrow[r]&\Omega^2B\arrow[r]&\Omega F\arrow[r]&\Omega E\arrow[r]&\Omega B\arrow[r]&F\arrow[r]&E\arrow[r]&B
\end{tikzcd}
\]
where continuing to the left is applying the loop space functor.
\end{theorem}
\end{tcolorbox}





\section{Serre Spectral Sequence}
For the following discussions, we will use the assumption that $B$ is simply connected and work over $R=\mathbb{Z}$ to simplify things. 
\begin{tcolorbox}[colback=purple!5!white,colframe=purple!75!black]
\begin{definition}
Given a Serre fibration $F\hookrightarrow X\to B$, with fiber $F$ path-connected and base $B$ simply-connected. Then, the \underline{\textbf{Serre cohomological spectral sequence}} is given by the $E_2$ page
\[
E_2^{p,q}=H ^{p}(B; H ^{q}(F; \mathbb{Z})) \Longrightarrow H ^{p+q}(X; \mathbb{Z})
\]
\end{definition}
\end{tcolorbox}
If the space $B$ is not simply connected, then the coefficient group $H ^{q}(F)$ is actually the local system on $B$ given by the fibers. This reduces to the integral cohomology when the action of $\pi_1(B)$ on the fibers are trivial.


\begin{tcolorbox}[colback=red!5!white,colframe=red!30!white]
\begin{theorem}
(Product Structure) The Serre cohomological spectral sequence has a bigraded $\mathbb{Z}$-algebra structure, given by the product 
\[
E_n^{p,q}\times E_n^{s,t}\to E_n ^{p+s,q+t}
\]
In particular, the product structure on $E_2$ page is given by the cup product, and the product on $E_n$ induces the one on $E_{n+1}$.
\end{theorem}
\end{tcolorbox}


\begin{tcolorbox}[colback=blue!5!white,colframe=blue!30!white]
\begin{proposition}
    The differentials $d_n: E_n ^{p,q}\to E_n ^{p+n, q-n+1}$ is a graded derivation with respect to the product structure. In other words, given $a\in E_n ^{p,q}$ and $b\in E_n^{s,t}$, we have 
    \[
        d_n(ab)=d_n(a)b+(-1)^{|p+q|}ad_n(b) 
    \]
\end{proposition}
\end{tcolorbox}
We are ready to compute $\pi_4(S^3)$.

\section{Computations}

\begin{tcolorbox}[colback=red!5!white,colframe=red!30!white]
\begin{theorem}
    $\pi_4(S^3)\cong \mathbb{Z}/2$
\end{theorem}
\end{tcolorbox}
We know $\mathbb{Z}=H^3(S^3;\mathbb{Z})=[S^3,K(\mathbb{Z},3)]=\pi_3(K(\mathbb{Z},3))$. In particular, we may choose a map $f: S^3\to K(\mathbb{Z},3)$ representing the generator of the group. Note that by construction, $f_*: \pi_3(S^3)\to \pi_3(K(\mathbb{Z},3))$ is an isomorphism. Let $F_f$ be the homotopy fiber of $f$,

\begin{tcolorbox}[colback=blue!5!white,colframe=blue!30!white]
\begin{proposition}
$H_4(F_f)\cong \pi_4(S^3)$.
\end{proposition}
\end{tcolorbox}

\begin{proof}
We have the the long exact sequence 
\[ ...\pi_{n+1}(K(\mathbb{Z},3))\to\pi_n(F_f)\to \pi_n(S^3)\to \pi_n(K(\mathbb{Z},3))\to...\]
For $n=3$, we note $\pi_{4}(K(\mathbb{Z},3))$ is trivial, and $\pi_n(S^3)\to \pi_n(K(\mathbb{Z},3))$ is an isomorphism, so $\pi_3(F_f)$ must be trivial; similar argument shows $\pi_n(F_f)$ is trivial for $0<n\leq 3$ by corollary $1.1.1$ and the fact that $\pi_k(K(\mathbb{Z},3))\neq 0$ iff $k=3$. Thus, $F_f$ is $3$-connected. Note that the long exact sequence at degree $4$ also gives us the isomorphism $\pi_4(F_f)\cong \pi_4(S^3)$. Apply Hurewicz Theorem gives us the desired result.
\end{proof}

We may extend the fiber sequence one step to the left, which is the next step in the Pupper sequence $\Omega K(\mathbb{Z},3)=K(\mathbb{Z},2)\to F_f\to S^3$. Our goal now is to use the spectral sequence to calculate the cohomology of $F_f$ using the the cohomology of $S^3$ and $K(\mathbb{Z},2)$, which is realized as $\mathbb{CP}^{\infty}$. 

Recall that the cohomology ring of $\mathbb{CP}^{\infty}$ is $\mathbb{Z}[v]$, with $|v|=2$. The $E_2$ page of the Serre spectral sequence looks like the following 


\begin{center}
    \begin{sseqdata}[ name= 4, xscale = 0.6,  classes = {draw = none } ]
    \class["\mathbb{Z}1"](0,0)
    \class["0"](0,1)
    \class["\mathbb{Z}v_1"](0,2)
    \class["0"](1,0)
    \class["0"](1,1)
    \class["0"](1,2)
    \class["0"](2,0)
    \class["0"](2,1)
    \class["0"](2,2)
    \class["\mathbb{Z}k"](3,0)
    \class["0"](3,1)
    \class["\mathbb{Z}v_1k"](3,2)
    \end{sseqdata}[]
    \end{center}






\end{document}