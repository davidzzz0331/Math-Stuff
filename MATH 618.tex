\documentclass{article}
\usepackage[utf8]{inputenc}
\usepackage{amsmath}
\usepackage{amsfonts}
\usepackage{amssymb}
\usepackage{tikz}
\usepackage{fullpage}
\usepackage{tikz-cd}
\usepackage{spectralsequences}
\usepackage{adjustbox}
\usepackage[backend=biber, style=alphabetic]{biblatex}
\usepackage{xfrac}
\usepackage{tcolorbox}
\usepackage{xcolor}
\usepackage{graphicx}
\graphicspath{ {D:/Chrome Downloads./} }
\usepackage[parfill]{parskip}
\usepackage{amsthm}
\theoremstyle{definition}
\newtheorem{theorem}{Theorem}[section]
\theoremstyle{definition}
\newtheorem{definition}{Definition}[theorem]
\theoremstyle{definition}
\newtheorem{remark}{Remark}[theorem]
\theoremstyle{definition}
\newtheorem{proposition}{Proposition}[theorem]
\theoremstyle{definition}
\newtheorem{lemma}[theorem]{Lemma}
\theoremstyle{definition}
\newtheorem{corollary}{Corollary}[theorem]
\theoremstyle{definition}
\newtheorem{example}{Example}[theorem]
\title{MATH 618 Algebraic Topology}
\author{David Zhu}

\begin{document}
\maketitle

\section{The Correct Category}
Let $T=$ compactly generated weakly Hausdorff. Let $T_2=$ pairs of spaces $(X,A), A\subseteq X$, and 
\[T_2((X,A),(Y,B))=\{ f\in T(X,Y): f(A)\subseteq B\}\]
We define $(X,A)\otimes (Y,B)=(X\times Y, X\times B\cup Y\times A)$. (Think about product of boundaries). We want tot understand the analogue of $T(X\times Y, Z)\cong T(X,T(Y,Z))$.


\begin{tcolorbox}[colback=red!5!white,colframe=red!30!white]
\begin{theorem}
Let $(X,A),(Y,B),(Z,C)\in T_2$, then 
\[T_2((X,A)\otimes (Y,B),(Z,C))\cong T_2((X,A), T_2(T_2((Y,B),(Z,C))), T(Y,C))\]
\end{theorem}
\end{tcolorbox}

Let $T_*$ be the full subcategory of $T_2$ consisting of pairs $(X,*)$. There exists a pair of functors $T_2\to T_*$ defined by $(X,A)\mapsto (X/A, A/A=*)$. 


\begin{tcolorbox}[colback=blue!5!white,colframe=blue!30!white]
\begin{proposition}
$q: X\to X/A$, we get 
\[q*: T_*(X/A,Y)\to T_2((X,A),(Y,*))\]
an isomorphism.
\end{proposition}
\end{tcolorbox}

We want a producgt in $T_*$ which works well with function spaces: 

\begin{tcolorbox}[colback=purple!5!white,colframe=purple!75!black]
\begin{definition}
Given $X,Y\in T_*$, defined $X\wedge Y=(X\times Y/X\vee Y, *=X\vee Y)$ called the \underline{\textbf{smash product}}.
\end{definition}
\end{tcolorbox}
Note that the smash product is not the categorical product here.(The categorical product is simply the cartesian product carring the canonical basepoint). 


\begin{tcolorbox}[colback=purple!5!white,colframe=purple!75!black]
\begin{definition}
The \underline{\textbf{reduced suspension}} $\Sigma X:=S^1\wedge X$. 
\end{definition}
\end{tcolorbox}


\begin{tcolorbox}[colback=red!5!white,colframe=red!30!white]
\begin{theorem}
The catgegory of based spaces $T_*$ has the following properties 
\begin{enumerate}
    \item $T_*(X,Y)\in T_*$, with basedpoint the constant map to basepoint.
    \item $T_*(X,Y)\wedge X\to Y$ is continuous.
    \item $T_*(X,Y)\wedge T_*(Y,Z)\to T_*(X,Z)$ is continuous.
    \item $T_*(X\wedge Y,Z)\cong T_*(X,T_*(Y,Z))$
    \item Small limits and colimits exists in $T_*$.
    \item The forgetful functor $T_*\to T$ preserves limits. 
\end{enumerate}
\end{theorem}
\end{tcolorbox}



\begin{tcolorbox}[colback=purple!5!white,colframe=purple!75!black]
\begin{definition}
The \underline{\textbf{reduced cone}} is a functor $C: T_*\to T_*$
defined by $X\mapsto I\smash X$, with the basepoint of $I$ being $1$. 
\end{definition}
\end{tcolorbox}


\begin{tcolorbox}[colback=blue!5!white,colframe=blue!30!white]
\begin{proposition}
There exists a pushout 
\[\begin{tikzcd}
X\arrow[r]\arrow[d]&CX\arrow[d]\\
CX\arrow[r]&\Sigma X
\end{tikzcd}\]
\end{proposition}
\end{tcolorbox}


\begin{tcolorbox}[colback=purple!5!white,colframe=purple!75!black]
\begin{definition}
The \underline{\textbf{loop space}} is defined to be $T_*(S^1,X)$.
\end{definition}
\end{tcolorbox}


\begin{tcolorbox}[colback=red!5!white,colframe=red!30!white]
\begin{theorem}
(Eckmann-Hilton duality) 
\[T_*(X,\Omega Y)\cong T_*(\Sigma X,Y)\] 
\end{theorem}
\end{tcolorbox}

There exists a functor $T\to T_*$ given by $X\mapsto X\coprod \{x\}$, which is left adjoint to the forgetful functor. If $X\in T_*$, then $\Omega X\in T_*$, and note that $\pi_0(\Omega X)=[S^1,X]_*=\pi_1(X)$. 

\end{document}