\documentclass{article}
\usepackage[utf8]{inputenc}
\usepackage{amsmath}
\usepackage{amsfonts}
\usepackage{amssymb}
\usepackage{tikz}
\usepackage{fullpage}
\usepackage{tikz-cd}
\usepackage{spectralsequences}
\usepackage{adjustbox}
\usepackage[backend=biber, style=alphabetic]{biblatex}
\usepackage{xfrac}
\usepackage{tcolorbox}
\usepackage{xcolor}
\usepackage{graphicx}
\graphicspath{ {D:/Chrome Downloads./} }
\usepackage[parfill]{parskip}
\usepackage{amsthm}
\theoremstyle{definition}
\newtheorem{theorem}{Theorem}[section]
\theoremstyle{definition}
\newtheorem{definition}{Definition}[theorem]
\theoremstyle{definition}
\newtheorem{remark}{Remark}[theorem]
\theoremstyle{definition}
\newtheorem{proposition}{Proposition}[theorem]
\theoremstyle{definition}
\newtheorem{lemma}[theorem]{Lemma}
\theoremstyle{definition}
\newtheorem{corollary}{Corollary}[theorem]
\theoremstyle{definition}
\newtheorem{example}{Example}[theorem]
\title{MATH 618 Algebraic Topology}
\author{David Zhu}

\begin{document}
\maketitle

\section{The Correct Category}
Let $T=$ compactly generated weakly Hausdorff. Let $T_2=$ pairs of spaces $(X,A), A\subseteq X$, and 
\[T_2((X,A),(Y,B))=\{ f\in T(X,Y): f(A)\subseteq B\}\]
We define $(X,A)\otimes (Y,B)=(X\times Y, X\times B\cup Y\times A)$. (Think about product of boundaries). We want tot understand the analogue of $T(X\times Y, Z)\cong T(X,T(Y,Z))$.


\begin{tcolorbox}[colback=red!5!white,colframe=red!30!white]
\begin{theorem}
Let $(X,A),(Y,B),(Z,C)\in T_2$, then 
\[T_2((X,A)\otimes (Y,B),(Z,C))\cong T_2((X,A), T_2(T_2((Y,B),(Z,C))), T(Y,C))\]
\end{theorem}
\end{tcolorbox}

Let $T_*$ be the full subcategory of $T_2$ consisting of pairs $(X,*)$. There exists a pair of functors $T_2\to T_*$ defined by $(X,A)\mapsto (X/A, A/A=*)$. 


\begin{tcolorbox}[colback=blue!5!white,colframe=blue!30!white]
\begin{proposition}
$q: X\to X/A$, we get 
\[q*: T_*(X/A,Y)\to T_2((X,A),(Y,*))\]
an isomorphism.
\end{proposition}
\end{tcolorbox}

We want a producgt in $T_*$ which works well with function spaces: 

\begin{tcolorbox}[colback=purple!5!white,colframe=purple!75!black]
\begin{definition}
Given $X,Y\in T_*$, defined $X\wedge Y=(X\times Y/X\vee Y, *=X\vee Y)$ called the \underline{\textbf{smash product}}.
\end{definition}
\end{tcolorbox}
Note that the smash product is not the categorical product here.(The categorical product is simply the cartesian product carring the canonical basepoint). 


\begin{tcolorbox}[colback=purple!5!white,colframe=purple!75!black]
\begin{definition}
The \underline{\textbf{reduced suspension}} $\Sigma X:=S^1\wedge X$. 
\end{definition}
\end{tcolorbox}


\begin{tcolorbox}[colback=red!5!white,colframe=red!30!white]
\begin{theorem}
The catgegory of based spaces $T_*$ has the following properties 
\begin{enumerate}
    \item $T_*(X,Y)\in T_*$, with basedpoint the constant map to basepoint.
    \item $T_*(X,Y)\wedge X\to Y$ is continuous.
    \item $T_*(X,Y)\wedge T_*(Y,Z)\to T_*(X,Z)$ is continuous.
    \item $T_*(X\wedge Y,Z)\cong T_*(X,T_*(Y,Z))$
    \item Small limits and colimits exists in $T_*$.
    \item The forgetful functor $T_*\to T$ preserves limits. 
\end{enumerate}
\end{theorem}
\end{tcolorbox}



\begin{tcolorbox}[colback=purple!5!white,colframe=purple!75!black]
\begin{definition}
The \underline{\textbf{reduced cone}} is a functor $C: T_*\to T_*$
defined by $X\mapsto I\smash X$, with the basepoint of $I$ being $1$. 
\end{definition}
\end{tcolorbox}


\begin{tcolorbox}[colback=blue!5!white,colframe=blue!30!white]
\begin{proposition}
There exists a pushout 
\[\begin{tikzcd}
X\arrow[r]\arrow[d]&CX\arrow[d]\\
CX\arrow[r]&\Sigma X
\end{tikzcd}\]
\end{proposition}
\end{tcolorbox}


\begin{tcolorbox}[colback=purple!5!white,colframe=purple!75!black]
\begin{definition}
The \underline{\textbf{loop space}} is defined to be $T_*(S^1,X)$.
\end{definition}
\end{tcolorbox}


\begin{tcolorbox}[colback=red!5!white,colframe=red!30!white]
\begin{theorem}
(Eckmann-Hilton duality) 
\[T_*(X,\Omega Y)\cong T_*(\Sigma X,Y)\] 
\end{theorem}
\end{tcolorbox}

There exists a functor $T\to T_*$ given by $X\mapsto X\coprod \{x\}$, which is left adjoint to the forgetful functor. If $X\in T_*$, then $\Omega X\in T_*$, and note that $\pi_0(\Omega X)=[S^1,X]_*=\pi_1(X)$. 


\begin{tcolorbox}[colback=purple!5!white,colframe=purple!75!black]
\begin{definition}
A functor $F: T_*\to C$ for some catgeory $C$ is called a \underline{\textbf{homotopy functor}} if $F(f)=F(g)$ when $f\cong g$
\end{definition}
\end{tcolorbox}



\begin{tcolorbox}[colback=purple!5!white,colframe=purple!75!black]
    \begin{definition}
    Define $\pi_n(X):= [S^n, X]_*$. Then, $\pi_n$ is a homotopy functor from $T_*$ to set.
    \end{definition}
    \end{tcolorbox}
    
    
    \begin{tcolorbox}[colback=blue!5!white,colframe=blue!30!white]
    \begin{proposition}
    $\pi_n$ is a group for $n\geq 1$ and abelian when $n\geq 2$. 
    \end{proposition}
    \end{tcolorbox}

    \begin{proof}
        The group structure is given by: suppose we have $\varphi,\psi: I^n\to X$. Then, $\varphi+\psi: I^n\to X$ is explicitly given by 
        \[\varphi+\psi(t_1,...,t_n)=
        \begin{cases}
            \varphi(2t_1,t_2,...,t_n)& t\in [0,\frac{1}{2}]\\
            \psi(2t_1-1,t_2,...,t_n)&t\in  [\frac{1}{2},1]\\
        \end{cases}
        \]
    \end{proof}

    
    \begin{tcolorbox}[colback=purple!5!white,colframe=purple!75!black]
    \begin{definition}
    Define $H$-space be a topological space $X$ with homotopy associative map 
    \[X\wedge X\to X\]
    \end{definition}
    \end{tcolorbox}

    
    \begin{tcolorbox}[colback=blue!5!white,colframe=blue!30!white]
    \begin{proposition}
    If $(Y,y_0)$ is an $H$-space, then $[X,Y]_*$ has a group structure. 
    \end{proposition}
    \end{tcolorbox}









\begin{tcolorbox}
\begin{lemma}
Let $C=(C^{*,*}, d,\delta)$, be a double complex, $(C^{*},d)$ a complex with $\epsilon: C^{*}\to C^{*,0}$ such that $d\epsilon+\epsilon d=0$. Furthermore, assume that the columns are exact(in particular, this implies that $H^q(C^{p,*}, \delta)\cong C^p$ for $q=0$). Then, $\epsilon: H^p(C,d)\to H^p(\textrm{Tot}(C),D)$ is an isomorphism.
\end{lemma}
\end{tcolorbox}
\begin{proof}
    standard augmentation/resolution lemma. Mattie you can do this. 
\end{proof}

Let $M$ be a smooth manifold. Suppose $M$ has a good cover $U=\{U_\alpha\}$ such that each intersection $U_{\alpha_1,...,\alpha_n}$ is either empty or contractible. (Note that you can always do this for a Riemannian manifold). We define a double complex as follows: Let 
\[C^{p,q}=\prod_{\alpha_0,...,\alpha_q}\Omega^q(U_{\alpha_0,...,\alpha_q})\]
The horizontal differentials $d: C^{p,q}\to C^{p+1,q}$ is $d(\omega\in \Omega^p(u_{\alpha_0,...,\alpha_q}))=d\omega\in \Omega^{p+1}(U_{\alpha_0,...,\alpha_q})$, which is just the De Rham differentials.

The vertical differential $\delta: C^{p,q}\to C^{p,q+1}$ is defined to by: let $\omega\in \prod_{\alpha_0,...,\alpha_q}\Omega^p(U_{\alpha_0,...,\alpha_q})$
\[\delta(\omega)_{\alpha_0,...,\alpha_{q+1}}=\sum_{j=0}^{q+1}(-1)^j\omega_{\alpha_0,...,\widehat{\alpha_j},...,\alpha_{q+1}}|_{U_{\alpha_0,...,\widehat{\alpha_j},...,\alpha_{q+1}}}\]
which is the Cech differential.

It is easy to check that $\delta^2=0$ and $d^2=0$ and $d\delta=\delta d$.

We have an agumentation of the double complex by the De Rham complex, and apply the previous lemma. The trick to checking the columns are exact is to let $\lambda_{\alpha}$ be a partition of unity for $U_{\alpha}$, and define a chain contraction $s: C^{p,q}\to C^{p,q-1}$ defined by $(s\omega)_{\alpha_0,...,\alpha_{q-1}}=\sum\lambda_{\alpha_0}\omega_{\alpha_0,...,\alpha_{q-1}}$. Thus, we get that $H^p(\Omega M)\cong H^p(\textrm{Tot}(C), D)$. 

Similarly, there is an augmentation of the Cech comomplex $C^{\vee,q}(U)=\prod_{\alpha_0,...,\alpha_q} \mathbb{R}$, with $\delta: C^{\vee q}\to C^{\vee q+1}$ defined by $\delta(c)_{\alpha_0,...,\alpha_{q+1}}:=\sum(-1)^jc_{\alpha_0,...,\widehat{\alpha_j},...,\alpha_{q+1}}$. We may check that the rows are exact directly this time, and that the cohomology of the complex is isomorphic to the cech cohomology $H^*(C^{\vee,*},\delta)$ by the row analog of the lemma. In the end, we have 


\begin{tcolorbox}[colback=red!5!white,colframe=red!30!white]
\begin{theorem}[De Rham Theorem]
\[H^{*}_{DR}(M)\cong H^*(\Omega^*M)\cong H^{\vee,*}(M)\]
\end{theorem}
\end{tcolorbox}



\begin{tcolorbox}[colback=purple!5!white,colframe=purple!75!black]
\begin{definition}
A \underline{\textbf{spectral sequence}} is a sequence of abelian groups and differentials $(A_n,d_n)$, such that $d_n: A_n\to A_n$, such that $A_n\cong H^*(A_n,d_n)$ (Block's notation $A_n$ is the page). A morphism of spectral sequences is a sequence of map $A_n\to E_n$ that commutes with all differentials and turning the page.
\end{definition}
\end{tcolorbox}


\begin{tcolorbox}[colback=red!5!white,colframe=red!30!white]
\begin{theorem}[Serre Spectral Sequence]
Let $\pi: Y\to X$ be a Serre fibration, with fiber $F$. Then, there exists a spectral sequence $(E_r^{p,q},d_r)$, with the differentials going the way 
\[d_r: E_r^{p,q}\to E_r^{p+r,q-r+1}\]
If $X$ is simply connected, then the $E_2$ page is given by 
\[E_2^{p,q}=H^{p}(X; H^q(F))\Rightarrow H^{p+q}(Y)\]
\end{theorem}
\end{tcolorbox}


\begin{tcolorbox}[colback=yellow!5!white,colframe=yellow!30!white]
\begin{example}
Consider the fibration $S^1\to S^{2n+1}\to \mathbb{CP}^n$. Then, $E_2^{p,q}=H^p(\mathbb{CP}^n,H^q(S^1))$. You can draw the $E_2$ page of the spectral sequence, and find that the non-trivial terms concentrate on row $0,1$ by our knowledge of the cohomology of $S^1$. Moreover, we know the cohomology of $\mathbb{CP}^n$ is only in even degrees, so we only get non-zero terms in the even columns. Combining the information, the $E_2$ page is pretty clear to see. You can work out when the spectral sequence collapses and how it converges.  
\end{example}
\end{tcolorbox}







\end{document}

