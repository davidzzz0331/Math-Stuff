\documentclass{article}
\usepackage[utf8]{inputenc}
\usepackage{amsmath}
\usepackage{amsfonts}
\usepackage{amssymb}
\usepackage{tikz}
\usepackage{fullpage}
\usepackage{tikz-cd}
\usepackage{spectralsequences}
\usepackage{adjustbox}
\usepackage{xfrac}
\usepackage{graphicx}
\usepackage[parfill]{parskip}
\usepackage{amsthm}
\newtheorem{theorem}{Theorem}[section]
\newtheorem{problem}{problem}[section]
\newtheorem{proposition}{Proposition}[section]
\newtheorem{lemma}[theorem]{Lemma}
\newtheorem{definition}{Definition}[section]
\newtheorem{corollary}{Corollary}[theorem]
\newtheorem{example}{Example}[section]
\title{MATH 603 Notes}
\author{David Zhu}

\begin{document}
\maketitle

\section{More on Commutative Rings}
Let $a,b\in R$. Then $a|b \iff \exists a'\in R, b=aa'$;
A semi ring on $R,\leq$ defined by $a\leq b \iff a|b$. Note that $\leq $ is usally not an ordering: let $b\in R^{\times}$, then $a\leq ab\leq a$, but $a\neq ab$.
\begin{proposition}
$a\equiv b$ iff $a\leq b$ and $b\leq a$ iff $(a)=(b)$ is an equivalence relaation.
\end{proposition}
For $R$ a domain, the induced relation gives a well-defined definition of greatest common divisor.
\begin{definition}
    The gcd of $a,b$, denoted by $gcd(a,b)$, if exists, is any $d\in R$ such that $d|a,b$ and for any other $d'$ satisfying the condition, $d'|d$.
\end{definition}
\begin{definition}
    The lcm of $a,b$, denoted by $lcm(a,b)$, if exists, is any  $d\in R$ such that $a,b|d$ and for any other $d'$ satisfying the condition, $d'|d$.
\end{definition}
Note that either may not exist, for example take $R=\mathbb Z[\sqrt{6}]$, then $gcd(2,\sqrt{6})$ does not exist. 

\begin{proposition}
    If $gcd(a,b)$ exists, then $gcd(a,b)=sup \{ d:d\leq a,b \}$. 
\end{proposition}
The dual notion for lcm also holds. Note that maximal/minimal elements always exists by Zorn's lemma. However, the unique supremum/infimum may not exist.  

\begin{proposition}
    Let $a,b\in R$ be given. Then the following hold: $gcd(a,b)=d$ exists iff $(d)$ is the unique maximal prinipal ideal such that $(a)+(b)\subset (d)$. Dually, $lcm(a,b)=c$ exists iff $(c)=(a)\cap (b)$. If both holds, then $a\cdot b=lcm(a,b)\cdot gcd(a,b)$
\end{proposition}
\begin{proof}
    Easy exercise for gcd. Note that the inclusion can be proper, for example, take $R=k[x,y]$ and ideals $(x),(y)$. Then $(1)$ is the gcd, but the containment is proper.
\end{proof}
Recall that $Id(R)$ is partially ordered by inclusion. 
\begin{definition}
    $Id(R), +,\cap,\cdot,\leq$ is the lattice of ideals of $R$.
\end{definition}
Note that $+,,\cap$ are simply the sums and intersection, but $\cdot$ is the ideal generated by the products. 

\begin{theorem}
TFAE for non-finitely generated ideals of $R$, which we denote $Id^{\infty}(R)$: $1.$ $Id^{\infty}(R)$ is non-empty; $2.$ there exists infinite non-stationary chains $(\sigma_i)$, where $\sigma_i\in Id(R)$;
\end{theorem}
\begin{proof}
    Easy exercise.
\end{proof}

\begin{theorem}
Cohen's lemma: Let $Id^{\infty}(R)\neq \emptyset$. Then, it has a maximal element and any such maximal element is prime. 
\end{theorem}
\begin{proof}
    Zorn's lemma implies $Id^{\infty}(R)$ has maximal elements. Let $a$ be maximal, and $xy\in a$. Suppose by contradiction that $x,y\not \in a$, then $I_1=a\subset (x)+a$ and $I_2=a\subset (y)+a$, which contradicts maximality by proving one of them must be infinitely generated. Consider $(a:x)=\{ \gamma\in R:\gamma\cdot x\in a \}$. Note $a,y\in (a:x)$, and $x\cdot(a;x)\subseteq a$. Hence $(a:x)\not \in Id^{\infty}(R)$ and $(a:x)$ is finitely generated. Thus, $a=I_0+(a:x)$ must be finitely generated. 
\end{proof}

\section{Euclidean Rings}
\begin{definition}
    A Principal Ideal Ring is any ring $R$ such that $Id(R)=Id^p(R)$. If $R$ is a domain, then $R$ is called a PID. 
\end{definition}
\begin{definition}
    A Factorial Ring is any ring $R$ in which all units can be written as a finite product of irreducible elements. If $R$ is domain, then it is called a UFD. (Note that if it is not a domain, weird things can happen)
\end{definition}
\begin{definition}
    A Noetherian Ring is any ring $R$ such that any ideal is finitely generated. 
\end{definition}



\begin{definition}
    Let $R$ be a domain. A Euclidean norm on $R$ is any map $\phi: R\to \mathbb N$ satisfying $\phi(x)=0 $ iff $x=0$ and for every $a,b\in R$ with $b\neq 0$, then there exists $q,r\in R$ such that $a=bq+r$ with $\phi(r)<\phi(b) $. A Euclidean domain is any domain equipped with a Euclidean norm.
\end{definition}
Example of Euclidean domains include $\mathbb Z,\mathbb Z[i]$. A non-trivial example $R=F[t]$, with $\phi(p(t))=2^{deg(p(t))}$. A non-example is $\mathbb Z[\sqrt{6}]$ for it is not a PID.

\begin{theorem}
    Eucldiean Domains are PIDs; The Euclidean Algorithm: $a,b\in R$, $b\neq 0$ and set $r_0=a,r_1=b$, and continue inductively $r_{i-1}=r_i\cdot q_i+r_{i+1}$. Then, $r_i=0$ for $i>\phi(b)$ and if $r_{i_0}\geq 1$ maximal with $r_{i_0}\neq 0$, then $r_{i_0}=gcd(a,b)$.
\end{theorem}
\begin{proof}
    Easy exercise.
\end{proof}

\section{Principal Ideal Domains}
\begin{theorem}
    (Charaterization) For A domain $R$ TFAE: $1.$ $R$ is a PID; $2.$ every $a\in R^{\times}, a\neq 0$ is a product of finitely many prime elements up unique up to permutation. $3.$ every $p\in Spec(R)$ is principal. 
\end{theorem}

\begin{proof}
    Let $a\in R$ such that $a$ is non-zero and not a unit. Then, there exists $p\in Spec(R)$ such that $(a)\subseteq p$. Hence $R$ being a PID implies $\exists \pi\in R $ such that $p=(\pi)$. Hence, $\pi$ must be prime and $\pi|a$. Set $a_1=a$, $\pi_1=\pi$, and let $a_2$ be the element such that $\pi_1a_2=a_1$. Continue inductively such that if $a_n$ is a unit, stop; otherwise repeat. Suppose by contradiction that the process does not stablize.

    Assuming that every prime is principal, Cohen's Lemma implies $Id^{\infty}(R)\neq \emptyset$; therefore, every ideal is finitely generated. We therefore can choose a minimal prime over a given finitely genearted ideal and build a chain of ideals whose union is prime and contradiction. 
\end{proof}
\begin{corollary}
    Let $R$ be a PID; let $P\subset R$ be a set of representatives for the prime elements up to association. For every $a\in R$, $\exists \epsilon\in R^{\times} $ and $e_{\pi}\in \mathbb{N} $ such that almost all $e_{\pi}=0$. Then, every $a\in R$ can be written as $a=\epsilon\prod_{\pi\in P}\pi^{e_{\pi}}$. We proceed to recover $gcd$ and $lcm$, up to associates. 
\end{corollary}
Note that the above corollary generalizes to the quotient field by replacing $\mathbb{N}$ with $\mathbb{Z}$.

\section{Unique Factorization Domains}
\begin{definition}
    A Unique Factorization Domain is a domain in which every non-zero, non-units is a product of prime elements. 
\end{definition}
\begin{proposition}
    TFAE: $(1.)$ $R$ is a UFD; $(2).$ every minimal prime ideal is principal and every non-zero, non-invertible elements in contained in finitely many primes. 
\end{proposition}
\begin{proof}
    Exercise. 
\end{proof}
Remark: we recover the $gcd$ and $lcm$ definition using the same factorization as Corollary $3.1.1$.



\begin{theorem}
    (Gauss Lemma)Let $R$ be a UFD; then $R[t]$ is a UFD. 
\end{theorem}
\begin{proof}
    Let $f(t)=a_0+...+a_nt^n$ be given. Then, the content of $f$, denoted $C(f)$, is the GCD of all coefficients. In particular, $C(f)|a_i$ for all $i$, and $f_0:=f/(C(f))$ has content $1$. 
\begin{lemma}
    Let $R$ be a UFD, then the following hold: $(1).$ $C(f): R[t]\to R$ given by $f\mapsto C(f)$ is multiplicative; in particular, if $C(f)=C(g)=1$, then $C(fg)=1$. 
\end{lemma}
proof of the lemma: given $f(t)=a_0+...+a_nt^n$ and $g(t)=b_0+b_mt^m$. If one of $f$, $g$ is constant, then it is easy exercise; suppose neither is constant, then set $f=f_0\cdot C(f)$ and $g=g_0\cdot C(g)$. Clearly we have $C(f)\cdot C(g)| C(fg)$. Hence it suffices to prove that $C(f_0g_0)=1$. Equivalently, let $\pi\in R$ be a prime element, then there exists a coefficient $c_k\in f_0g_0$ such that $\pi$ does not divide $c_k$. Suppose $\pi| C_k=\sum_{i+j=k}a_ib_j$ for all $k$. Then, $\pi|a_0b_0$ and WLOG, $\pi|a_0$. Because $C(f_0)=C(g_0)=1$, then there exists minimal $a_i,b_j$ such that $\pi$ does not divide $a_{i_0},b_{j_0}$. Then, $\pi$ does not divede $C_{i_0+j_0}$.

\end{proof}
The proof goes similarly for quotient fields. 
\begin{theorem}
    For $f(t)\in R[t]$, TFAE
    $1.$ $f(t)$ is prime
    $2.$ is irreducible
    $3$. If $f=a_0\in R$ and $a_0$ is prime or $C(f)=1$ and $f$ is irreducible. 
\end{theorem}


\end{document}
