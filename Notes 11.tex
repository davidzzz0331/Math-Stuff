\documentclass{article}
\usepackage[utf8]{inputenc}
\usepackage{amsmath}
\usepackage{amsfonts}
\usepackage{amssymb}
\usepackage{tikz}
\usepackage{fullpage}
\usepackage{tikz-cd}
\usepackage{spectralsequences}
\usepackage{adjustbox}
\usepackage{tcolorbox}
\usepackage{xfrac}
\usepackage{graphicx}
\usepackage[parfill]{parskip}
\usepackage{amsthm}
\theoremstyle{definition}
\newtheorem{theorem}{Theorem}[section]
\theoremstyle{definition}
\newtheorem{problem}{problem}[section]
\theoremstyle{definition}
\newtheorem{proposition}{Proposition}[section]
\theoremstyle{definition}
\newtheorem{lemma}[theorem]{Lemma}
\theoremstyle{definition}
\newtheorem{definition}{Definition}[section]
\theoremstyle{definition}
\newtheorem{corollary}{Corollary}[theorem]
\theoremstyle{definition}
\newtheorem{example}{Example}[section]

\title{MATH 603 Notes}
\author{David Zhu}

\begin{document}
\maketitle

\section{More on Commutative Rings}
Let $a,b\in R$. Then $a|b \iff \exists a'\in R, b=aa'$;
A semi ring on $(R,\leq)$ defined by $a\leq b \iff a|b$. Note that $\leq $ is usally not a partial order: let $b\in R^{\times}$, then $a\leq ab\leq a$, but $a\neq ab$.


\begin{tcolorbox}[colback=blue!5!white,colframe=blue!30!white]
\begin{proposition}
    $a\sim b$ iff $a\leq b$ and $b\leq a$ iff $(a)=(b)$ is an equivalence relation.
\end{proposition}
\end{tcolorbox}

For $R$ a domain, the induced relation gives a well-defined definition of greatest common divisor.
\begin{tcolorbox}[colback=red!5!white,colframe=red!30!white]
    \begin{definition}
        The \underline{\textbf{gcd}} of $a,b$, denoted by $gcd(a,b)$, if exists, is any $d\in R$ such that $d|a,b$ and for any other $d'$ satisfying the condition, $d'|d$.
    \end{definition}    
\end{tcolorbox}


\begin{tcolorbox}[colback=purple!5!white,colframe=purple!75!black]
\begin{definition}
    The \underline{\textbf{lcm}} of $a,b$, denoted by $lcm(a,b)$, if exists, is any  $d\in R$ such that $a,b|d$ and for any other $d'$ satisfying the condition, $d|d'$.
\end{definition}
\end{tcolorbox}


\begin{tcolorbox}[colback=blue!5!white,colframe=blue!30!white]
\begin{proposition}
    If $gcd(a,b)$ exists, then $gcd(a,b)=sup \{ d:d\leq a,b \}$.  If $lcm(a,b)$ exists, then $lcm(a,b)=\textrm{inf} \{ d :a,b\leq d \}$. 
\end{proposition}
\end{tcolorbox}

Note that maximal/minimal elements always exists by Zorn's lemma. However, the unique supremum/ \ infimum may not exist. We have our following example:

\begin{tcolorbox}[colback=yellow!5!white,colframe=yellow!30!white]
\begin{example}
Take $R=[\sqrt{-3}]$. Let $a=4=(1+\sqrt{-3})(1-\sqrt{-3})$ and $b=2(1+\sqrt{-3})$. Then, $(1+\sqrt{-3})$ and $2$ are both maximal divisors, but they are not comparable since the only divisors of $2$ are $\{ \pm 1, \pm 2 \}$ by norm reasons, and none divides $1+\sqrt{-3}$.
\end{example}
\end{tcolorbox}


\begin{tcolorbox}[colback=blue!5!white,colframe=blue!30!white]
\begin{proposition}
    Let $a,b\in R$ be given. Then the following hold: $gcd(a,b)=d$ exists iff $(d)$ is the unique maximal prinipal ideal such that $(a)+(b)\subseteq (d)$. Dually, $lcm(a,b)=c$ exists iff $(c)=(a)\cap (b)$. If both holds, then $a\cdot b=lcm(a,b)\cdot gcd(a,b)$
\end{proposition}
\end{tcolorbox}

\begin{proof}
    Easy exercise. Note that the inclusion can be proper, for example, take $R=k[x,y]$ and ideals $(x),(y)$. Then $(1)$ is the gcd, but the containment is proper.
\end{proof}
Recall that $Id(R)$ is partially ordered by inclusion. 


\begin{tcolorbox}[colback=purple!5!white,colframe=purple!75!black]
\begin{definition}
    ($Id(R), +,\cap,\cdot,\leq$) is the lattice of ideals of $R$.
\end{definition}
\end{tcolorbox}

Note that $+,,\cap$ are simply the sums and intersection, but $\cdot$ is the ideal generated by the products, i.e the set of finite sums of products. 


\begin{tcolorbox}[colback=red!5!white,colframe=red!30!white]
\begin{theorem}
    Let $Id^{\infty}(R)$ be the set of non-finitely generated ideals for $R$; the following are equivalent:
\begin{enumerate}
    \item $Id^{\infty}(R)$ is non-empty; 
    \item  There exists an infinite non-stationary chain of ideals $(\sigma_i)$, where $\sigma_i\in Id(R)$;
\end{enumerate}
\end{theorem}
\end{tcolorbox}


\begin{proof}
    For $1\implies 2$, let $I$ be a non-finitely generated ideal of $R$ and pick $x_1\in \in I$. Let $\sigma_1=(x_1)$. Because the ideal is non-finitely generated, we can pick $x_2\in I$ such that $x_2\not \in \sigma_1$. Let $\sigma_2=(x_1,x_2)$. Continue inductively gives us an infinite non-stationary chain of ideals. 

    For $2\implies 1$, take the union of all the ideals in the infinite non-stationary chain. It is an ideal and it cannot be finitely generated. 
\end{proof}


\begin{tcolorbox}[colback=red!5!white,colframe=red!30!white]
\begin{theorem}
    (Cohen's lemma): Let $Id^{\infty}(R)\neq \emptyset$. Then, it has a maximal element and any such maximal element is prime. 
\end{theorem}
\end{tcolorbox}
Before proving Cohen's lemma, we need the following technical lemma:\

\begin{tcolorbox}
\begin{lemma}
    Let $I$ be an ideal. Define $(I:a):=\{ b\in R: ab\in I \}$. If $I+(x)$ and $(I:x)$ are both finitely generated, then $I$ is finitely generated.
\end{lemma}
\end{tcolorbox}
\begin{proof}[Proof of Lemma 1.3]
    By assumption, there is finite set $\{ \alpha_i:=a_i+f_ix: a_i\in I, f_i\in R, i=1,...,k \}$ that generate $I+(x)$, and a finite set $\{ b_j: j=1,...,l \}$ that generate $(I:x)$. We claim that the set $\{ a_i, b_jx: i\in I, j\in J\}$ generate the entire $I$: since $I\subseteq I+(x)$, we can write any element $\pi\in I$ as a finite linear combination $\pi=\sum_{i=1}^{k}g_i\alpha_i=\sum_{i=1}^{k}g_i(a_i+f_ix)$, where $g_i\in R$. We note that $\pi-\sum_{i=1}^{k}g_ia_i=\sum_{i=1}^{k}g_if_ix$ is in $I$; it follows that $\sum_{i=1}^{k}g_if_i\in (I:x)$, so $\sum_{i=1}^{k}g_if_ix$ is generated by the set $ \{b_jx\}$, and we are done. 
\end{proof}

With the lemma in hand, now we can prove Theorem $1.2$
\begin{proof}[Proof of Theorem $1.2$]
    Zorn's lemma implies $Id^{\infty}(R)$ has maximal elements.  Let $I$ one such maximal element, and suppose it is not prime. Then, there exists $xy\in I$ and WLOG suppose $x\not \in I$. By maximality, $I+(x)$ must be finitely generated. By definition, we have $y\in (I:x)$. Lemma $1.3$ implies $(I:x)$ is not finitely generated, and in particular, $I\subseteq (I:x)$. Applying maximality again, we have $I=(I:x)$, which forces $y\in I$, a contradiction.
\end{proof}

\section{Euclidean Rings}


\begin{tcolorbox}[colback=purple!5!white,colframe=purple!75!black]
\begin{definition}
    A \underline{\textbf{Principal Ideal Ring}} is any ring $R$ i which every ideal is principally generated. If $R$ is a domain, then $R$ is called a \underline{\textbf{PID}}. 
\end{definition}
\end{tcolorbox}


\begin{tcolorbox}[colback=purple!5!white,colframe=purple!75!black]
\begin{definition}
    A \underline{\textbf{Factorial Ring}}  is any ring $R$ in which all units can be written as a finite product of irreducible elements, unique up to a unit. If $R$ is domain, then it is called a \underline{\textbf{UFD}}. 
\end{definition}
\end{tcolorbox}
Note that if the ring $R$ it is not a domain, $x|y$ and $y|x$ does not imply $x=uy$ for some unit $u$. Let us prove that this holds for a domain: suppose $x=ys$ and $y=xt$, and $x,y\neq 0$ then $x=xts$, which implies $x(1-ts)=0$. This forces $1-ts=0$, and $t,s$ are then units. We can concoct counterexamples when $R$ is not a domain accordingly: let $R=k[x]/(x^3-x)$ and take $a=x$, $b=x^2$. Clearly, $a|b$ and $b=x^2\cdot x=x^3$, so $b|a$. However, $x$ is not a unit. 



\begin{tcolorbox}[colback=purple!5!white,colframe=purple!75!black]
\begin{definition}
    A \underline{\textbf{Noetherian Ring}} is any ring $R$ such that any ideal is finitely generated. 
\end{definition}
\end{tcolorbox}


\begin{tcolorbox}[colback=purple!5!white,colframe=purple!75!black]
\begin{definition}
    Let $R$ be a domain. A \underline{\textbf{Euclidean norm}} on $R$ is any map $\phi: R\to \mathbb N$ satisfying $\phi(x)=0 $ iff $x=0$ and for every $a,b\in R$ with $b\neq 0$, then there exists $q,r\in R$ such that $a=bq+r$ with $\phi(r)<\phi(b) $. A \underline{\textbf{Euclidean Domain}} is any domain equipped with a Euclidean norm.
\end{definition}
\end{tcolorbox}

Example of Euclidean domains include $\mathbb Z,\mathbb Z[i]$. A non-trivial example $R=F[t]$, with $\phi(p(t))=2^{deg(p(t))}$. A non-example is $\mathbb Z[\sqrt{6}]$ for it is not a PID.



\begin{tcolorbox}[colback=blue!5!white,colframe=blue!30!white]
\begin{proposition}
    Eucldiean Domains are PIDs.
\end{proposition}
\end{tcolorbox}
\begin{proof}
    By the well-ordering principal, for every ideal $I$ in a Euclidean domain, there exists an element other than $0$ of the smallest norm. It is easy exercise that such element generate the entire ideal.
\end{proof}


\begin{tcolorbox}[colback=blue!5!white,colframe=blue!30!white]
\begin{proposition}
    (The Euclidean Algorithm): Given $a,b\in R$, $b\neq 0$. Set $r_0=a,r_1=b$, and continue inductively $r_{i-1}=r_i\cdot q_i+r_{i+1}$. Then, $r_i=0$ for $i>\phi(b)$ and if $r_{i_0}\geq 1$ maximal with $r_{i_0}\neq 0$, then $r_{i_0}=gcd(a,b)$.
\end{proposition}
\end{tcolorbox}
\begin{proof}
    Note that the remainder is strictly decreasing, so $r_i$ must become $0$ after $\phi(b)$ steps. Note that once $r_{i+1}=0$, we have $r_i|r_{n}$ for all $n\leq i$. Coversely, it is clear that any divisor of $a,b$ divdes all $r_n$ for $n\leq i$.
\end{proof}


\section{Principal Ideal Domains}

\begin{tcolorbox}[colback=red!5!white,colframe=red!30!white]
\begin{theorem}
    (Charaterization) For A domain $R$, the following are equivalent: 
\begin{enumerate}
    \item  $R$ is a PID.
    \item every $p\in Spec(R)$ is principal. 
\end{enumerate}
\end{theorem}
\end{tcolorbox}
\begin{proof}
    One direction is trivial; for the other direction, assume that every prime is principal. Then, Cohen's Lemma implies $Id^{\infty}(R)\neq \emptyset$; In particular, every ideal is finitely generated, so the ring is Noetherian. We may apply Zorn's lemma on the set of non-principally generated ideal (since every chain stablizes and has a maximal element), and let $P$ be a maximal non-principally generated ideal. Suppose it is not prime, and let $xy\in P$ with $x\not \in P$. Then, $P\subset (P:x)$ and $P\subset P+(x)$ properly. By maximality, we have $(P:x)=(c)$, and $(I:c)=(d)$. By definition, we have $cd\in I$; moreover, suppose $x\in I$, then $x=cr=cdt$ for some $r,t\in R$. Thus, $I=(cd)$ is principal, a contradiction.  
\end{proof}


\begin{tcolorbox}[colback=blue!5!white,colframe=blue!30!white]
\begin{proposition}
PIDs are UFDs.
\end{proposition}
\end{tcolorbox}


\begin{proof}
    Let $a\in R$ such that $a$ is non-zero and not a unit. Then, there exists $p\in Spec(R)$ such that $(a)\subseteq p$. Hence $R$ being a PID implies $\exists \pi\in R $ such that $p=(\pi)$. Hence, $\pi$ must be prime and $\pi|a$. Set $a_1=a$, $\pi_1=\pi$, and let $a_2$ be the element such that $\pi_1a_2=a_1$. If $a_2$ is not a unit, find $(a_2)\subset (\pi_2)$, where $\pi_2$ is prime. Let $a_3$ be the element such that $\pi_2a_3=a_2$. Continue inductively until $a_n$ is a unit. The process must terminate, for otherwise we get an infinite chain of distinct principal ideals $(a_i)$ that does not stablize( stablizing is equivalent to $(a_n)=(a_{n+1})$ for some $n$, which implies they differ by a unit).
\end{proof}



\begin{tcolorbox}[colback=green!5!white,colframe=green!30!white]
\begin{corollary}
    Let $R$ be a PID; let $P\subset R$ be a set of representatives for the prime elements up to association. For every $a\in R$, $\exists \epsilon\in R^{\times} $ and $e_{\pi}\in \mathbb{N} $ such that almost all $e_{\pi}=0$. Then, every $a\in R$ can be written as $a=\epsilon\prod_{\pi\in P}\pi^{e_{\pi}}$. We proceed to recover $gcd$ and $lcm$, up to associates.
\end{corollary}
\end{tcolorbox}

Note that the above corollary generalizes to the quotient field by replacing $\mathbb{N}$ with $\mathbb{Z}$.

\section{Unique Factorization Domains}


\begin{tcolorbox}[colback=purple!5!white,colframe=purple!75!black]
\begin{definition}
  The following are equivalent for a domain $R$:
  \begin{enumerate}
    \item $R$ is a UFD.
    \item Every  minimal prime ideal (prime of height $1$) is principal and every non-zero, non-invertible elements in contained in finitely many primes. 
  \end{enumerate}
\end{definition}
\end{tcolorbox}


\begin{proof}
   $1\implies 2$: For every non-zero prime $P$, pick $x\in P$ has factor. One of the prime factors must be in $P$, and it follows by minimality that $P$ must be generated by such prime factor. For the second part, the finite factorization of the element gives precisely the finite primes that it is contained in. 
   $2\implies 1$:given $x\in R$, the finitely many primes containing $x$ are principally generated by prime elements, which gives a factorization.

   
\end{proof}
Remark: we recover the $gcd$ and $lcm$ definition using the same factorization as Corollary $3.1.1$.


\begin{tcolorbox}[colback=red!5!white,colframe=red!30!white]
\begin{theorem}
    (Gauss Lemma)Let $R$ be a UFD; then $R[t]$ is a UFD. 
\end{theorem}
\end{tcolorbox}
To prove the theorem, we need the following lemma on contents:


\begin{tcolorbox}[colback=purple!5!white,colframe=purple!75!black]
\begin{definition}
    Let $f(t)=a_0+...+a_nt^n$ be given. Then, the \underline{\textbf{content}} of $f$, denoted $C(f)$, is the GCD of all coefficients. In particular, $C(f)|a_i$ for all $i$, and $f_0:=f/(C(f))$ has content $1$. 
\end{definition}
\end{tcolorbox}


\begin{tcolorbox}
\begin{lemma}
    Let $R$ be a UFD, then the following hold: $(1).$ $C(f): R[t]\to R$ given by $f\mapsto C(f)$ is multiplicative; in particular, if $C(f)=C(g)=1$, then $C(fg)=1$. 
\end{lemma}
\end{tcolorbox}

\begin{proof}[Proof of lemma $4.2$]
 given $f(t)=a_0+...+a_nt^n$ and $g(t)=b_0+...+b_mt^m$. If one of $f$, $g$ is constant, then it is easy exercise; suppose neither is constant, then set $f=f_0\cdot C(f)$ and $g=g_0\cdot C(g)$. Clearly we have $C(f)\cdot C(g)| C(fg)$. Hence it suffices to prove that $C(f_0g_0)=1$. Equivalently, let $\pi\in R$ be a prime element, we want to show there exists a coefficient $c_k\in f_0g_0$ such that $\pi$ does not divide $c_k$. Suppose $\pi| c_k=\sum_{i+j=k}a_ib_j$ for all $k$. Because $C(f_0)=C(g_0)=1$, then there exists minimal $a_i,b_j$ such that $\pi$ does not divide $a_{i_0},b_{j_0}$. Then, $\pi$ does not divede $C_{i_0+j_0}$.

\end{proof}



\begin{tcolorbox}[colback=blue!5!white,colframe=blue!30!white]
\begin{proposition}
    Let $K:=\textrm{Quot}(R)$, and $f\in K[t]$. Then, let $d$ be the least common multiple of the denominators of the coefficients of $f$. Then, $f=df/d$, and $df\in R[t]$. Define $C_{K}(f)=C(df)/d$. It is standard to check the analog for lemma $4.2$ holds for $C_K$ as well. 
\end{proposition}
\end{tcolorbox}


\begin{tcolorbox}[colback=blue!5!white,colframe=blue!30!white]
\begin{proposition}
Let $R$ be a UFD. For any irreducible $f\in R[t]$, either $f$ is a constant and thus prime in $R$, or $f$ is primitive, i.e $C(f)=1$.
\end{proposition}
\end{tcolorbox}
\begin{proof}
    If $f$ is a constant, the first part of the proposition is obvious; now suppose $f$ has degree $>0$; then $f$ can be factored into its primitive part and content; if $C(f)\neq 1$, we either have a non-trivial factorization of $f$ or $f$ will be a constant multiplied by a unit, a contradction.
\end{proof}

\begin{tcolorbox}[colback=red!5!white,colframe=red!30!white]
\begin{theorem}
    Let $R$ be a UFD. For $f(t)\in R[t]$, let $K:=\textrm{Quot}(R)$. Then, the following are equivalent: 
    \begin{enumerate}
        \item $f(t)$ is prime
        \item  $f(t)$ is irreducible
        \item  Either $f$ is an irreducible constant in $R$ or $f$ is irreducible in $K[t]$ and $C_K(f)=1$. 
    \end{enumerate} 
\end{theorem}
\end{tcolorbox}

\begin{proof}
    $1 \implies 2$ holds in every domain: suppose $a$ is prime and $a=bc$. Then by primeness, we have $a|b$ or $a|c$. WLOG, suppose $a|b$, such that $ax=b$ and $a=axc$, so $cx-1=0$, which implies $c$ is a unit. 

    $2 \implies 1$ in UFDs: suppose $f$ is an irreducible and $f|gh$, then we have some $l$ such that $fl=gh$. Because $g,h,l$ can be uniquely written as a product of irreducibles up to permutation and units, we see that the irreducible $f$ must appear on the RHS once, i.e $f|g$ or $f|h$. 

    For $2\implies 3$:  If $f$ is a constant, then it become a unit in the field of fractions; suppose $deg(f)>0$, so irreducibility implies $C(f)=1$. Suppose by contradiction that $f$ is reducible over $K[t]$, and let $f=gh$ for $g,h\in K[t]$ be a factorization in $K[t]$. Note that given $g,h\in K[t]$, there is some $x_g,x_h\in K$ such that $x_gg,x_hh\in R[t]$ and $C(x_hh)=C(x_gg)=1$. Then, $x_gx_hf=(x_gg)(x_hh)\in R[t]$. By Proposition $4.2$, we have $C(x_gx_hf)=x_gx_hC(f)=1$, which implies $x_gx_h=1$ (up to a unit in $R$). However, this implies $f=(x_gg)(x_hh)$, a contradiction. 
    
    So we are left to prove $3\implies 2$. Suppose $f$ is not a constant and $f$ primitive and irreducible. Suppose $f = gh \in R[x]$. WLOG $g$ is a unit in $K[x]$, so $g$ is a nonzero element of $R$. Now $g$ divides all the coefficients of $f$, so $g$ is a unit in $R$. 
\end{proof}
 

\begin{tcolorbox}[colback=blue!5!white,colframe=blue!30!white]
\begin{proposition}
$R[t_i]_{i\in I}$ is UFD if $R$ is UFD.
\end{proposition}
\end{tcolorbox}
\begin{proof}
   By induction it suffices to show that $R[t]$ is a UFD. The idea is that $K[t]$ is PID so it is a UFD. A factorization in $K[t]$ will correspond to a factorization in $R[t]$ by the equivalence of $2$ and $3$ in Theorem $4.3$.
\end{proof}



\section{Noetherian Rings}



\begin{tcolorbox}[colback=purple!5!white,colframe=purple!75!black]
\begin{definition}
A commutative ring $R$ is called a \underline{\textbf{Noetherian}} ring if every chain of ideals in $R$ is stationary.
\end{definition}
\end{tcolorbox}




\begin{tcolorbox}[colback=blue!5!white,colframe=blue!30!white]
\begin{proposition} The following are equivalent:
    \begin{enumerate}
        \item Every chain of ideals is stationary.
        \item All ideals are finitely generated.
        \item $Spec(R)\subseteq Id^f(R)$.    
    \end{enumerate}
    Terminology: the condition $1$ is called the ACC (Ascending Chain Condition).
\end{proposition}
\end{tcolorbox}
\begin{proof}
    By Cohen's lemma, we deduce $2\iff 3$; $1\iff 2$ is an easy exercise.
\end{proof}


For non-commutative rings, it is possible that a ring is left Noetherian but not right Noetherian.
 
 \begin{tcolorbox}[colback=yellow!5!white,colframe=yellow!30!white]
 \begin{example}
 $R=\{
 \begin{bmatrix}
 p&q\\
 0&m
 \end{bmatrix}: p,q\in \mathbb{Q}; m\in \mathbb{Z} \}$ is left Noetherian but not right Noetherian. 
 \end{example}
 \end{tcolorbox}





\begin{tcolorbox}[colback=blue!5!white,colframe=blue!30!white]
\begin{proposition}
(Basic Properties) Let $R$ be a Noetherian ring. The the following hold: 
\begin{enumerate}
    \item If $\mathfrak{a}$ is an ideal of $R$, then $R/\mathfrak{a}$ is Noetherian if $R$ is Noetherian.
    \item If $\Sigma\subset R$ is a multiplicative system, then $R_{\Sigma}$ is Noetherian. 
    \item The radical of an ideals $\mathfrak{a}$, $rad(\mathfrak{a})$, has a power contained in $\mathfrak{a}$.
    \item  Let $Spec_{min}(\mathfrak{a}):=\{ p\in Spec(R): \mathfrak{a}\subseteq p, \ p \ \textrm{minimal}  \}$ is finite.
\end{enumerate}
\end{proposition}
\end{tcolorbox}
\begin{proof}
    To $1$. Ideals in $R/\mathfrak{a}$ corresponds bijectively to ideals in $R$ that contains $\mathfrak{a}$. Finite generation of ideals in $R$ clearly implies the finite generation of ideals in the quotient. 

    To $2$. $Spec(R_{\Sigma})$ corresponds bijectively to primes in $Spec(R)$ with empty intersection with $\Sigma$. We also have $p$ finite generated implies $p^e$ f.g.

    To $3$. Suppose $rad(\mathfrak{a})=(r_1,.,,,r_n)$ f.g. For every $i$, we have $r_i^{n_i}\in \mathfrak{a}$ for some $n_i$. Take $n=\sum n_i$ and $nil(\mathfrak{a})^{n}\subset \mathfrak{a}$. 
    
    To $4$. The first method to prove this is by contradiction: let $A=\{ \mathfrak{a}: Spec_{min}(\mathfrak{a}) \ \textrm{is infinite} \}$. Then $A$ has maximal elements. Let $\mathfrak{a_0}$ be maximal. Note that $\mathfrak{a_0}$ cannot be prime for it is over itself. Suppose it is not prime, then there exists $xy\in \mathfrak{a}$ with both $x$ and $y$ not in $\mathfrak{a}$; for every prime ideal $P$ containing $\mathfrak{a}$, $P$ contains either $x$ or $y$. By pigeonhole, there must be infinite such primes containing either $\mathfrak{a}+(x)$ or $\mathfrak{a}+(y)$, which contradicts maximality. 

    The second method is using the fact that $Spec(R)$ is a Noetherian topological space, which has finitely many irreducible components.

\end{proof}
The third method is through primary decomposition. An ideal $I$ is irreducible if $I=a_1\cap a_2$ then, $I=a_1$ or $I=a_2$. For principal ideals, this is equivalent to the generator being irreducible. 


\begin{tcolorbox}[colback=blue!5!white,colframe=blue!30!white]
\begin{proposition}
If $R$ is Noetherian, then every ideal $I\in R$ is in the finite intersection of irreducible ideals in $R$. 
\end{proposition}
\end{tcolorbox}
\begin{proof}
    By contradction, let $X$ be the set of ideals that does not satisfy the proposition. Then, $X$ is non-empty, and by Noetherian assumption, there is a maximal element $\mathfrak{a_0}$. Then, $\mathfrak{a_0}$ is not irreducible, for it would be the intersection of itself. Therefore, there exists $I_0,I_1$ such that $a_0=I_0\cap I_1$, where $a_0$ is properly contained in both. By maximality, $I_0,I_1$ are both finite intersection of irreducibles, and we can decompose $a_0$ based on such, a contradction.
\end{proof}


\begin{tcolorbox}[colback=purple!5!white,colframe=purple!75!black]
\begin{definition}
Let $R$ be a commutative ring. Then an ideal $I\subset R$ is primary if for all $x,y\in R$ we have: if $xy\in I$, $x\not \in I$, then ther exists $n\in \mathbb{N}$ such that $y^n\in I$.
\end{definition}
\end{tcolorbox}
In general, a power of prime ideal is not primary. If $I=\mathfrak{m}^n$ for some maximal ideal $\mathfrak{m}$, then $I$ is in fact primary. 


\begin{tcolorbox}[colback=blue!5!white,colframe=blue!30!white]
\begin{proposition}
Let $R$ be Noetherian, and $\mathfrak{a}\in Id(R)$ be a irreducible ideal. Then, $\mathfrak{a}$ is primary, and $nil(\mathfrak{a})$ is prime. 
\end{proposition}
\end{tcolorbox}
\begin{proof}
    Exercise
\end{proof}
These two facts imply $Spec_{min}$ must be finite. In general, quotient of $UFD$ and $PID$ are not $UFD$ or $PID$. but this holds for Noetherian rings. 


\begin{tcolorbox}[colback=red!5!white,colframe=red!30!white]
\begin{theorem}
Let $R$ be a Noetherian ring. Then the following hold:
\begin{enumerate}
    \item (Hilbert Basis Theorem): $R[t_1,...,t_n]$ is Noetherian. 
    \item Every finitely generated $R$-algebra $S$ is Noetherian. 
\end{enumerate}
\end{theorem}
\end{tcolorbox}
\begin{proof}
    Note that $1\implies 2$ since every finitely generated algebra is a quotient of polynomial rings over finitely many variable. To prove $1$, by induction it suffices to show for $i=1$. Let $\mathfrak{a}\in R[t]$ be an ideal. Claim: $\mathfrak{a}$ is f.g. Inductively, we may choose elements $f_i\in I$ with $deg(f_{i+1})$ being minimal in $I\setminus (f_1,...,f_{i-1})$. If the process terminates, then we are done; otherwise, let $a_i$ be the leading coefficient of $f_i$, and the chain of ideals $(I_i:=(a_1,...,a_i))$ must stablizes by Noetherian assumption on $R$. Suppose it stablizes at step $N$, and moreover suppose by contradction that $f_1,...,f_N$ does not generate $\mathfrak{a}$. Then, consider the elment $f_{N+1}$, which by our argument is not contained in $(f_1,...,f_N)$ and of minimal degree. The leading coefficient of $f_{N+1}$ is expressed as $a_{N+1}=\sum_{i=1}^{N}\mu_ia_i$. Then, we cook up 

    \[g=\sum_{i=1}^{N}\mu_if_ix^{deg(f_{N+1})-deg(f_i)}\]
    where $g\in (f_1,...,f_N)$ by construction, and $f_{N+1}-g\not\in (f_1,...,f_N)$. However, $f_{N+1}-g$ has degree strictly less than $f_{N}$ since we cancelled the leading term, which is impossible. 
\end{proof}

\section{Valuation Rings}


\begin{tcolorbox}[colback=blue!5!white,colframe=blue!30!white]
\begin{proposition}
    Let $R$ be a domain. Then the following are equivalent:
    \begin{enumerate}
        \item The ideals in $R$ are totally ordered by inclusion.
        \item The principal ideals in $R$ is totally ordered by inclusion, i.e $id(R)$ is a chain
        \item For every $x\in \textrm{Quot}(R)$, if $x\not \in R$ then $x^{-1}\in R$. 
    \end{enumerate}
\end{proposition}
\end{tcolorbox}
\begin{proof}
    $1\implies 2$ is trivial; $$
\end{proof}

\begin{tcolorbox}[colback=purple!5!white,colframe=purple!75!black]
\begin{definition}
A ring $R$ satisfy one of the conditions above is called a (Krull) \underline{\textbf{Valutation Ring}}.
\end{definition}
\end{tcolorbox}


\begin{tcolorbox}[colback=yellow!5!white,colframe=yellow!30!white]
\begin{example}
$\mathbb{Z}_{(p)}=\{ \frac{q}{l}\in \mathbb{Q}: gcd(l,p)=1 \}$ is a valuation ring with maximal ideal $(p)$. The valuation on $v_p$ is defined by $v(\frac{q}{l})=r$ where $r$ is the maximal natural number such that $p^r|q$. 
\end{example}
\end{tcolorbox}


\begin{tcolorbox}[colback=blue!5!white,colframe=blue!30!white]
\begin{proposition}
(Properties) Let $R$ be a valuation ring, and $K$ be its quotient field. The the following hold:
\begin{enumerate}
    \item $R$ is local, and $m=\{x\in R: x^{-1}\not \in R\}$. The maximal ideal is called \underline{\textbf{valuation ideal}} of $R$.
    \item $\Gamma_R:=K^{\times}/R^{\times}$ is totally ordered by $xR^{\times}\leq yR^{\times}$ iff $yR\subset xR$. The group is called the \underline{\textbf{value group}} of $R$.
    \item The natural map $v_R: K\to \Gamma_R\cup \{\infty\}$, $v(0)=\infty$ satisfies $v(xy)=v(x)+v(y)$ and $v(x+y)\geq min(v(x),v(y))$. Such map is called the (canonical )\underline{\textbf{valuation}} of $R$.
\end{enumerate}
\end{proposition}
\end{tcolorbox}
\begin{proof}
    Exercise.
\end{proof}
Note $R$ is the set $\{ x\in K:v_R(x)\geq 0 \}$; $\mathfrak{m}$ is the set $\{ x\in K:v_R(x)> 0 \}$;



\begin{tcolorbox}[colback=purple!5!white,colframe=purple!75!black]
\begin{definition}
    Let $R$ be a domain, and $K$ be a field, $(\Gamma,+,\leq )$ be a totally orderedd abelian group. Let $v: K\to \Gamma\cup \{\infty\}$ be a map satisfying 
    \begin{enumerate}
        \item $v(x)=\infty$ iff $x=0$
        \item $v(xy)=v(x)+v(y)$
        \item $v(x+y)\geq min(v(x),v(y))$
    \end{enumerate}
    Then, the map $v$ is called a \underline{\textbf{valuation}} of $K$.
\end{definition}
\end{tcolorbox}

\begin{tcolorbox}[colback=blue!5!white,colframe=blue!30!white]
\begin{proposition}
$R_v=\{ x\in K:v(x)\geq 0 \}$ is a valuation ring. The map $\tau: \Gamma_{R_v}\to \Gamma$, given by $xR_v^{\times}\mapsto v(x)$ is an order preserving embedding. Moreover, $v=\tau \circ v_{R_v}: K\to \Gamma\cup \{\infty\}$.
\end{proposition}
\end{tcolorbox}
\begin{proof}
    Exercise.
\end{proof}

Given a valuation ring, $R\subset K$, every embedding of totally ordered groups $\Gamma_R\to \Gamma$ gives rise to a valuation.


\begin{tcolorbox}[colback=purple!5!white,colframe=purple!75!black]
\begin{definition}
\begin{enumerate}
    \item Two valuations $v,w$ on $K$ are equivalent if $R_v=R_w$. If $v: K\to \Gamma_v\cup \{\infty\}$ and $w: K\to \Gamma_w\cup \{\infty\}$, with embeddings $\tau_v: \Gamma_{R_v}\to \Gamma_v$ $\tau_w: \Gamma_{R_w}\to \Gamma_w$. There exists an order preserving commutative diagram
   \[
   \begin{tikzcd}
   \Gamma_{R_v}\arrow[r]&\tau_v(\Gamma_{R_v})\arrow[d,"\tau_{vw}"]\arrow[r]&\Gamma_v\\
   \Gamma_{R_w}\arrow[r]&\tau_w(\Gamma_{R_w})\arrow[r]&\Gamma_w
   \end{tikzcd}\]
   \item  Two valuations $v,w$ on $K$ are equivalent iff $\mathfrak{m}_v=\mathfrak{m}_w$
\end{enumerate}
\end{definition}
\end{tcolorbox}

A valuation ring $R$ is called \underline{\textbf{discrete}}, if $v_R(K)\cong \mathbb{Z}$ as ordered abelian groups. If $\pi\in R$ has $v_R(\pi)$ minimal since $\mathbb{Z}$ has minimal elements, then $\pi$ is called a uniformizing parameter.


\begin{tcolorbox}[colback=yellow!5!white,colframe=yellow!30!white]
\begin{example}
$\mathbb{Z}_{(p)}\subset \mathbb{Q}$ is a discrete valuation ring. The uniformation parameter is $p\epsilon$ with $\epsilon$ a unit. 
\end{example}
\end{tcolorbox}
A valuation ring $R$ is called rank $1$ if $v_r(K)$ satisfies the Archimedian axiom, i.e for $\forall \gamma_1,\gamma_2\in \Gamma_R, \gamma_1>0$, $\exists n\in \mathbb{N}$ such that $\gamma_2\leq n\cdot \gamma_2$. A totally ordered group $\Gamma$ is Archimedian if there is an ordered preserving embedding into the reals. In relation to absolute values,


\begin{tcolorbox}[colback=purple!5!white,colframe=purple!75!black]
\begin{definition}
An absolute value of a field $K$ is any map $|-|: K\to \mathbb{R}_{\geq 0}^+$ iff it satisfies the norm axioms. An absolute value is called \underline{\textbf{non-Archimedian}} or \underline{\textbf{ultra-metric}} if $|x+y|\leq max\{|x|,|y|\}$.
\end{definition}
\end{tcolorbox}

Let $|-|: K\to R$ be a non-Archimedian absolute value. Then $v_{|-|}:=- log\circ |-|:K\to \mathbb{R}\cup \{\infty\} $ is rank $1$ valuation. Conversely, let $v: K\to R\cup \{\infty\}$ be a rank one valuation, then $|-|_{v}:=e^{-v}: K\to \mathbb{R}_{\geq 0}$ is a non-Archimedian absolute value. 


\begin{tcolorbox}[colback=red!5!white,colframe=red!30!white]
\begin{theorem}
    The following facts about possible valuations
\begin{enumerate}
    \item If $K|F_p$ algebraic, then no non-trivial valuations.
    \item If $v$ is a valuation of $F(t)$ such $v$ is trivial on $F$, then $R_v=F[t]_{p(t)}$, where $p(t)$ irreducible or $R_v=F[\frac{1}{t}]_{(\frac{1}{t})}$. thus all valuations are discrete. 
    \item If $v$ is a non-trivial valuation on $\mathbb{Q}$, then $R_v=\mathbb{Z}_{(p)}$ for some $p$ prime. Morever, all non-archimedian absolute values of $\mathbb{Q}$ corresponds to the valuation above. (Ostrowskis Theorem).
\end{enumerate}
\end{theorem}
\end{tcolorbox}
In general, the space of all valuations on $K$, denoted $Val(K)$, is called the Zariski-Riemann space. Moreover, $Val(K)$ carries a topology called a patch topology, or constrcutible topology, that makes the space compact totally disconnected. The space is usually very complicated.


\begin{tcolorbox}[colback=red!5!white,colframe=red!30!white]
\begin{theorem}
(Chevalley's Theorem for Existence of Valuations) Let $A$ be a domain, $p\in Spec(a)$ a prime ideal, $\kappa(p)Quot(A/p)\subset \Omega$, with $\Omega$ algebraically closed. Then, there exists a valuation ring $R$ of $K=Quot(A)$ such that $\mathfrak{m}_R\cap A=p$, and $R/\mathfrak{m}\hookrightarrow \Omega$.
\end{theorem}
\end{tcolorbox}
\begin{proof}
    Set up for Zorn's lemma: $H=\{ (B,q): A\subset B, q\in Spec(B), q\cap A=p \}$ such that the embedding into the closure $\Omega$ commutes. Prove that $H$ has maximal elements $R$. If $R$ is not local, then $R_{\mathfrak{m}}$ is local and bigger than $R$. Thus, $R$ is local. let $x\in K$, we want to show $x\not \in R$ implies $x^{-1}\in R$. Claim, if $m[x]=R[x]$, then $m[x^{-1}]\subset R[x^{-1}]$. If so m then $R_x:=R[x^{-1}]$ is greater than $R$, a contradictin. By contradiction, let $m[x]=R[x]$, $m[x^{-1}]=R[x^{-1}]$. Then, there exists coefficients in $m$ such that polynomials are $1$.  
\end{proof}

\section{Artin Rings}

\begin{tcolorbox}[colback=purple!5!white,colframe=purple!75!black]
\begin{definition}
A commutative ring $R$ is called \underline{\textbf{Artin}}, if every descending chain of ideals $(I_n)$ is stationary. 
\end{definition}
\end{tcolorbox}

\begin{tcolorbox}[colback=blue!5!white,colframe=blue!30!white]
\begin{proposition}
Let $R$ be Artinian. Then the following hold:
\begin{enumerate}
    \item If $\Sigma$ is a multiplicative system, then $\Sigma^{-1}R$ is also Artinian.
    \item If $I\subset R$ is an ideal. Then, $R/I$ is Artinian. 
    \item $Spec(R)=Max(R)$ is finite. 
\end{enumerate}
\end{proposition}
\end{tcolorbox}
\begin{proof}
    To $1$. pull back of ideals respects inclusion. To $2$. obvious. To $3$, let $p\in Spec(R)$. Then, $R/p$ is an integral Artinian ring. Then, $R/p$ must be a field. Thus, all primes are maximal. If $\mathfrak{m_1},...,\mathfrak{m}_n$, then $\mathfrak{m}_1\subset \mathfrak{m}_1\mathfrak{m}_2\subset ...\subset \mathfrak{m_1}\mathfrak{m}_2\mathfrak{m}_3....$
\end{proof}


\begin{tcolorbox}[colback=red!5!white,colframe=red!30!white]
\begin{theorem}
Let $R$ be an Artinian ring. Then the following hold:
\begin{enumerate}
    \item J(R)=N(R) is nilpotent.
    \item (Structure Theorem) Let $Max(R)=\{m_1,...,m_r\}$. Then, $R\to R/(m_1)^n\times...\times R/m_r^n$. Hence, $R$ is a product of local Artinian rings. 
\end{enumerate}
\end{theorem}
\end{tcolorbox}
\begin{proof}
    Look at $J(R)\subset J^2(R)\subset ...$ becomes stationary. Thus, there exists $n$ minimal such that $I=J^n(R)$ such that $I^k=I$ for all $k$. Let $H$ be the set of ideals in $R$ sich that  
\end{proof}


\begin{tcolorbox}[colback=red!5!white,colframe=red!30!white]
\begin{theorem}
(Relations of Artin Rings and Noether Rings) Let $R$ be a commutative ring. The the following are equivalent: 
\begin{enumerate}
    \item $R$ is an Artin ring
    \item $R$ is Noether and Krull dimension of $R$ is $0$. 
\end{enumerate}
\end{theorem}
\end{tcolorbox}
\begin{proof}
    Step one is reduce to $R$ local as follows: $R$ Artin ring implies $R=\prod R_i$, where $R_i$ are local Artin rings. In particular, the product is Noetherian iff each $R_i$ is Noetherian. $R$ is Noetherian and Krull dimension $0$ implies $Max(R)=Spec(R)$, hence $rad(R)=Jac(R)$. Thus, $R$ has only finitely many prime ideals. Hence, $Max(R)$ is finite and we reduce to the finite case. 

    Step two: now assume $R$ is local. Assume $(R,\mathfrak{m})$ is a local Artin ring. To prove $R$ is Noetherian, it suffices to show $\mathfrak{m}$ is finitely generated. For $k>0$, we have the exact spectralsequences
    \[\begin{tikzcd}
    0\arrow[r]&\mathfrak{m}^k/\frak{m}^{k+1}\arrow[r,"f"]&R/\mathfrak{m}^{k+1}\arrow[r,"g"]&R/\mathfrak{m}^k\arrow[r]&0
    \end{tikzcd}\]
    Note $\kappa:= R/\mathfrak{m}$ is a field, $R$ acts on $\mathfrak{m}^k/\mathfrak{m}^{k+1}$ through the projection $R\to \kappa$. In particular, there is a bijection 
    \[
        \{ \kappa -\textrm{vector subspaces of }\ \mathfrak{m}^k/\mathfrak{m}^{k+11} : \} \iff \{ R-\textrm{ideals} \mathfrak{n} \ : \mathfrak{m}^{k+1}\subset \mathfrak{n}\subset \mathfrak{m}^k \}=\epsilon
    \]
    Note $R$ Artinian implies $R/\mathfrak{m}^{k+1}$ is Artinian. Thus, the set $\epsilon$ is finite, and $\mathfrak{m}^{k}/\mathfrak{m}^{k+1}$ is finite dimensional. Then induction on $k$: suppose $R/\mathfrak{m}^k$ is Noetherian. Then, $\mathfrak{m}^k/\mathfrak{m}^{k+1}$ is finite $R$-module. 
    
    
    For the reverse implication: Given $(R,\mathfrak{m})$ Noetherian and $\frak{m}^k=0$ for $k$ sufficiently large. If $k=0$ then the statement holds; suppose it holds for $k$; consider the exact sequence above and $R/\mathfrak{m}^k$ is Artin. Let $(\mathfrak{a}_i)$ be a descending sequence of $Id(R/\mathfrak{m}^{k+1})$. Hence, there exists $i_0$ such that $\pi_k(\frak{a}_i)\pi_k(\mathfrak{a_{i_0}})$ for all $i>i_0$. Then, look at 
    \[\begin{tikzcd}
        0\arrow[r]&\mathfrak{a}_i\cap\mathfrak{m}^k/\frak{m}^{k+1}\arrow[r,"f"]&\mathfrak{a}_i/\mathfrak{m}^{k+1}\arrow[r,"g"]&\mathfrak{a}_i/\mathfrak{m}^k\arrow[r]&0
        \end{tikzcd}\]
\end{proof}

\section{Krull's Theorem on Noetherian Rings}



\begin{tcolorbox}[colback=purple!5!white,colframe=purple!75!black]
\begin{definition}
    Let $R$ be a commutative ring; $\mathfrak{a}\subset R$ a proper ideal. Consider $\mathfrak{a}^n$ and the projection $p_n: R/\mathfrak{a}^{n+1}\to R/\mathfrak{a}^n$. Then, $(R/\mathfrak{a}^n, p_n)$ is a projective system. The limit $\widehat{R}:=\varprojlim R/\mathfrak{a}^n$, together with $i: R\to \widehat{R}$ is called $\frak{a}$-adic completion of $R$. 
\end{definition}
\end{tcolorbox}


\begin{tcolorbox}[colback=blue!5!white,colframe=blue!30!white]
\begin{proposition}
    The kernel of the inclusion $i: R\to \widehat{R}$ is the intersection of all $\mathfrak{a}^n$.

\end{proposition}
\end{tcolorbox}



\begin{tcolorbox}[colback=red!5!white,colframe=red!30!white]
\begin{theorem}
(The Intersection Theorem) Let $R$ be a Noetherian ring that is local or integral. Let $\mathfrak{a}\subset R$ be a proper ideal. Then, $\cap \mathfrak{a}^n=0$. In particular, the inclusion map in the $\mathfrak{a}$-adic completion is injection.
\end{theorem}
\end{tcolorbox}
\begin{proof}
 Suppose $R$ is local with maximal ideal $\mathfrak{m}$. Let $\mathfrak{a}_0=\cap \mathfrak{a}^k$ is a finitely generated ideal. Let $\mathfrak{m}_0=\cap \mathfrak{m}^k$ is f.g with $\mathfrak{a}_0\subset \mathfrak{m}_0$. We then have $\mathfrak{m}\cdot \mathfrak{m}_0=\frak{m}_0$, and apply Nakayama's lemma, we get $\mathfrak{m}_0=(0)$. 
 
 Now suppose $\mathfrak{m}$ be a maximal ideal over $\mathfrak{a}$, and let $\phi: R\to R_{\mathfrak{m}}$ be the localization is injective. 
\end{proof}

Much more is true.

\begin{tcolorbox}[colback=red!5!white,colframe=red!30!white]
\begin{theorem}
(Artin-Reese Lemma) Let $R$ be a Noetherian ring. Given ideals $\mathfrak{a,b}, I$, there exists $k$ such that for $k'>k$, we have $\mathfrak{a}^k \mathfrak{b}\cap I=(\mathfrak{a}^{k'-k})$. 
\end{theorem}
\end{tcolorbox}


\begin{tcolorbox}[colback=red!5!white,colframe=red!30!white]
\begin{theorem}
If $R$ is Noetherian, then all $\mathfrak{a}$-adic completions of $R$ is Noetherian. 
\end{theorem}
\end{tcolorbox}



\begin{tcolorbox}[colback=purple!5!white,colframe=purple!75!black]
\begin{definition}
Let $R$ be a ring. For $r\in R$, let $Spec_{min}(r)=\{p\in Spec(R) : (r)\subset p \ \textrm{minimal}\}$. Definition goes similarly for a set of elements. 
\end{definition}
\end{tcolorbox}



\begin{tcolorbox}[colback=purple!5!white,colframe=purple!75!black]
\begin{definition}
    For $p\in Spec(R)$, the \underline{\textbf{height}} of $p$ is the krull dimension of $R_{p}$. The \underline{\textbf{coheight}} is the krull dimension is the krull dimension of $R/p$. 
\end{definition}
\end{tcolorbox}


\begin{tcolorbox}[colback=blue!5!white,colframe=blue!30!white]
\begin{proposition}
$height(p)+coheight(p)\leq$ Krull dimension of $p$.
\end{proposition}
\end{tcolorbox}


\begin{tcolorbox}[colback=red!5!white,colframe=red!30!white]
\begin{theorem}
(Krull's Principal Ideal Theorem/ Hauptideasatz) Let $R$ be a Noetherian ring. Then, for all non-units $r\in R$, one has $height(p)\leq 1$ for all $p\in Spec_{min}(r)$, with equality when $r$ is not a zero-divisor.
\end{theorem}
\end{tcolorbox}
\begin{proof}
    Suppose by contradction that $height(p)>1$. Equivalently, there exists a chain $q_0\subset q$ prime ideals. Reduction step: can replace $R$ by $R/p_0$, and $q/q_0$ and $p/q_0$. Can replace $R,q,p$ by $R_p, p_p, q_p$. So WLOG, $R$ is local, and $p$ is maximal. Hence, $(r)\subset p$, and $rad(r)=p$. Conclude $\overline{R}:= R/(r)$ is Artin. Hence, $q^{(n)}$ gives $(\overline{q ^{(n)}})_n$ a descending sequence, so it must stablize. 
\end{proof}



\begin{tcolorbox}
\begin{lemma}
For $q\in Spec(R)$, let $q^{(n)}:=(q^nR_{q})^c$. Then, $(q^{(n)})^e=q^nR_q=(qR_q)^n$. $q^{(n)}$ is called the symbolic $n$-th power of $q$.
\end{lemma}
\end{tcolorbox}
\begin{proof}
    exercise
\end{proof}


\begin{tcolorbox}
\begin{lemma}
$q^{(n)}$ is primary if $ax\in q^{(n)}$, and $x\not \in q$, then $a\in q ^{(n)}$. 
\end{lemma}
\end{tcolorbox}
\begin{proof}
    exercise
\end{proof}


\begin{tcolorbox}[colback=purple!5!white,colframe=purple!75!black]
\begin{definition}
Let $r=(r_1,..,r_n)$ be given. It is called a \underline{\textbf{regular }} sequence if $r_i$ is not a zerodivisor in $R/(r_1,...,r_{i-1})$. 
\end{definition}
\end{tcolorbox}


\begin{tcolorbox}[colback=red!5!white,colframe=red!30!white]
\begin{theorem}
(Krull's Dimension Theorem) Let $R$ be a Noether ring, and $r=(r_1,...,r_m)$ a system. Then $Spec_{min}(r)$ contains prime ideals of height $\leq m$, with equality when $r$ is regular.
\end{theorem}
\end{tcolorbox}
\begin{proof}


    Induction: $n=1$ is PIT; Given $r=(r_1,...,r_{m+1})$, and $p\in Spec_{min}(r)$. Let $q\subset p$ be a maximal with this property. Claim: $ht(q)=m$. Hence $ht(p)=m+1$,
   By contradiction, let $ht(q)>m$ and consider $q_0\subset q_1\subset ...\subset q_m=q$. Reduction step $1$: replace $R,p,q$ by $\overline{R}:=R/q_0$,$\overline{P}:=p/q_0$, $\overline{q}:=q/q_0$. WLOG, $R=\overline{R}$ is a Noetherian domain. Reduction step $2$ Replace $R,p,q$ with their localization at $p$. Then, $R$ is local with $max(R)=p$.
    
    Since $q$ is properly contained in $p$, there exists $r_i\not\in q$ by minimality. WLOG, $r=m+1$. Consider $\mathfrak{a}=q+(r_{m+1})$, which implies $q\subset \mathfrak{a}\subseteq p$. Then, $nil(\mathfrak{a})=p$ since $p$ is the only prime ideal containing $a$. Conclude that $r_i\in p$ implies $r_i\in nil(\mathfrak{a})$, hence for every $i=1,...,m$, there exists $a_i\in R$ and $s_i\in q$ such that $r_i^{n_i}=s_i+a_ir_{m+1}$. Conclude that $s=(s_1,...,s_m)$ and $nil(s)=q$. In particular, $q$ is a minimal prime ideal containing elements $s_1,...,s_m$. Look at $R\to R/s$, and $\overline{p}=p/s$. Hence, $ht(\overline{p})$ is at most $1$, and $1$ $\overline{r}_{m+1}$ is not a zero divisor. 
    
\end{proof}


\begin{tcolorbox}[colback=green!5!white,colframe=green!30!white]
\begin{corollary}
Let $R$ be Noether. Then, the following hold:
\begin{enumerate}
    \item Every descending sequence of prime ideals is staionary. 
    \item if $ht(p)=m$, then there exists a regular system of length $m$ with $p$ a minimal prime over it.
\end{enumerate}
\end{corollary}
\end{tcolorbox}

\begin{proof}
    exercise
\end{proof}

\section{Modules over special classes of rings}
\subsection{Modules over PIDs}


\begin{tcolorbox}
\begin{lemma}
Let $C=(\gamma_1,..,\gamma_n)$ be an $R$-basis of a free $R$-module $P$, and $\gamma'_1= \sum_{i=1}^{m}r_i\gamma_i$ such that $gcd(r_1,...,r_m)=1$. Then there exists a basis $C'=(\gamma_1',...,\gamma_m')$
\end{lemma}
\end{tcolorbox}
\begin{proof}
    Induction: $n=1$ trivial; let $x=\sum_{i=1}^{m+1}r_i\gamma_i=\sum_{i=1}^{m}r_i\gamma_i+r_{m+1}\gamma_{m+1}$. We can write $\sum_{i=1}^{m}r_i\gamma_i$ with respect to the new basis by induction step. since the gcd of the sequence is $gcd(d,r_{m+1})$, which must be $1$. 
\end{proof}

\begin{tcolorbox}[colback=red!5!white,colframe=red!30!white]
\begin{theorem}
Let $R$ be a PID. Then, the following hold:
\begin{enumerate}
    \item If $M$ is a free $R$ module, every $R$-submodule of $M$ is $R$-free. 
    \item  A finite torsion free $R$-module is $R$-free
\end{enumerate}
\end{theorem}
\end{tcolorbox}
\begin{proof}
    To $1$: let $A=(\alpha_i)$ be a basis for $M$. Suppose $I$ is well-ordered(otherwise use Zorn's lemma), we make transfinite induction: let $i_0$ be the minimal element of $I$, $M_{i_0}=\langle \alpha_{i_0} \rangle$ be the cyclic submodule. Let$M_{i'}=\langle \alpha_{i'}: i\leq i'  \rangle$ and $N_{i'}=N\cap M_i'$. Note that $N_{i_0}=N\cap R_{\alpha_{i_0}}$, and the submodules of a PID is precisely the principal ideals, which are free. 

    To $2$: let $(x_1,...,x_m)$ be a system of generators and let $R^m\to M$ be the natural surjection. If the system is linearly independent, we are done. Let $\sum_{i=1}^{k}r_1x_i=0$. Divide out the $gcd$, and if there is no torsion, we can apply the lemma. 
\end{proof}




\begin{tcolorbox}[colback=red!5!white,colframe=red!30!white]
\begin{theorem}
(Invariant Factors Theorem) Let $R$ be a principal ideal domain and $M$ a free $R$-module, $N\subset M$ a submodule. Then, there exists $R$-basis $A=(\alpha_1,...,\alpha_m)$ of $M$ and $\delta_1|\delta_2|...|\delta_n$ in $R$ such that $\delta_1\alpha_1...,\delta_n\alpha_n$ is an $R$-basis for $n$.
\end{theorem}
\end{tcolorbox}
\begin{proof}
    Let $D=\{ d\in R: \exists y\in N, \textrm{basis} (\beta_1,...,\beta_m) \textrm{of} M st d\beta_1=y \}=\{ d\in R:\exists y=d \sum_{i=1}^{k}r_i'\alpha_i'\in N, gcd(r_i)=1 \}$.
\end{proof}

\begin{tcolorbox}
    \begin{lemma}
    $d_1=inf D$ exists, where the ordering is by divisibility. 
    \end{lemma}
    \end{tcolorbox}

    \begin{proof}
        Obviously, $D$ has minimal elements. Let $d',d''$ be minimal elements, and $d=gcd(d',d'')$, and show $d\in D$. 
    \end{proof}


\end{document}
