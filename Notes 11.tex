\documentclass{article}
\usepackage[utf8]{inputenc}
\usepackage{amsmath}
\usepackage{amsfonts}
\usepackage{amssymb}
\usepackage{tikz}
\usepackage{fullpage}
\usepackage{tikz-cd}
\usepackage{spectralsequences}
\usepackage{adjustbox}
\usepackage{tcolorbox}
\usepackage{xfrac}
\usepackage{graphicx}
\usepackage[parfill]{parskip}
\usepackage{amsthm}
\theoremstyle{definition}
\newtheorem{theorem}{Theorem}[section]
\theoremstyle{definition}
\newtheorem{problem}{problem}[section]
\theoremstyle{definition}
\newtheorem{proposition}{Proposition}[section]
\theoremstyle{definition}
\newtheorem{lemma}[theorem]{Lemma}
\theoremstyle{definition}
\newtheorem{definition}{Definition}[section]
\theoremstyle{definition}
\newtheorem{corollary}{Corollary}[theorem]
\theoremstyle{definition}
\newtheorem{example}{Example}[section]

\title{MATH 603 Notes}
\author{David Zhu}

\begin{document}
\maketitle

\section{More on Commutative Rings}
Let $a,b\in R$. Then $a|b \iff \exists a'\in R, b=aa'$;
A semi ring on $(R,\leq)$ defined by $a\leq b \iff a|b$. Note that $\leq $ is usally not a partial order: let $b\in R^{\times}$, then $a\leq ab\leq a$, but $a\neq ab$.


\begin{tcolorbox}[colback=blue!5!white,colframe=blue!30!white]
\begin{proposition}
    $a\sim b$ iff $a\leq b$ and $b\leq a$ iff $(a)=(b)$ is an equivalence relation.
\end{proposition}
\end{tcolorbox}

For $R$ a domain, the induced relation gives a well-defined definition of greatest common divisor.
\begin{tcolorbox}[colback=red!5!white,colframe=red!30!white]
    \begin{definition}
        The \underline{\textbf{gcd}} of $a,b$, denoted by $gcd(a,b)$, if exists, is any $d\in R$ such that $d|a,b$ and for any other $d'$ satisfying the condition, $d'|d$.
    \end{definition}    
\end{tcolorbox}


\begin{tcolorbox}[colback=purple!5!white,colframe=purple!75!black]
\begin{definition}
    The \underline{\textbf{lcm}} of $a,b$, denoted by $lcm(a,b)$, if exists, is any  $d\in R$ such that $a,b|d$ and for any other $d'$ satisfying the condition, $d|d'$.
\end{definition}
\end{tcolorbox}


\begin{tcolorbox}[colback=blue!5!white,colframe=blue!30!white]
\begin{proposition}
    If $gcd(a,b)$ exists, then $gcd(a,b)=sup \{ d:d\leq a,b \}$.  If $lcm(a,b)$ exists, then $lcm(a,b)=\textrm{inf} \{ d :a,b\leq d \}$. 
\end{proposition}
\end{tcolorbox}

Note that maximal/minimal elements always exists by Zorn's lemma. However, the unique supremum/ \ infimum may not exist. We have our following example:

\begin{tcolorbox}[colback=yellow!5!white,colframe=yellow!30!white]
\begin{example}
Take $R=[\sqrt{-3}]$. Let $a=4=(1+\sqrt{-3})(1-\sqrt{-3})$ and $b=2(1+\sqrt{-3})$. Then, $(1+\sqrt{-3})$ and $2$ are both maximal divisors, but they are not comparable since the only divisors of $2$ are $\{ \pm 1, \pm 2 \}$ by norm reasons, and none divides $1+\sqrt{-3}$.
\end{example}
\end{tcolorbox}


\begin{tcolorbox}[colback=blue!5!white,colframe=blue!30!white]
\begin{proposition}
    Let $a,b\in R$ be given. Then the following hold: $gcd(a,b)=d$ exists iff $(d)$ is the unique maximal prinipal ideal such that $(a)+(b)\subseteq (d)$. Dually, $lcm(a,b)=c$ exists iff $(c)=(a)\cap (b)$. If both holds, then $a\cdot b=lcm(a,b)\cdot gcd(a,b)$
\end{proposition}
\end{tcolorbox}

\begin{proof}
    Easy exercise. Note that the inclusion can be proper, for example, take $R=k[x,y]$ and ideals $(x),(y)$. Then $(1)$ is the gcd, but the containment is proper.
\end{proof}
Recall that $Id(R)$ is partially ordered by inclusion. 


\begin{tcolorbox}[colback=purple!5!white,colframe=purple!75!black]
\begin{definition}
    ($Id(R), +,\cap,\cdot,\leq$) is the lattice of ideals of $R$.
\end{definition}
\end{tcolorbox}

Note that $+,,\cap$ are simply the sums and intersection, but $\cdot$ is the ideal generated by the products, i.e the set of finite sums of products. 


\begin{tcolorbox}[colback=red!5!white,colframe=red!30!white]
\begin{theorem}
    Let $Id^{\infty}(R)$ be the set of non-finitely generated ideals for $R$; the following are equivalent:
\begin{enumerate}
    \item $Id^{\infty}(R)$ is non-empty; 
    \item  There exists an infinite non-stationary chain of ideals $(\sigma_i)$, where $\sigma_i\in Id(R)$;
\end{enumerate}
\end{theorem}
\end{tcolorbox}


\begin{proof}
    For $1\implies 2$, let $I$ be a non-finitely generated ideal of $R$ and pick $x_1\in \in I$. Let $\sigma_1=(x_1)$. Because the ideal is non-finitely generated, we can pick $x_2\in I$ such that $x_2\not \in \sigma_1$. Let $\sigma_2=(x_1,x_2)$. Continue inductively gives us an infinite non-stationary chain of ideals. 

    For $2\implies 1$, take the union of all the ideals in the infinite non-stationary chain. It is an ideal and it cannot be finitely generated. 
\end{proof}


\begin{tcolorbox}[colback=red!5!white,colframe=red!30!white]
\begin{theorem}
    (Cohen's lemma): Let $Id^{\infty}(R)\neq \emptyset$. Then, it has a maximal element and any such maximal element is prime. 
\end{theorem}
\end{tcolorbox}
Before proving Cohen's lemma, we need the following technical lemma:\

\begin{tcolorbox}
\begin{lemma}
    Let $I$ be an ideal. Define $(I:a):=\{ b\in R: ab\in I \}$. If $I+(x)$ and $(I:x)$ are both finitely generated, then $I$ is finitely generated.
\end{lemma}
\end{tcolorbox}
\begin{proof}[Proof of Lemma 1.3]
    By assumption, there is finite set $\{ \alpha_i:=a_i+f_ix: a_i\in I, f_i\in R, i=1,...,k \}$ that generate $I+(x)$, and a finite set $\{ b_j: j=1,...,l \}$ that generate $(I:x)$. We claim that the set $\{ a_i, b_jx: i\in I, j\in J\}$ generate the entire $I$: since $I\subseteq I+(x)$, we can write any element $\pi\in I$ as a finite linear combination $\pi=\sum_{i=1}^{k}g_i\alpha_i=\sum_{i=1}^{k}g_i(a_i+f_ix)$, where $g_i\in R$. We note that $\pi-\sum_{i=1}^{k}g_ia_i=\sum_{i=1}^{k}g_if_ix$ is in $I$; it follows that $\sum_{i=1}^{k}g_if_i\in (I:x)$, so $\sum_{i=1}^{k}g_if_ix$ is generated by the set $ \{b_jx\}$, and we are done. 
\end{proof}

With the lemma in hand, now we can prove Theorem $1.2$
\begin{proof}[Proof of Theorem $1.2$]
    Zorn's lemma implies $Id^{\infty}(R)$ has maximal elements.  Let $I$ one such maximal element, and suppose it is not prime. Then, there exists $xy\in I$ and WLOG suppose $x\not \in I$. By maximality, $I+(x)$ must be finitely generated. By definition, we have $y\in (I:x)$. Lemma $1.3$ implies $(I:x)$ is not finitely generated, and in particular, $I\subseteq (I:x)$. Applying maximality again, we have $I=(I:x)$, which forces $y\in I$, a contradiction.
\end{proof}

\section{Euclidean Rings}


\begin{tcolorbox}[colback=purple!5!white,colframe=purple!75!black]
\begin{definition}
    A \underline{\textbf{Principal Ideal Ring}} is any ring $R$ i which every ideal is principally generated. If $R$ is a domain, then $R$ is called a \underline{\textbf{PID}}. 
\end{definition}
\end{tcolorbox}


\begin{tcolorbox}[colback=purple!5!white,colframe=purple!75!black]
\begin{definition}
    A \underline{\textbf{Factorial Ring}}  is any ring $R$ in which all units can be written as a finite product of irreducible elements, unique up to a unit. If $R$ is domain, then it is called a \underline{\textbf{UFD}}. 
\end{definition}
\end{tcolorbox}
Note that if the ring $R$ it is not a domain, $x|y$ and $y|x$ does not imply $x=uy$ for some unit $u$. Let us prove that this holds for a domain: suppose $x=ys$ and $y=xt$, and $x,y\neq 0$ then $x=xts$, which implies $x(1-ts)=0$. This forces $1-ts=0$, and $t,s$ are then units. We can concoct counterexamples when $R$ is not a domain accordingly: let $R=k[x]/(x^3-x)$ and take $a=x$, $b=x^2$. Clearly, $a|b$ and $b=x^2\cdot x=x^3$, so $b|a$. However, $x$ is not a unit. 



\begin{tcolorbox}[colback=purple!5!white,colframe=purple!75!black]
\begin{definition}
    A \underline{\textbf{Noetherian Ring}} is any ring $R$ such that any ideal is finitely generated. 
\end{definition}
\end{tcolorbox}


\begin{tcolorbox}[colback=purple!5!white,colframe=purple!75!black]
\begin{definition}
    Let $R$ be a domain. A \underline{\textbf{Euclidean norm}} on $R$ is any map $\phi: R\to \mathbb N$ satisfying $\phi(x)=0 $ iff $x=0$ and for every $a,b\in R$ with $b\neq 0$, then there exists $q,r\in R$ such that $a=bq+r$ with $\phi(r)<\phi(b) $. A \underline{\textbf{Euclidean Domain}} is any domain equipped with a Euclidean norm.
\end{definition}
\end{tcolorbox}

Example of Euclidean domains include $\mathbb Z,\mathbb Z[i]$. A non-trivial example $R=F[t]$, with $\phi(p(t))=2^{deg(p(t))}$. A non-example is $\mathbb Z[\sqrt{6}]$ for it is not a PID.



\begin{tcolorbox}[colback=blue!5!white,colframe=blue!30!white]
\begin{proposition}
    Eucldiean Domains are PIDs.
\end{proposition}
\end{tcolorbox}
\begin{proof}
    By the well-ordering principal, for every ideal $I$ in a Euclidean domain, there exists an element other than $0$ of the smallest norm. It is easy exercise that such element generate the entire ideal.
\end{proof}


\begin{tcolorbox}[colback=blue!5!white,colframe=blue!30!white]
\begin{proposition}
    (The Euclidean Algorithm): Given $a,b\in R$, $b\neq 0$. Set $r_0=a,r_1=b$, and continue inductively $r_{i-1}=r_i\cdot q_i+r_{i+1}$. Then, $r_i=0$ for $i>\phi(b)$ and if $r_{i_0}\geq 1$ maximal with $r_{i_0}\neq 0$, then $r_{i_0}=gcd(a,b)$.
\end{proposition}
\end{tcolorbox}
\begin{proof}
    Note that the remainder is strictly decreasing, so $r_i$ must become $0$ after $\phi(b)$ steps. Note that once $r_{i+1}=0$, we have $r_i|r_{n}$ for all $n\leq i$. Coversely, it is clear that any divisor of $a,b$ divdes all $r_n$ for $n\leq i$.
\end{proof}


\section{Principal Ideal Domains}

\begin{tcolorbox}[colback=red!5!white,colframe=red!30!white]
\begin{theorem}
    (Charaterization) For A domain $R$, the following are equivalent: 
\begin{enumerate}
    \item  $R$ is a PID.
    \item every $p\in Spec(R)$ is principal. 
\end{enumerate}
\end{theorem}
\end{tcolorbox}
\begin{proof}
    One direction is trivial; for the other direction, assume that every prime is principal. Then, Cohen's Lemma implies $Id^{\infty}(R)\neq \emptyset$; In particular, every ideal is finitely generated, so the ring is Noetherian. We may apply Zorn's lemma on the set of non-principally generated ideal (since every chain stablizes and has a maximal element), and let $P$ be a maximal non-principally generated ideal. Suppose it is not prime, and let $xy\in P$ with $x\not \in P$. Then, $P\subset (P:x)$ and $P\subset P+(x)$ properly. By maximality, we have $(P:x)=(c)$, and $(I:c)=(d)$. By definition, we have $cd\in I$; moreover, suppose $x\in I$, then $x=cr=cdt$ for some $r,t\in R$. Thus, $I=(cd)$ is principal, a contradiction.  
\end{proof}


\begin{tcolorbox}[colback=blue!5!white,colframe=blue!30!white]
\begin{proposition}
PIDs are UFDs.
\end{proposition}
\end{tcolorbox}


\begin{proof}
    Let $a\in R$ such that $a$ is non-zero and not a unit. Then, there exists $p\in Spec(R)$ such that $(a)\subseteq p$. Hence $R$ being a PID implies $\exists \pi\in R $ such that $p=(\pi)$. Hence, $\pi$ must be prime and $\pi|a$. Set $a_1=a$, $\pi_1=\pi$, and let $a_2$ be the element such that $\pi_1a_2=a_1$. If $a_2$ is not a unit, find $(a_2)\subset (\pi_2)$, where $\pi_2$ is prime. Let $a_3$ be the element such that $\pi_2a_3=a_2$. Continue inductively until $a_n$ is a unit. The process must terminate, for otherwise we get an infinite chain of distinct principal ideals $(a_i)$ that does not stablize( stablizing is equivalent to $(a_n)=(a_{n+1})$ for some $n$, which implies they differ by a unit).
\end{proof}



\begin{tcolorbox}[colback=green!5!white,colframe=green!30!white]
\begin{corollary}
    Let $R$ be a PID; let $P\subset R$ be a set of representatives for the prime elements up to association. For every $a\in R$, $\exists \epsilon\in R^{\times} $ and $e_{\pi}\in \mathbb{N} $ such that almost all $e_{\pi}=0$. Then, every $a\in R$ can be written as $a=\epsilon\prod_{\pi\in P}\pi^{e_{\pi}}$. We proceed to recover $gcd$ and $lcm$, up to associates.
\end{corollary}
\end{tcolorbox}

Note that the above corollary generalizes to the quotient field by replacing $\mathbb{N}$ with $\mathbb{Z}$.

\section{Unique Factorization Domains}


\begin{tcolorbox}[colback=purple!5!white,colframe=purple!75!black]
\begin{definition}
  The following are equivalent for a domain $R$:
  \begin{enumerate}
    \item $R$ is a UFD.
    \item Every  minimal prime ideal (prime of height $1$) is principal and every non-zero, non-invertible elements in contained in finitely many primes. 
  \end{enumerate}
\end{definition}
\end{tcolorbox}


\begin{proof}
   $1\implies 2$: For every non-zero prime $P$, pick $x\in P$ has factor. One of the prime factors must be in $P$, and it follows by minimality that $P$ must be generated by such prime factor. For the second part, the finite factorization of the element gives precisely the finite primes that it is contained in. 
   $2\implies 1$:given $x\in R$, the finitely many primes containing $x$ are principally generated by prime elements, which gives a factorization.

   
\end{proof}
Remark: we recover the $gcd$ and $lcm$ definition using the same factorization as Corollary $3.1.1$.


\begin{tcolorbox}[colback=red!5!white,colframe=red!30!white]
\begin{theorem}
    (Gauss Lemma)Let $R$ be a UFD; then $R[t]$ is a UFD. 
\end{theorem}
\end{tcolorbox}
To prove the theorem, we need the following lemma on contents:


\begin{tcolorbox}[colback=purple!5!white,colframe=purple!75!black]
\begin{definition}
    Let $f(t)=a_0+...+a_nt^n$ be given. Then, the \underline{\textbf{content}} of $f$, denoted $C(f)$, is the GCD of all coefficients. In particular, $C(f)|a_i$ for all $i$, and $f_0:=f/(C(f))$ has content $1$. 
\end{definition}
\end{tcolorbox}


\begin{tcolorbox}
\begin{lemma}
    Let $R$ be a UFD, then the following hold: $(1).$ $C(f): R[t]\to R$ given by $f\mapsto C(f)$ is multiplicative; in particular, if $C(f)=C(g)=1$, then $C(fg)=1$. 
\end{lemma}
\end{tcolorbox}

\begin{proof}[Proof of lemma $4.2$]
 given $f(t)=a_0+...+a_nt^n$ and $g(t)=b_0+...+b_mt^m$. If one of $f$, $g$ is constant, then it is easy exercise; suppose neither is constant, then set $f=f_0\cdot C(f)$ and $g=g_0\cdot C(g)$. Clearly we have $C(f)\cdot C(g)| C(fg)$. Hence it suffices to prove that $C(f_0g_0)=1$. Equivalently, let $\pi\in R$ be a prime element, we want to show there exists a coefficient $c_k\in f_0g_0$ such that $\pi$ does not divide $c_k$. Suppose $\pi| c_k=\sum_{i+j=k}a_ib_j$ for all $k$. Because $C(f_0)=C(g_0)=1$, then there exists minimal $a_i,b_j$ such that $\pi$ does not divide $a_{i_0},b_{j_0}$. Then, $\pi$ does not divede $C_{i_0+j_0}$.

\end{proof}



\begin{tcolorbox}[colback=blue!5!white,colframe=blue!30!white]
\begin{proposition}
    Let $K:=\textrm{Quot}(R)$, and $f\in K[t]$. Then, let $d$ be the least common multiple of the denominators of the coefficients of $f$. Then, $f=df/d$, and $df\in R[t]$. Define $C_{K}(f)=C(df)/d$. It is standard to check the analog for lemma $4.2$ holds for $C_K$ as well. 
\end{proposition}
\end{tcolorbox}


\begin{tcolorbox}[colback=blue!5!white,colframe=blue!30!white]
\begin{proposition}
Let $R$ be a UFD. For any irreducible $f\in R[t]$, either $f$ is a constant and thus prime in $R$, or $f$ is primitive, i.e $C(f)=1$.
\end{proposition}
\end{tcolorbox}
\begin{proof}
    If $f$ is a constant, the first part of the proposition is obvious; now suppose $f$ has degree $>0$; then $f$ can be factored into its primitive part and content; if $C(f)\neq 1$, we either have a non-trivial factorization of $f$ or $f$ will be a constant multiplied by a unit, a contradction.
\end{proof}

\begin{tcolorbox}[colback=red!5!white,colframe=red!30!white]
\begin{theorem}
    Let $R$ be a UFD. For $f(t)\in R[t]$, let $K:=\textrm{Quot}(R)$. Then, the following are equivalent: 
    \begin{enumerate}
        \item $f(t)$ is prime
        \item  $f(t)$ is irreducible
        \item  Either $f$ is an irreducible constant in $R$ or $f$ is irreducible in $K[t]$ and $C_K(f)=1$. 
    \end{enumerate} 
\end{theorem}
\end{tcolorbox}

\begin{proof}
    $1 \implies 2$ holds in every domain: suppose $a$ is prime and $a=bc$. Then by primeness, we have $a|b$ or $a|c$. WLOG, suppose $a|b$, such that $ax=b$ and $a=axc$, so $cx-1=0$, which implies $c$ is a unit. 

    $2 \implies 1$ in UFDs: suppose $f$ is an irreducible and $f|gh$, then we have some $l$ such that $fl=gh$. Because $g,h,l$ can be uniquely written as a product of irreducibles up to permutation and units, we see that the irreducible $f$ must appear on the RHS once, i.e $f|g$ or $f|h$. 

    For $2\implies 3$:  If $f$ is a constant, then it become a unit in the field of fractions; suppose $deg(f)>0$, so irreducibility implies $C(f)=1$. Suppose by contradiction that $f$ is reducible over $K[t]$, and let $f=gh$ for $g,h\in K[t]$ be a factorization in $K[t]$. Note that given $g,h\in K[t]$, there is some $x_g,x_h\in K$ such that $x_gg,x_hh\in R[t]$ and $C(x_hh)=C(x_gg)=1$. Then, $x_gx_hf=(x_gg)(x_hh)\in R[t]$. By Proposition $4.2$, we have $C(x_gx_hf)=x_gx_hC(f)=1$, which implies $x_gx_h=1$ (up to a unit in $R$). However, this implies $f=(x_gg)(x_hh)$, a contradiction. 
    
    So we are left to prove $3\implies 2$. Suppose $f$ is not a constant and $f$ primitive and irreducible. Suppose $f = gh \in R[x]$. WLOG $g$ is a unit in $K[x]$, so $g$ is a nonzero element of $R$. Now $g$ divides all the coefficients of $f$, so $g$ is a unit in $R$. 
\end{proof}
 

\begin{tcolorbox}[colback=blue!5!white,colframe=blue!30!white]
\begin{proposition}
$R[t_i]_{i\in I}$ is UFD if $R$ is UFD.
\end{proposition}
\end{tcolorbox}
\begin{proof}
   By induction it suffices to show that $R[t]$ is a UFD. The idea is that $K[t]$ is PID so it is a UFD. A factorization in $K[t]$ will correspond to a factorization in $R[t]$ by the equivalence of $2$ and $3$ in Theorem $4.3$.
\end{proof}



\section{Noetherian Rings}



\begin{tcolorbox}[colback=purple!5!white,colframe=purple!75!black]
\begin{definition}
A commutative ring $R$ is called a \underline{\textbf{Noetherian}} ring if every chain of ideals in $R$ is stationary.
\end{definition}
\end{tcolorbox}




\begin{tcolorbox}[colback=blue!5!white,colframe=blue!30!white]
\begin{proposition} The following are equivalent:
    \begin{enumerate}
        \item Every chain of ideals is stationary.
        \item All ideals are finitely generated.
        \item $Spec(R)\subseteq Id^f(R)$.    
    \end{enumerate}
    Terminology: the condition $1$ is called the ACC (Ascending Chain Condition).
\end{proposition}
\end{tcolorbox}
\begin{proof}
    By Cohen's lemma, we deduce $2\iff 3$; $1\iff 2$ is an easy exercise.
\end{proof}


For non-commutative rings, it is possible that a ring is left Noetherian but not right Noetherian.
 
 \begin{tcolorbox}[colback=yellow!5!white,colframe=yellow!30!white]
 \begin{example}
 $R=\{
 \begin{bmatrix}
 p&q\\
 0&m
 \end{bmatrix}: p,q\in \mathbb{Q}; m\in \mathbb{Z} \}$ is left Noetherian but not right Noetherian. 
 \end{example}
 \end{tcolorbox}





\begin{tcolorbox}[colback=blue!5!white,colframe=blue!30!white]
\begin{proposition}
(Basic Properties) Let $R$ be a Noetherian ring. The the following hold: 
\begin{enumerate}
    \item If $\mathfrak{a}$ is an ideal of $R$, then $R/\mathfrak{a}$ is Noetherian if $R$ is Noetherian.
    \item If $\Sigma\subset R$ is a multiplicative system, then $R_{\Sigma}$ is Noetherian. 
    \item The radical of an ideals $\mathfrak{a}$, $rad(\mathfrak{a})$, has a power contained in $\mathfrak{a}$.
    \item  Let $Spec_{min}(\mathfrak{a}):=\{ p\in Spec(R): \mathfrak{a}\subseteq p, \ p \ \textrm{minimal}  \}$ is finite.
\end{enumerate}
\end{proposition}
\end{tcolorbox}
\begin{proof}
    To $1$. Ideals in $R/\mathfrak{a}$ corresponds bijectively to ideals in $R$ that contains $\mathfrak{a}$. Finite generation of ideals in $R$ clearly implies the finite generation of ideals in the quotient. 

    To $2$. $Spec(R_{\Sigma})$ corresponds bijectively to primes in $Spec(R)$ with empty intersection with $\Sigma$. We also have $p$ finite generated implies $p^e$ f.g.

    To $3$. Suppose $rad(\mathfrak{a})=(r_1,.,,,r_n)$ f.g. For every $i$, we have $r_i^{n_i}\in \mathfrak{a}$ for some $n_i$. Take $n=\sum n_i$ and $nil(\mathfrak{a})^{n}\subset \mathfrak{a}$. 
    
    To $4$. The first method to prove this is by contradiction: let $A=\{ \mathfrak{a}: Spec_{min}(\mathfrak{a}) \ \textrm{is infinite} \}$. Then $A$ has maximal elements. Let $\mathfrak{a_0}$ be maximal. Note that $\mathfrak{a_0}$ cannot be prime for it is over itself. Suppose it is not prime, then there exists $xy\in \mathfrak{a}$ with both $x$ and $y$ not in $\mathfrak{a}$; for every prime ideal $P$ containing $\mathfrak{a}$, $P$ contains either $x$ or $y$. By pigeonhole, there must be infinite such primes containing either $\mathfrak{a}+(x)$ or $\mathfrak{a}+(y)$, which contradicts maximality. 

    The second method is using the fact that $Spec(R)$ is a Noetherian topological space, which has finitely many irreducible components.

\end{proof}
The third method is through primary decomposition. An ideal $I$ is irreducible if $I=a_1\cap a_2$ then, $I=a_1$ or $I=a_2$. For principal ideals, this is equivalent to the generator being irreducible. 


\begin{tcolorbox}[colback=blue!5!white,colframe=blue!30!white]
\begin{proposition}
If $R$ is Noetherian, then every ideal $I\in R$ is in the finite intersection of irreducible ideals in $R$. 
\end{proposition}
\end{tcolorbox}
\begin{proof}
    By contradction, let $X$ be the set of ideals that does not satisfy the proposition. Then, $X$ is non-empty, and by Noetherian assumption, there is a maximal element $\mathfrak{a_0}$. Then, $\mathfrak{a_0}$ is not irreducible, for it would be the intersection of itself. Therefore, there exists $I_0,I_1$ such that $a_0=I_0\cap I_1$, where $a_0$ is properly contained in both. By maximality, $I_0,I_1$ are both finite intersection of irreducibles, and we can decompose $a_0$ based on such, a contradction.
\end{proof}


\begin{tcolorbox}[colback=purple!5!white,colframe=purple!75!black]
\begin{definition}
Let $R$ be a commutative ring. Then an ideal $I\subset R$ is primary if for all $x,y\in R$ we have: if $xy\in I$, $x\not \in I$, then ther exists $n\in \mathbb{N}$ such that $y^n\in I$.
\end{definition}
\end{tcolorbox}
In general, a power of prime ideal is not primary. If $I=\mathfrak{m}^n$ for some maximal ideal $\mathfrak{m}$, then $I$ is in fact primary. 


\begin{tcolorbox}[colback=blue!5!white,colframe=blue!30!white]
\begin{proposition}
Let $R$ be Noetherian, and $\mathfrak{a}\in Id(R)$ be a irreducible ideal. Then, $\mathfrak{a}$ is primary, and $nil(\mathfrak{a})$ is prime. 
\end{proposition}
\end{tcolorbox}
\begin{proof}
    Exercise
\end{proof}
These two facts imply $Spec_{min}$ must be finite. In general, quotient of $UFD$ and $PID$ are not $UFD$ or $PID$. but this holds for Noetherian rings. 


\begin{tcolorbox}[colback=red!5!white,colframe=red!30!white]
\begin{theorem}
Let $R$ be a Noetherian ring. Then the following hold:
\begin{enumerate}
    \item (Hilbert Basis Theorem): $R[t_1,...,t_n]$ is Noetherian. 
    \item Every finitely generated $R$-algebra $S$ is Noetherian. 
    \item The power series ring $R[[x]]$
\end{enumerate}
\end{theorem}
\end{tcolorbox}
\begin{proof}
    Note that $1\implies 2$ since every finitely generated algebra is a quotient of polynomial rings over finitely many variable. To prove $1$, by induction it suffices to show for $i=1$. We now present a proof that applies for both $1$ and $3$. Let $I\in R[t]$ be an ideal. Claim: $I$ is f.g. Inductively, we may choose elements $f_i\in I$ with $deg(f_{i})$ being minimal in $I\setminus (f_1,...,f_{i-1})$. If the process terminates, then we are done; otherwise, let $a_i$ be the leading coefficient of $f_i$, and the chain of ideals $(I_i:=(a_1,...,a_i))$ must stablizes by Noetherian assumption on $R$. Suppose it stablizes at step $N$, and moreover suppose by contradction that $f_1,...,f_N$ does not generate $\mathfrak{a}$. Then, consider the elment $f_{N+1}$, which by our argument is not contained in $(f_1,...,f_N)$ and of minimal degree. The leading coefficient of $f_{N+1}$ is expressed as $a_{N+1}=\sum_{i=1}^{N}\mu_ia_i$. Then, we cook up 

    \[g=\sum_{i=1}^{N}\mu_if_ix^{deg(f_{N+1})-deg(f_i)}\]
    where $g\in (f_1,...,f_N)$ by construction, and $f_{N+1}-g\not\in (f_1,...,f_N)$. However, $f_{N+1}-g$ has degree strictly less than $f_{N}$ since we cancelled the leading term, which is impossible. 
\end{proof}

\section{Valuation Rings}


\begin{tcolorbox}[colback=blue!5!white,colframe=blue!30!white]
\begin{proposition}
    Let $R$ be a domain. Then the following are equivalent:
    \begin{enumerate}
        \item The ideals in $R$ are totally ordered by inclusion.
        \item The principal ideals in $R$ are totally ordered by inclusion, i.e $id(R)$ is a chain
        \item For every $x\in \textrm{Quot}(R)$, if $x\not \in R$ then $x^{-1}\in R$. 
    \end{enumerate}
\end{proposition}
\end{tcolorbox}
\begin{proof}
    $1\implies 2$ is trivial;  for $2 \implies 3$, suppose $\frac{a}{b}\not \in R$; then since the principal ideals are totally ordered, the elements are totally ordred by divisibility. Hence, $b\not | a$ implies $a|b$, so $\frac{b}{a}\in R$. For $3 \implies 1$, suppose we are given ideals $I,J$. If there exists $j\in J$ such that $j\not \in I$, then $\frac{i}{j}\in R$ for all $i\in I$, for otherwise there exists $i'$ such that $\frac{j}{i'}\in R$, which implies $j\in I$. Thus, $I\subseteq J$. 
\end{proof}

\begin{tcolorbox}[colback=purple!5!white,colframe=purple!75!black]
\begin{definition}
A ring $R$ satisfy one of the conditions above is called a (Krull) \underline{\textbf{Valutation Ring}}.
\end{definition}
\end{tcolorbox}


\begin{tcolorbox}[colback=yellow!5!white,colframe=yellow!30!white]
\begin{example}
$\mathbb{Z}_{(p)}=\{ \frac{q}{l}\in \mathbb{Q}: gcd(l,p)=1 \}$ is a valuation ring with maximal ideal $(p)$. The valuation on $v_p$ is defined by $v(\frac{q}{l})=r$ where $r$ is the maximal natural number such that $p^r|q$. The natural extension of such valuation on the entire $\mathbb{Q}$ is $v(\frac{p}{q})=v(p)-v(q)$. 
\end{example}
\end{tcolorbox}


\begin{tcolorbox}[colback=blue!5!white,colframe=blue!30!white]
\begin{proposition}
(Properties) Let $R$ be a valuation ring, and $K$ be its quotient field. The the following hold:
\begin{enumerate}
    \item $R$ is local, and $m=\{x\in R: x^{-1}\not \in R\}$. The maximal ideal is called \underline{\textbf{valuation ideal}} of $R$.
    \item $\Gamma_R:=K^{\times}/R^{\times}$ is totally ordered by $xR^{\times}\leq yR^{\times}$ iff $yR\subseteq xR$ iff $x|y$ in $R^{\times}$. The group is called the \underline{\textbf{value group}} of $R$.
    \item The natural map $v_R: K\to \Gamma_R\cup \{\infty\}$, $v(0)=\infty$ satisfies $v(xy)=v(x)+v(y)$ and $v(x+y)\geq min(v(x),v(y))$. Such map is called the (canonical) \underline{\textbf{valuation}} of $R$.
\end{enumerate}
\end{proposition}
\end{tcolorbox}
\begin{proof}
    To $1$, note that by Proposition $6.1.1$, the ideals are linearly ordered, so there exists a unique maximal ideal, and the ring is local. In a local ring, the maximal ideal is precisely the non-units. 

    To $2$, the statement is obvious from $6.1.2$ that elements in $R$ are totally ordered by divisibility.

    To $3$, it is clear that if $x|y$, then $x|x+y$. Therefore, $v(x+y)\geq min\{v(x),v(y)\}$. 
\end{proof}
Note $R$ is the set $\{ x\in K:v_R(x)\geq 0 \}$; $\mathfrak{m}$ is the set $\{ x\in K:v_R(x)> 0 \}$;



\begin{tcolorbox}[colback=purple!5!white,colframe=purple!75!black]
\begin{definition}
    Let $R$ be a domain, and $K$ be a field, $(\Gamma,+,\leq )$ be a totally orderedd abelian group. Let $v: K\to \Gamma\cup \{\infty\}$ be a map satisfying 
    \begin{enumerate}
        \item $v(x)=\infty$ iff $x=0$
        \item $v(xy)=v(x)+v(y)$
        \item $v(x+y)\geq min(v(x),v(y))$
    \end{enumerate}
    Then, the map $v$ is called a \underline{\textbf{valuation}} of $K$.
\end{definition}
\end{tcolorbox}

\begin{tcolorbox}[colback=blue!5!white,colframe=blue!30!white]
\begin{proposition}
$R_v=\{ x\in K:v(x)\geq 0 \}$ is a valuation ring. The map $\tau: \Gamma_{R_v}\to \Gamma$, given by $xR_v^{\times}\mapsto v(x)$ is an order preserving embedding. Moreover, $v=\tau \circ v_{R_v}: K\to \Gamma\cup \{\infty\}$.
\end{proposition}
\end{tcolorbox}
\begin{proof}
    It is easy to check $R_v=\{ x\in K:v(x)\geq 0 \}$ is a ring from the definition of a valuation above. To see that it is valuation ring, note that $v(\frac{x}{y})=v(x)-v(y)=-v(\frac{y}{x})$. Therefore one of them is $\geq 0$ and thus in $R_v$. The order on $\Gamma_{R_v}$ is given by $xR_v^{\times}\leq yR_v^{\times}$ iff $x|y$ in $R_v^{\times}$ iff $v(\frac{y}{x})\geq 0$ iff $v(x)\leq v(y)$. The final composition is easy to check by definition. 
\end{proof}

Given a valuation ring, $R\subset K$, every embedding of totally ordered groups $\Gamma_R\to \Gamma$ gives rise to a valuation.


\begin{tcolorbox}[colback=purple!5!white,colframe=purple!75!black]
\begin{definition}
    The following are equivalent definitions for equivalence of valuations on $K$: 
    \begin{enumerate}
    \item Two valuations $v,w$ on $K$ are equivalent if $R_v=R_w$. 
   \item  Two valuations $v,w$ on $K$ are equivalent if $\mathfrak{m}_v=\mathfrak{m}_w$
   \item Given $v: K\to \Gamma_v\cup \{\infty\}$ and $w: K\to \Gamma_w\cup \{\infty\}$, with embeddings $\tau_v: \Gamma_{R_v}\to \Gamma_v$ $\tau_w: \Gamma_{R_w}\to \Gamma_w$. Then, $v,w$ are equivalent if there exists an order preserving isomorphism $\tau_{vw}: \tau_v(\Gamma_{R_v})\to \tau_w(\Gamma_{R_w})$ that fits into the following commutative diagram
   \[
   \begin{tikzcd}
   \Gamma_{R_v}\arrow[r]&\tau_v(\Gamma_{R_v})\arrow[d,"\tau_{vw}"]\arrow[r]&\Gamma_v\\
   \Gamma_{R_w}\arrow[r]&\tau_w(\Gamma_{R_w})\arrow[r]&\Gamma_w
   \end{tikzcd}\]
\end{enumerate}
\end{definition}
\end{tcolorbox}
To see that the above definitions are indeed equivalent, note that $1\implies 2$ is trivial; for $2 \implies 1$, suppose there exists $a\in R_v-\mathfrak{m}_v$ such that $a\not \in R_w-\mathfrak{m}_w$. Then, by properties of a valuation ring, $a^{-1}\in R_w$ and in particular, it is not in the maximal ideal, so it is a unit, and $a\in R_w$. For $1\implies 3$: if $R_v=R_w$, then $\Gamma_{R_v}=\Gamma_{R_w}$ by definition. For $k\in \tau_v(\Gamma_{R_v})$, pick a representative $\tau_v ^{-1}(k)\in \Gamma_{R_v}=\Gamma_{R_w}$, and define $\tau_{vw}(k)=\tau_w(\tau_v ^{-1}(k))$. It is standard to verify the map is an order-preserving isomorphism. For the converse, the map is also easy to construct given the isomorphism $\tau_{vw}$. 



\begin{tcolorbox}[colback=purple!5!white,colframe=purple!75!black]
\begin{definition}
    A valuation ring $R$ is called \underline{\textbf{discrete}}, if $v_R(K)\cong \mathbb{Z}$ as ordered abelian groups. An element $\pi$ such that $v_R(\pi)$ generates $\mathbb{Z}$ is called a \underline{\textbf{uniformizing parameter}}.
\end{definition}
\end{tcolorbox}




\begin{tcolorbox}[colback=yellow!5!white,colframe=yellow!30!white]
\begin{example}
$\mathbb{Z}_{(p)}\subset \mathbb{Q}$ is a discrete valuation ring. The uniformation parameter is $p\epsilon$ with $\epsilon$ a unit. 
\end{example}
\end{tcolorbox}
A valuation ring $R$ is called rank $1$ if $v_r(K)$ satisfies the Archimedian axiom, i.e for $\forall \gamma_1,\gamma_2\in \Gamma_R, \gamma_1>0$, $\exists n\in \mathbb{N}$ such that $\gamma_2\leq n\cdot \gamma_2$. A totally ordered group $\Gamma$ is Archimedian if there is an ordered preserving embedding into the reals. In relation to absolute values,


\begin{tcolorbox}[colback=purple!5!white,colframe=purple!75!black]
\begin{definition}
An absolute value of a field $K$ is any map $|-|: K\to \mathbb{R}_{\geq 0}^+$ iff it satisfies the norm axioms. An absolute value is called \underline{\textbf{non-Archimedian}} or \underline{\textbf{ultra-metric}} if $|x+y|\leq max\{|x|,|y|\}$.
\end{definition}
\end{tcolorbox}


\begin{tcolorbox}[colback=yellow!5!white,colframe=yellow!30!white]
\begin{example}
    Let $|-|: K\to \mathbb{R}$ be a non-Archimedian absolute value. Then $v(-):=- log(|-|):K\to \mathbb{R}\cup \{\infty\} $ is rank $1$ valuation. Conversely, let $v: K\to \mathbb{R}\cup \{\infty\}$ be a rank one valuation, then $|-|_{v}:=e^{-v(-)}: K\to \mathbb{R}_{\geq 0}$ is a non-Archimedian absolute value. 
\end{example}
\end{tcolorbox}


\begin{tcolorbox}[colback=red!5!white,colframe=red!30!white]
\begin{theorem}
    The following facts about possible valuations
\begin{enumerate}
    \item If $K|F_p$ algebraic, then no non-trivial valuations exists on $K$.
    \item If $v$ is a valuation of $F(t)$ such $v$ is trivial on $F$, then $R_v=F[t]_{p(t)}$, where $p(t)$ irreducible or $R_v=F[\frac{1}{t}]_{(\frac{1}{t})}$. 
    \item If $v$ is a non-trivial valuation on $\mathbb{Q}$, then $R_v=\mathbb{Z}_{(p)}$ for some $p$ prime. 
\end{enumerate}
\end{theorem}
\end{tcolorbox}
\begin{proof}
    For $1$, let $K|F_p$ be an algebraic extension. Then, any element $a\in K$ is a root to the polynomial of the form $x^{p^k-1}-1$. A valuation on $K$ satisfies $0=v(1)=v(a^{p^k-1})=(p^k-1)v(a)=v(a)$. Thus, the valuation must be trivial.

    For $2,3$, refer to HW$7$ problem $6$.
\end{proof}



\begin{tcolorbox}[colback=red!5!white,colframe=red!30!white]
\begin{theorem}
(Ostrowski's Theorem)Every non-trivial absolute value on $\mathbb{Q}$ is equivalent to either the usual real absolute value or a p-adic absolute value.
\end{theorem}
\end{tcolorbox}

In general, the space of all valuations on $K$, denoted $Val(K)$, is called the Zariski-Riemann space. Moreover, $Val(K)$ carries a topology called a patch topology, or constrcutible topology, which makes the space compact and totally disconnected. The space is usually very complicated.


\begin{tcolorbox}[colback=red!5!white,colframe=red!30!white]
\begin{theorem}
(Chevalley's Theorem for extension of Valuations) Let $A$ be a domain, $p\in Spec(a)$ a prime ideal, Then, there exists a valuation ring $R$ of $K=Quot(A)$ such that $\mathfrak{m}_R\cap A=p$. 
\end{theorem}
\end{tcolorbox}
\begin{proof}
     Replace $A$ with $A_p$ if needed, so that we may assume $A$ is local with maximal ideal $p$. Let $H=\{ B\subset K: B \ \textrm{local}, \mathfrak{m}_B\cap A=p \}$. Then, it is easy to check that the union of a chain of ascending local rings is again a local ring, with maximal ideal containing $p$. Applying Zorn's lemma gives us the maximal local ring $R$ containing $A$ such that $\mathfrak{m}_R\cap A=p$. It remains to show that $R$ is local. 
     
     Suppose $x\in K$ but $x\not \in R$. Suppose neither $x,\frac{1}{x}$ is in $R$; if either $x,\frac{1}{x}$ is integral over $R$, then $R[x]$ has a maximal ideal lying over $p$. After localization, we get a local ring lying over $A$ that strictly contains $R$, which contradicts maximality. In particular, $\frac{1}{x}$ is not integral over $R$, and we claim that $\mathfrak{p}^e$ in $R[\frac{1}{x}]$ is not the entire ring: suppose other wise, then $1=a_0+\frac{a_1}{x}+...+\frac{a_n}{x^n}$, where $a_i\in p$. Multiplying $x^n$ to both sides yields $(1-a_0)x^n+a_1x^{n-1}+...+a_n=0$, and since $1-a_0$ is a unit, this shows $x$ is integral over $R$, a contradiction. Thus, $R[\frac{1}{x}]$ localized at $p^e$ gives us a local ring with maximal ideal $\mathfrak{m}'$ lying over $p$. ($p\subseteq A\cap \mathfrak{m}'$, then apply maximality ). This contradicts maximality of $R$, therefore one of $x,\frac{1}{x}$ is in $R$. 
\end{proof}

\section{Artin Rings}

\begin{tcolorbox}[colback=purple!5!white,colframe=purple!75!black]
\begin{definition}
A commutative ring $R$ is called \underline{\textbf{Artin}}, if every descending chain of ideals $(I_n)$ is stationary. 
\end{definition}
\end{tcolorbox}

\begin{tcolorbox}[colback=blue!5!white,colframe=blue!30!white]
\begin{proposition}
Let $R$ be Artinian. Then the following hold:
\begin{enumerate}
    \item If $\Sigma$ is a multiplicative system, then $\Sigma^{-1}R$ is also Artinian.
    \item If $I\subset R$ is an ideal. Then, $R/I$ is Artinian.
    \item An integral Artinian domain is a field. 
    \item $Spec(R)=Max(R)$ is finite. 
\end{enumerate}
\end{proposition}
\end{tcolorbox}
\begin{proof}
    To $1,2$, ideals under localization and quotients have nice correspondence with those in $R$ that respects inclusion. 

    To $3$, given any $a\neq 0 \in R$, where $R$ is an Artinian domain, the chain $(a)\subseteq (a^2)\subseteq (a^3)...$ must stablize, so $(a^{n+1})=(a^n)$ for some $n$. But this implies $a^n=a^{n+1}r$, which implies $a^n(1-ar)=0$. By $R$ being a domain, we get $a$ is invertible. 
    
    To $4$, let $p\in Spec(R)$. Then, $R/p$ is an Artinian domain. Then, $R/p$ must be a field. Thus, all primes are maximal.
    

    If $\mathfrak{m_1},\mathfrak{m_2}...$ is infinite, then we claim $\mathfrak{m}_1\supset \mathfrak{m}_1\mathfrak{m}_2\supset ...\supset \mathfrak{m_1}\mathfrak{m}_2\mathfrak{m}_3....$ does not stabilize: suppose otherwise $\mathfrak{m}_1 \mathfrak{m}_2...\mathfrak{m}_k=\mathfrak{m_1}...\mathfrak{m}_{k+1}\subseteq \mathfrak{m}_{k+1}$ for some $k$. By primeness, this implies $\mathfrak{m}_j\subseteq \mathfrak{m}_{k+1}$ for some $1\leq j\leq k $, which contradicts maximality. 
\end{proof}



\begin{tcolorbox}
\begin{lemma}
    If $R$ is Artin or Noetherian of Krull dimension $0$, then $J(R)=N(R)$ is nilpotent.
\end{lemma}
\end{tcolorbox}
\begin{proof}
    In Artinian rings or any ring of Krull dimension $0$, all prime ideals are maximal, and we get the equality $J(R)=N(R)$.
    
    In the case of $R$ is Artin, by DCC, $(N^n(R))_{n\in \mathbb{N}}$ stablizes at an ideal $I$ where $I\subseteq N(R) $. Suppose $I\neq 0$. Then, let $H$ be the set of all ideals of $R$ whose product with $I$ is not $0$. The set is non-empty since $I$ is in $H$; by artinian assumption, the set has a minimal element, call it $\mathfrak{a}$. By construction, there exists $x\in a$ such that $(x)I\neq 0$, so we must have $(x)=\mathfrak{a}$ by minimality. However, $((x)I)I=(x)I$, so $(x)I=(x)$. In particular, this implies $xi=x$ and consequently $xi^n=x$ for some $i\in N(R)$ and $n\in \mathbb{N}$. However, $i$ is nilpotent, which contradicts the assumption that $x\neq 0$. 

    In the case where $R$ is Noetherian, we simply note that $N(R)=rad((0))$, and $nil((0))^k\subseteq (0)$ for $k$ large enough by proposition $5.2.3$,
\end{proof}



If $R$ is Artin or Noetherian of dimension $0$, then every prime is both maximal and minimal, which means $Max(R)$ is finite. We now present a proof of structure theorem for Artin rings, with an arguement that also applies for Noetherian rings of dimension $0$ without knowing a priori that they are in fact equivalent. 
\begin{tcolorbox}[colback=red!5!white,colframe=red!30!white]
\begin{theorem}
    (Structure Theorem) If $R$ is Artin or Noetherian of dimension $0$ with $Max(R)=\{m_1,...,m_r\}$ is finite. Moreover, $R\cong R/(m_1)^n\times...\times R/m_r^n$. Hence, $R$ is a product of local Artinian rings. 
\end{theorem}
\end{tcolorbox}
\begin{proof}
    We know the $J(R)^n=(\cap_{i=1}^k \mathfrak{m}_i)^n=0$ for some $n$ by Lemma $7.1$. The goal is to use the Chinese Remainder Theorem and shwo that $R\cong R/(0)=R/J(R)$ has the desired form. First, we note that $\mathfrak{m}_i+\mathfrak{m}_j=1$ by maximality, so $(\mathfrak{m}_i)$ are pairwise coprime. Furthermore, this implies that $\mathfrak{m}_i^n+\mathfrak{m}^n_j=1$ for all $i,j$: if not, then there exists minimal prime $p$ over $\mathfrak{m}_i^n+\mathfrak{m}_j^n$, which implies $\mathfrak{m}_i^n\subseteq p$ and $\mathfrak{m}_i^n\subseteq p$, which in turn implies $\mathfrak{m}_i\subseteq p$ and $\mathfrak{m}_j\subseteq p$, which is impossible. Thus, $(\mathfrak{m}_i^n)$ are also pairwise coprime. It follows that $0=(J(R))^n=\prod \mathfrak{m}_i^n$, since intersection of ideals is product of ideals when the ideals are coprime. It is then a straight application of Chinese Remainder Theorem that $R\cong R/(m_1)^n\times...\times R/m_r^n$. 
    
    Lastly, note that each ring of the form $R/(\mathfrak{m}^k)$ is local: any suppose $\mathfrak{m}^k\subset p$ for $p$ prime, then for every $m\in \mathfrak{m}$, we have $m^k\in p$, so by primeness we have $m\in p$, and $\mathfrak{m}\subseteq p$. Thus, the only prime ideal is the image of $\mathfrak{m}$.
\end{proof}


\begin{tcolorbox}[colback=red!5!white,colframe=red!30!white]
\begin{theorem}
(Relations of Artin Rings and Noether Rings) Let $R$ be a commutative ring. The the following are equivalent: 
\begin{enumerate}
    \item $R$ is an Artin ring
    \item $R$ is Noether and Krull dimension of $R$ is $0$. 
\end{enumerate}
\end{theorem}
\end{tcolorbox}
\begin{proof}


Step one is reduce to the case where $R$ is local by structure theorem, since product of Noetherian rings is Noetherian and product of Artin rings is Artin. 
    
    
 Now assume $(R,\mathfrak{m})$ is a local Artin ring. For $k>0$, we have the exact sequence of $R$-modules
    \[\begin{tikzcd}
    0\arrow[r]&\mathfrak{m}^k/\mathfrak{m}^{k+1}\arrow[r,"i"]&R/\mathfrak{m}^{k+1}\arrow[r,"p"]&R/\mathfrak{m}^k\arrow[r]&0
    \end{tikzcd}\]
    where $i$ is the inclusion map and $p$ is the canonical projection. By proposition $9.2$, which we will prove latter, $R/ \mathfrak{m}^{k+1}$ is Noetherian provided both $R/ \mathfrak{m}^k$ and $\mathfrak{m}^k/\mathfrak{m}^{k+1}$ are Noetherian. Moreover, $R$ being Artinian implies $\mathfrak{m}^k=0$ for $k$ large enough, and we have $R/\mathfrak{m}^k\cong R$ for $k$ large enough. Our goal is to inductively show $R/\mathfrak{m}^{k}$ Noetherian for all $k$:  when $k=1$, $R/m$ is a field and thus Noetherian; now suppose $R/\mathfrak{m}^n$ is Noetherian. 

    Note $\kappa:= R/\mathfrak{m}$ is a field, and $\kappa$ acts on $\mathfrak{m}^n/\mathfrak{m}^{n+1}$ in the following way: $\overline{r}\cdot \overline{m} :=\overline{rm}$, So, $\mathfrak{m}^n/\mathfrak{m}^{n+1}$ has a canonical $\kappa$-vector space structure. 
    
    
    In particular, there is an inclusion preserving bijection 
    \[
        \{ \kappa -\textrm{vector subspaces of }\ \mathfrak{m}^n/\mathfrak{m}^{n+1} \} \iff \{ R-\textrm{ideals} \ \mathfrak{n} \ : \mathfrak{m}^{n+1}\subseteq \mathfrak{n}\subseteq \mathfrak{m}^n \}=\epsilon
    \]
    

    Note $R$ Artinian implies $R/\mathfrak{m}^{n+1}$ is Artinian. Thus, the set $\epsilon$ is finite, and $\mathfrak{m}^{n}/\mathfrak{m}^{n+1}$ is a finite dimensional vector space. This condition forces $\epsilon$ to satisfy both ACC and DCC, and by ideal correspondence, $\mathfrak{m}^k/\mathfrak{m}^{k+1}$ as an $R$-module satisfies ACC and is thus Noetherian. 


    
    
   For the converse, let $(R,\mathfrak{m})$ be a Noetherian local ring of dimension $0$. Note we also have $\mathfrak{m}^k=0$ for $k$ large enough, since $\mathfrak{m}=N(R)$, which is nilpotent by proposition $7.1$. 
    
    We proceed inductively as before: if $k=0,1$ then $R/\mathfrak{m}^k$ is clealy Artin. Now suppose it holds for $k=n$ such that $R/ \mathfrak{m}^n$ is Artin. By using the same arguemnt as before, $R/\mathfrak{m}^{n+1}$ is Noetherian and satisfies ACC, so $\mathfrak{m}^n/\mathfrak{m}^{n+1}$ is again finite dimensional, which forces $\mathfrak{m}^n/\mathfrak{m}^{n+1}$ satisfying DCC as well. 
\end{proof}

\section{Krull's Theorem on Noetherian Rings}



\begin{tcolorbox}[colback=purple!5!white,colframe=purple!75!black]
\begin{definition}
    Let $R$ be a commutative ring; $\mathfrak{a}\subset R$ a proper ideal. Consider $\mathfrak{a}^n$ and the projection $p_n: R/\mathfrak{a}^{n+1}\to R/\mathfrak{a}^n$. Then, $(R/\mathfrak{a}^n, p_n)_{n\in \mathbb{N}}$ is a projective system. The limit $\widehat{R}:=\varprojlim R/\mathfrak{a}^n$, together with $i: R\to \widehat{R}$ is called $\mathfrak{a}$\underline{\textbf{-adic completion}} of $R$. 
\end{definition}
\end{tcolorbox}


\begin{tcolorbox}[colback=blue!5!white,colframe=blue!30!white]
\begin{proposition}
    The kernel of the inclusion $\pi: R\to \widehat{R}$ is the intersection of all $\mathfrak{a}^n$.

\end{proposition}
\end{tcolorbox}
\begin{proof}
    We note $a\in ker(\pi)$ iff $\pi(a)=0$ iff $pr_n(a)=0$ for all $n$ iff $a\in \cap_{i=0}^{\infty}\mathfrak{a}^n$. 
\end{proof}
The reason we refer $i$ as the inclusion map is because when $R$ is Noetherian and local/integral, the kernel of $i$ is trivial by the following theorem by Krull.


\begin{tcolorbox}[colback=red!5!white,colframe=red!30!white]
\begin{theorem}
(The Intersection Theorem) Let $R$ be a Noetherian ring that is local or integral. Let $\mathfrak{a}\subset R$ be a proper ideal. Then, $\cap_{n=0}^{\infty} \mathfrak{a}^n=0$. In particular, the inclusion map in the $\mathfrak{a}$-adic completion is injection.
\end{theorem}
\end{tcolorbox}
\begin{proof}
 Suppose $R$ is Notherian and local with maximal ideal $\mathfrak{m}$. By Noetherian assumption, the ideal $\mathfrak{a}_0:=\cap \mathfrak{a}^k$ is finitely generated. Moreover, $\mathfrak{m}_0:=\cap \mathfrak{m}^k$ is f.g with $\mathfrak{a}_0\subseteq \mathfrak{m}_0$. We then have $\mathfrak{m}\cdot \mathfrak{m}_0=\mathfrak{m}_0$, and apply Nakayama's lemma, we get $\mathfrak{m}_0=(0)$. 
 
 Now suppose $R$ is Noetherian and integral, and choose $\mathfrak{m}$ be a maximal ideal over $\mathfrak{a}$. The integral assumption implies $\phi: R\to R_{\mathfrak{m}}$ is injective, and we reduce to the local case.
\end{proof}


\begin{tcolorbox}[colback=yellow!5!white,colframe=yellow!30!white]
\begin{example}
The intersection theorem does not hold for generic Noetherian Rings. For example, in $\mathbb{Z}/6$, which is not a domain nor local, and the ideal $I=(2)$ is idempotent. Thus, $\cap_{i=0}^{\infty}=I$. 
\end{example}
\end{tcolorbox}


\begin{tcolorbox}[colback=purple!5!white,colframe=purple!75!black]
\begin{definition}
Given a ring $R$ and $I$ an ideal, we equip $R$ with $I$\underline{\textbf{-adic topology}} given by the following basis $\{x+I^n: x\in R, n\in \mathbb{N}\}$. Moreover, a sequence of points $(x_n)$ is called \underline{\textbf{Cauchy}} if for every $k>0$, there exists $N$ such that for $m,n>N$, we have $x_n-x_m\in I^k$. 
\end{definition}
\end{tcolorbox}
It is standard to verify that this is well-defined basis. Heuristically, the larger $n$ the smaller the open neighborhood is. In particular, the intersection theorem says if $R$ is Noetherian and integral/local, then the $I$-adic topology is Hausdorff. (an element evetually lives outside of $I^n$ for $n$ large enough). Then, the $I$-adic completion $\widehat{R}_I$ is the topological completion of $R$. 

It is easy to extend the whole package of definitions up to this point to $R$-modules. Given an $R$-modules $M$ equipped with a choice of $I$-adic topology and a submodule $N$, it is natural to ask whether the subspace topology and $I$-adic topology on $N$ agrees. The Artin-Rees lemma gives us a positive answer in the case when the ring is Noetherian and $M$ is finitely generated. 


\begin{tcolorbox}[colback=red!5!white,colframe=red!30!white]
\begin{theorem}
(Artin-Rees Lemma) Let $R$ be a Noetherian ring and $I$ an ideal. Let $M$ be a finitely generated $R$-module and $N\subset M$ a submodule. Then, there exists an integer $k\geq 1$ such that for $n\geq k$, we have 
\[I^nM\cap N=I^{n-k}(I^kM\cap N)\]
\end{theorem}
\end{tcolorbox}

Before proving Theorem $8.2$, we first set up some necessary tools. 


\begin{tcolorbox}[colback=purple!5!white,colframe=purple!75!black]
\begin{definition}
Let $R$ be a ring and $I\subset R$ an ideal. Then, the \underline{\textbf{blow-up algebra}} of $R$ is the graded $R$-algebra
\[B_IR:=\oplus_{i=0} ^{\infty} I^i\]
\end{definition}
\end{tcolorbox}
Note that when $R$ is Noetherian, $I$ is finitely generated as an $R$-module, and the generators generate $B_IR$ as an $R$-algebra, which implies $B_IR$ is a Noetherian ring as well. 



\begin{tcolorbox}[colback=purple!5!white,colframe=purple!75!black]
\begin{definition}
Let $R$ be a ring and $I\subset R$ an ideal, and let $M$ be an $R$-module. A filtration $M=M_0\supset M_1\supset...$ is called an $I$\underline{\textbf{-filtration }} if $IM_n\subset M_{n+1}$ for all $n$. The filtration is called $I$-stable if $IM_n=M_{n+1}$ for $n$-large enough. Given an $I$-filtration $J$ of $M$, define the \underline{\textbf{blow-up}} module as $B_JM:=\oplus_{i=1}^{\infty}M_i$.
\end{definition}
\end{tcolorbox}
Note that $B_JM$ has a natural $B_IR$-module structure. We now introduce a proposition that relates stability and finite generation of blow-up modules. 


\begin{tcolorbox}[colback=blue!5!white,colframe=blue!30!white]
\begin{proposition}
Let $R$ be a ring, $I\subset R$ an ideal, and let $M$ a finitely generated $R$-module with $I$-filtration $J: M=M_0\supset M_1\supset...$, where each $M_i$ is finitely generated. Then, the filtration $J$ is $I$-stable iff the $B_IR$-module $B_JM$ is finitely generated. 
\end{proposition}
\end{tcolorbox}
\begin{proof}
    Easy Exercise. 
\end{proof}

We are now ready to prove Artin-Rees: 
\begin{proof}[Proof of Theorem 8.2]
    Note $ B_J M\cap N$ has a natural $B_I R$-module structure, which makes it a submodule of $B_JM$. In particular, if $J$ is an $I$-stable filtration of $M$, then $B_jM$ is finitely generated over a Noetherian ring $B_IR$, so the submodule $ B_J M\cap N$ is a finitely generated $B_I R$-module, which implies the desired equality. 
\end{proof}







\begin{tcolorbox}[colback=red!5!white,colframe=red!30!white]
\begin{theorem}
If $R$ is Noetherian, then all $\mathfrak{a}$-adic completions of $R$ is Noetherian. 
\end{theorem}
\end{tcolorbox}
\begin{proof}
    Let $(f_1,...,f_n)$ be a set of generators for a given $\mathfrak{a}$. There is a natural surjection from the power series ring $R[[x_1,...,x_n]]\to \widehat{R}_{\mathfrak{a}}$ given by the map $x_i\mapsto f_i$. Then, $\widehat{R}_{\mathfrak{a}}$ is a quotient of a Noetherian ring and is thus Noetherian.
\end{proof}






\begin{tcolorbox}[colback=purple!5!white,colframe=purple!75!black]
\begin{definition}
Let $R$ be a ring. For $r\in R$, define $Spec_{min}(r):=\{p\in Spec(R) : (r)\subset p \ \textrm{minimal}\}$. For a set of elements $\{r_1,...,r_n\}$, define similarly $Spec_{min}(r)=\{p\in Spec(R) : (r_1,..,r_n)\subset p \ \textrm{minimal}\}$
\end{definition}
\end{tcolorbox}



\begin{tcolorbox}[colback=purple!5!white,colframe=purple!75!black]
\begin{definition}
    For $p\in Spec(R)$, the \underline{\textbf{height}} of $p$ is the krull dimension of $R_{p}$. The \underline{\textbf{coheight}} is the krull dimension of $R/p$. 
\end{definition}
\end{tcolorbox}


\begin{tcolorbox}[colback=blue!5!white,colframe=blue!30!white]
\begin{proposition}
$height(p)+coheight(p)\leq$ Krull dimension of $R$.
\end{proposition}
\end{tcolorbox}
\begin{proof}
    Trivial. 
\end{proof}


\begin{tcolorbox}[colback=purple!5!white,colframe=purple!75!black]
\begin{definition}
    For $q\in Spec(R)$, the symbolic $n$-th power of $q$ is defined as $q^{(n)}:=q^nR_{q}\cap R$. In other words, $q^{(n)}=\{ r\in R: sr\in q^n \ \textrm{for some} \ s\in R\setminus q : \}$
\end{definition}
\end{tcolorbox}

\begin{tcolorbox}
    \begin{lemma}
    $q^{(n)}R_q=(qR_q)^{n}$. '

    \end{lemma}
    \end{tcolorbox}
    \begin{proof}
        Suppose $x\in (qR_q)^{n}$, then $x=x_1...x_n$ where $x_i=\frac{r}{s}$, where $r_i\in q$ and $s_i\in R\setminus q$. It is clear that $(\prod s_i)x\in q^n$, so  
        $x=\frac{\prod x_i}{\prod s_i}\in q^{(n)}R_q$; on the other hand, if $y\in q^{(n)}R_q$, then $y=\frac{m}{n}$ where $m\in q^{(n)}$ and $n\in R\setminus q$. By definition, there exists $s\in R/q$ such that $sm= q_1...q_n\in q^n$, where $q_i\in q$. Then, $y=\frac{m}{n}=\frac{q_1...q_n}{sn}=\prod \frac{q_i}{sn}\in (qRq)^n$.
    \end{proof}
    
    
    \begin{tcolorbox}
    \begin{lemma}
     For $q\in Spec(R)$, the $n$th symbolic power $q^{(n)}$ is primary. If $ax\in q^{(n)}$, and $x\not \in q$, then $a\in q ^{(n)}$. 
    \end{lemma}
    \end{tcolorbox}
    \begin{proof}
        Note that $q^{(n)}$ is the contraction of the ideal $q^nR_q$, which is a power of maximal ideal and thus primary. Thus, $q^{(n)}$ is primary as well. By definition, if $ax\in q^{(n)}$, then $a(sx)\in q^n$ with $s,x\not \in q$, which implies $a\in  q^{(n)}$.
    \end{proof}



\begin{tcolorbox}[colback=red!5!white,colframe=red!30!white]
\begin{theorem}
(Krull's Principal Ideal Theorem/ Hauptidealsatz) Let $R$ be a Noetherian ring. Then, for all non-units $r\in R$, one has $height(q)\leq 1$ for all $q\in Spec_{min}(r)$, with equality when $r$ is not a zero-divisor.
\end{theorem}
\end{tcolorbox}
\begin{proof}
    Suppose there exists a chain $q_0\subset q$ of prime ideals, and we want to show that $height(q_0)=0$, so that $height(q)\leq 1$. We may localize at $q$ so that we may assume $R$ is local with maximal ideal $q$. By the assumption that $p$ is minimal over $r$, the ring $R/(x)$ is Noetherian and of dimension $0$, hence Artinian. Thus, the chain 
    \[(r)+q_0^{(n)}\]
    stablizes. Say we have $(r)+q_0^{(k)}=(r)+q_0^{(k+1)}$. It follows that $q_0^{(k)}\subset (r)+q_0^{(k+1)}$, so for any $f\in q_0^{(k)}$ we may write $f=ar+g$ with $g\in q_0^{(k+1)}$. It is immediate that $ar\in q_0^{(k)}$, but $r\not \in q_0$ by minimality, so $a\in q_0^{(k)}$
    
    From this we have $q_0^{(k)}=(x)q_0^{(k)}+q_0^{(k+1)}$. Taking things modulo $q_0^{(k+1)}$, we have $x\in J(R)$, and an application of Nakayama's lemma says $q_0^{(k)}=q_0^{(k+1)}$. We further localize to $R_{q_0}$, and Lemma $8.4$ and another application of  Nakayama's lemma gives us $(q_0R_{q_0})^k=0$. In other words, the maximal ideal $q_0R_{q_0}$ is nilpotent in the local ring $R_{q_0}$. It follows that $q_0R_{q_0}\subseteq N(R_{q_0})$, which forces $q_0R_{q_0}$ to be the unique prime ideal. We have $R_{q_0}$ is of dimension $0$, as desired. 

    For the second part of the statement, if $height(q)=0$, then $q$ is nilpotent in $R_q$, and let $n$ be minimal such that $r^n=0\in R_q$, which implies $sr^n=0\in R$ for some $s\neq 0$. By minimality, $sr^{n-1}\neq 0$, so $r$ must be a zero divisor. 

\end{proof}






\begin{tcolorbox}[colback=purple!5!white,colframe=purple!75!black]
\begin{definition}
A sequence of elements $r_1,...,r_n$ is called a \underline{\textbf{regular}} sequence if $(x_1,...,x_d)$ is a proper ideal for all $d\leq n$, and $r_i$ is not a zerodivisor in $R/(r_1,...,r_{i-1})$ for all $i\leq n$. 
\end{definition}
\end{tcolorbox}

We have a generalization of the PIT for a system of elements:
\begin{tcolorbox}[colback=red!5!white,colframe=red!30!white]
\begin{theorem}
(Krull's Dimension Theorem) Let $R$ be a Noether ring, and $r=(r_1,...,r_m)$ a system. Then $Spec_{min}(r)$ contains prime ideals of height $\leq m$, with equality when $r$ is regular.
\end{theorem}
\end{tcolorbox}
\begin{proof}


    We proceed by induction: $n=1$ is PIT; now assume the dimension theorem holds for $n=m$. Given $r=(r_1,...,r_{m+1})$, and $p\in Spec_{min}(r)$, let $q\subset p$ be a maximal prime ideal contained in $p$. Our goal is to show that $ht(q)=m$, which immediately implies that $ht(p)=m+1$. By localizing at $p$, we may assume that $R$ is local with maximal ideal $p$. 

    Since $q$ is properly contained in $p$, we have WLOG that $r_{m+1}\not\in q$ by minimality.  Consider $\mathfrak{a}=q+(r_{m+1})$, $q\subset \mathfrak{a}\subseteq p$. Then, $nil(\mathfrak{a})=p$ since $p$ is the only prime ideal containing $a$. By definition, we have $r_i\in p$ for all $i=1,...,m+1$, and there exists $a_i\in R$ and $s_i\in q$ such that $r_i^{n_i}=s_i+a_ir_{m+1}$. Thus, we have $r_i^{n_i}\in (s_1,...s_m,r_{m+1})$, and a prime containing $(s_1,...s_m,r_{m+1})$ will contain all $r_i$ as well. It follows that $p$ is minimal over $(s_1,...s_m,r_{m+1})$. Let $s=(s_1,...,s_m)$. The image of $p$ under the quotient map $R\to R/s$ is minimal over $r_{m+1}$. Therefore by PIT, $\overline{p}$ has height at most $1$, which forces the image of $q$ having height $0$, which means $q$ is minimal over $(s_1,...,s_m)$. By induction hypothesis, we are done. 
    
    Note that in our proof, $\overline{p}$ has height $1$ when $r_{m+1}$ is not a zero-divisor under the quotient by PIT, which is equivalent to saying the system is regular. 
    
\end{proof}


\begin{tcolorbox}[colback=green!5!white,colframe=green!30!white]
\begin{corollary}
Let $R$ be Noether. Then, the following hold:
\begin{enumerate}
    \item Every descending sequence of prime ideals is staionary. 
    \item if $ht(p)=m$, then there exists a regular system of length $m$ with $p$ a minimal prime over it.
\end{enumerate}
\end{corollary}
\end{tcolorbox}

\begin{proof}
    To $1$: every prime ideal in a Noetherian ring is finitely generated. In particular, given $p$ we can find a system of generators $(r_1,...,r_m)$ for $p$ such that $p$ is minimal over the system by definition. Then, $ht(p)\leq m$ by dimension theorem.

    To $2$: we proceed by induction: it is trivial if $m=1$ by taking the system $r=(0)$. Inductively suppose $m=k+1$. Let $p_1\subset...\subset p_k\subset p_{k+1}=p$ be a chain of length $k+1$. Then, $p_k$ is minimal over a regular system $(x_1,...,x_k)$. First, quotient out the bottom prime so the ring is assumed to be integral. By Noetherian assumption, there is only a finite set of primes $\{q_i\}$ minmial over $(x_1,...,x_k)$. Then by prime avoidance, $p$ cannot be contained in the union of $\{q_i\}$, otherwise contradicting minimality. Therefore, we may choose an element $x_{k+1}\not \in (x_1,...,x_k)$ such that $p$ is minimal over $(x_1,...,x_{k+1})$, and it is regular. 

\end{proof}

\section{Modules over special classes of rings}
\subsection{Modules over PIDs}






The motivating fact for the following lemma is this: given a non-zeron functional on a finite dimensional real vector space $\phi: V\to \mathbb{R}$, the range is one dimensional, say generated by $v\in V$. Then, we may decompose $V$ as $V=span(v)\oplus ker(\phi)$.
\begin{tcolorbox}
\begin{lemma}
Let $R$ be a PID and $M$ a free $R$-module. Given a submodule $N\subset M$, there exists $y,y_1\in N $ and $v\in Hom_R(M,R)$ such that the following hold:
\begin{enumerate}
    \item $M=Ry_1\oplus ker(v)$;
    \item $N=Ry\oplus (N\cap ker(v))$
\end{enumerate}
\end{lemma}
\end{tcolorbox}
\begin{proof}
    The proof is trivial if $N=0$, so assume $N$ is not trivial. First, note that for any $\phi\in Hom_R(M,R)$, the image $\phi(N)$ is an ideal of $R$ and thus principally generated by some element $a_{\phi}\in R$. Let \[\Sigma= \{ a_{\phi}:\phi\in Hom_R(M,R) \}\]
    Then, $\Sigma$ is not empty becuase $0\in \Sigma$. Since PID are noetherian, $\Sigma$ has a maximal element. Let $v$ be the homomorphism such that $v(N)=(a_v)$ is maximal, and $y\in N$ be the element such that $v(y)=a_1$. To see that $a_1$ is not trivial, it suffices to demonstrate one homomorphism where $N$ is not contained in the kernel. Let $(x_1,...,x_n)$ be a basis for $M=\oplus_{i=1}^n Rx_i$. Since $N\neq 0$, there must the projection map onto the $i$th summand restricts to a homomorphism where $N$ is not contained in the kernel. 
 
    The next step is to demonstrate $a_1$ divides all $\phi(y)$ for $\phi\in Hom_R(M,R)$. Note that ideal generated by $a_1$ and $\phi(y)$ is principal, and let $b$ be its generator. Then, we may write $b=r_1a_1+r_2\phi(y)$ for some $r_1,r_2\in R$. Consider the homomorphism $r_1v+r_2\phi\in Hom_R(M,R)$, which sends $y$ to $r_1v(y)+r_2\phi(y)=b$. Therefore by maximality, we must have $(a_1)=(b)$, and it follows that $a_1| \phi(y)$.
 
    In particular, we have $a_1|\pi_i(y)$, where $\pi_i$ is the projection onto the $Rx_i$ summand. In other words, $y=\sum_{i=1}^n(a_1b_i)x_i$ for $b_i\in R$. By factoring out the $a_1$ term from the coefficients, we get $y_1:=\sum_{i=1}^n(b_i)x_i$ where $v(y_1)=1$. The claim is that $M=Ry_1\oplus ker(v)$ and $N=Ry\oplus (N\cap ker(v))$. For the first equality, we note that every $x\in M$ can be written as $x=v(x)y_1+(x-v(x)y_1)$, where $(x-v(x)y_1)\in ker(v)$ by a direct verification. For the second equality, for every $x'\in N$, we have $x'=v(x')y_1+(x'-v(x')y_1)$. Note that $a_1|v(x')$, so $v(x')y_1\in Ry$; by similarly reasoning, we have $(x'-v(x')y_1)=\in N$ and $v(x'-v(x')y_1)=0$. Both sums are easily seen to be direct.
\end{proof}



\begin{tcolorbox}[colback=red!5!white,colframe=red!30!white]
\begin{theorem}
Every submodule of a finitely generated free module over PID is free.
\end{theorem}
\end{tcolorbox}
We use induction on rank. Suppose $N\subset M$ is of rank $0$, then it must be torsion and any non-zero submodule of a free module is torsion free. Thus, $N=0$ and it is free. Suppose the statement holds for submodules of rank $m$. For submodule $N$ of rank $m+1$, we decompose $N= Ry\oplus N\cap ker(v)$, where $ N\cap ker(v)$ must be of rank $m$. It follows from the induction hypothesis that $N$ is a direct sum of free modules and thus free. 

Note that we may alter the proof slightly by choosing a well-ordered basis for $M$ if it is not finitely generated and use transfinite induction to prove the result in general.



\begin{tcolorbox}[colback=red!5!white,colframe=red!30!white]
\begin{theorem}
(Invariant Factors Theorem) Let $R$ be a principal ideal domain and $M$ a free $R$-module, $N\subset M$ a submodule. Then, there exists $R$-basis $A=(\alpha_1,...,\alpha_m)$ of $M$ and $\delta_1|\delta_2|...|\delta_n$ in $R$ such that $\delta_1\alpha_1...,\delta_n\alpha_n$ is an $R$-basis for $N$, unique up to association. 
\end{theorem}
\end{tcolorbox}
\begin{proof}
We induct on rank of $M$: if rank of $M=0$, then there is nothing to prove. Suppose the statement holds for $rk(M)=n$. Since $M=Ry_1\oplus ker(v)$, we know there is a basis $y_2,...,y_{n}$ of $ker(v)$ and $\delta_2|...|\delta_n$ such that $\delta_2\alpha_2...,\delta_n\alpha_n$ is an $R$-basis for $N\cap ker(v)$. We are left to show that $\delta_1:=a_1$ divides all $\delta_i$, and in particular it suffices to prove $\delta_1|\delta_2$. The proof follows from the similar vein as in Lemma $9.1$, based on the maximality of $\delta_1$.  
   
\end{proof}

    \begin{tcolorbox}[colback=red!5!white,colframe=red!30!white]
    \begin{theorem}
    (Structure Theorem) Let $R$ be a PID, and $M$ a finite $R$-module. Then, there exists non-units $\delta_1|...|\delta_n$ unique up to association such that $M\cong \oplus R/(\delta_i)\oplus R^f$
    \end{theorem}
    \end{tcolorbox}
\begin{proof}
    Let $(x_1,...,x_n)$ be a system of generators for $M$. Let $f:R^n\to M$ be the morphism given by $e_i\mapsto x_i$. Then, the kernel is a submodule of $R^n$, so by invariant factors theorem we get a basis $(e_1',...,e_n')$ for $R^n$ and a basis  $\delta_1e'_1,...,\delta_me'_m$ for $ker(f)$. By isomorphism theorem, we have 
    \[M\cong R^n/ker(f)=Re'_1\oplus ...\oplus Re'_n/R\delta_1e'_1\oplus ...\oplus \delta_me'_m\cong \oplus R/(\delta_i)\oplus R^{n-m}\]
    
    
    For uniqueness, given $M\cong \oplus R/(\delta_i)\oplus R^f$ and the projection $p:R^n\to M$. We get $
    N=ker(p)$ has basis required in the invariant factors theorem, which is unique. 
\end{proof}


\begin{tcolorbox}[colback=green!5!white,colframe=green!30!white]
\begin{corollary}
The following hold for finitely generated modules over PID:
\begin{enumerate}
    \item $M$ is torsion free iff $M$ is free.
    \item The torsion submodule of $M$ is finitely generated.
\end{enumerate}
\end{corollary}
\end{tcolorbox}


\begin{tcolorbox}[colback=yellow!5!white,colframe=yellow!30!white]
\begin{example}
For a finitely generated abelian group $A$, $A\cong \mathbb{Z}/(d_1)\oplus...\oplus \mathbb{Z}/(d^r)\oplus \mathbb{Z}^f$
\end{example}
\end{tcolorbox}


\begin{tcolorbox}[colback=yellow!5!white,colframe=yellow!30!white]
\begin{example}
(Smith Normal Form) Given an $n\times m$ matrix $A$ with entries in PID, there exists a decomposition $A=LDR$, where $L,R$ are invertible matrices representing row and column operations, and $D$ is a diagonal matrix of the form 
\[
\begin{bmatrix}
\delta_1\\
&\delta_2\\
&&...\\
&&&\delta_n
\end{bmatrix}\]
where $\delta_1|...|\delta_n$. The diagonal matrix is called the Smith Normal Form of $A$. In the context of Invariant factor theorem, the decomposition says that under the basis change to $y_1,...,y_m$ given by $R$, the vectors $\delta_1y_1,...\delta_ny_m$ spans the range, under the base change $L$.
\end{example}
\end{tcolorbox}
The algorithm of reducing a matrix $A$ to the smith normal form is as follows: starting with the first column, we may use elementary row operations to reduce the $1,1$ entry to the $d=gcd(a_{1,1},a_{2,1})$: $R$ being a PID implies there exists $r_1,r_2\in R$ such that $r_1a_{1,1}+r_2a_{2,1}=d$ (note that having a Euclidean Algorithm will make this actually algorithmically computable instead of theoretically exists). The row operation corresponds to the matrix 
\[
\begin{bmatrix}
r_1& r_2\\
-a_{1,1}/d& a_{2,1}/d
\end{bmatrix}\]
which has determinant $1$ and thus invertible. Now we can subtract and get rid of all entries in the first column other than $a_{1,1}=d$. Do the same for the first row, which possibly adds new entries back to the first column, but the number of prime factors of $a_{1,1}$ reduces, therefore the algorithm must terminate.

Now we have obtained a diagnonal matrix. To put it into the desired form, suppose $\delta_1 \not| \ \delta_2$. Then, we may add $\delta_2$ back to the first column, and perform the same operations to turn $\delta_1$ into $gcd(\delta_1,\delta_2)$. By the same reasoning, the process terminates. 

\begin{tcolorbox}[colback=yellow!5!white,colframe=yellow!30!white]
\begin{example}
    (Rational Canonical Form) Let $k$ be a field and $V$ a finite dimensional vector space over $k$. Fix some $\varphi\in End(V)$. Then, $V$ becomes a $k[t]$-module by 
    \[p(t)\cdot v=p(\varphi)(v)\]
     By Cayley-Hamilton, $V$ is a finite-torsion $F[t]$ module.  Hence, $V\cong F[t]/(\delta_1)\oplus...\oplus F[t]/(\delta_n)$, with $\delta_1|...|\delta_n$. Let $\delta_i=t^{n_i}+a_{n_i-1}t^{n_i-1}...+a_0$. Then $R_i:=R/(\delta_i)$ has basis $(1,t,...,t^{n_i-1}) $, The action of $t$ on the basis vectors is $t\cdot x^k= x^{k+1}$ for $k<n_i$ and $t\cdot x^{n_i}= -(a_{n_i-1}t^{n_i-1}...+a_0)$. Thus, each $R_i$ has the matrix form
     \[
     \begin{bmatrix}
     &&&&&&-a_0\\
     1&&&&&&-a_1\\
    &1&&&&& -a_2\\
    &&1&&&& -a_3\\
    &&&...\\
    &&&&&1& -a_{n_i-1}
     \end{bmatrix}\]
     and $V=R_1\oplus ...\oplus R_n$.
\end{example}
\end{tcolorbox}
 

\begin{tcolorbox}[colback=blue!5!white,colframe=blue!30!white]
\begin{proposition}
Given a $n\times n$ matrix $A$, the invariant factors $\delta_1,...,\delta_n\in k[t]$ can be determined by reducing the matrix $A-tI$ to the smith normal form.   
\end{proposition}
\end{tcolorbox}
\begin{proof}
    It is easy to prove the lemma that each block $R_i-tI$ has a smith normal form \[
    \begin{bmatrix}
    \delta_i\\
    &1\\
    &&1\\
    &&&\dots \\
    &&&&1
    \end{bmatrix}\]
    The proposition follows from the fact that two matrices $A,B$ are similar iff their characteristic matrices are equivalent (i.e you can get from the other through elementary matrix operations). In particuar, the minimal polynomial of $\phi$ is $\delta_n$, and the characteristic polynomial is the product $\prod_{i=1}^{n}\delta_i$.
\end{proof}


\begin{tcolorbox}[colback=yellow!5!white,colframe=yellow!30!white]
\begin{example}
We find the invariant factors of the matrix 
\[
    A=
    \begin{bmatrix}
    1&0&0&0\\
    0&1&1&0\\
    0&0&1&0\\
    -1&0&0&0
    \end{bmatrix}
\]
It is easy to calculate the characteristic polynomial to be $t(t-1)^3$, and the minimal polynomial is $t(t-1)^2$. We see that this forces the invariant factors to be $t-1$ and $t(t-1)^2$. The may sanity check by reducing the characteristic matrix to the smith normal form
\[
    A-tI \Longrightarrow
    \begin{bmatrix}
    1&0&0&0\\
    0&1&1&0\\
    0&0&t-1&0\\
    0&0&0&t(t-1)^2
    \end{bmatrix}
\]
This means the rational canonical form of $A$ is 
\[
 \begin{bmatrix}
 1&0&0&0\\
 0&0&0&0\\
 0&1&0&-1\\
 0&0&1&2\\
 \end{bmatrix}  
\]
\end{example}
\end{tcolorbox}



\subsection{Noetherian/Artinian Modules}
Let $R$ be a (not necessarily commutative) ring, and $M$ be a (left/right/bi) module. We say that $M$ satisfies ACC/DCC iff thet set of submodules satisfies ACC/BCC with respect to inclusion.


\begin{tcolorbox}[colback=yellow!5!white,colframe=yellow!30!white]
\begin{example}
If $R$ is a Noetherian/artinian ring. Then it is a Noetherian/Artinian module over itself. 
\end{example}
\end{tcolorbox}


\begin{tcolorbox}[colback=blue!5!white,colframe=blue!30!white]
\begin{proposition}
(Characterization) Let $M$ be an $R$-module. Then the following hold: 
\begin{enumerate}
    \item $M_{\cdot}$ satisfies ACC/DCC if every subset of submodules has maximal/minimal elements with respect to inclusions.
    \item $M_{\cdot}$ satisfies ACC iff every submodule is finitely generated. 
\end{enumerate}
\end{proposition}
\end{tcolorbox}
\begin{proof}
    To $1$: Suppose $X$ is a subset of submodules. If the subset has no maximal/minimal elements, then there exists a non-stablizing ascending/descending chain of submodules, so $M$ cannot satisfy ACC/DCC. Conversely, if there is a infinite ascending/descending chain of submodules of $M$, then collection of the submodules in the chain is a subset with no maximal/minimal elements. 

    To $2$: If $N\subseteq M$ is not finitely generated, we may inductivly choose elements in $x_i\in M\setminus M_{i-1}$, where $ M_{i-1}:=(x_1,...,x_{i-1})$ is the module generated by the elements in the parenthesis. Then, $(M_i)_{i\in \mathbb{N}}$ is a non-stablizing ascending chain. Conversely, if $(M_i)_{i\in \mathbb{N}}$ is a non-stablizing ascending chain of submodules, then $\cup_{i=0}^{\infty} M_i$ is a submodule that is not finitely generated. 
\end{proof}



\begin{tcolorbox}[colback=blue!5!white,colframe=blue!30!white]
\begin{proposition}
(Properties) The following hold:
\begin{enumerate}
    \item If $M$ satisfies ACC/DCC, then every submodule of $M$ and quotient module of $M$ satisfies ACC/DCC.
    \item The category of $R$-modules satisfiying ACC/DCC has finite products and coproducts. 
    \item Localization preserves ACC/DCC. 
 \end{enumerate}
\end{proposition}
\end{tcolorbox}
\begin{proof}
    To $1$: Trivial.
    To $2$: Consider the projection $p: M\to M/IM$. The inverse image $p^{-1}$ takes a submodule to a submodule, and it is (proper) inclusion preseving. Thus, every ascending/descending chain in $M/IM$, $M/IM$ lifts to an ascending/descending chain in $M$.  
    To $3$: In $\textbf{R-Mod}$, finite product and coproducts agree, and it suffices to consider the direct product $M\times N$. If $M\times N$ has ascending/descending chain of submodules, then the projection map onto $M$ and $N$ takes the chain to ascending/descending chains as well. If both chains stablize after some finite degree $n$, then it is clear that the original chain stablize after degree $n$ as well.
    To $4$: consider the inclusion $i: M\to \Sigma ^{-1}M$. The inverse image $i^{-1}$ takes a submodule to a submodule, and it is (proper) inclusion preseving(a submodule in $\Sigma ^{-1}M$ is equal to the localization of its contraction). Thus, every ascending/descending chain in $\Sigma ^{-1}M$ lifts to an ascending/descending chain in $M$. 
\end{proof}


\begin{tcolorbox}[colback=blue!5!white,colframe=blue!30!white]
\begin{proposition}
    For $R$-module $M$, the following hold:
\begin{enumerate}
    \item Given a short exact sequence \[\begin{tikzcd}
    0\arrow[r]&M_0\arrow[r]&M_1\arrow[r,"p"]&M_2\arrow[r]&0
    \end{tikzcd}\]
    We have $M_1$ satisfies ACC/DCC iff $M_0$ and $M_2$ satisfies ACC/DCC.
    \item Let \[\begin{tikzcd}
        0\arrow[r]&M_0\arrow[r]&M_1\arrow[r]&...\arrow[r]&M_n\arrow[r]&0
        \end{tikzcd}\]
       Then, $(M_{2k})$ satisfies ACC/DCC iff $(M_{2k+1})$ does so.
\end{enumerate}
\end{proposition}
\end{tcolorbox}
\begin{proof}
    To $1$: assume $M_1$ satisfies ACC/DCC: then $M_0$ is canonically a submodule of $M_1$ and $M_2$ is a quotient $M_1$, so they satisfy ACC/DCC by Proposition $9.2.1$; now supoose $M_1$ does not satisfy ACC/DCC, which means there is a non-stablizing ascending/descending chain $C=(C_n)$ of submodules. Now $C\cap M_1$ is naturally a chain of submodules of $M_0$, and $p(C)$ is an ascending/descending chain of submodules of $M_2$. Suppose by contradiction that both chain stabilizes, which means there exists $N$ such that $C_{N}+M_0=C_{N+1}+M_0$ and $C_{N}\cap M_0=C_{N+1}\cap M_0$. However, the first equality implies $C_{N+1}-C_{N}\subset M_0$ for ascending ($C_{N}-C_{N-1}$ for descending), and combined with the second equality we have $C_N=C_{N-1}$, a contraction. 

    To $2$, we may break the long exact sequence to short exact sequences by adding in the kernel and cokerknel terms. The result is then a simple corollary of part $1$. 

\end{proof}




Recall the discussion on composition series of $R$-modules. If a composition series exist, then all such have the same length and the same simple factors up to permutation. $0\subseteq M_1\subseteq M_2\subseteq ...\subseteq M_n=M$ such that $\overline{M_i}=M_i/M_{i-1}$ is simple. 


\begin{tcolorbox}[colback=blue!5!white,colframe=blue!30!white]
\begin{proposition}
Let $M$ be a (left) modules. Then, $M$ has a (left) composition series iff $M$ satisfies ACC and DCC. 
\end{proposition}
\end{tcolorbox}
\begin{proof}
    Let $0\subseteq M_1\subseteq M_2\subseteq ...\subseteq M_n=M$ be a composition series, and make induction on $n$. For $n=1$, the module is simple and it automatically satisfies ACC and DCC. For inductive step, suppose $0\subseteq M_1\subseteq M_2\subseteq ...\subseteq M_n$ is a composition series, so $M_n$ satisfies ACC and DCC. Then, there exists the exact sequence 
    \[\begin{tikzcd}
    0\arrow[r]&M_n\arrow[r,"f"]&M_{n+1}\arrow[r,"g"]&M_{n+1}/M_n\arrow[r]&0
    \end{tikzcd}\]
    and by proposition $9.2$, $M_{n+1}$ satisfies ACC and DCC sicne $M_{n+1}/M_n$ is simple. 

    Suppose $M$ satisfies ACC and DCC. In particular, $M$ has minimal submodules $M_1$ by DCC, which must be simple. Proceed inductively, and consider the set $M'=\{N|M_1\subset N\}$, which also has minimal elements, say $M_2$. Then, $M_2/M_1$ must be simple. By ACC, the sequence must terminate and we get a finite composition series. 
\end{proof}

\section{Integral extensions}

\subsection{Basic Facts}
\begin{tcolorbox}[colback=purple!5!white,colframe=purple!75!black]
\begin{definition}
A commutative ring extension is any injective ring homomorphism $R\hookrightarrow S$. We denote such an entension by $S|R$. An element $x\in S$ is called \underline{\textbf{integral}} or \underline{\textbf{algebraic}} if it is a root of a monic polynomial in $R[t]$. 
\end{definition}
\end{tcolorbox}


\begin{tcolorbox}[colback=yellow!5!white,colframe=yellow!30!white]
\begin{example}
The canonical embedding $\mathbb{Z}\hookrightarrow \mathbb{Q}$ is a ring extension. The only integral elements are elements in $\mathbb{Z}$. In general, if $R$ is a UFD, then $x\in S$ integral over $R$ iff $x\in R$. For example, $\mathbb{Z}[t]\hookrightarrow \mathbb{Q}[t]$. 
\end{example}
\end{tcolorbox}


\begin{tcolorbox}[colback=blue!5!white,colframe=blue!30!white]
\begin{proposition}
Let $S|R$ be a ring extension. Then, the following are equivalent:
\begin{enumerate}
    \item $x\in S$ is integral over $R$.
    \item $R[x]$ is a finite $R$-module
    \item There exists a subring $T$ such that $R\subseteq T\subseteq S$ where $x\in T$, and $T$ is a finite $R$-module. 
\end{enumerate}
\end{proposition}
\end{tcolorbox}
\begin{proof}
    For $1\implies 2$, suppose $x$ satisfies the minimal monic polynomial $p(t)=t^n+a_{n-1}t^{n-1}+...+a_0$. Then, $R[x]\cong R[t]/(p(t))$, which is a finite $R$ module generated by $1,t,...,t ^{n-1}$. 
    $2\implies 3$ is trivial. 
    
    
    For $3\implies 1$, suppose $T$ as an $R$ module is finitely generate by $(v_1,...,v_n)$. Then, $x$ acts on $T$ by left multiplication, and suppose $xv_i=\sum_{j=1}^{n}a_{i,j}v_j$. Let $A=(a_{i,j})$ and $v$ be the column vector $(v_1,...,v_n)^T$. Then, the equation says $(A-xI)v=0$, which implies $det(A-xI)=0$ (for $R$ a domain, we may enlarge to the quotient field and the statement is purely linear algebra; for general $R$, this can be proven using Cramer's rule). Thus, $x$ satisfies the characteristic polynomial of $A$, which makes it integral. 
\end{proof}
The proof of proposition $10.1.3$ generalizes to the famous Cayley-Hamilton Theorem.

\begin{tcolorbox}[colback=red!5!white,colframe=red!30!white]
\begin{theorem}
(Cayley-Hamilton) Let $R$ be a ring, $I\subset R$ an ideal, $M$ a finitely generated $R$-module that is generated by $n$ elements. Fix $\varphi\in End_R(M)$. If 
\[\varphi(M)\subset IM\]
then there exists a monic polynomial $p(x)=x^n+p_1x^{n-1}+...+p_n$ with $p_i\in I^i$ such that $p(\varphi)=0$.
\end{theorem}
\end{tcolorbox}

As a direct corollary of Cayley-Hamilton by taking $I=R$ (or seen more directly from the proof of proposition $10.1$), we have a converse of the interplay between integral extension and finiteness as a module described in Proposition $10.1$:
\begin{tcolorbox}[colback=green!5!white,colframe=green!30!white]
\begin{corollary}
If $S$ is a finitely generated $R$-module, then it is generated by finitely many elements integral over $R$.
\end{corollary}
\end{tcolorbox}



\begin{tcolorbox}[colback=blue!5!white,colframe=blue!30!white]
\begin{proposition}
Let $S|R$ be a ring extension. 
\begin{enumerate}
    \item If $x_1,...,x_n\in S$ are integral over $R$. Then, $R[x_1,...,x_n]$ is a finite $R$-module. 
    \item $\tilde{R}:=\{ x\in S: x \ \textrm{integral over} \ R \}$ is a subring containing $R$. 
    \item If $I\in Id(R)$, and $\tilde{I}=\{ x\in S: x \  \textrm{integral over} \ I \}$ is an ideal of $\tilde{R}$ containing $I$. In particular, it is $N(I\tilde{R})$. 
\end{enumerate}
\end{proposition}
\end{tcolorbox}
\begin{proof}
    To $1$: direct corollary of Proposition $10.1.2$.
    
    To $2$. Note that if $a,b$ are integral over $R$, then $a+b$ and $ab$ are contained in the ring $R[a,b]$, which by $1$ is finite over $R$. By Proposition $10.1.3$, we are done. 
    
    
    To $3$: For the $\tilde{I}\subseteq N(I\Tilde{R})$ direction, let $x\in \tilde{R}$ be integral over $I$, which means there is $x^n+a_{n-1}x^{n-1}+....=0$. We may rewrite the equation as $x^n=(-a_{n-1}x^{n-1}+....)\in I\tilde{R}$, hence $x\in N(I\Tilde{R})$. For the other direction, let $y\in N(I\tilde{R})$, which implies $y^k=\sum_{i=1}^{n}b_ix_i$, where $b_i\in I$ and $x_i\in \tilde{R}$. Then, $M=R[x_1,...,x_n]$ is finite module over $R$, and $y^kM\subset IM$. Left multiplication by $y^k$ again is again an endomorphism of $IM$, and we finish by Cayley-Hamilton. 
\end{proof}


\begin{tcolorbox}[colback=purple!5!white,colframe=purple!75!black]
\begin{definition}
Let $S|R$ be a ring extension. The ring $\tilde{R}=\{ x\in S: x \  \textrm{algebraic over } R \}$ is called  \underline{\textbf{integral closure}} of $R$. $S|R$ is called \underline{\textbf{integral}} if $\tilde{R}=S$. $R$ is called \underline{\textbf{integrally closed in}} $S$ if $\tilde{R}=R$.
\end{definition}
\end{tcolorbox}


\begin{tcolorbox}[colback=purple!5!white,colframe=purple!75!black]
\begin{definition}
Let $R$ be a domain and $K$ its quotient field. $R$ is called \underline{\textbf{integrally closed}} if $R$ is integrally closed in $K$.
\end{definition}
\end{tcolorbox}


\begin{tcolorbox}[colback=blue!5!white,colframe=blue!30!white]
\begin{proposition}
UFDs are integrally closed.
\end{proposition}
\end{tcolorbox}
\begin{proof}
    Suppose $R$ is a UFD. Let $f(t)=a_0+...x^n$ be a monic polymial in $R[x]$. Suppose $\frac{p}{q}\in \textrm{Quot}(R)$ is a root to $f(t)$ with $gcd(p,q)=1$. Then, $q^nf(\frac{p}{q})=p^n+a_{n-1}p^{n-1}q+...+a_0q^n=0$. In particular, we see $q$ clearly divides every term of degree $<n$, so it must divide $p^n$ as well. By $gcd$ assumption, $q$ must be a unit, and $\frac{p}{q}\in R$. We conclude $R$ is integrally closed.
\end{proof}


\begin{tcolorbox}[colback=blue!5!white,colframe=blue!30!white]
\begin{proposition}
    Valuation rings are integrally closed.
\end{proposition}
\end{tcolorbox}
\begin{proof}
 Let $R$ be a valuation ring, and $f(t)=a_0+...x^n$ be a monic polymial in $R[x]$. If $b$ is a root to $f(t)$, we know one of $b$ and $b^{-1}$ is in $R$; if $b\in R$ we are done; if $b ^{-1}\in R$, by $f(b)=0$ we get $b+a_{n-1}+a_{n-2}b^{-1}...+a_0b ^{-n+1}=0$, so $b\in R$ as well. We conclude $R$ is integrally closed.
\end{proof}
\begin{tcolorbox}[colback=red!5!white,colframe=red!30!white]

\begin{theorem}
Let $R$ be a domain. Then, $R$ is integrally closed iff $R=\cap R_v$ where $R_v$ is a valuation ring over $R$ in the quotient field.  
\end{theorem}
\end{tcolorbox}
\begin{proof}

 The intersection of integrally closed subrings of a common quotient field is clearly integrally closed in the quotient field: an element integral over the intersection is integral over every ring in the intersection, thus contained in every ring in the intersection. 

Suppose $R$ is integrally closed in the quotient field $K$. Clearly, $R\subseteq \cap R_v$ where $R_v$ are valuation rings lying over $R$ since they are integrally closed. Conversely, if $x\in K$ is an element not integral over $R$, then $x^{-1}\not \in R$ by the same argument as above. Let $\mathfrak{m}\subset R$ be a maximal ideal containing $x ^{-1}$. By Chevalley's extension theorem, there is a valuation ring $(V,\mathfrak{m}_V)$ over $R$ with $\mathfrak{m}_V\cap R= \mathfrak{m}$. In particular, $V$ is a valuation ring containing $x^{-1}$ but not $x$. Thus, $R= \cap R_v$

\end{proof}


\begin{tcolorbox}[colback=blue!5!white,colframe=blue!30!white]
\begin{proposition}
The following hold:
\begin{enumerate}
    \item (Transitivity) Let $S_2|S_1|R$ be ring extensions. Then, $S_2|R$ is integral iff $S_2|S_1$ is integral and $S_1|R$ is integral as well.
    \item (Functoriality) Suppose $S|R$ is integral, $b$ a proper ideal of $S$, and let $a:=b\cap R$. Then $S/b|R/a$ is integral.
    \item Let $S|R$ be integral, and $\Sigma$ be a multiplicative system of $R$. Then, $S_{\Sigma}|R_{\Sigma}$ is integral.
\end{enumerate}
\end{proposition}
\end{tcolorbox}
\begin{proof}
    To $1$: if $S_2|R$ is integral, then clearly both $S_2|S_1$ and $S_1| R$ are integral. Conversely, suppose $S_2|S_1$ and $S_1|R$ are integral. Given $x\in S_2$, we have $x$ satisfying $x^n+s_{n-1}x^{n-1}+...+s_0=0$. Consider the subring $R':=R[s_0,...,s_{n-1}]$, which is a finite module over $R$. Then, $R[x]$ is finite over $R'$, therefore finite over $R$ as well.

    To $2$: Suppose we have $x\in S$, then $x^n+p_{n-1}x^{n-1}+...+p_0=0$ for $p_i\in R$. Reduce the equation mod $b$ gives us a monic polymial over $R/a$. 
    
    To $3$: Let $\frac{s}{b}\in S_{\Sigma}$. By integral assumption, $s^n+p_{n-1}s^{n-1}+...+p_0=0$. Replace $p_i$ with $p_i/b^i$ gives us a monic polynomial with coefficients in $R_{\Sigma}$ and $\frac{s}{b}$ a root.
\end{proof}


\subsection{Going-Up Theorem}

\begin{tcolorbox}[colback=blue!5!white,colframe=blue!30!white]
\begin{proposition}
    Let $S|R$ be a integral extension. If $S$ is a domain, then $S$ is a field iff $R$ is a field. In particular, if $\mathfrak{m}$ is maximal in $S$ iff $\mathfrak{m}\cap R$ is maximal. 
\end{proposition}
\end{tcolorbox}
    \begin{proof}
        First, assume $R$ is a field. Take $x\neq 0\in S$, and there exists $a_0,...,a_{n-1}\in R$ such that $a_0+...+a_{n-1}x^{n-1}+x^n=0$. By the domain assumption, we may assume $a_0\neq 0$, for otherwise we may factor out $x^k$ as necessary. Then $x(x^{n-1}+...+a_{1})=-a_0$ is invertible, so $x\in S^{\times}$. Conversely, suppose $x\in R$. Then, $x^{-1}\in S$ is integral over $R$, satisfying $x^{-n}+p_{n-1}x^{-n+1}+...p_0=0$. Multiplying both sides with $x^{n-1}$ shows $x^{-1}$ is in $R$ as well.
    
        Finally, if $\mathfrak{m}\in Max(S)$ and $\mathfrak{n}=\mathfrak{n}\cap R$. Then, $S/\mathfrak{m}$ is integral over $R/\mathfrak{n}$ by Proposition $10.5.2$. Therefore, $R/\mathfrak{n}$ is a field and thus $\mathfrak{n}$ is maximal. 
    \end{proof}
    

\begin{tcolorbox}[colback=blue!5!white,colframe=blue!30!white]
\begin{proposition}
Let $S|R$ be an integral extension. Suppose we have $q_1,q_2\in Spec(S)$ with $q_1\subseteq q_2$ such that $q_1\cap R=q_2\cap R$. Then, $q_1=q_2$. 
\end{proposition}
\end{tcolorbox}
\begin{proof}
  We may localize both ring at $q_1\cap R$. Then, $q_1,q_2$ are still primes in $S$, while $q_1\cap R$ is maximal in $R$. By Proposition $10.6$, we have $q_1,q_2$ both maximal, so we have $q_1=q_2$.
\end{proof}

\begin{tcolorbox}[colback=red!5!white,colframe=red!30!white]
\begin{theorem}
 Let $S|R$ be a integral ring extension. Then, the following hold: 
\begin{enumerate}
    \item (\underline{\textbf{Lying-over}}) For every $p\in Spec(R)$, there exists $q\in Spec(S)$ such that $q\cap R=p$. 
    \item (\underline{\textbf{Going-up}}) Let $p_1\subseteq p_2\subseteq ....\subseteq p_n$ be a chain in $Spec(R)$, $p_1\subseteq p_2\subseteq ....\subseteq p_m$ a chain in $Spec(S)$, such that $m<n$ and $q_j\cap R=p_j$ for all $j\leq m$. Then, the chain in $Spec(S)$ can be extened to length $n$. In particular, Krull dimension of $R$ equals the Krull dimension of $S$. 
\end{enumerate}
\end{theorem}
\end{tcolorbox}

\begin{proof}
   To prove the lying over property: let $p\in Spec(R)$ be given. Consider $R_p\subset S_p$. Then, $S_p$ over $R_p$ is integral. In particular, $R_p$ is local and $p$ is maximal. Using Proposition $10.6$, taking any maximal ideal in $S_q$ finishes. 

   To prove going-up, it suffices to show $n=2$ and $m=1$: suppose we have $p_1\subset p_2$ with $q_1$ such that $q_1\cap R=p_1$. Then, consider $R':=R/p_1$ and $S':=S/q_1$. By lying over, there is a prime in $S'$ lying over $\overline{p_2}$. Lift back to $S$ finishes.
   
   
   

\end{proof}


\begin{tcolorbox}[colback=green!5!white,colframe=green!30!white]
\begin{corollary}
Given a integral extension $i: R\to S$, the induced map $i^*: Spec(S)\to Spec(R)$ is surjective.
\end{corollary}
\end{tcolorbox}


\section{Noether Normalization Theorem}
Let $k$ be a field; $R|k$ be a algebra of finite type. We first prove two lemma regarding change of variables. The first one applies when normalizing $k$-algebras when $k$ is infinite.


\begin{tcolorbox}
\begin{lemma}
Suppose $k$ is an infinite field. Given $q\in k[x_1,...,x_n]$, there is a choice of $a_1,...,a_n\in k$ such that the change of variables $x_i\mapsto x_i+a_ix_n$ for $i<n$ and $x_n\mapsto a_nx_n$ take $q$ to the form 
\[q=cx_n^{d}+(\textrm{terms in which }x_n\textrm{ has exponent less than }d)\]
\end{lemma}
\end{tcolorbox}
\begin{proof}
    It suffices to consider the homogeneous part of highest total degree $d$, denoted by $q_d$, since the linear change of variables does not change the total degree. The coefficients of $x_n^d$ is precisely $q_d(a_1,...,a_n)$. Thus, the lemma reduces to whether a homogeneous polynomial $q_d$ does not vanish everywhere on $k^n$. When $k$ is infinite, this is standard to prove by induction on number of variables, and the fact that a polynomial over $R[x]$ has only finitely many roots.

    Note that since $k$ is a field and the polynomial is homogeneous, we can divide out an appropriate constant so that $c=q_d(a_1,...,a_n)=1$.
\end{proof}


\begin{tcolorbox}[colback=yellow!5!white,colframe=yellow!30!white]
\begin{example}
This argument fails when $k$ is finite. For example, the polynomial $x^3+2x=x(x-1)(x-2)$ vanishes everywhere on $F_3$. 
\end{example}
\end{tcolorbox}

Now we prove a more generalized change of variables that applies without the assumption of the infinitude of $k$.
\begin{tcolorbox}
\begin{lemma}
    Suppose $D$ is a integral domain. Given $q\in D[x_1,...,x_n]$, there is a choice of $m_1,...,m_n\in k$ such that the change of variables $x_i\mapsto x_i+x_n^{m_i}$ for $i<n$ and $x_n\mapsto x_n^{m_n}$ take $q$ to the form 
    \[q=cx_n^{m}+(\textrm{terms in which }x_n\textrm{ has exponent less than }d)\]
\end{lemma}
\end{tcolorbox}
\begin{proof}
    Let $N$ be a natural number larger than the highest exponent of $x_i$ anywhere in $q$, and take $m_i=N^i$. Then, let $c\underline{X}$ be a term in $q$, where the exponents of $\underline{X}$ is given by the multindices $\alpha:=(\alpha_0,...,\alpha_n)$. Then, the term having the highest exponent of $x_n$ is the monomial $x_n^{T_{\alpha}}$, where $T_{\alpha}=\sum_{i=0}^{n}\alpha_iN^i$. Note that given a different set of multindices $\alpha'$, we have $T_{\alpha}\neq T_{\alpha'}$ by uniqueness of $N$-ary representation. Therefore, we obtain a monomial $cx_n^{m}$, with $m$ being the unique largest exponent after the change of variables. 

    Again, if $D$ is a field, we may take $c=1$ by dividing out an appropriate constant. 
\end{proof}



\begin{tcolorbox}[colback=red!5!white,colframe=red!30!white]
\begin{theorem}
(Noether Normalization Theorem) Let $R=k[x_1,...,x_n]$ be a $k$-algebra of finite type. Then, there exists $t_1,...,t_d\in R$, $d\leq n$ such that $\{t_i\}$ algebraically independent over $R$ and $R$ is integral over $R_0:=k[t_1,...,t_d]$, a polynomial ring over $d$ variables.  
\end{theorem}
\end{tcolorbox}
\begin{proof}
We induct on $n$. If $n=1$, and $x_1$ is algebraic over $k$, then $k[x]$ is a vector space and thus we take $R_0=k$ as well; if $x_1$ is not algebraic over $k$, then take $t$ be a transcendental variable and $k[x]=k[t]$. Now suppose the theorem holds for $n=m$. For $n=m+1$, if all elements $x_1,...,x_m$ again are algebraically independent, we may just take $t_i=x_i$. Otherwise, there exists a non-zero polynomial $p$ such that $p(x_1,...,x_n)=0$. Using Lemma $11.2$, we may let $x'_i=x_i-x_n^{m_i}$ for $i<n$, and $x_n$ is integral over $R_0:=k[x'_1,...,x'_{n-1}]$. Then, $x_n^{m_i}$ are integral over $R_0$, and the sum of integral elements $x_i+x_n^{m_i}=x'_i$ are integral as well. Therefore, $R|R_0$ is integral. By transitivity of integrality and the inductive hypothesis, we are done. 
\end{proof}



\begin{tcolorbox}[colback=purple!5!white,colframe=purple!75!black]
\begin{definition}
The new set of variables $t_1,...,t_n$ is called a \underline{\textbf{Noether Basis}} of $R$ over $k$. 
\end{definition}
\end{tcolorbox}



\begin{tcolorbox}[colback=purple!5!white,colframe=purple!75!black]
\begin{definition}
Let $R$ be a commutative ring, and $f\in R$. Then, $V(f):=\{\mathfrak{m}\in Max(R), f\in \mathfrak{m}\}$. This is called the \underline{\textbf{zero set}} of $f$ in $R$. More generally, given $I\in Id(R)$, we denote $V(I):=\{\mathfrak{m}\in Max(R), I\subset \mathfrak{m}\}$
\end{definition}
\end{tcolorbox}


\begin{tcolorbox}[colback=purple!5!white,colframe=purple!75!black]
\begin{definition}
A commutative ring $R$ is called \underline{\textbf{Jacobson}} if for every $p\in Spec(R)$, we have $p=J(p)$. In other words, every prime is the intersection of maximal ideals lying above it.  
\end{definition}
\end{tcolorbox}


\begin{tcolorbox}[colback=blue!5!white,colframe=blue!30!white]
\begin{proposition}
The following are equivalent:
\begin{enumerate}
    \item $R$ is Jacobson
    \item For every $\mathfrak{a}\in Id(R)$, we have $N(\mathfrak{a})=J(\mathfrak{a})$;
    \item For every surjective ring homomorphism $R\to S$, we have $N(S)=J(S)$.
\end{enumerate}
\end{proposition}
\end{tcolorbox}
\begin{proof}
    Easy exercise.
\end{proof}


\begin{tcolorbox}[colback=red!5!white,colframe=red!30!white]
\begin{theorem}
(Hilbert Nullstellensatz) Let $R=k[x_1,.,x_n]$ be a $k$-algebra of finite type. 
\begin{enumerate}
    \item If $R$ is a field, then $R$ is a finite algebraic extension of $k$. In particular, given $\mathfrak{m}\in Max(R)$, we have $R/\mathfrak{m}$ algebraic over $k$.
    \item $R$ is a Jacobson ring.
    \item Let $g, f_1,...,f_m\in R$ be given. Then, the zero set of $f_1,...,f_m$ are contained in the zero set of $g$ iff there exists $N>0$, $\lambda_i\in R$ such that $g^N=f_1\lambda_1+...+f_m\lambda_m$. 
\end{enumerate}
\end{theorem}
\end{tcolorbox}
\begin{proof}
    To $1$: By Noether Normalization, $R$ is an integral extension over a polynomial ring over $k$. Since $R$ is a field and integral extension preserves krull dimension, the polynomial ring must in fact be $k$, and the result follows.

    To $2$: Given $p\in Spec(R)$ and let $f\not \in p$. Then, out goal is to construct a maximal ideal containing $p$ and not containing $f$. If $p$ is maximal then we are done; otherwise let $S=R/p$, and $\Sigma=\{1,f^1,f^2,...\}$. Then, $S_{\Sigma}=S[\frac{1}{f}]$ is a $k$-algebra of finite type. Let $\mathfrak{m}_{\Sigma}$ be a maxmimal ideal, which must avoid $f$. Then, $S/(\mathfrak{m}_{\Sigma})$ is a finite algebraic extension of $k$, thus also a finite module over $R$. Thus, $\mathfrak{m}_{\Sigma}\cap R$ must be maximal in $R$ by integrality. 

    To $3$: Set $\mathfrak{a}=(f_1,...,f_m)$. We have $V(\mathfrak{a})\subseteq V(g)$ is equivalent to $g\in \mathfrak{m}$ for every maximal $\mathfrak{m}$ containing $\mathfrak{a}$. In other words, we have $g\in J(\mathfrak{a})=N(\mathfrak{a})$.  
\end{proof}
\begin{tcolorbox}[colback=yellow!5!white,colframe=yellow!30!white]
    \begin{example}
    Let $k$ be algebraically closed, and $R=k[x_1,...,x_n]$. Then, every maximal ideal of $R$ is of the form $(x_1-a_1,...,x_n-a_n)$ for $a_i\in k$. The proof is follows: it is easy to show ideals of the forms are maximal; for the reverse, note that if $\mathfrak{m}$ is maximal, then $R/\mathfrak{m}\cong k$ by Theorem 11.4.1 and the algebraically closed assumption. Let $a_i\in k$ be the image of $x_i$ under the quotient map $R\to R/m$, and it is easy to see the kernel is precisely $(x_1-a_1,...,x_n-a_n)$.
    \end{example}
    \end{tcolorbox}
    
\section{Integral Extensions over Integral Domains}

\begin{tcolorbox}[colback=blue!5!white,colframe=blue!30!white]
\begin{proposition}
Let $S|R$ be an integral extension where $R$ is integral domain. Let $K$ be the quotient field of $R$, $L$ be the quotient field of $S$. Then, $L|K$ is an algebraic field extension and there exists a basis in $S$. 
\end{proposition}
\end{tcolorbox}
\begin{proof}
    To $1$: For the first part, it suffices to show if $s\in S$, then $\frac{1}{s}$ is algebraic over $K$. By definition, $s$ satisfies a monic polynomial such that $s^n+a_{n-1}s^{n-1}+...+a_0=0$. Divide both sides by $s^n$ and we get $\frac{1}{s}$ satisfiying an algebraic equation over $K$. For the second part, suppose $\beta=\frac{r}{s}$ is a given basis vector, we know $s$ satisfies a algebraic equation over $K$. WLOG we can assume the constant term is $0$ since we are in a domain, and we have $s^n+a_{n-1}s^{n-1}+...+a_1s=-a_0$. Divide both sides by $-\frac{a_0}{s}$ and we express $\frac{1}{s}$ as a $K$-linear combination of $1,s,...,s^{n-1}$. Thus, we can turn an arbitrary basis into a set contained in $S$ that span $L$, and we may reduce it to a basis if needed. 

\end{proof}


\begin{tcolorbox}
\begin{lemma}
Suppose $R$ is a Noetherian integrally closed domain with, and let $K=Quot(R)$. Suppose $L|K$ is a finite Galois extension, then the integral closure of $R$ in $L$ is a finitely generated $R$-module.
\end{lemma}
\end{tcolorbox}
\begin{proof}
    Choose a basis $\beta_1,...,\beta_n$ for $L|K$, and let $Gal(L|K)=\{\sigma_1,...,\sigma_n\}$. Let $M=(\sigma_ib_j)_{1\leq i,j\leq n}$. Then, we claim that 
   \[\tilde{R}\subset  det(M)^{-2}\sum_{i=1}^nRb_i\]
   By Noetherian assumption, it follows immediately that $\tilde{R}$ is finitely generated over $R$. To prove the claim, first note that by linear indepedence of characters the determinant is not zero. Let $d:=det(M)\in \tilde{R}$. Then, it suffices to show that given $r=\sum_{i=1}^{n}a_n\beta_n\in \tilde{R}$, we have $d^2a_n\in R$ for all $n$. Let $a$ be the column vector with entry $a_i$. Then, we have 
\[(Ma)_i=\sum_{j}a_j\sigma_i(a_j)=\sigma_ib\in \tilde{R}\]
By cofactor formula, we have $da_i\in \tilde{R}$ as well. On the other hand, $\sigma_id$ is the determinant of the matrix resulted from a permutation of the rows of $M$, so $\sigma_id=\pm d$. Thus, $d^2\in K$ since $d^2$ is invariant under $G$. (Alternatively, we may choose the basis to be primitively generated, so the matrix is vandermonde matrix and the determinant is more visibly invariant under $G$). It now follows that $d^2a_i\in K\cap \tilde{R}$, and by integrally closed assumption, we have $d^2a_i\in R$. 
\end{proof}


\begin{tcolorbox}[colback=red!5!white,colframe=red!30!white]
    \begin{theorem}
    Let $R=k[x_1,..,x_n]$ be an integral finitely generated $k$-algebra. Let $K$ be the quotient field of $R$ and $L|K$ a finite field extension, $S$ the integral closure of $R$ in $L$. Then, $S$ is a finite $R$-module.
    \end{theorem}
    \end{tcolorbox}
    \begin{proof}
    By Noether normalization, $R$ is a finite module over a polynomial ring $k[t_1,..,t_k]$. Thus, we may assume $R$ is Noetherian and integrally closed to begin with. By enlarging $L$ to the normal closure if necessary, a direct application of Lemma $12.1$ finishes the case where $L|K$ is separable.  

    For the purely inseparable case, note that purely inseparable extension of $k[t_1,..,t_k]$ must be of the form of adjoining a $q_i=p^{n_i}$th root, where $p=char(k)$. Then, the field $L$ is contained in $L':=k'(t_1^{\frac{1}{q_1}},...,t_n^{\frac{1}{q_n}})$, where $k'$ is obtained by adjoining the $p$th roots necessary. Note that the integral closure of $k[t_1,...,t_n]$ in $L'$ is $R':=k'[t_1^{\frac{1}{q_1}},...,t_n^{\frac{1}{q_n}}]$, since it is a UFD and each element of $R'$ is visibly integral over $k$. By characterization of integrality, $R'$ is generated by finitely many integral elments over $R$ (and $k'$ is a finite field extension of $k$), so $R'$ is finite over $R$. 
    \end{proof}
    

\begin{tcolorbox}[colback=red!5!white,colframe=red!30!white]
\begin{theorem}
(Going Down) Let $S|R$ be a integral ring extension of domains, with $R$ integrally closed, then the extension satisfies going down property: given a chain of prime $p_1\supseteq p_2...\supseteq p_m $ in $R$, and a chain $q_1\supseteq q_2...\supseteq q_n$ in $S$ with $m>n$, then we may extend the chain in $S$ such that $q_i\cap R=p_i$ for all $1 \leq i\leq m$.
\end{theorem}
\end{tcolorbox}
\begin{proof}
    By induction, reduce to the case where $R$ is local. 
\end{proof}

\begin{tcolorbox}[colback=green!5!white,colframe=green!30!white]
    \begin{corollary}
        Let $R:=k[x_1,...,x_n]$ be an integral finitely generated $k$-algebra. \\
        Then, $R$ is \underline{\textbf{strongly catenary}}, i.e every maximal sequence of prime ideals has length $n=d$. 
    \end{corollary}
    \end{tcolorbox}





\section{Introduction to Hilbert Decomposition Theory}
Let $R$ be an integrally closed domain. $K$ the quotient field. $L|K$ the algebraic separable extension. $S$ the integral closure of $R$ in $L$. The question is describe the behaviour of $Spec(R)$ under the integral ring extension $S|R$. Especially when $L|K$ is Galois. In some sense, this extends the usual Galois theory for field extension to ring extensions. 


\begin{tcolorbox}[colback=blue!5!white,colframe=blue!30!white]
\begin{proposition}
    Let $G$ be a profinite group, acting continuously on a discrete ring $S$.  Then, the oribit for every $x\in S$ is finite under $G$. 
\end{proposition}
\end{tcolorbox}


\begin{tcolorbox}[colback=blue!5!white,colframe=blue!30!white]
\begin{proposition}
Let $G_i=g/N_i$. Then, $S_i=S^{G_i}:= \{ x\in S: G_ix=x \}$. 
\begin{enumerate}
    \item Then, $\cup S_i=S$
\item Let $R=S^G$. Then, $S_i|G$ is integral. Moreover, $G_i\subset Aut_R(S_i)$, with $S_i^{G_i}=R$.
\end{enumerate}
\end{proposition}
\end{tcolorbox}


\begin{tcolorbox}[colback=blue!5!white,colframe=blue!30!white]
\begin{proposition}
    Let $R=S^G$. Then, $S_i|R$ and $S|R$ are integral ring extensions. Hence, $S_j|S_i$ for $N_j\subset N_i$ is integral extension.
\end{proposition}
\end{tcolorbox}


\begin{tcolorbox}[colback=blue!5!white,colframe=blue!30!white]
\begin{proposition}
Let $p\in Spec(R)$, and denote $X^i_p=\{ q\in Spec(S_i):q_i\cap R=p \}$ and $X_p=\{ q\in Spec(S):q\cap R=p \}$. Then, $G$ acts Transitivity on $X_p$, and $G_i$ acts transitively on $X^i_p$.
 \end{proposition}
\end{tcolorbox}

Fact: Let $L|K$ be Galois; let $G$ be the galois group. Then, $\sigma\in G$ implies $\sigma(S)=S$. 



The ring extensions $S_j|R$ satisfies $Spec S_i\to Spec(R)$ given by $q_i\mapsto q_i\cap R$ is surjective. Then, $Spec(R)=G_i\ Spec(S_i)$ is the $G_i$ orbits of $Spec(S_i)$. Finally, $Spec(S)$ is the projective limit of $Spec(S_i)$ as $G$-spaces. 


\begin{tcolorbox}[colback=purple!5!white,colframe=purple!75!black]
\begin{definition}
Given $q\in Spec(X_p)$, the \underline{\textbf{q-decomposition group}} $D_{q|p}=st_G(q)=\{ \sigma\in G:\sigma q=q \}$. 
\end{definition}
\end{tcolorbox}


\begin{tcolorbox}[colback=blue!5!white,colframe=blue!30!white]
\begin{proposition}
Let $\Sigma\subset R$ be a multiplicative system. Then, $G$ acts on $S_{\Sigma}$ $\sigma(\frac{q}{r})=\frac{\sigma(q)}{r}$. And, $S^G_{\Sigma}=R_{\Sigma}$ Moreover, $p\cap \Sigma=\emptyset$. Then, $X_{P_{\Sigma}}=\{ q_{\Sigma}: q\in X_p \}$. And $D_{q_{\Sigma}|p_{\Sigma}}=D_{q|p}$. 

\end{proposition}
\end{tcolorbox}


\begin{tcolorbox}[colback=blue!5!white,colframe=blue!30!white]
\begin{proposition}
Let $H\subset G$ be an open subgroup. Define $q^H=q\cap S^H$. 

\begin{enumerate}
    \item  $q^H\cap R=p$ and  $D_{q|q^H}=D_{q|p}\cap H$.
    \item If H is normal, let $\overline{G}: G/H$. Then, $\overline{G}$ acts on $S^H$ by $\overline{\sigma}(h)=\sigma(h)$.  Then, $(S^H)^{\overline{G}}=R$, and $D_{q^H|p}=\overline{D_{q|p}}:=D_{q|p}/H$.
\end{enumerate}
\end{proposition}
\end{tcolorbox}



\begin{tcolorbox}[colback=green!5!white,colframe=green!30!white]
\begin{corollary}
Let $G$, $S|R$ be as usual. Then,
\begin{enumerate}
    \item The going-down for $S|R$ holds. Given $p_1\subset ...\subset p_n$ a chain in $Spec(R)$, and $q_m\subset ...q_n$ with $m\leq n$, then there exists $q_1\subset ... \subset q_m$ in $Spec(S)$ that prolongs the sequence. 
\end{enumerate}
\end{corollary}
\end{tcolorbox}
\begin{proof}
    By induction, it suffice to consider $n=2$ and $m=1$. Hence $p_1\subset p_2$ and $q_2\cap R=p_2$. Recall there exists $q_1'\in Spec(S)$ such that $q_1'\cap R=p_1$. By going-up, there exists $q_2'\in Spec(S)$ such that $q_2'\cap R=p_2$. Since $G$ acts transitively  on $X_{p_1}$ and $X_{p_2}$ such that $\exists \sigma\in G$ such that $\sigma'q_2=q_2'$. Take $\sigma ^{-1}$ finishes. 
\end{proof}


\begin{tcolorbox}[colback=red!5!white,colframe=red!30!white]
\begin{theorem}
The restriction map $Spec(S)\to Spec(R)$ by $q\mapsto q\cap R$ defines a bijection between the maximal chain of prime ideals in the two rings. 
\end{theorem}
\end{tcolorbox}

Let $H\subset G$ be a closed subgroup. $S^H$ be the fixed ring. $Spec(S^H)\to Spec(R)$ be the restriction map. Then, the previous theorem also holds. 


\begin{tcolorbox}[colback=yellow!5!white,colframe=yellow!30!white]
\begin{example}
Main example of the theory is: Let $R$ be an integrally closed domain. $K$ be the quotient field of $R$ and $L|K$ galois extensions. Take $S$ be the integral closure of $R$ in $L$. Recall $G$ acts on $S$ via $L$ (check $\sigma s\in S$). Then, $S|R$ is quasi-galois extension with $G=Aut_R(S)$. Hence, the Hilbert Decomposition Theory works for $S|R$, the going down. 
\end{example}
\end{tcolorbox}



\begin{tcolorbox}[colback=yellow!5!white,colframe=yellow!30!white]
\begin{example}
Let $R=\mathbb{Z}$, $K=\mathbb{Q}$ and $L=\mathbb{Q}[\sqrt{d}]$. Then, $S=\mathbb{\sqrt{d}}$ iff $d\neq 1$ mod $4$. Otherwise $S=\mathbb{Z}[\frac{1+\sqrt{d}}{2}]$.
Then $G=\mathbb{Z}/2$. Let $\mathfrak{P}\in X_p\in Spec(S)$. Then the decomposition group is either $1$ or the entire $G$. Look at $x^2-d(mod p)$. If reducible, then $p$ cannot be prime.

\end{example}
\end{tcolorbox}



\begin{tcolorbox}[colback=yellow!5!white,colframe=yellow!30!white]
\begin{example}
$R=k[t]$ and $K=k(t)$, $L=k(\sqrt{f})$.
\end{example}
\end{tcolorbox}

\section{Dedekind domains}


\begin{tcolorbox}[colback=purple!5!white,colframe=purple!75!black]
\begin{definition}
A domain $R$ is called a \underline{\textbf{Dedekind}} domain if every proper ideal can be written as a product of prime ideals.
\end{definition}
\end{tcolorbox}
\begin{tcolorbox}[colback=yellow!5!white,colframe=yellow!30!white]
    \begin{example}
    All PIDs are Dedekind Domains. The integral closure of a dedekind doamin under the galois extension is still a Dedekind domain.
    \end{example}
    \end{tcolorbox}
    
    



\begin{tcolorbox}[colback=purple!5!white,colframe=purple!75!black]
\begin{definition}
Let $R$ be a commutative ring and $K$ be its total ring of fractions. If $R$ is a domain, then $K$ is a field. A fractional ideal of $R$ is any $R$-submodule $M\subset K$ such that there exists non-zero divisor $r$ such that $rM\subset R$. 
\end{definition}
\end{tcolorbox}


\begin{tcolorbox}[colback=purple!5!white,colframe=purple!75!black]
\begin{definition}
Given $M\subset K$ a fractional ideal, it is called \underline{\textbf{invertible}} if there exists $N\subset K$ fractional ideal such that $MN=R$.
\end{definition}
\end{tcolorbox}


\begin{tcolorbox}[colback=blue!5!white,colframe=blue!30!white]
\begin{proposition}
Let $I'_R$ be the set of fractional ideals of $R$. Then, the set is closed under addition and multiplication. 
\end{proposition}
\end{tcolorbox}



\begin{tcolorbox}[colback=blue!5!white,colframe=blue!30!white]
\begin{proposition}
Every invertibe ideal $M$ is finitely generated. 
\end{proposition}
\end{tcolorbox}


\begin{tcolorbox}[colback=purple!5!white,colframe=purple!75!black]
\begin{definition}
$I_R\subset S'_R$ is the set of invertible fractional ideals.
\end{definition}
\end{tcolorbox}


\begin{tcolorbox}[colback=blue!5!white,colframe=blue!30!white]
\begin{proposition}
The following hold:
\begin{enumerate}
    \item $I_R$ is the group of invertible elements in $I'_R$. 
    \item $M\subset I_R$ implies $M$ finitely generated as $R$-module, and there exists $s\in \Sigma_R^{0}\cap M$, meaning there must be non-zero divisors of $M$. 
    \item Let $N':=(R: M)$, then $N'M\subset R$ and $N'M=R$ iff $M\in I_R$ 
\end{enumerate}
\end{proposition}
\end{tcolorbox}
\begin{proof}
    Exercise. To $2$: let $N\in I_R$ such that $NM=R$; then there exists $n_i\in N$, $m_i\in M$ such that $\sum_{i=1}^{k}n_im_i=1\in R$.  
\end{proof}


\begin{tcolorbox}[colback=blue!5!white,colframe=blue!30!white]
\begin{proposition}
Being invertible is a local property. 
\end{proposition}
\end{tcolorbox}
\begin{proof}
    Exercise. The hint is $N_p:=(R_p: M_p)$ for all prime $p$. 
\end{proof}


\begin{tcolorbox}[colback=red!5!white,colframe=red!30!white]
\begin{theorem}
(Characterization of Dedekind Domains) The following are equivalent:
\begin{enumerate}
    \item $R$ is a Dedekind domain,
    \item $I'_R=I_R$. In other words, all non-zero fractional ideals are invertible.
    \item All non-zero prime ideals of $R$ are invertible.
    \item For all $m\in Max(R)$, we have $R$ Noetherian, and $R_m$ is a DVR.
    \item $R$ is Noetherian, integrally closed, and Krull dimension is $1$.   
\end{enumerate}

\end{theorem}
\end{tcolorbox}
\begin{proof}
    Let $M\in I_R',$ $M\neq 0$, and $r\in R$ such that $rM\in Id(R)$. Then, $M\in I_R$ iff $rM\subset R$ is invertible. 

    For $a\in Id(R)$, choose $r\neq 0$, $r\in a$. Then, $a\subset (r)$ both have a prime decompsotion, with exponenets comparable. 

    For $iii\implies iv$, if $M\in I_R$, then $M$ is finitely generated. Hence, all $p\in Spec(R)$ are finitely generated and we get Noetherian. The claim now is $dim(R_m)=1$. Let $p\in Spec(R_m)$, and $p\neq m$. Look at $a=pm^{-1}$. Then, $p=ma$. Thus, we have $a\subset p$. On the other hand $p\subset qm^{-1}$, we have $p=a$. By Nakayama, we have $p=pm$ in a noetherian local ring and $p$ must be $0$. Check that the maximal ideal is principal, so that the ring is a DVR.

    For $iv\equiv v$: see lemma below.
    
    For $iv\implies i$: let $a\in Id(R)$ be a proper ideal. For every $m\in Max(R)$, there exists $N_m>0$ such that $a_m=m_m^{N_m}$. Since the ring is noetherian and of krull dimension $1$, the set $\{ N_m:N_m>0 \}=Spec_{min}(a)$ is finite.  
\end{proof}


\begin{tcolorbox}
\begin{lemma}
Let $R$ be a local domain. Then the following are equivalent:
\begin{enumerate}
    \item $R$ is integrally closed and of Krull dimension $1$.
    \item $R$ is Noetherian and the maximal ideal is principal
    \item $R$ is a PID
    \item $R$ is DVR.
\end{enumerate}
\end{lemma}
\end{tcolorbox}

\begin{proof}
    For $r\neq 0$, there exists minimal $N$ such that $m^N\subset (r)$. Conclude that there exists $a\in m^{N-1}$ such that $\frac{a}{r}$ is not in $R$. But, $\frac{r}{a}m\subseteq R$. But $m$ is finitely generated, we get $\frac{r}{a}$ is integral over $R$.  
\end{proof}


\begin{tcolorbox}[colback=red!5!white,colframe=red!30!white]
\begin{theorem}
(Permanence Properties) Let $R$ be a Dedekind domain. Then, the following hold:
\begin{enumerate}
    \item If $\Sigma\in R$ is a multiplicative system, and $R_{\Sigma}$ not a field. Then, $R_{\Sigma}$ is a dedekind domain. 
    \item Let $K=Quot(R)$, and $L|K$ a separable extension, $S$ be the inetrgal closure of $R$ in $L$. Then, $S$ is Dedekind domain.
\end{enumerate}
\end{theorem}
\end{tcolorbox}
\begin{proof}
     To $1$: Exercise. To $2$: it suffices to prove for $n\in Max(S)$, then $S_n$ is a DVR. Given $n$, let $m=n\cap R$, so $m\in Max(R)$. Then, $R_m$ is a DVR and $R_n$ lies over $R_m$ by the fundamental inequality.
\end{proof}


\begin{tcolorbox}[colback=red!5!white,colframe=red!30!white]
\begin{theorem}
(Fundamental Inequality) Let $R_v$ be a valuation ring of $K$, and $L|K$ a finite extension. Let $X_v=\{ w\in Val(L): L \textrm{ w|k=v} \}$. Then 
\[\sum_{w\in X_v}(wL:vK)[k_w:k_v]\leq [L:K]\] 
\end{theorem}
\end{tcolorbox}

\begin{tcolorbox}[colback=blue!5!white,colframe=blue!30!white]
\begin{proposition}
Some remarks: Let $R$ be Dedekind domain. 
\begin{enumerate}
    \item If $Spec(R)$ is finite. Then, $R$ is a PID. 
    \item If $a\in Id(R)$ there exists $x,y\in a$ such that $a=(x,y)$. 
\end{enumerate}
\end{proposition}
\end{tcolorbox}

\begin{proof}
    Exercise.
\end{proof}


\begin{tcolorbox}[colback=green!5!white,colframe=green!30!white]
\begin{corollary}
Let $R$ be a Notherian ring. $p\in Spec(R)\cap I_R$ then, $R_p$ is a DVR.
\end{corollary}
\end{tcolorbox}

Let $R$ be a Dedekind domain, $p$ a maximal ideal, and $R_p$ is a valuation ring. Let $v_p: K\to \mathbb{Z}$ be the canonical valuation. Given $M\in I_R$, $M=p_1^{e_1}...p_n^{e_n}$. Define $v_p(M)=e_{p_i}$. In particular, if $M,N\in I_R$, then $v_p(M+N)=min(v_p(M)+v_p(N))$ Suppose $\pi_n(R_p)$. 

Remark: taking integral closure does not preserve Noetherian property in general. The integral closure is usually not a finitely generated $R$-modules other than know special cases: $R=k[x_1,...,x_n]$, and when $dim(R)=1$.


\begin{tcolorbox}[colback=red!5!white,colframe=red!30!white]
\begin{theorem}
(Pronlongation of valuation)Let $(R,m)$ be a valuation ring, $K$ be the quotient field of $R$, and $L|K$ an algebraic extension, $S$ the integral closure of $R$ in $L$. Then, let $X_m$ be the set of maximal ideals in $S$ lying over $m$. In fact $X_m=Max(S)$. Then, the following hold:
\begin{enumerate}
    \item $S_n$ is a valuation ring such that $S_m\cap K=R$ for $n\in Max(S)$.  
    \item $S_w$ is a valuation ring of $S$ such that $S_w\cap K=R$, then $S_w=S_n$ for a unique $n\in Max(S)$.
\end{enumerate}
\end{theorem}
\end{tcolorbox}
\begin{proof}
    HW. 
\end{proof}




\begin{tcolorbox}[colback=purple!5!white,colframe=purple!75!black]
\begin{definition}
The set of invertible ideals of $R$ is callled the \underline{\textbf{Cartier divisors}} of $R$. The group of fractional ideals is called the \underline{\textbf{divisor group}}; If $R$ is Dedekind, then the group is also the group of invertible idals. The group $H_R$ is the subgroup of divisor group of $R$ that consists of principal ideals. 
\end{definition}
\end{tcolorbox}


\begin{tcolorbox}[colback=purple!5!white,colframe=purple!75!black]
\begin{definition}
The \underline{\textbf{Ideal class group}} is the quotient $Div(R)/H_R$. 
\end{definition}
\end{tcolorbox}

\begin{tcolorbox}[colback=yellow!5!white,colframe=yellow!30!white]
\begin{example}
$cl(\mathbb{Z})=1$; $cl(k[t])=1$; If $K|\mathbb{Q}$ a quadratic number field, $O_K$ be the ring of integers. Then, $cl(O_k)$ not always $1$. 
\end{example}
\end{tcolorbox}


\end{document}