\documentclass{article}
\usepackage[utf8]{inputenc}
\usepackage{amsmath}
\usepackage{amsfonts}
\usepackage{amssymb}
\usepackage{tikz}
\usepackage{fullpage}
\usepackage{tikz-cd}
\usepackage{spectralsequences}
\usepackage{adjustbox}
\usepackage{tcolorbox}
\usepackage{xfrac}
\usepackage{graphicx}
\usepackage[parfill]{parskip}
\usepackage{amsthm}
\theoremstyle{definition}
\newtheorem{theorem}{Theorem}[section]
\theoremstyle{definition}
\newtheorem{problem}{problem}[section]
\theoremstyle{definition}
\newtheorem{proposition}{Proposition}[section]
\theoremstyle{definition}
\newtheorem{lemma}[theorem]{Lemma}
\theoremstyle{definition}
\newtheorem{definition}{Definition}[section]
\theoremstyle{definition}
\newtheorem{corollary}{Corollary}[theorem]
\theoremstyle{definition}
\newtheorem{example}{Example}[section]

\title{MATH 603 Notes}
\author{David Zhu}

\begin{document}
\maketitle

\section{More on Commutative Rings}
Let $a,b\in R$. Then $a|b \iff \exists a'\in R, b=aa'$;
A semi ring on $(R,\leq)$ defined by $a\leq b \iff a|b$. Note that $\leq $ is usally not a partial order: let $b\in R^{\times}$, then $a\leq ab\leq a$, but $a\neq ab$.


\begin{tcolorbox}[colback=blue!5!white,colframe=blue!30!white]
\begin{proposition}
    $a\sim b$ iff $a\leq b$ and $b\leq a$ iff $(a)=(b)$ is an equivalence relation.
\end{proposition}
\end{tcolorbox}

For $R$ a domain, the induced relation gives a well-defined definition of greatest common divisor.
\begin{tcolorbox}[colback=red!5!white,colframe=red!30!white]
    \begin{definition}
        The \underline{\textbf{gcd}} of $a,b$, denoted by $gcd(a,b)$, if exists, is any $d\in R$ such that $d|a,b$ and for any other $d'$ satisfying the condition, $d'|d$.
    \end{definition}    
\end{tcolorbox}


\begin{tcolorbox}[colback=purple!5!white,colframe=purple!75!black]
\begin{definition}
    The \underline{\textbf{lcm}} of $a,b$, denoted by $lcm(a,b)$, if exists, is any  $d\in R$ such that $a,b|d$ and for any other $d'$ satisfying the condition, $d|d'$.
\end{definition}
\end{tcolorbox}


\begin{tcolorbox}[colback=blue!5!white,colframe=blue!30!white]
\begin{proposition}
    If $gcd(a,b)$ exists, then $gcd(a,b)=sup \{ d:d\leq a,b \}$.  If $lcm(a,b)$ exists, then $lcm(a,b)=\textrm{inf} \{ d :a,b\leq d \}$. 
\end{proposition}
\end{tcolorbox}

Note that maximal/minimal elements always exists by Zorn's lemma. However, the unique supremum/ \ infimum may not exist. We have our following example:

\begin{tcolorbox}[colback=yellow!5!white,colframe=yellow!30!white]
\begin{example}
Take $R=[\sqrt{-3}]$. Let $a=4=(1+\sqrt{-3})(1-\sqrt{-3})$ and $b=2(1+\sqrt{-3})$. Then, $(1+\sqrt{-3})$ and $2$ are both maximal divisors, but they are not comparable since the only divisors of $2$ are $\{ \pm 1, \pm 2 \}$ by norm reasons, and none divides $1+\sqrt{-3}$.
\end{example}
\end{tcolorbox}


\begin{tcolorbox}[colback=blue!5!white,colframe=blue!30!white]
\begin{proposition}
    Let $a,b\in R$ be given. Then the following hold: $gcd(a,b)=d$ exists iff $(d)$ is the unique maximal prinipal ideal such that $(a)+(b)\subseteq (d)$. Dually, $lcm(a,b)=c$ exists iff $(c)=(a)\cap (b)$. If both holds, then $a\cdot b=lcm(a,b)\cdot gcd(a,b)$
\end{proposition}
\end{tcolorbox}

\begin{proof}
    Easy exercise. Note that the inclusion can be proper, for example, take $R=k[x,y]$ and ideals $(x),(y)$. Then $(1)$ is the gcd, but the containment is proper.
\end{proof}
Recall that $Id(R)$ is partially ordered by inclusion. 


\begin{tcolorbox}[colback=purple!5!white,colframe=purple!75!black]
\begin{definition}
    ($Id(R), +,\cap,\cdot,\leq$) is the lattice of ideals of $R$.
\end{definition}
\end{tcolorbox}

Note that $+,,\cap$ are simply the sums and intersection, but $\cdot$ is the ideal generated by the products, i.e the set of finite sums of products. 


\begin{tcolorbox}[colback=red!5!white,colframe=red!30!white]
\begin{theorem}
    Let $Id^{\infty}(R)$ be the set of non-finitely generated ideals for $R$; the following are equivalent:
\begin{enumerate}
    \item $Id^{\infty}(R)$ is non-empty; 
    \item  There exists an infinite non-stationary chain of ideals $(\sigma_i)$, where $\sigma_i\in Id(R)$;
\end{enumerate}
\end{theorem}
\end{tcolorbox}


\begin{proof}
    For $1\implies 2$, let $I$ be a non-finitely generated ideal of $R$ and pick $x_1\in \in I$. Let $\sigma_1=(x_1)$. Because the ideal is non-finitely generated, we can pick $x_2\in I$ such that $x_2\not \in \sigma_1$. Let $\sigma_2=(x_1,x_2)$. Continue inductively gives us an infinite non-stationary chain of ideals. 

    For $2\implies 1$, take the union of all the ideals in the infinite non-stationary chain. It is an ideal and it cannot be finitely generated. 
\end{proof}


\begin{tcolorbox}[colback=red!5!white,colframe=red!30!white]
\begin{theorem}
    (Cohen's lemma): Let $Id^{\infty}(R)\neq \emptyset$. Then, it has a maximal element and any such maximal element is prime. 
\end{theorem}
\end{tcolorbox}
Before proving Cohen's lemma, we need the following technical lemma:\

\begin{tcolorbox}
\begin{lemma}
    Let $I$ be an ideal. Define $(I:a):=\{ b\in R: ab\in I \}$. If $I+(x)$ and $(I:x)$ are both finitely generated, then $I$ is finitely generated.
\end{lemma}
\end{tcolorbox}
\begin{proof}[Proof of Lemma 1.3]
    By assumption, there is finite set $\{ \alpha_i:=a_i+f_ix: a_i\in I, f_i\in R, i=1,...,k \}$ that generate $I+(x)$, and a finite set $\{ b_j: j=1,...,l \}$ that generate $(I:x)$. We claim that the set $\{ a_i, b_jx: i\in I, j\in J\}$ generate the entire $I$: since $I\subseteq I+(x)$, we can write any element $\pi\in I$ as a finite linear combination $\pi=\sum_{i=1}^{k}g_i\alpha_i=\sum_{i=1}^{k}g_i(a_i+f_ix)$, where $g_i\in R$. We note that $\pi-\sum_{i=1}^{k}g_ia_i=\sum_{i=1}^{k}g_if_ix$ is in $I$; it follows that $\sum_{i=1}^{k}g_if_i\in (I:x)$, so $\sum_{i=1}^{k}g_if_ix$ is generated by the set $ \{b_jx\}$, and we are done. 
\end{proof}

With the lemma in hand, now we can prove Theorem $1.2$
\begin{proof}[Proof of Theorem $1.2$]
    Zorn's lemma implies $Id^{\infty}(R)$ has maximal elements.  Let $I$ one such maximal element, and suppose it is not prime. Then, there exists $xy\in I$ and WLOG suppose $x\not \in I$. By maximality, $I+(x)$ must be finitely generated. By definition, we have $y\in (I:x)$. Lemma $1.3$ implies $(I:x)$ is not finitely generated, and in particular, $I\subseteq (I:x)$. Applying maximality again, we have $I=(I:x)$, which forces $y\in I$, a contradiction.
\end{proof}

\section{Euclidean Rings}


\begin{tcolorbox}[colback=purple!5!white,colframe=purple!75!black]
\begin{definition}
    A \underline{\textbf{Principal Ideal Ring}} is any ring $R$ i which every ideal is principally generated. If $R$ is a domain, then $R$ is called a \underline{\textbf{PID}}. 
\end{definition}
\end{tcolorbox}


\begin{tcolorbox}[colback=purple!5!white,colframe=purple!75!black]
\begin{definition}
    A \underline{\textbf{Factorial Ring}}  is any ring $R$ in which all units can be written as a finite product of irreducible elements, unique up to a unit. If $R$ is domain, then it is called a \underline{\textbf{UFD}}. 
\end{definition}
\end{tcolorbox}
Note that if the ring $R$ it is not a domain, $x|y$ and $y|x$ does not imply $x=uy$ for some unit $u$. Let us prove that this holds for a domain: suppose $x=ys$ and $y=xt$, and $x,y\neq 0$ then $x=xts$, which implies $x(1-ts)=0$. This forces $1-ts=0$, and $t,s$ are then units. We can concoct counterexamples when $R$ is not a domain accordingly: let $R=k[x]/(x^3-x)$ and take $a=x$, $b=x^2$. Clearly, $a|b$ and $b=x^2\cdot x=x^3$, so $b|a$. However, $x$ is not a unit. 



\begin{tcolorbox}[colback=purple!5!white,colframe=purple!75!black]
\begin{definition}
    A \underline{\textbf{Noetherian Ring}} is any ring $R$ such that any ideal is finitely generated. 
\end{definition}
\end{tcolorbox}


\begin{tcolorbox}[colback=purple!5!white,colframe=purple!75!black]
\begin{definition}
    Let $R$ be a domain. A \underline{\textbf{Euclidean norm}} on $R$ is any map $\phi: R\to \mathbb N$ satisfying $\phi(x)=0 $ iff $x=0$ and for every $a,b\in R$ with $b\neq 0$, then there exists $q,r\in R$ such that $a=bq+r$ with $\phi(r)<\phi(b) $. A \underline{\textbf{Euclidean Domain}} is any domain equipped with a Euclidean norm.
\end{definition}
\end{tcolorbox}

Example of Euclidean domains include $\mathbb Z,\mathbb Z[i]$. A non-trivial example $R=F[t]$, with $\phi(p(t))=2^{deg(p(t))}$. A non-example is $\mathbb Z[\sqrt{6}]$ for it is not a PID.



\begin{tcolorbox}[colback=blue!5!white,colframe=blue!30!white]
\begin{proposition}
    Eucldiean Domains are PIDs.
\end{proposition}
\end{tcolorbox}
\begin{proof}
    By the well-ordering principal, for every ideal $I$ in a Euclidean domain, there exists an element other than $0$ of the smallest norm. It is easy exercise that such element generate the entire ideal.
\end{proof}


\begin{tcolorbox}[colback=blue!5!white,colframe=blue!30!white]
\begin{proposition}
    (The Euclidean Algorithm): Given $a,b\in R$, $b\neq 0$. Set $r_0=a,r_1=b$, and continue inductively $r_{i-1}=r_i\cdot q_i+r_{i+1}$. Then, $r_i=0$ for $i>\phi(b)$ and if $r_{i_0}\geq 1$ maximal with $r_{i_0}\neq 0$, then $r_{i_0}=gcd(a,b)$.
\end{proposition}
\end{tcolorbox}
\begin{proof}
    Note that the remainder is strictly decreasing, so $r_i$ must become $0$ after $\phi(b)$ steps. Note that once $r_{i+1}=0$, we have $r_i|r_{n}$ for all $n\leq i$. Coversely, it is clear that any divisor of $a,b$ divdes all $r_n$ for $n\leq i$.
\end{proof}


\section{Principal Ideal Domains}

\begin{tcolorbox}[colback=red!5!white,colframe=red!30!white]
\begin{theorem}
    (Charaterization) For A domain $R$, the following are equivalent: 
\begin{enumerate}
    \item  $R$ is a PID.
    \item every $p\in Spec(R)$ is principal. 
\end{enumerate}
\end{theorem}
\end{tcolorbox}
\begin{proof}
    One direction is trivial; for the other direction, assume that every prime is principal. Then, Cohen's Lemma implies $Id^{\infty}(R)\neq \emptyset$; In particular, every ideal is finitely generated, so the ring is Noetherian. We may apply Zorn's lemma on the set of non-principally generated ideal (since every chain stablizes and has a maximal element), and let $P$ be a maximal non-principally generated ideal. Suppose it is not prime, and let $xy\in P$ with $x\not \in P$. Then, $P\subset (P:x)$ and $P\subset P+(x)$ properly. By maximality, we have $(P:x)=(c)$, and $(I:c)=(d)$. By definition, we have $cd\in I$; moreover, suppose $x\in I$, then $x=cr=cdt$ for some $r,t\in R$. Thus, $I=(cd)$ is principal, a contradiction.  
\end{proof}


\begin{tcolorbox}[colback=blue!5!white,colframe=blue!30!white]
\begin{proposition}
PIDs are UFDs.
\end{proposition}
\end{tcolorbox}


\begin{proof}
    Let $a\in R$ such that $a$ is non-zero and not a unit. Then, there exists $p\in Spec(R)$ such that $(a)\subseteq p$. Hence $R$ being a PID implies $\exists \pi\in R $ such that $p=(\pi)$. Hence, $\pi$ must be prime and $\pi|a$. Set $a_1=a$, $\pi_1=\pi$, and let $a_2$ be the element such that $\pi_1a_2=a_1$. If $a_2$ is not a unit, find $(a_2)\subset (\pi_2)$, where $\pi_2$ is prime. Let $a_3$ be the element such that $\pi_2a_3=a_2$. Continue inductively until $a_n$ is a unit. The process must terminate, for otherwise we get an infinite chain of distinct principal ideals $(a_i)$ that does not stablize( stablizing is equivalent to $(a_n)=(a_{n+1})$ for some $n$, which implies they differ by a unit).
\end{proof}



\begin{tcolorbox}[colback=green!5!white,colframe=green!30!white]
\begin{corollary}
    Let $R$ be a PID; let $P\subset R$ be a set of representatives for the prime elements up to association. For every $a\in R$, $\exists \epsilon\in R^{\times} $ and $e_{\pi}\in \mathbb{N} $ such that almost all $e_{\pi}=0$. Then, every $a\in R$ can be written as $a=\epsilon\prod_{\pi\in P}\pi^{e_{\pi}}$. We proceed to recover $gcd$ and $lcm$, up to associates.
\end{corollary}
\end{tcolorbox}

Note that the above corollary generalizes to the quotient field by replacing $\mathbb{N}$ with $\mathbb{Z}$.

\section{Unique Factorization Domains}


\begin{tcolorbox}[colback=purple!5!white,colframe=purple!75!black]
\begin{definition}
  The following are equivalent for a domain $R$:
  \begin{enumerate}
    \item $R$ is a UFD.
    \item Every  minimal prime ideal (prime of height $1$) is principal and every non-zero, non-invertible elements in contained in finitely many primes. 
  \end{enumerate}
\end{definition}
\end{tcolorbox}


\begin{proof}
   $1\implies 2$: For every non-zero prime $P$, pick $x\in P$ has factor. One of the prime factors must be in $P$, and it follows by minimality that $P$ must be generated by such prime factor. For the second part, the finite factorization of the element gives precisely the finite primes that it is contained in. 
   $2\implies 1$:given $x\in R$, the finitely many primes containing $x$ are principally generated by prime elements, which gives a factorization.

   
\end{proof}
Remark: we recover the $gcd$ and $lcm$ definition using the same factorization as Corollary $3.1.1$.


\begin{tcolorbox}[colback=red!5!white,colframe=red!30!white]
\begin{theorem}
    (Gauss Lemma)Let $R$ be a UFD; then $R[t]$ is a UFD. 
\end{theorem}
\end{tcolorbox}
To prove the theorem, we need the following lemma on contents:


\begin{tcolorbox}[colback=purple!5!white,colframe=purple!75!black]
\begin{definition}
    Let $f(t)=a_0+...+a_nt^n$ be given. Then, the \underline{\textbf{content}} of $f$, denoted $C(f)$, is the GCD of all coefficients. In particular, $C(f)|a_i$ for all $i$, and $f_0:=f/(C(f))$ has content $1$. 
\end{definition}
\end{tcolorbox}


\begin{tcolorbox}
\begin{lemma}
    Let $R$ be a UFD, then the following hold: $(1).$ $C(f): R[t]\to R$ given by $f\mapsto C(f)$ is multiplicative; in particular, if $C(f)=C(g)=1$, then $C(fg)=1$. 
\end{lemma}
\end{tcolorbox}

\begin{proof}[Proof of lemma $4.2$]
 given $f(t)=a_0+...+a_nt^n$ and $g(t)=b_0+...+b_mt^m$. If one of $f$, $g$ is constant, then it is easy exercise; suppose neither is constant, then set $f=f_0\cdot C(f)$ and $g=g_0\cdot C(g)$. Clearly we have $C(f)\cdot C(g)| C(fg)$. Hence it suffices to prove that $C(f_0g_0)=1$. Equivalently, let $\pi\in R$ be a prime element, we want to show there exists a coefficient $c_k\in f_0g_0$ such that $\pi$ does not divide $c_k$. Suppose $\pi| c_k=\sum_{i+j=k}a_ib_j$ for all $k$. Because $C(f_0)=C(g_0)=1$, then there exists minimal $a_i,b_j$ such that $\pi$ does not divide $a_{i_0},b_{j_0}$. Then, $\pi$ does not divede $C_{i_0+j_0}$.

\end{proof}
Note that proof goes similarly for quotient fields. 


\begin{tcolorbox}[colback=red!5!white,colframe=red!30!white]
\begin{theorem}
    Let $R$ be a UFD. For $f(t)\in R[t]$, the following are equivalent: 
    \begin{enumerate}
        \item $f(t)$ is prime
        \item  $f(t)$ is irreducible
        \item  If $f=a_0\in R$ and $a_0$ is prime or $C(f)=1$ and $f$ is irreducible. 
    \end{enumerate} 
\end{theorem}
\end{tcolorbox}


\end{document}
