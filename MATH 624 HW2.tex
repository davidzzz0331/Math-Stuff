\documentclass{article}
\usepackage[utf8]{inputenc}
\usepackage{amsmath}
\usepackage{amsfonts}
\usepackage{amssymb}
\usepackage{tikz}
\usepackage{fullpage}
\usetikzlibrary{calc}
\usepackage{tikz-cd}
\usepackage{spectralsequences}
\usepackage{adjustbox}
\usepackage[backend=biber, style=alphabetic]{biblatex}
\usepackage{xfrac}
\usepackage{tcolorbox}
\usepackage{xcolor}
\usepackage{graphicx}
\graphicspath{ {D:/Chrome Downloads./} }
\usepackage[parfill]{parskip}
\usepackage{amsthm}
\addbibresource{sample.bib}
\theoremstyle{definition}
\newtheorem{theorem}{Theorem}[section]
\theoremstyle{definition}
\newtheorem{definition}{Definition}[theorem]
\theoremstyle{definition}
\newtheorem{remark}{Remark}[theorem]
\theoremstyle{definition}
\newtheorem{proposition}{Proposition}[theorem]
\theoremstyle{definition}
\newtheorem{lemma}[theorem]{Lemma}
\theoremstyle{definition}
\newtheorem{corollary}{Corollary}[theorem]
\theoremstyle{definition}
\newtheorem{example}{Example}[theorem]
\tikzset{curve/.style={settings={#1},to path={(\tikztostart)
    .. controls ($(\tikztostart)!\pv{pos}!(\tikztotarget)!\pv{height}!270:(\tikztotarget)$)
    and ($(\tikztostart)!1-\pv{pos}!(\tikztotarget)!\pv{height}!270:(\tikztotarget)$)
    .. (\tikztotarget)\tikztonodes}},
    settings/.code={\tikzset{quiver/.cd,#1}
        \def\pv##1{\pgfkeysvalueof{/tikz/quiver/##1}}},
    quiver/.cd,pos/.initial=0.35,height/.initial=0}
\title{MATH 624 Homeworks}
\author{David Zhu}

\begin{document}
\maketitle

\section*{Problem 1}
\subsection*{(a)}
The fixed field of the aboslute galois group $G_k$ acting on $k^s$ is $k$; The fixed field of the aboslute galois group acting on $\overline{k}$ is $k^i$. The assertion follows immediately. 


\subsection*{(b)}
$(i)\implies (ii)$ $V$ minimal implies $I(V)$ is maximal since $I(-)$ reverses inclusion. 

$(ii)\implies (iv)+\textrm{irreducible}$: $I(V)$ maximal implies $K[V]$ is a finite field extension of $k$. In particular, for each $x_i$, $I(V)\cap k[x_i]\neq 0$. We may thus pick $f_1(x_1),..,f_n(x_n)\in I(V)$, so that the $i$th  coordinate of $a\in V$ must satisfy $f_i$, and there are only finitely many roots to choose from. 

$(iv)+\textrm{irreducible}\implies (iii)$: Points on $V(\overline{k})$ corresponds bijectively to $k$-algebra homomorphisms $k[V]\to \overline{k}$ by evaluation. If $V$ has only finitely many points, then $k[V]$ must be a finite field extension of $k$, so points on $V$ correspond to $Gal(K[V]|k)$, which is a quotient subgroup of $G_k$. Clearly the action of $G_k$ on the quotient is transitive, and thus $V$ is a $G_k$ orbit. 

$(iii)\implies (i)$: a galois orbit is necessarily finite, thus $k[V]$ is a finite extension of $K$ and thus $I(V)$ is maximal, which implies $V$ is minimal.

\section*{Problem 3}
$2x_1^2-3x_2x_3$ is irreducible by degree reasons, therefore $V_1:=V(2x_1^2-3x_2x_3)$ is irreducible. Clearly irreducible implies connected. 

$x_1^2x_2^2-x_3^4=(x_1x_2-x_3^2)(x_1x_2+x_3^2)$, therefore $V_2:=V(x_1^2x_2^2-x_3^4)=V(x_1x_2-x_3^2)\cup V(x_1x_2+x_3^2)$ is not irreducible. The irreducible components are connected and $(0,0,0)$ is in the intersection, so $V_2$ is connected. 

Clearly the intersection of an irreducible set and a reducible set is reducible; the intersection $V_3:V_1\cap V_2$ is the union $(V_1\cap V(x_1x_2-x_3^2))\cup (V_1\cap V(x_1x_2+x_3^2))$, where both are connected curves and the intersection contains $(0,0,0)$, so it is connected.

The union is still reducible since $V_2$ is reducible, and $V((x_1^2x_2^2-x_3^4)(2x_1^2-3x_2x_3))$ does not have a prime radical. The union is connected since the intersection is non-empty.

\section*{Problem 5}
\subsection*{(a)}
If the closure of $x$ is contained in $V_1\cup V_2$, where $V_1$, $V_2$ are closed and disjoint, then clearly $x$ must belong to one of them. Thus, the closure of a singleton must be irreducible. Note that the argument does not 

\subsection*{(b)}
Take any variety of dimension $0$ to $n$, and choose a generic point on that variety suffices.

\subsection*{(c)}
Consider the evaluation homomorphism $\varphi_x: k[x_1,...,x_n]\to K$ by plugging in $x$. The coordinate ring of $V_x$ is precisely $k[x_1,...,x_n]/ker(\varphi_x)$. 

\subsection*{(d)}
For example, consider $V_x=\{x\}=(\sqrt[p]{t},1)=V(x^p-t,y-1)$ in $\mathbb{A}^2_{F_p(t)}$. 


\section*{Problem 6}
\subsection*{(c)}
Suppose $V$ is irreducible, then $k[V]$ is a domain and admits an embedding into $K$, and one such embedding determines a generic point given by the image of $(x_1,...,x_n)$. 
\subsection*{(d)}
If $dim(V)>0$, and suppose $V$ admits a generic point. Then, $trdeg(K|k)\geq 1$ and $k[V]\to K$ admits infinitely many embeddings.  

\section*{Problem 7}
Since closed sets are finite union of points in $\mathbb{A}^1$, ${A}^1-\{0\}$ clearly cannot be written as a union of two disjoint closed sets. Clearly the disconnected subsets are finite union of singletons with more than $1$ element, so all infinite subsets are connected. 

Take $V((x-2)(x-3))$, which is a disconnected infinite set in $\mathbb{A}^2$.

\section*{Problem 8}
\subsection*{(a),(c)}
Trivial exercise. See 602 Notes.

\subsection*{(e)}
The variety $V$ can be parallel to $H_f$, so the intersection is trivial. Otherwise, the assertion follows from the krull dimension of $k[x_1,...,x_n]/(I(V),f)$, which has dimension $dim(V)-1$. 



\section*{Homework 2}
\subsection*{Problem 1b}
Suppose $U_f$ is not empty. Let $W=\{ a\in V: k(a)|k \textrm{ is a finite algebraic extension} \}$, which corresponds to the vanishing locus of maximal ideals of $k[V]$. Clearly $W\subset V(\overline{k})$, so it suffices to show that $W\cap U_f$ is dense in in $U_f$ for every $f$, which is equivalent to every open $U_f$ containing a point in $W$. To see this, consider a maximal ideal in $k[V]_f$, which must be the image of a maximal ideal in $k[V]$ under localization: suppose otherwise, then every maximal ideal of $k[V]$ contains $f$, which implies $f$ is in the Jacobson radical of $k[V]$. However, $k[V]$ has trivial Jacobson radical since $k[X]$ is Jacobson, which implies $f=0$ and $U_f$ is empty, and contradiction. Then, the locus of the maximal ideal is contyained in $U_f\cap W$.


\subsection*{Problem 2b}
A representative of $\tilde{O}_a$ is given by a pair $(W_1, \frac{f_1}{g_1})$, with $g_1\neq 0$ on $W_1$, and $(W_1, \frac{f_1}{g_1})\sim (W_2, \frac{f_2}{g_2})$ iff there exists a open $U_{h'}\subset W_1\cap W_2$ such that $\frac{f_1}{g_1}=\frac{f_2}{g_2}$ on $U_{h'}$. On the other hand, a representative of $k[V]_{\mathfrak{p}_a}$ is given by some $\frac{f}{g}$, where $g(a)\neq 0$. By continuity, there exists a basic open $U_h$ containing $a$ on which $g$ does not vanish. We define the $k$-algebra homomorphism: 
\[i: k[V]_{\mathfrak{p}_a}\to \tilde{O}_a  \ \ \ \frac{f}{g}\mapsto (U_h, \frac{f}{g})\]

Surjectivity is obvious by construction, so there are two things to check: well-definedness (it is clearly that this will be a $k$-algebra morphism once we check well-definedness) and injectivity.

Well-definedness: suppose $\frac{f}{g}\sim \frac{f'}{g'}$ in $k[V]_{\mathfrak{p}_a}$, which means there exists some $h'\in K[V]$ such that $h'(fg'-f'g)=0$, which implies $\frac{f}{g}=\frac{f'}{g'}$ on $U_{h'}$. Thus, both will be mapped to the equivalence class $(U_{h'},\frac{f}{g})$.

Injectivity: suppose $i(\frac{f}{g})=(U_h,\frac{f}{g})$ represents the $0$ element. WLOG, we may assume that $f$ vanishes on $U_h$, for otherwise we may replace $U_h$ with a smaller basic open. Then, $\frac{f}{g}\sim \frac{0}{1}$ in $ k[V]_{\mathfrak{p}_a}$ since $h(f\cdot 1-g\cdot 0)$ is identically $0$ on $V$.

\subsection*{Problem 3b}
By problem 2b, the stalk is isomorphic to $k[V]_{p_a}$, which is always local. In regards to when $k[V]_{p_a}$ is a not a domain, it will be when there exists an $x\in p_a$ such that $\exists y\in p_a$ and $xy=0$, but $xz\neq 0$ for every non-zero $z\not \in p_a$. For example, let $V=V(xy)$. Then, $k[V]=k[x,y]/(xy)$. Take $a=(0,0)$, then $p_a=(x,y)$, and we have $xy=0$ but $xz\neq 0$ for every non-zero $z$ not in $(x,y)$. 

Note that a reduced Noetherian ring is integral iff it has a unique minimal prime. Another method of detection for integrality is iff $p_a$ contains a unique minimal prime of $k[V]$ (because it is reduced Notherian), which corresponds to $a$ belonging to a unique irreducible component.

\subsection*{Problem 4}
\subsubsection*{(a)}
$V$ is irreducible iff $I(V)$ is prime iff $k[V]$ is a domain iff $k(V)$ is a field. The Krull dimension of $k(V)$ and the trascendence degree are the same by Noether normalization.
\subsubsection*{(b)}
Take the finite set of minimal primes $\{p_1,...,p_n\}$ of $k[V]$, and recall that the union of the minimal primes is precisely the zero-divisors of $k[V]$, and the intersection is the trivial nilradical. Then, localize at $S=k[V]\setminus \cup p_i$, and $S^{-1}k[V]$ has unique maximal primes $S^{-1}p_1,...,S^{-1}p_n$, which are coprime. By chinese remainder, we have
\[k(V)=S^{-1}k[V]/(0)=S^{-1}k[V]/\cap S^{-1}p_i\cong \prod k(V_i)  \]
\subsubsection*{(c)}
Suppose $V$ is irreducible. Note that $k[V_{k^s}]\cong k[V]\otimes_k k^s$, so $k(V_{k^s})\cong k(V)\otimes_k k^s$ after taking the field of fractions. Thus, absolute irreducibility of $V$ is equivalent to the integrality of $k(V_{k^s})\cong k(V)\otimes_k k^s$. Suppose $\overline{k}\cap k(V)$ is not purely inseparable over $k$, so there exists $\alpha$ algebraic over $k$, and $k(\alpha)\otimes_k k(\alpha)$ is a subring of $k(V)\otimes_k k^s$, which is not integral. To see this, note, let $p(t)$ be a minimal polynomial of $\alpha$, then
\[k(\alpha)\otimes_k k[t]/p(t)\cong k(\alpha)[t]/p(t)\]
cleary has $(x-\alpha)$ as a zero-divisor. 

Conversely, suppose $k(V)\cap \overline{k}$ is purely inseparable. It suffices to show that $k(V)\otimes_k k[t]/p(t)\cong  k(V)[t]/p(t)$ is integral for every irreducible $p(t)$. If there is $q(t)\in k(V)[t]$ that divides $p(t)$, then $q(t)$ is also contained in $k^s[t]$, so $q(t)\in (k^s\cap k(V))[t]=k[t]$,  which forces it to be $1$ or $p(t)$, and the ring is still integral. 

\subsubsection*{(d)}

\subsection*{Problem 5}
\subsubsection*{(a)}
The correct statement should be $\tilde{O}_x$ is a domain iff $x$ is contained in a unique irreducible component, and the proof is given in problem $3$.
\subsubsection*{(b)}
It is a standard point-set topology argument that finite intersection of open dense sets is still open and dense.

\subsubsection*{(c)}
The colimit is the function field of $V$. The detail proofs are given in HW3 problem $10$.

\subsection*{Problem 8}
\subsubsection*{(a)}
Clearly the empty set and the whole line is open affine, so the only non-trivial case is the line minus a finite set of points. Let $a_1,...,a_n$ be a finite number of points, and $\mathbb{A}^n\setminus \{a_1,..,a_n\}$ is isomorphic to the affine algebraic set $V(y(x-a_1)...(x-a_n)-1)\subset \mathbb{A}^{n+1}$ given by the map 
\[\varphi: \mathbb{A}^n\setminus \{a_1,..,a_n\}\to V(y(x-a_1)...(x-a_n)-1) \ \ \ \ t\mapsto (t,\frac{1}{(t_1-a_1)...(t_n-a_n)})\]
with inverse $\psi: (x,y)\mapsto x$. Both functions are Zariski continuous since they are rational functions. Let $T$ be an open of $\mathbb{A}^n$ and $U$ be an open of $\mathbb{A}^{n+1}$ such that $f(T)\subset U$. Then, given any regular function $\frac{f(x,y)}{g(x,y)}$ on $U$, the pullback $\frac{f(x,\frac{1}{x})}{g(x,\frac{1}{x})}$ is a regular function on $T$ by multiplying large enough powers of $x$ to the numerator and denominator. The other direction is trivial since the pullback will be the same function on one variable. Thus, $\varphi$ and $\psi$ are $k$-isomorphisms. 
\subsubsection*{(b)}
The open $U:=\mathbb{A}^2\setminus \{(0,0)\}$ is not affine. Note that $U$ is covered by $U_1=D_{f(x,y)=x}$ and $U_2=D_{f(x,y)=y}$, whose ring of regular functions are $k[x,y]_x$ and $k[x,y]_y$. On the overlap, the ring of regular functions is $k[x,y]_{x,y}$. Let $f$ be a regular function on $U$, which restricts to a regular function of the form $p_1/x^m$ on $U_1$ and $p_2/y^n$ on $U_2$. The compatibility condition on $U_1\cap U_2$ implies that $p_1/x^m=p_2/y^n$, which implies $x^mp_2=y^np_1$. Since $k[x,y]$ is a UFD, $x_m| p_1$, and $f$ is in $k[x,y]$. Thus, $O(U)\cong k[x,y]\cong O(\mathbb{A}^2)$. Thus, if $U$ were affine, the inlusion map $i: U\to \mathbb{A}^2$ is an isomorphism, which is false. 



\section*{Homework 3}
\subsection*{Problem 1}
\subsubsection*{(a)}
Suppose $X$ is separated, and $Z$ is closed in $X$. Since $Z\to X$ is a closed immersion, $Z\times Y\to X\times Y$ is a closed immersion for all $Y$. In particular, this implies the composition $Z\times Z\to Z\times X\to X\times X$ is a closed immersion. We then have the commutative triangle
\[\begin{tikzcd}
Z\arrow[r,"\Delta_Z"]\arrow[d]& Z\times Z\arrow[d]\\
X\arrow[r]& X\times X
\end{tikzcd}
\]
where all arrows except $\Delta_Z$ is a closed immersion. It follows that $\Delta_Z$ is a closed immersion as well. Suppose now $X$ is proper. Let $T$ be the terminal object in our category (Spec(k) for the category of $k$-prevarieties ). We have the pullback squares
\[\begin{tikzcd}
	W\cong Z\times Y & {X\times Y} & Y \\
	Z & X & T
	\arrow["f", from=1-1, to=1-2]
	\arrow[from=1-1, to=2-1]
	\arrow["g", from=1-2, to=1-3]
	\arrow[from=1-2, to=2-2]
	\arrow[from=1-3, to=2-3]
	\arrow["i"', from=2-1, to=2-2]
	\arrow[from=2-2, to=2-3]
\end{tikzcd}\]
where the two smaller squares are pullbacks by definition, and the outer rectangle is also a pullback by general categorical nonsense. Note that closed immersions is stable under pullback, so $f$ is also a closed immersion, and $W\to Y$ is a closed map by composition. 

\subsubsection*{(b)}
Following the hint, we have canonical isomorphisms $(X\times X)\times (Y\times Y)\cong (X\times Y)\times (X\times Y)$, which induces an isomorphism $\Delta_X\times \Delta_Y\cong \Delta_{X\times Y}$. We see that $\Delta_{X\times Y}$ is closed iff both $\Delta_X$ and $\Delta_Y$ are closed, so $X\times Y$ is separated iff $X,Y$ are both separated. 

Note that universally closed morphisms are stable under pullbacks by definition, so proper morphisms are stable under pullbacks. Moreover, composition of proper morphisms is also proper. In particular, the product of two proper morphisms is proper since it case be written as the composition of two proper morphisms from pullback.

\subsection*{Problem 2}
\subsubsection*{(c)}
Checking $R_f^0$ is an $R_0$-algebra is trivial; for the second part, first recall the canonical homeomorphism $D_f\cong Spec(R_f)$. Then, $D_f^+$ is the subspace of homogeneous primes of $Spec(R_f)$, i.e $Proj(R_f)$. Thus, it suffices to show that $Proj(R_f)$ is homeomorphic to $Spec(R_f^0)$. Consider the map 
$Proj(R_f)\to Spec(R_f^0)$ given by $\oplus_{d\geq 0}I_d\mapsto I_0$, which is easily see to be well-defined and continuous since it is induced by the inclusion $R_f^0\to R_f$. We will explicitly construct an inverse $f^{-1}: Spec(R_f^0)\to Proj(R_f)$, given by $p_0\mapsto \sqrt{\oplus_{d\geq 0} p_0S_d}$. It is standard to check the image is a homogeneous prime ideal. Let $g=\sum_i g_i$ be an element in $R_f$, and $W_g$ be a basic open in $Proj(R_f)$. Then, the inverse image of $W_g$ is the finite intersection of basic opens $\cap \tilde{W}_{g_i}$ in $Spec(R_f^0)$, which is open, and we have continuity. The composition $f\circ f^{-1}$ is clearly the identity, and we are left to show that $f^{-1}\circ f(\oplus_{d\geq 0} I_d)=\oplus_{d\geq 0} I_d$. For simplicity, assume $deg(f)=1$ so we don't have keep track of it. Take $s\in I_d$, then $\frac{s}{f^d}\in I_0$, and it follows that $s\in f^{-1}\circ f(\oplus_{d\geq 0} I_d)$; conversely, suppose $q\in  \sqrt{\oplus_{d\geq 0} p_0S_d}$ where $deg(q)=d$, then $\frac{q}{f^d}=\frac{q'}{f^{e}}$ for some $q'\in I_e$. We then have 
\[f^k(f^eq-f^dq')=0\]
which implies $q\in \sqrt{\oplus_{d\geq 0} p_0S_d}$ by primeness as $f^{k+e}$ is not in the prime ideal.

\subsection*{Problem 3b}
It is not the coproduct since there are no canonical graded morphism from $R\to R\otimes^{gr}_A S$. Given graded algebras $P,Q$, then correct coproduct is the graded-algebra 
\[P\otimes_A Q:=\oplus_{m+n=d}P_m\otimes_A Q_n\]
with coordinate-wise multiplication structure and bilinear $A$-action, together with canonical inclusions $P\to P\otimes_A Q$ and $Q\to P\otimes_A Q$.


\subsection*{Problem 4}
$Hom_k(\mathbb{A}_K^1,\mathbb{A}_K^1)$ is in bijection with $Hom_k(k[x],k[x])$, which is specified by the image of $x$. Thus, \[Hom_k(\mathbb{A}_K^1,\mathbb{A}_K^1)\cong k[x]\]. Automorphisms of $\mathbb{A}^1$ corresponds to automorphisms of $k[x]$, and which correponds to mapping $x$ to a linear polynomial $ax+b$ with $a\neq 0$. 

\subsection*{Problem 5}
Let $U_1,U_2$ be the affine open covers of the line with two origins. The diagonal of the two affine opens are of the form $U_1\times_k U_1$ and $U_2\times_k U_2$. The closure of the two sets must contain $U_1\times_k U_2$ and $U_2\times_k U_1$, which forces the closure to be the entire product. 

\subsection*{Problem 6}
\subsubsection*{(a)}
Since the product of $k$-prevarieties is the categorical product, it is automatically associative and commutative up to isomorphism by general abstract nonsense.
\subsubsection*{(b)}
The finite product of affine variety $Spec(k[V])$ and $Spec(k[W])$ is isomorphic to $Spec(k[V]\otimes_k k[W])$, which is affine. Note that all affine varieties are separated, since the multiplication map $A\otimes A\to A$ is surjective, so the map $Spec(A)\to Spec(A\otimes A)$ is a closed immersion. The properness of the product follows from the fact that proper morphisms are stable under pullbacks. 
\subsubsection*{(c)}
The statement follows from the algebraic fact that 
\[dim(k[V])+ dim(k[W])=dim(k[V]\otimes_k k[W])\]
To see this, use Noether normalization so that the tensor product of coordinate rings is a finite module over tensor product of polynomial rings, which is again a polynoimal ring whose krull dimension is the sum of that of $k[V]$ and $k[W]$. 



\subsection*{Problem 7}
Choose an affine covering $X=\cup V_i$. Then, the sets $\{\prod_{n}(V_{i})_n\}$ is an affine covering of $X^n$, and it suffices to check for the affine opens. It is clear that a product of affine varieties is absolutely irreducible/geometrically integral iff every factor is so. 

\subsection*{Problem 8}
\subsection*{(a,b)}
We want separatedness for this question. If $X$ is separated, then $\Delta$ is closed in $X\times_k X$, and $\Delta\cap(U_1\times_k U_2)\subset X\times_k X\cong $ is a closed subset of $(U_1\times U_2)$, and is also isomorphic to $\Delta(U_1\cap_k U_2)$ and thus $U_1\cap U_2$ since it is an open immersion, which implies it is affine.


\subsection*{Problem 10}
\subsubsection*{(a)}
The part is done in problem $5b$ HW2 and Problem $8$ HW3.
\subsubsection*{(b)}
This part simply follows from the definition of a colimit.
\subsubsection*{(cd)}
In general, if $U$ is an dense set, then the colimit taken over open subsets of $U$ coincides with the colimit taken over open susbets of $X$: for every open $W\subset X$, we have $W\cap U\neq \emptyset$ open in $U$, so the directed system is cofinal. Thus, $\kappa(X)\cong \kappa(U)$ in this case. If $U$ were affine, then $\kappa(U)\cong k(U)$, which is a field iff $U$ were irreducible. Moreover, the trascendence degree of $k(U)$ is precisely the dimension of the affine variety.

Gnerally, each irreducible component of $X$ admits a dense open affine subset $U_i$ whose pairwise intersection is empty. The assertion $k(X)\cong \prod k(X_i)$ where $X_i$ are irreducible components follows. 



\section*{Homework 4}

\subsection*{Problem 1}
Given $f,g: Y\to X$, the universal property of the product gives a morphism $h: Y\to X\times_k X$. It is immediate that $\Delta_{f,g}=h^{-1}(\Delta_X(X))$, which is closed if $X$ is separated. Conversely, take $Y=X\times_kX$ with $f,g$ being the two projection maps. Then, $\Delta_{f,g}$ is the diagonal which is assumed to be closed and $X$ is then separated. 

\subsection*{Problem 2}
\subsubsection*{(a)}
The first part of the problem is given in Problem 8, HW3.

\subsection*{Problem 3}
\subsubsection*{(a)}
Follows from the fact that degree of polynomials is multiplicative. 
\subsubsection*{(b)}
We note that the degree does not change after homogenization, so $D_i\circ H_i(f)=(x_i^{deg(f)}f)/(x_{i})^{deg(f)}=f$. For the other direction, write $g=x_i^Ng_0$, where $x_i\nmid g_0$. Note that $deg(g_0)=deg(D_i(g_0))$, so it is clear that $H_i\circ D_i(g_0)=g_0$. It is easy to see that $H_i\circ D_i((x_i^n))=1$, so $H_i\circ D_i(g)=g_0$ by multiplicativity.


and $H_i\circ D_i(x_i)=1$.


\subsubsection*{(c)}
We will make the definitions clear: the ith homogenization of an ideal $\mathfrak{a}$ is the ideal generated by $\langle H_i(f): f\in \mathfrak{a} \rangle$, and the ith dehomogenization of a homogeneous ideal $\mathfrak{b}$ is the ideal generated by $\langle D_i(f): f\textrm{ homogeneous in } \mathfrak{b} \rangle$. In this case, clearly we have $D_i\circ H_i(\mathfrak{a})\supset \mathfrak{a}$ by part $b$. To see the other direction, it suffices to show that for every homogeonous $f=\sum a_iH_i(g_i)$ with $g_i\in \mathfrak{a}$, we have $D_i(f)\in \mathfrak{a}$, and this is straightforward to check. 




\subsubsection*{(d)}
$H_i\circ D_i(\mathfrak{a})$ for $\mathfrak{a}$ homogeneous is the direct sum $\oplus a_i^{0}$, where $a_i^{0}=\{f\in a_i: x_i\nmid f\}$.



\subsection*{Problem 4}
The gluing data amount to identifying the $n+1$ open sets $U_i$, which are isomorphic to $\mathbb{A}^n$ by the identification $(a_1,...,a_n)\mapsto [a_1:...a_{i-1}:1:a_{i+1}:...:a_n]$.

WLOG, suppose $i<j$. The open sets $U_{ij}$ are then the set with homogeneous coordinates $x_i, x_j\neq 0$, which is identified with the subset of $U_i=\mathbb{A}^n$ with the $j$th affine coordinate non-zero, and $U_{ji}$ the subset of $U_j=\mathbb{A}^n$ with the $i$ th affine coordinate non-zero The transition function $U_{ij}\to U_{ji}$ is then defined by 
\[(a_1,...,a_n)\mapsto (\frac{a_1}{a_i},..,\frac{a_{j-1}}{a_i},\frac{1}{a_i},...,\frac{a_n}{a_i})\]



\subsection*{Problem 6}
\subsubsection*{(a)}
For $U_1=V(\langle2x_1^2-x_2x_3 \rangle)$ and $U_2=V(\langle x_1x_2-x_1 \rangle)$, the defining ideals are principal, so we may simply consider the projective ideals defined by their homogenization $\overline{U_1}=V(2x_1^2-x_2x_3)$ and $\overline{U_2}=V(x_1x_2-x_1x_0)$. The points are infinity for $U_1$ is $[0,:a_1:a_2:a_3]$, where $(a_1,a_2,a_3)\in U_1\setminus \{0\}$. 

The points at infinity for $\overline{U_2}=V(x_1x_2-x_1x_0)=V(x_1)\cup V(x_2-x_0)$ are $[0:0:a_2:a_3]$ where $a_2,a_3$ not both $0$, and $[0:a_1:0:a_3]$ where $a_1,a_3$ not both $0$. 


The intersection $U_1\cap U_2=V(x_2)\cup V(x_3)\cup V(2x_1-x_3)$. The closure of union is the union of the closures, so we have $\overline{U_1\cap U_2}=V(x_2)\cup V(x_3)\cup V(2x_1-x_3)$. The points at infinity are $[0:a_1:0:a_3]$ where where $a_1,a_3$ not both $0$, $[0:a_1:a_2:0]$ where where $a_1,a_2$ not both $0$, and $[0,a_1,a_2,\frac{a_1}{2}]$ where $a_1,a_2$ not both zero. 

\subsubsection*{(b)}
Recall that the twisted cubic is defined by $V(x_1^2-x_2,x_1x_2-x_3,x_2^2-x_1x_3)\subset \mathbb{A}^3$. The closure is $V(x_1^2-x_0x_2, x_1x_2-x_3,x_2^2-x_1x_3)\subset \mathbb{P}^3$, since it has one extra point, and any affine variety is not compact. The points at infinity is $[0:0:0:1]$.

\subsection*{Problem 7}
\subsection*{(a)}
Let $X$ be the irreducible, and $U_i$ be the standard affine opens. Then, $I(X)=H_i(I(U_i\cap X))$. To see that, note $\overline{X\cap U_i}\subseteq X$, so $I(X)\subseteq I(\overline{X\cap U_i})=H_i(I(U_i\cap X))$. 

Conversely, since $X\subseteq \cup X\cap U_i$, we have 
\[I(X)\supseteq I(\cup X\cap U_i)=\cap I( X\cap U_i)=\cap H_i(I( X\cap U_i))\].

\subsection*{Problem 8}
\subsection*{(b)(c)(d)}
b,c are easy to see. To prove $\mathbb{P}^n$ is separated, it suffices to show that for every $x,y\in \mathbb{P}^n$, there exists an affine open that contains $x,y$. Using standard reduction, it suffices to show that a basic open $D^+_{x_1+x_2}$ is affine, and that follows from the automorphism of $\mathbb{P}^n$ that sends $x_1+x_2$ to $x_1$, and the basic open $D^+_{x_1+x_2}$ is then isomorphic to the standard affine open $D^+_{x_1}$.


\subsection*{Problem 10}
\subsection*{(a)}
The result follows from part (b).
\subsection*{(b)}
It suffice to show this for every irreducible component, and for an dense affine open subset. Then the claim follows from the fact that for affine varieties, the each irreducible component of $X\cap Y$ has dimension at least $\textrm{dim}(X)+\textrm{dim}(Y)- n$.


\section*{Homework 5}
\subsection*{Problem 1}
Suppose $k$ is algebraically closed. Recall that a regular function $\varphi$ on $\mathbb{P}^n$ is locally of the form $\frac{p}{q}$ on some $U$, where $p,q$ are homogeneous of the same degree, with no common factors. If $q$ is not a constant, then it vanishes at some point $a\in \mathbb{P}^n$. But for any open set $U'$ containing $a$, $\varphi$ is of the form $\frac{p'}{q'}$. On $U\cap U'$, we have 
\[\frac{p}{q}=\frac{p'}{q'}\]
so we must have $qp'=pq'$, which implies $q|q'$, and $\varphi$ is not regular at $p$. Thus, the only regular functions on $\mathbb{P}^n$ is constants. 

If $k$ is not algebraically closed, we can have non-trivial regular functions. For example, $\frac{x^2}{y^2+x^2}$ is regular on $\mathbb{P}_{\mathbb{R}}^1$.

\subsection*{Problem 2}
For the following problems, it is useful to prove the following proposition:

\begin{tcolorbox}[colback=blue!5!white,colframe=blue!30!white]
\begin{proposition}
A $k$-morphism $f:\mathbb{P}^n\to \mathbb{P}^m$ is of the form 
\[x\mapsto [f_0(x):...:f_m(x)]\]
where $f_i$ are homogeneous polynomials of the same degree and $V(f_0,...f_m)=\emptyset$.
\end{proposition}
\end{tcolorbox}
\begin{proof}
	By abuse of notation, let $\mathbb{A}_i^m$ denote the standard $i$th affine open cover of $\mathbb{P}^m$, and let $X_i:=f^{-1}(\mathbb{A}_i^m)$, which is dense open. The restriction $f|_{X_i}: X_i\to \mathbb{A}_i^m$ is of the form $(\varphi_0,...\varphi_{i-1},\varphi_{i+1},...,\varphi_m)$, where $\varphi_k=\frac{p_k}{q_k}$  are elements in $O_{\mathbb{P}^n}(X_i)$. By multiplying the common denominator, we can turn this back to homogeneous coordinates, so that $f|_{X_i}$ is given by $x\mapsto (f_0:....:f_m)$. Suppose we do the same procedure and get $f|_{X_j}$ given by $x\mapsto (g_0:....:g_m)$, then on $X_i\cap X_j$ they must agree. Since $k[X]$ is a UFD, the two expression are the same modulo a constant. 
\end{proof}

Using the result of problem $3$, we see that $\mathbb{P}^n\times \mathbb{P}^m$ has a non-trivial map to $\mathbb{P}^n$ given by the projection, but $\mathbb{P}^{m+n}\to \mathbb{P}^n$ must be constant. 



\subsection*{Problem 3}
By previous proposition, it suffices to show that the intersection of $m+1$-hyperplanes in $\mathbb{P}^n$ is non-empty. But this follows from the dimension formula
\[\textrm{dim}(H_1\cap...\cap H_m)\geq (m+1)(n-1)-mn=n-m>0\]
so a $k$-morphism $f: \mathbb{P}^m\to \mathbb{P}^m$ when $n>m$ must be a constant map. 

\subsection*{Problem 5}
\subsubsection*{(b)}
Note that the function field of a irreducible variety is isomorphic to the function field of any of its dense open subset. So, we identify $k(t)\cong k(U_0)$, where $U_0$ is the standard affine open where $x_0\neq 0$. 
By proposition $0.0.1$, a morphism $\mathbb{P}^1\to \mathbb{P}^1$ on $U_0$ is given by a map $[1: \frac{y}{x}]\mapsto [1: \frac{p(\frac{y}{x},1)}{g(\frac{y}{x},1)}]$. We thus have a natural map to automorphism of $k(t)$ defined by $t\mapsto \frac{f_0(t,1)}{f_1(t,1)}$.  Thus, an automorphism of $\mathbb{P}^1$ corresponds to an automorphism of $k(t)$. 

\subsubsection*{(c)}
An automorphism of $k(t)$ will be a Moebius transform, as will we demonstrate in problem $6$ to be induced by projective linear transformations. 



\subsection*{Problem 6}
We have an obvious choice of map 
\[\rho: GL_{n+1}(k)\to Aut(\mathbb{P}_k^n) \]
by the clear action on the homogeneous coordinates. It is clear that this is a group homomorphism and the scalar multiples of the identity matrix form the kernel of this homomorphism.

By Bezout's theorem, an automorphism of $\mathbb{P}^n$ takes hyperplanes to hyperplanes. Moreover, on a dense open subset of $\mathbb{P}^n$, a morphism will take the form $x\mapsto [f_0(x):...:f_n(x)]$. Since such morphism takes hyperplanes to hyperplanes, each $f_i$ must be of degree $1$, and thus the automorphism must be induced by linear transforms, so the homomorphism is surjective.
\subsection*{Problem 7}
\subsubsection*{(c)}
it is clear that matrix multiplication and taking inverse have each coordinate functions polynoimials, therefore define $k$-morphism of affine varieties.
\subsubsection*{(d)}
We note that $PGL_n(k)$ is the quasi-projective variety that is the complement of the projetive variety $V(det)\subset \mathbb{P}^{n^2-1}$, where $det$ is the homogeneous polynomial defining the determinant. It is irreducible since it is an open subset of a irreducible projective variety. It has dimension $n^2-1$ since it is open dense in $\mathbb{P}^{n^2-1}$.

\subsection*{Problem 8}
It suffices to show that it is open on each affine chart, where $f$ restricts to a rational function $\frac{f}{g}$. It belongs to the image of the stalk $O_{X,x}$ iff $g$ does not vanish at $x$, and such $x$ is open. 
\subsection*{Problem 9}
Note $X=V(2x_1^2-3x_2x_3)$ is irreducible, so its function field is the field of fraction of $k[X]=\textrm{Quot}(k[x_1,x_2,x_3]/(2x_1^2-3x_2x_3))$.

Note $Y=V(x_1x_2-x_1)$ is has irreducible components $V(x_1)$ and $V(x_2-1)$, so the function field is $k(Y)\cong k(x_2,x_3)\times \textrm{Quot}(k[x_1,x_2]/x(x_2-1))$.

Note $X\cap Y$ has irreducible components $V(x_2)$, $V(x_3)$ and $V(2x_1^2-3x_3)$, so the function field is $k(X\cap Y)\cong k(x_1,x_3)\times k(x_1,x_2)\times \textrm{Quot}(k[x_1,x_2,x_3]/(2x_1^2-3x_3))$


\subsection*{Problem 10}
\subsubsection*{(a)}
The function field of the cuspidal curve is \[\textrm{Quot}(k[x_1,x_2]/(x_1^2-x_2^3))\cong \textrm{Quot}(k[x_1,x_2]/(x_1^2-x_2^3))\cong \textrm{Quot}(k[t^2,t^3]\cong k(t)\], so the cuspidal cubic is ratinal.

\subsubsection*{(c)}
An immediate consequence of rationality is that the variety $X$ is birationally equivalent to $\mathbb{A}_k^n$ for some $n$. An immediate consequence of this is that a dense open subset $X$ is isomorphic to a dense open subset of $\mathbb{A}_k^n$, whose $k$-points are clearly dense. 


\section*{Homwork 6}

\subsection*{Problem 1}
\subsubsection*{(a)}
By Chevalley's extension theorem, every local ring $(R, \mathfrak{m})$ is dominated by a valuation ring whose field of fraction is $K(X)$. By valuative criterion, separatedness is equivalent to having at most one point whose stalk is dominated by the valuation ring. Since domination is transitive, we are done.   
\subsubsection*{(b)}
It suffices to show this fact for $U$ affine open, and $U'=D(f)$ a basic open. Then, the restriction map $O_X(U)\to O_X(U')$ corresponds to the localization map $A\to A_f$, which is injective if $A$ is integral.  

\subsection*{Problem 2}
\subsubsection*{(a)}
Obvious to check.

\subsubsection*{(b)}
Let $R=\oplus_{*}R_*$ and $S=\oplus_{*}R_{d*}$. We will show that $Proj R\to Proj S$ is a isomorphism on an open cover. By definition, $Proj S$ is covered by homogeneous elements $f$ of degree $d$, and let $D^S_+(f)$ denote such a basic open. Note that $\{D^R_+(f)_+|def(f)=d\}$ also covers $R$, as $D_+(g)=D_+(g^n)$ for any $g$. We have the canonical isomorphism $D^S_+(f)=Spec S_f^0$ and $D^R_+(f)=Spec(R_f^0)$, and note that $S_f^0=\{\frac{s}{f^k}|s\in R_{kd}\}=R_f^0$ are canonically isomorphic.

\subsubsection*{(c)}
Consider the Veronese embedding: $\mathbb{P}^1\to \mathbb{P}^2$ given by $[s:t]\mapsto [s^2:st:t^2]$. Then, the induced map on homogeneous coordinate rings is the inclusion $k[x,y,z]/(xz-y^2)\cong k[s^2,st,t^2]\to k[s,t]$, which is not an isomorphism of $k$-algebras. 




\subsection*{Problem 4}
\subsubsection*{(a)}
The isomorphism between $V(q)$ and $V(\delta)$ is induced by the automophism of $\mathbb{P}_k^n$ by linear transformations given by the diagonalization of a symmetric matrix. 

\subsubsection*{(b)}
If there is only one $a_i\neq 0$, then it is clear that the algebraic set is not reduced; if there are two, then we have the factorization $x^2+y^2=(x+iy)(x-iy)$. If the number is greater or equal to $3$, then we claim that the Fermat curve defined by $V(\sum_{i=1}^{k}x_i^2)$ is irreducible over any field of characteristic not $2$. In the case where $k=3$, we see by eisenstein that $x_1^2+(x_2-ix_3)(x_2+ix_3)$ is irreducible when we view the polynomial over $k[x_2,x_3]$, since $(x_2-ix_3)|x_2^2+x_3^2$ but $(x_2-ix_3)^2$ does not. Inductively, we have $x_1^2+(x_2^2+...x_n^2)$ over $k[x_2,...,x_n]$, where $(x_2^2+...x_n^2)$ is irreducible by hypothesis. By Eisenstein, it suffices to show that $x_2^2+...x_n^2)\not \in (x_2^2+...x_n^2)^2$, but this is straighforward to see by degree reasons.

\subsection*{Problem 5}
\subsubsection*{(a)}
This assertion is clearly false. Consider the affine variety $V(x^2+y^2)$, which is irreducible over $\mathbb{R}$, and $(0,0)$ is a $\mathbb{R}$-rational point. However, it is not geometrically irreducible since $x^2+y^2=(x+iy)(x-iy)$ over $\mathbb{C}$.


\subsubsection*{(b)}
This is done in problem $10.c$ in HW5.

\subsubsection*{(c)}
We are left to show that if $X(k)$ is non-empty, then $X$ is a rational variety. Note that since $X$ is absolutely irreducible by problem $4b$, we know that the intersection $k(X)\cap \overline{k}=k$, which means that the function field is a purely transcendetal extension over $k$ by HW2 problem c, as desired.


\subsection*{Problem 7}
Suppose $f$ is a $k$-isomorphism with inverse $f^{-1}$. Then, the Jacobians satisify $Id=j(f\circ f^{-1})=j(f)j(f^{-1})$. Since the determinant is multiplicative, we know that the determinant of $j(f)$ must be invertible. 




\subsection*{Problem 9}

\subsubsection*{(a)}
The closed subschemes of an affine scheme $Spec R$ corresponds to closed immersions of the form $Spec(R/I)\to Spec(R)$, where $I$ is an ideal of $R$. The points of $Spec(\mathbb{Z}[t])$ are prime ideals of $\mathbb{Z}[t]$, which is of the following three forms: $(p)$ where $p$ is a prime number, $(f(t))$ where $f$ is irreducible, and $(p,f(t))$ where $f$ is irreducible mod $p$. The residue field of $(p)$ is precisely $(\mathbb{Z}/p)(t)$; the residue fields of $f(t)$ corresponds to the quotient field of $\mathbb{Z}[t]/(f(t))$; the residue field of $(p,f(t))$ is isomorphic to $\mathbb{F}_p[t]/(f(t))$, which is the finite field $\mathbb{F}_{p^n}$, where $n$ is the degree of $f(t)$.
\subsubsection*{(b)}
Let $p$ be a choice of point in $Spec(\mathbb{Z}[t])$. Then, the fiber is computed as the pullback of the diagram
\[\begin{tikzcd}
	{\pi^{-1}(p)} & {Spec(\mathbb{Z}[t])\times_{\mathbb{Z}}Spec(\mathbb{Z}[x_1,...,x_n])} \\
	{Spec(\kappa(p))} & {Spec(\mathbb{Z}[t])}
	\arrow[from=1-1, to=1-2]
	\arrow[from=1-1, to=2-1]
	\arrow["\pi"', from=1-2, to=2-2]
	\arrow[from=2-1, to=2-2]
\end{tikzcd}\]
since everything is affine, we move to the ring side and compute that the fiber is 
\[Spec(\kappa(p)\otimes_{\mathbb{Z}[t]}\mathbb{Z}[t,x_1,...,x_n])\cong Spec(\kappa(p)[x_1,...,x_n])\]

\section*{Homework 7}

\subsection*{Problem 3}
\subsubsection*{(a)}
For each pair $x,y$, let $U$ be the affine open set that contains $x,y$, so we get a covering $\{U\times_k U\}$ of $X\times_k X$, and it suffices to show that the diagonal map $U\to U\times_k U$ is a closed immersion. But this is standard since the map $A\otimes_k A\to A $ is always surjective, thus inducing a closed immerison of affine schemes. 

\subsubsection*{(b)}
Let $f: X\to S$ be the structure map, $\{U_i\}$ be an affine covering of $X$, and choose $V_i$ to be an affine subscheme of $S$ that contains $f(U_i)$. Then, $U_i\times_{V_i}U_i$ is affine open in $X\times_S X$, and $\Delta^{-1}(U_i\times_{V_i}U_i)=U_i$. If $\Delta (X)$ is closed, then to check that $\Delta: X\to \Delta(X)\subset X\times_{S}X$ is a closed immersion, it suffices to show that $\Delta: U_i\to U_i\times_{V_i}U_i$ is a closed immersion for every $i$, but we have checked this true for affine schemes so we are done. 




\subsection*{Problem 4}
\subsubsection*{(a)}
If direction is trivial, since we can take $X=Y$. For the other direction, we have the pullback square
\[\begin{tikzcd}
X\arrow[d]\arrow[r]& X\times_S Y\arrow[d]\\
Y\arrow[r]&Y\times_S Y
\end{tikzcd}\]
and the bottom arrow is a closed immersion by assumption, so the top arrow is a closed immersion as well since it is stable under pullback.


\subsubsection*{(b)}
This is done in HW4, problem 1.



\subsection*{Problem 6}
\subsubsection*{(a)}
\underline{\textbf{Quasi-compact}}: It suffices to find an affine open covering of $T$ such that the inverse image of each affine open in the cover under the map $g:X\times_S T\to T$ is quasi-compact. Let $\phi: T\to S$ be the stucture morphism, and for every $x\in T$, we have an affine open $U$ containing $\phi(x)\in S$. Moreover, we can find an affine open $U'\subset \phi^{-1}(U)$ containing $x$. Using this procedure, we can find a (not necessarily finite) affine open covering $\{U'_i\}$ of $T$, with a collection $\{U_i\}$ of affine opens in $S$ such that $\phi(U'_i)\subset U_i$. Let $f:X\to S$ be the structure morphism of $X$. Then, $f^{-1}(U_i)$ can be covered by finitely many affine open sets of $X$, say $\{V_j\}$. Then, 
\[g^{-1}(U_i')=f^{-1}(U_i)\times_{U_i}U_i'=\cup_J V_j\times_{U_i} U_i' \]
so it is a finite union of quasi-compact open subschemes and is thus quasi-compact, and we are done. 

\underline{\textbf{Noetherian}}
Not stable under base change. For example, $\mathbb{C}\times_{\mathbb{Q}} \mathbb{C}$ is not a Noetherian $k$-algebra.


\underline{\textbf{Locally of finite type}}
Note that to show a morphism $f: X\to T$ is of finite type, it suffices to find an open covering $\{U_i\}$ of $T$, such that ther exists an open affine covering $\{T_{i,j}\}$ for each $f^{-1}(U_i)\subset X$, and each ring map $O_X(U_i)\to O_X(V_i)$ is of finite type. Then by similar argument to part $a$, we may reduce the problem into showing that base change of an affine scheme locally of finite type is locally of finite type, which follows from the statement from ring theory and base change along a finite type ring morphism is of finite type.


\underline{\textbf{Finite over Base}}
I don't know what this means.

\subsubsection*{(b)}

\underline{\textbf{Integral}}: Not stable under base change. See below for irreducible/reduced.

\underline{\textbf{Irreducible}}: Take any non absolutely irreducible variety, and base change to $\overline{k}$.

\underline{\textbf{Reduced}}: Consider the variety $\mathbb{F}_p(t)[x]/(x^p-t)$, which is reduced, and base change to the closure, where it becomes no longer reduced since $x-\sqrt[p]{t}$ is a nilpotent. 

\underline{\textbf{Normal}}
Base change does not preserve integrality, thus cannot preserve normality. 





\subsection*{Problem 7}
Every morphism except dominant is stable under base change. 



\subsection*{Problem 8}
\subsubsection*{(a)}
Note that $R=k[x_1,x_2,x_3]/((x_1-x_2)^2,x_1^3)$ has the obvious reduction to $R'=k[x_1,x_2,x_3]/(x_1-x_2,x_1)\cong k[x_3]$. Thus, $X^{\textrm{red}}=Spec(k[x_3])$.



\subsection*{Problem 9}
We have $k[x_1,x_2,x_3]/(x_1^2-2x_1x_2-x_2^2-x_1^3)\cong k[t_1,t_2,x_3]/(t_2^2-t_1^3)\cong k[t^2,t^3,x_3]$ by a linear change of variables $t_1\to x_1$ and $t_2\to x_1-x_2$. The integral closure of $k[t^2,t^3,x+3]$ is clearly $k[t,x_3]$ since it is a UFD.
\subsection*{Problem 10}
To compute normal locus and normalization, it suffices to compute these on each affine chart and glue. We let $U_i$ denote the standard affine chart on which $x_i\neq 0$. 

\subsubsection*{(a)}
On $U_0$, we have the affine curve $V(ax_1^3-x_2^2)$. Assuming $char (k)\neq 3$ and $k$ algebraically closed, we have $k[x_2,x_3]/(ax_1^3-x_2^2)\cong k[t^2,t^3]$, whose normal closure is $k[t]$. On $U_1$, we have the affine curve $V(a-x_0x_2^2)$. Consider the map $k[x_0,x_2]\to k[t,\frac{1}{t}]$ by $x_0\mapsto at^2$ and $x_2\mapsto \frac{1}{t}$. The map induces an isomorphism between $k[x_0,x_2]/(a-x_0x_2^2)$ and $k[t,\frac{1}{t}]$, which is integrally closed. Alternatively, is it easy to compute that the curve is smooth. On $U_2$, we have the affine curve $V(ax_1^3-x_0)$, which is smooth and thus normal.

\subsubsection*{(b)}
For $X$, we may first perform a change of variable so that $a=b=1$. Then, $k[x_1,x_2,x_3]/(x_1^2-x_2x_3)\cong k[s^2,st,t^2]\subset k[s,t]$ using the same map as the veronese embedding. The normalization is thus $k[s,t]$. 

For $Y$, the curve is smooth so it is normal. 

Note that $X\cap Y=V(x_1,x_3)\cup V(x_1-\frac{3}{2}x_2^2,x_3-x_1x_2)$. The second irreducible component is parameterized by $t\mapsto (t,\frac{3}{2}t^2,\frac{3}{2}t^3)$, which is a twisted cubic and is smooth.

\subsubsection*{(c)}
On $U_0$, we have the affine nodal curve $V(ax_1^3-bx_1^2-x_2^2)$, whose normalization is $k[t]$ by considering $t=\frac{x_2}{x_1}$; On $U_1$, we have the affine curve $V(a-bx_0-x_0x_2^2)$, which is smooth; On $U_2$, we have the affine curve $V(ax_1^3-bx_0x_1^2-x_0)$, which is smooth.




\section*{Homework 8}

\subsection*{Problem 1}
Every property listed is local in natural, thus inherited by the open/closed subschemes. 

\subsection*{Problem 2}
\subsubsection*{(a)}
Note that Noetherian valuation rings have principal maximal ideals, since finitely generated ideals in valuation rings must be principal. By Krull Hauptidealsatz, every ideal of a notherian valuation ring is a power of its maximal ideal, therefore must be discrete. Conversely, a DVR is a PID and is surely Noetherian.

\subsubsection*{(b)}
Suppose we are given ideals $I,J$. If there exists $j\in J$ such that $j\not \in I$, then $\frac{i}{j}\in R$ for all $i\in I$, for otherwise there exists $i'$ such that $\frac{j}{i'}\in R$, which implies $j\in I$. Thus, $I\subseteq J$, and in particular $Spec(R)$ is a chain. For the second part of the question, it suffices to show that every subring of $K$ containing the valuation ring $R$ is a localization of $R$ at a prime. This follows from the fact that valuation rings are maximal under domination. 
\subsection*{Problem 4}
\subsubsection*{(a)}
$X\times_T Y$ and $X\times_S Y$ are canonically isomorphic as $S$-schemes by the uniqueness of pullback, since they are both the pullback of the same diagram 
\[\begin{tikzcd}
	Z & Y \\
	X & T \\
	&& S
	\arrow[from=1-1, to=1-2]
	\arrow[from=1-1, to=2-1]
	\arrow[from=2-1, to=2-2]
	\arrow[from=1-2, to=2-2]
	\arrow[curve={height=12pt}, from=2-1, to=3-3]
	\arrow[curve={height=-12pt}, from=1-2, to=3-3]
	\arrow[from=2-2, to=3-3]
\end{tikzcd}\]

\subsubsection*{(b)}
Assuming all schemes are Noetherian. One direction follows from the lemma that if $f: X\to Y$ and $g:Y\to Z$ are morphism of schemes, and if $g\circ f$ is separated, then $f$ is separated. (Hartshorne corllary $4.6$)

The other direction depends on the morphism $T\to S$.

\subsubsection*{(c)}
Assuming all schemes are Noetherian. One direction follows from the lemma that if $f: X\to Y$ and $g:Y\to Z$ are morphism of schemes, and if $g\circ f$ is universally closed, then $f$ is universally closed. (Hartshorne corllary $4.8$). The other direction depends on the morphism $T\to S$. 

\subsection*{Problem 6}
\subsubsection*{(a)}
Suffices to show that the closed sets are stable under specialization. Suppose $x\in \overline{\{y\}}$. Then, let $y\in U$, where $U$ is closed. It is immediate that $\overline{\{y\}}\subset U$, and thus $x\in U$. Conversely, by definition of closure we have $x\in \overline{\{y\}}$ if $x$ is in the every closed set that contains $t$. Taking the complement shows that open sets are stable under generalization. 

\subsubsection*{(b)}
Suffice to show that for a scheme $X$, every irreducible closed set $Z$ has a generic point. But this is clear since there exists some dense affine open such that $U\cap Z$ is a closed irreducible subset of an affine scheme, which has a generic point given by the defining prime ideal.

\subsection*{Problem 8}
\subsubsection*{(a)}
By problem $4b$, it suffices to show that $Proj R\to Spec(\mathbb{Z})$ is a separated. To show this, recall the lemma that a scheme $X$ is separated if for every pair of affine open $U,V$, the intersection $U\cap V$ is affine and the induced map $O_X(U)\otimes O_X(V)\to O_X(U\cap V)$ is surjective. For $Proj R$, the standard affine open $D_+(f)$ suffices, since $D+(f)\cap D_+(g)=D_+(fg)$ is affine, and 
\[R^0_{(f)}\otimes R^0_{(g)}\to R^0_{(fg)}\]
is easily checked to be surjective. 

\subsubsection*{(b)}
Suffices to show that the morphism $Proj S\to Spec(S_0)$ is quasi-compact, and satisfies the hypothesis of the valuative criterion for universally closedness. A proof can be found in Stacks project lemma $27.8.11$. 

\subsection*{Problem 9}
\subsubsection*{(a)}
Suffices to show for an affine scheme $Spec(R)$, the normal locus and reduced locus are open.
\subsubsection*{(b)}
No. It is easy to show that the singular locus of a curve is affine since it is defined by the vanishing of formal derivatives; The normal locus is then the complement of a closed affine variety of codimension $1$, which is never affine. 



\subsubsection*{(c)}




















\end{document}