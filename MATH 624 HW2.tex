\documentclass{article}
\usepackage[utf8]{inputenc}
\usepackage{amsmath}
\usepackage{amsfonts}
\usepackage{amssymb}
\usepackage{tikz}
\usepackage{fullpage}
\usepackage{tikz-cd}
\usepackage{spectralsequences}
\usepackage{adjustbox}
\usepackage[backend=biber, style=alphabetic]{biblatex}
\usepackage{xfrac}
\usepackage{tcolorbox}
\usepackage{xcolor}
\usepackage{graphicx}
\graphicspath{ {D:/Chrome Downloads./} }
\usepackage[parfill]{parskip}
\usepackage{amsthm}
\addbibresource{sample.bib}
\theoremstyle{definition}
\newtheorem{theorem}{Theorem}[section]
\theoremstyle{definition}
\newtheorem{definition}{Definition}[theorem]
\theoremstyle{definition}
\newtheorem{remark}{Remark}[theorem]
\theoremstyle{definition}
\newtheorem{proposition}{Proposition}[theorem]
\theoremstyle{definition}
\newtheorem{lemma}[theorem]{Lemma}
\theoremstyle{definition}
\newtheorem{corollary}{Corollary}[theorem]
\theoremstyle{definition}
\newtheorem{example}{Example}[theorem]
\title{MATH 624 HW2}
\author{David Zhu}

\begin{document}
\maketitle

\section*{Homework 2}

\subsection*{Problem 1b}
Suppose $U_f$ is not empty. Let $W=\{ a\in V: k(a)|k \textrm{ is a finite algebraic extension} \}$, which corresponds to the vanishing locus of maximal ideals of $k[V]$. Clearly $W\subset V(\overline{k})$, so it suffices to show that $W\cap U_f$ is dense in in $U_f$ for every $f$, which is equivalent to every open $U_f$ containing a point in $W$. To see this, consider a maximal ideal in $k[V]_f$, which must be the image of a maximal ideal in $k[V]$ under localization: suppose otherwise, then every maximal ideal of $k[V]$ contains $f$, which implies $f$ is in the Jacobson radical of $k[V]$. However, $k[V]$ has trivial Jacobson radical since $k[X]$ is Jacobson, which implies $f=0$ and $U_f$ is empty, and contradiction. Then, the locus of the maximal ideal is contyained in $U_f\cap W$.


\subsection*{Problem 2b}
A representative of $\tilde{O}_a$ is given by a pair $(W_1, \frac{f_1}{g_1})$, with $g_1\neq 0$ on $W_1$, and $(W_1, \frac{f_1}{g_1})\sim (W_2, \frac{f_2}{g_2})$ iff there exists a open $U_{h'}\subset W_1\cap W_2$ such that $\frac{f_1}{g_1}=\frac{f_2}{g_2}$ on $U_{h'}$. On the other hand, a representative of $k[V]_{\mathfrak{p}_a}$ is given by some $\frac{f}{g}$, where $g(a)\neq 0$. By continuity, there exists a basic open $U_h$ containing $a$ on which $g$ does not vanish. We define the $k$-algebra homomorphism: 
\[i: k[V]_{\mathfrak{p}_a}\to \tilde{O}_a  \ \ \ \frac{f}{g}\mapsto (U_h, \frac{f}{g})\]

Surjectivity is obvious by construction, so there are two things to check: well-definedness (it is clearly that this will be a $k$-algebra morphism once we check well-definedness) and injectivity.

Well-definedness: suppose $\frac{f}{g}\sim \frac{f'}{g'}$ in $k[V]_{\mathfrak{p}_a}$, which means there exists some $h'\in K[V]$ such that $h'(fg'-f'g)=0$, which implies $\frac{f}{g}=\frac{f'}{g'}$ on $U_{h'}$. Thus, both will be mapped to the equivalence class $(U_{h'},\frac{f}{g})$.

Injectivity: suppose $i(\frac{f}{g})=(U_h,\frac{f}{g})$ represents the $0$ element. WLOG, we may assume that $f$ vanishes on $U_h$, for otherwise we may replace $U_h$ with a smaller basic open. Then, $\frac{f}{g}\sim \frac{0}{1}$ in $ k[V]_{\mathfrak{p}_a}$ since $h(f\cdot 1-g\cdot 0)$ is identically $0$ on $V$.

\subsection*{Problem 3b}
By problem 2b, the stalk is isomorphic to $k[V]_{p_a}$, which is always local. In regards to when $k[V]_{p_a}$ is a not a domain, it will be when there exists an $x\in p_a$ such that $\exists y\in p_a$ and $xy=0$, but $xz\neq 0$ for every non-zero $z\not \in p_a$. For example, let $V=V(xy)$. Then, $k[V]=k[x,y]/(xy)$. Take $a=(0,0)$, then $p_a=(x,y)$, and we have $xy=0$ but $xz\neq 0$ for every non-zero $z$ not in $(x,y)$. 

Note that a reduced Noetherian ring is integral iff it has a unique minimal prime. Another method of detection for integrality is iff $p_a$ contains a unique minimal prime of $k[V]$ (because it is reduced Notherian), which corresponds to $a$ belonging to a unique irreducible component.

\subsection*{Problem 4}
\subsubsection*{(a)}
$V$ is irreducible iff $I(V)$ is prime iff $k[V]$ is a domain iff $k(V)$ is a field. The Krull dimension of $k(V)$ and the trascendence degree are the same by Noether normalization.
\subsubsection*{(b)}
Take the finite set of minimal primes $\{p_1,...,p_n\}$ of $k[V]$, and recall that the union of the minimal primes is precisely the zero-divisors of $k[V]$, and the intersection is the trivial nilradical. Then, localize at $S=k[V]\setminus \cup p_i$, and $S^{-1}k[V]$ has unique maximal primes $S^{-1}p_1,...,S^{-1}p_n$, which are coprime. By chinese remainder, we have
\[k(V)=S^{-1}k[V]/(0)=S^{-1}k[V]/\cap S^{-1}p_i\cong \prod k(V_i)  \]
\subsubsection*{(c)}

\subsubsection*{(d)}

\subsection*{Problem 5}




\subsection*{Problem 10}
\subsubsection*{(a)}
For $\mathbb{A}^1\to \mathbb{A}^2$ given by $a\mapsto (a,\frac{1}{a})$, the general situation is discussed in problem 8; for $a\mapsto (a^2,a^3)$, the domain in the entire $\mathbb{A}^1$, and the image is a affine algebraic set given by $V(x^3-y^2)$. The map is clearly a bijection and a homeomorphism. However, the $k$-morphism is not an isomorphism, as the coordinate rings $k[t^2,t^3]$ and $k[t]$ are not isomorphic.
\subsubsection*{(b)}
As in part $(a)$, we see that it is possible for the $k-$morphism to not be an isomorphism. However in the case $\mathbb{A}^1\mapsto \mathbb{A}^3$ given by $a\mapsto (a^1,a^2,a^3)$, the $k$-morphism is an isomorphism. 


\section*{Problem 4c}
Suppose $V$ is irreducible. Note that $k[V_{k^s}]\cong k[V]\otimes_k k^s$, so $k(V_{k^s})\cong k(V)\otimes_k k^s$ after taking the field of fractions. Thus, absolute irreducibility of $V$ is equivalent to the integrality of $k(V_{k^s})\cong k(V)\otimes_k k^s$. Suppose $\overline{k}\cap k(V)$ is not purely inseparable over $k$, so there exists $\alpha$ algebraic over $k$, and $k(\alpha)\otimes_k k(\alpha)$ is a subring of $k(V)\otimes_k k^s$, which is not integral. To see this, note, let $p(t)$ be a minimal polynomial of $\alpha$, then
\[k(\alpha)\otimes_k k[t]/p(t)\cong k(\alpha)[t]/p(t)\]
cleary has $(x-\alpha)$ as a zero-divisor. 

Conversely, suppose $k(V)\cap \overline{k}$ is purely inseparable. It suffices to show that $k(V)\otimes_k k[t]/p(t)\cong  k(V)[t]/p(t)$ is integral for every irreducible $p(t)$. If there is $q(t)\in k(V)[t]$ that divides $p(t)$, then $q(t)$ is also contained in $k^s[t]$, so $q(t)\in (k^s\cap k(V))[t]=k[t]$,  which forces it to be $1$ or $p(t)$, and the ring is still integral. 


\section*{Problem 5}
\subsection*{(a)}
The correct statement should be $\tilde{O}_x$ is a domain iff $x$ is contained in a unique irreducible component, and the proof is given in problem $3$.
\subsection*{(b)}
It is a standard point-set topology argument that finite intersection of open dense sets is still open and dense.

\subsection*{(c)}
The colimit is the function field of $V$. The detail proofs are given in HW3 problem $10$.

\section*{Problem 8}
\subsection*{(a)}
Clearly the empty set and the whole line is open affine, so the only non-trivial case is the line minus a finite set of points. Let $a_1,...,a_n$ be a finite number of points, and $\mathbb{A}^n\setminus \{a_1,..,a_n\}$ is isomorphic to the affine algebraic set $V(y(x-a_1)...(x-a_n)-1)\subset \mathbb{A}^{n+1}$ given by the map 
\[\varphi: \mathbb{A}^n\setminus \{a_1,..,a_n\}\to V(y(x-a_1)...(x-a_n)-1) \ \ \ \ t\mapsto (t,\frac{1}{(t_1-a_1)...(t_n-a_n)})\]
with inverse $\psi: (x,y)\mapsto x$. Both functions are Zariski continuous since they are rational functions. Let $T$ be an open of $\mathbb{A}^n$ and $U$ be an open of $\mathbb{A}^{n+1}$ such that $f(T)\subset U$. Then, given any regular function $\frac{f(x,y)}{g(x,y)}$ on $U$, the pullback $\frac{f(x,\frac{1}{x})}{g(x,\frac{1}{x})}$ is a regular function on $T$ by multiplying large enough powers of $x$ to the numerator and denominator. The other direction is trivial since the pullback will be the same function on one variable. Thus, $\varphi$ and $\psi$ are $k$-isomorphisms. 
\subsection*{(b)}
The open $U:=\mathbb{A}^2\setminus \{(0,0)\}$ is not affine. Note that $U$ is covered by $U_1=D_{f(x,y)=x}$ and $U_2=D_{f(x,y)=y}$, whose ring of regular functions are $k[x,y]_x$ and $k[x,y]_y$. On the overlap, the ring of regular functions is $k[x,y]_{x,y}$. Let $f$ be a regular function on $U$, which restricts to a regular function of the form $p_1/x^m$ on $U_1$ and $p_2/y^n$ on $U_2$. The compatibility condition on $U_1\cap U_2$ implies that $p_1/x^m=p_2/y^n$, which implies $x^mp_2=y^np_1$. Since $k[x,y]$ is a UFD, $x_m| p_1$, and $f$ is in $k[x,y]$. Thus, $O(U)\cong k[x,y]\cong O(\mathbb{A}^2)$. Thus, if $U$ were affine, the inlusion map $i: U\to \mathbb{A}^2$ is an isomorphism, which is false. 

\section*{Problem 10}
\subsection*{(a)}
\subsection*{(b)}

\section*{Homework 3}
\subsection*{Problem 1}
\subsubsection*{(a)}
First, note that all closed/open immersions $i: Z\to X$ are separated morphisms: the diagonal map to the fiber product $Z\to Z\times_X Z\cong Z$ is an isomorphism. 







\end{document}