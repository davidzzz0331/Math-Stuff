\documentclass{article}
\usepackage[utf8]{inputenc}
\usepackage{amsmath}
\usepackage{amsfonts}
\usepackage{amssymb}
\usepackage{tikz}
\usepackage{fullpage}
\usepackage{tikz-cd}
\usepackage{spectralsequences}
\usepackage{adjustbox}
\usepackage[backend=biber, style=alphabetic]{biblatex}
\usepackage{xfrac}
\usepackage{tcolorbox}
\usepackage{xcolor}
\usepackage{graphicx}
\graphicspath{ {D:/Chrome Downloads./} }
\usepackage[parfill]{parskip}
\usepackage{amsthm}
\addbibresource{sample.bib}
\theoremstyle{definition}
\newtheorem{theorem}{Theorem}[section]
\theoremstyle{definition}
\newtheorem{definition}{Definition}[theorem]
\theoremstyle{definition}
\newtheorem{remark}{Remark}[theorem]
\theoremstyle{definition}
\newtheorem{proposition}{Proposition}[theorem]
\theoremstyle{definition}
\newtheorem{lemma}[theorem]{Lemma}
\theoremstyle{definition}
\newtheorem{corollary}{Corollary}[theorem]
\theoremstyle{definition}
\newtheorem{example}{Example}[theorem]
\title{MATH 624 HW2}
\author{David Zhu}

\begin{document}
\maketitle

\section*{Problem 1b}
Suppose $U_f$ is not empty. Let $W=\{ a\in V: k(a)|k \textrm{ is a finite algebraic extension} \}$, which corresponds to the vanishing locus of maximal ideals of $k[V]$. Clearly $W\subset V(\overline{k})$, so it suffices to show that $W\cap U_f$ is dense in in $U_f$ for every $f$, which is equivalent to every open $U_f$ containing a point in $W$. To see this, consider a maximal ideal in $k[V]_f$, which must be the image of a maximal ideal in $k[V]$ under localization: suppose otherwise, then every maximal ideal of $k[V]$ contains $f$, which implies $f$ is in the Jacobson radical of $k[V]$. However, $k[V]$ has trivial Jacobson radical since $k[X]$ is Jacobson, which implies $f=0$ and $U_f$ is empty, and contradiction. Then, the locus of the maximal ideal is contyained in $U_f\cap W$.


\section*{Problem 2b}
A representative of $\tilde{O}_a$ is given by a pair $(W_1, \frac{f_1}{g_1})$, with $g_1\neq 0$ on $W_1$, and $(W_1, \frac{f_1}{g_1})\sim (W_2, \frac{f_2}{g_2})$ iff there exists a open $U_{h'}\subset W_1\cap W_2$ such that $\frac{f_1}{g_1}=\frac{f_2}{g_2}$ on $U_{h'}$. On the other hand, a representative of $k[V]_{\mathfrak{p}_a}$ is given by some $\frac{f}{g}$, where $g(a)\neq 0$. By continuity, there exists a basic open $U_h$ containing $a$ on which $g$ does not vanish. We define the $k$-algebra homomorphism: 
\[i: k[V]_{\mathfrak{p}_a}\to \tilde{O}_a  \ \ \ \frac{f}{g}\mapsto (U_h, \frac{f}{g})\]

Surjectivity is obvious by construction, so there are two things to check: well-definedness (it is clearly that this will be a $k$-algebra morphism once we check well-definedness) and injectivity.

Well-definedness: suppose $\frac{f}{g}\sim \frac{f'}{g'}$ in $k[V]_{\mathfrak{p}_a}$, which means there exists some $h'\in K[V]$ such that $h'(fg'-f'g)=0$, which implies $\frac{f}{g}=\frac{f'}{g'}$ on $U_{h'}$. Thus, both will be mapped to the equivalence class $(U_{h'},\frac{f}{g})$.

Injectivity: suppose $i(\frac{f}{g})=(U_h,\frac{f}{g})$ represents the $0$ element. WLOG, we may assume that $f$ vanishes on $U_h$, for otherwise we may replace $U_h$ with a smaller basic open. Then, $\frac{f}{g}\sim \frac{0}{1}$ in $ k[V]_{\mathfrak{p}_a}$ since $h(f\cdot 1-g\cdot 0)$ is identically $0$ on $V$.

\section*{Problem 3b}
By problem 2b, the stalk is isomorphic to $k[V]_{p_a}$, which is always local. In regards to when $k[V]_{p_a}$ is a not a domain, it will be when there exists an $x\in p_a$ such that $\exists y\in p_a$ and $xy=0$, but $xz\neq 0$ for every non-zero $z\not \in p_a$. For example, let $V=V(xy)$. Then, $k[V]=k[x,y]/(xy)$. Take $a=(0,0)$, then $p_a=(x,y)$, and we have $xy=0$ but $xz\neq 0$ for every non-zero $z$ not in $(x,y)$. 

Note that a reduced Noetherian ring is integral iff it has a unique minimal prime. Another method of detection for integrality is iff $p_a$ contains a unique minimal prime of $k[V]$ (because it is reduced Notherian), which corresponds to $a$ belonging to a unique irreducible component.

\section*{Problem 4}
\subsection*{(a)}
$V$ is irreducible iff $I(V)$ is prime iff $k[V]$ is a domain iff $k(V)$ is a field. The Krull dimension of $k(V)$ and the trascendence degree are the same by Noether normalization.
\subsection*{(b)}
Take the finite set of minimal primes $\{p_1,...,p_n\}$ of $k[V]$, and recall that the union of the minimal primes is precisely the zero-divisors of $k[V]$, and the intersection is the trivial nilradical. Then, localize at $S=k[V]\setminus \cup p_i$, and $S^{-1}k[V]$ has unique maximal primes $S^{-1}p_1,...,S^{-1}p_n$, which are coprime. By chinese remainder, we have
\[k(V)=S^{-1}k[V]/(0)=S^{-1}k[V]/\cap S^{-1}p_i\cong \prod k(V_i)  \]
\subsection*{(c)}

\subsection*{(d)}

\section*{Problem 5}








\end{document}