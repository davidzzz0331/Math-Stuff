\documentclass{article}
\usepackage[utf8]{inputenc}
\usepackage{amsmath}
\usepackage{amsfonts}
\usepackage{amssymb}
\usepackage{tikz}
\usepackage{fullpage}
\usepackage{tikz-cd}
\usepackage{spectralsequences}
\usepackage{adjustbox}
\usepackage{xfrac}
\usepackage{tcolorbox}
\usepackage{xcolor}
\usepackage{graphicx}
\graphicspath{ {D:/Chrome Downloads./} }
\usepackage[parfill]{parskip}
\usepackage{amsthm}
\theoremstyle{definition}
\newtheorem{theorem}{Theorem}[section]
\theoremstyle{definition}
\newtheorem{definition}{Definition}[theorem]
\theoremstyle{definition}
\newtheorem{problem}{problem}[theorem]
\theoremstyle{definition}
\newtheorem{proposition}{Proposition}[theorem]
\theoremstyle{definition}
\newtheorem{lemma}[theorem]{Lemma}
\theoremstyle{definition}
\newtheorem{corollary}{Corollary}[theorem]
\theoremstyle{definition}
\newtheorem{example}{Example}[theorem]
\title{$K_0$ and Wall's Finiteness Obstruction}
\author{David Zhu}

\begin{document}
\maketitle

This note will follow the original paper by C.T.C Wall, which discusses the algebraic criteria determining when a CW-complex is homotopy equivalent to one of finite type/dimension.


\begin{tcolorbox}[colback=purple!5!white,colframe=purple!75!black]
\begin{definition}
A CW-complex is \underline{\textbf{finite}}, or equivalently of \underline{\textbf{finite type}}, if it is constructed with finitely many cells. 
\end{definition}
\end{tcolorbox}



\begin{tcolorbox}[colback=purple!5!white,colframe=purple!75!black]
\begin{definition}
A map $\varphi: K\to X$ is $n$\underline{\textbf{-connected}} if the relative homotopy group $\pi_i(\varphi):= \pi_i(M_{\varphi},K\times 1)$ is trivial for $0\leq i\leq n$, where $M_{\varphi}$ is the mapping cylinder equivalent to $X$.
\end{definition}
\end{tcolorbox}

For the following discussion, we will use the notation $\Lambda$ for the integral group ring on the fundamental group of $X$.
\section{Complexes of Finite Type}
For the condition of $X$ being equivalent to a complex with finite $n$-skeleton, we associate an algerbaic condition $F_n$, defined as follows: 

\begin{itemize}
    \item $F_1$ is $\pi_1(X)$ being finitely generated.
    \item $F_2$ is $\pi_1(X)$ being finitely presented, and for any finite complex $K^2$ and a map $\varphi: K^2\to X$ inducing isomorphism on fundamental groups, $\pi_2(\varphi)$ is a finitely generated $\Lambda$-module. 
    \item  For $n\geq 3$, $F_n$ stands for $F_{n-1}$ holds and for any finite complex $K^{n-1}$ and an $n-1$ connected map $\varphi: K^{n-1}\to X$, $\pi_n(\varphi)$ is a finitely generated $\Lambda$-module.
\end{itemize}


\begin{tcolorbox}[colback=red!5!white,colframe=red!30!white]
\begin{theorem}
A CW complex $X$ is equivalent to a complex with finite $n$-skeleton iff it satisfies $F_n$.
\end{theorem}
\end{tcolorbox}


\begin{proof}[Proof of Theorem $1.1$]
$\Rightarrow$    Let us first deal with the case where $n=1,2$. Recall that the fundamental group of $X$ is completely determined by its $2$-skeleton. The proof using Van-Kampen directly tells us that if the $1$-skeleton being finite implies $\pi_1(X)$ is finitely generated, and $2$-skeleton being finite implies $\pi_1(X)$ is finitely presented. For the second-part of $F_2$ and the rest of the theorem, by Hurewicz we have 
\[\pi_n(\varphi)\cong \pi_n(X,K)\cong \pi_n(\tilde{X},\Tilde{K})\cong H_n(\tilde{X},\Tilde{K})\]

By cellular approximation and a lemma of whitehead, we may assume $\tilde{K}$ is the $n-1$ skeleton of $\tilde{X}$. Then, $\cong H_n(\tilde{X},\Tilde{K})$ is a quotient of $\cong H_n(\tilde{X}^n,\Tilde{K})\cong C_n(\tilde{X})$, which is a finite $\Lambda$-module.

$\Leftarrow$: Before proving the direction, we make the following observation: for $n\geq 3$, given an $n-1$-connected map $\varphi: K^{n-1}\to X$, the LES of homotopy groups shows that it induces isomorphism on $\pi_i$ for $i< n-1$ and $\pi_{n-1}(K^{n-1})\to \pi_{n-1}(X)$ is a surjection. Then, we may attch $n$-cells to kill the kernel, each corresponding to a class in $\pi_n(K^{n-1},X)$. Specifically, consider the LES of the triple $(X,K^n,K^{n-1})$, where $K^{n}$ is the complex obtained by attaching $n$-cells to $K^{n-1}$:
\[\begin{tikzcd}
\pi_n(K^n,K^{n-1})\arrow[r]&\pi_n(X,K^{n-1})\arrow[r]&\pi_n(X,K^n)
\end{tikzcd}\]

Our goal is to build $K^n$ such that $\pi_n(K^n,X)$ vanishes. Note that $\pi_n(K^n,K^{n-1})\cong C_n(\tilde{L})$. So by finite generation assumption, only finitely many cells are needed, each corresponding to a generator of $\pi_n(K^{n-1},X)$ as a $\Lambda$-module (The module structure exists when $K^n\to X$ induces isomorphism on fundamental group, which holds for $n\geq 3$).

Now to the proof: If $X$ satisfies $F_1$, we may start with a finite wedge $K^1:=\bigvee S^1$, each copy corresponding to a generator of $\pi_1(X)$, and a map $K^1\to X$ inducing surjection on fundamental group. Then, attach cells of dimension $\geq 2$ to make it a homotopy equivalence. If $X$ satisfies $F_2$, by finite presentation of $\pi_1$ we can add finitely many $2$-cells to $K^1$ and form a new complex $K^2$. There is then a map $\varphi: K^2\to X$ that induces isomorphism of $\pi_1$ and thus is $2$-connected by LES.
We finish by using the construction outlined int he previous paragraph and continue inductively.


\end{proof}

We also have a stronger result that we will not prove here:
\begin{tcolorbox}[colback=red!5!white,colframe=red!30!white]
\begin{theorem}
$X$ is equivalent to a complex with finite $n$-skeleton iff $X$ is a homotopy retract of one.
\end{theorem}
\end{tcolorbox}


\section{Complexes of Finite Dimension}



Similar to the previous section, we associate an algebraic condition $D_n$ to the topological condition that $X$ is equivalent to a CW complex of dimension $n$.

\begin{itemize}
  \item $D_n$: $H_i(\tilde{X})=0$ for $i>n$, and $H^{n+1}(X;B)=0$ for all local coefficient $B$. 
\end{itemize}

Let us give a quick summary of the construction of the previous section, but without finiteness restriction on the number of cells allowed: given a CW complex $X$, we may approximate it by inductively building an $n$-connected map $\varphi_n: K^n\to X$ by attaching $n$-cells to $K^{n-1}$. Assuming path-connectedness, we start with $1$-cells corresponding to generators of $\pi_1(X)$, $2$-cells according to presentation of $\pi_1(X)$, and $n\geq 3$ cells according to generators of $\pi_n(K^{n-1},X)$. Now, if the module $\pi_n(K^{n-1},X)$ were free, then by construction and the LES,
\[\begin{tikzcd}
  \pi_{n+1}(X,K^{n})\arrow[r]&\pi_n(K^n,K^{n-1})\cong C_n(\tilde{K}^n)\arrow[r]&\pi_n(X,K^{n-1})\arrow[r]&\pi_n(X,K^n)
  \end{tikzcd}\]

we will also get $H_{n+1}(\tilde{X},\tilde{K}^{n})=0$ and $H_{n}(\tilde{X},\tilde{K}^{n})=0$ for free by Hurewicz. This comes in handy when we look at the homology LES of the triple $(X,K^{n},K^{n-1})$:

\[
  \begin{tikzcd}
  H_i(\tilde{K}^n,\tilde{K}^{n-1})\arrow[r]&H_i(\tilde{X},\tilde{K}^{n-1})\arrow[r]&H_i(\tilde{X},\Tilde{K}^n)
\end{tikzcd}
\]
If $X$ satisfies $D_n$, then $H_i(\tilde{X},\tilde{K}^{n-1})$ vanishes for $i\neq n$ by LES of the pair; $H_i(\tilde{K}^n,\tilde{K}^{n-1})=0$ for $i\neq n$ by cellular homology. Combined with the results in dimension $n$ and $n+1$, we get $H_*(\tilde{X},\Tilde{K}^n)=0$, which implies $\tilde{K}^n$ is equivalent to $X$ (inductive argument by Hurewicz since by construction $K^2\to X$ induces isomorphism on fundamental group). What this tells us is that if $X$ satisfies $D_n$ and $\pi_n(X,K^{n-1})$ is free, then we may stop at dimension $n$ and already get a homotopy equivalence. The result of this section is that (for $n\geq 3$) we may alway add $n-1$-cells to $K^{n-1}$ to make $\pi_n(X,K^{n-1})$ free, thus proving part $3$ of the following theorem

\begin{tcolorbox}[colback=red!5!white,colframe=red!30!white]
\begin{theorem}The following hold:
  \begin{itemize}
    \item $X$ satisfies $D_1$ iff it is a wedge of circles.
    \item $X$ satisfies $D_2$ iff it is equivalent to a $3$-dimensional complex.
    \item $X$ satisfies $D_n$ ($n\geq 3$) iff it is equivalent to a $n$-dimensional complex. 
  \end{itemize}

\end{theorem}
\end{tcolorbox}

The key proposition is the following: 

\begin{tcolorbox}[colback=blue!5!white,colframe=blue!30!white]
\begin{proposition}
For $n\geq 3$, suppose $X$ satisfies $D_n$. Then, given an $n-1$ connected map of CW complexes $\varphi: K^{n-1}\to X$, we have $\pi_n(\varphi)$ a projective $\Lambda$-module. 
\end{proposition}
\end{tcolorbox}
\begin{proof}
  We have the usual isomorphism $\pi_n(\varphi)\cong H_n(\tilde{X},\tilde{K}^{n-1})$. By cellular homology, $H_n(\tilde{X},\tilde{K}^{n-1})$ is isomorphic $C_n(\tilde{X})/B_n(\tilde{X})$, and we have the commutative diagram
  \[\begin{tikzcd}
  C_{n+2}\arrow[r,"d_{n+2}"]&C_{n+1}\arrow[dr,twoheadrightarrow,"j"]\arrow[rr,"d_{n+1}"]&&C_n\\
  &&B_n\arrow[ur,hookrightarrow,"i"]
  \end{tikzcd}\]
where $j$ is the induced surjective map from $d_{n+1}$. Note $B_n$ is naturally a $\Lambda$-module, and let us take $B_n$ to be the local coefficient system. Then, $j$ represents a class in $H^{n+1}(X;B_n)$, which by assumption is trivial. Therefore, $j$ is a coboundary, so there exists a $s: C_n\to B_n$ such that $j=s\circ d_{n+1}=s\circ i\circ j$. Since $j$ is epic, we have $s\circ i=Id$, so the short exact sequence 
\[\begin{tikzcd}
0\arrow[r]&B_n(\tilde{X})\arrow[r]&C_n(\tilde{X})\arrow[r]&\pi_n(\varphi)\arrow[r]&0
\end{tikzcd}\] 
splits. But $C_n(\tilde{X})$ is free, so $\pi_n(\varphi)$ is projective. 
\end{proof}

Here is a nice algebraic fact on projective modules:

\begin{tcolorbox}
\begin{lemma}
(Eilenberg Swindle) Given a projective $R$ module $M$, there exists a free module $N$ (in general not finitely generated) such that $M\oplus N\cong N$.
\end{lemma}
\end{tcolorbox}
\begin{proof}
  By projectivity, there exists a module $L$ such that $F:=M\oplus L$ is free. Then, take 
  \[N:=M\oplus L\oplus M\oplus L...\]
  By associativity, $N$ is isomorphic to both $F^{\mathbb{N}}$ and $M\oplus F^{\mathbb{N}}$.
\end{proof}
The topological construction corresponding to Eilenberg Swindle is the following: $\pi_n(\varphi_{n-1})=\pi_n(K^{n-1},X)$ measures the kernel of the map $\pi_{n-1}(K^{n-1})\to \pi_{n-1}(X)$, and our goal is to introduce addition kernel to make $\pi_n(\varphi_{n-1})$ free. Let $F$ be the free $\Lambda$-module such that $F\oplus \pi_n(\varphi_{n-1})$ is free by the Eilenberg Swindle. Then, attach $n-1$ cells by the constant boundary map, one for each generator of $F$, to $K^{n-1}$ (equivalent to wedging $n-1$ spheres). Let $K'^{n-1}$ denote the new complex, and we may extend the map to $\varphi'_{n-1}: K'^{n-1}\to X$ by collapsing the new $n-1$ spheres to the basepoint. It is easily verified that the construction makes $\pi_n(\varphi'_{n-1})$ free. 

Now let us finish off Theorem $2.1$:
\begin{proof}[Proof of Theorem $2.1$]
 For $D_1$: by universal coefficient theorem, $D_1$ implies all homologies of $\tilde{X}$ vanishes, so $X$ is a $K(\pi,1)$. In particular, $H^2(X;B)\cong H^2_{\textrm{Grp}}(\pi;B)=0$ implies every extension of $B$ by $\pi$ is split. Now let $B=ker(F\to \pi)$, where $F\to \pi$ is any surjection from a free group $F$ to $B$. Then, $\pi$ is realized as a direct summand of a free group and thus free. 

 For $D_2$: we construct the $2$-connected map $\varphi_2: K^2\to X$, it is easily showed that $\pi_3(\varphi_2)$ is projective usuing the same argument as in Proposition $2.1.1$ with $\pi_3(\varphi_2)\cong B_2$. We finish by first applying the topological Eilenberg Swindle, and then applying the argument outlined in the beginning paragraphs of the section.
\end{proof}

\section{$\tilde{K}_0$ and Obstruction to Finiteness}


Note that the construction of the complexes of finite $n$-skeleton, corresponding to $F_n$, the construction of $n$-dimensional complexes, corresponding to $D_n$, are not "compatible constructions". Namely, when we were turning the projective $\pi_n(\varphi)$ to a free module in the construction of finite dimensionality, we added possibly infinitely many $n-1$ cells. The idea is that this is the only obstruction to $X$ being finite $n$-dimensional when it satisfies $D_n$ and $F_n$. Algebraically, the obstruction measures how far the projective $\pi_n(\varphi)$ is from being stably free.

\begin{tcolorbox}
\begin{lemma}
Suppose $X$ satisfies $F_n$ and $D_n$. Let $\varphi_1: K_1\to X$ and $\varphi_2: K_2\to X$ be two $n-1$-connected maps, with $K_1,K_2$ finite $n$-dimensional. Then, $\pi_n(\varphi_1)$ and $\pi_n(\varphi_2)$ represents the same class in $\Tilde{K}_0(\mathbb{Z}[\pi_1(X)])$
\end{lemma}
\end{tcolorbox}
\begin{proof}
  First, show that if $X$ satifies $D_n$, then both $\varphi_i$ has a homotopy right inverse by basic obstruction theory. Then, composing one of $\varphi$ with the right inverse of the other gives a map between $K_1$ and $K_2$. It is easy to see that the map is $n-1$-connected. We may view $K_1$ as subcomplex of a complex equivalent to $K_2$, with possibly extra cells only in dimension $n$ and $n+1$. Then, we can identify $\pi_n(\varphi_1)$ and $\pi_n(\varphi_2)$ with the $n$th and $n+1$th homology of $(\tilde{K_2},\tilde{K_1})$, and an algebraic argument from there finishes. 
\end{proof}

The lemma says that $\pi_n(\varphi)$ is an invariant of $X$, and determines a class in $\Tilde{K}_0(\Lambda)$. From the argument of section $2$, we may conclude with the obstruction theorem:
\begin{tcolorbox}[colback=red!5!white,colframe=red!30!white]
\begin{theorem}
If $X$ satisfies $D_n$ and $F_n$ for $n\geq 3$, then there is a obstruction to finiteness $w(X):=[\pi_n(\varphi)]\in \Tilde{K}_0(\Lambda)$. Specifically, $X$ is equivalent to a finite $n$-dimensional CW complex iff $w(X)$ vanishes.  
\end{theorem}
\end{tcolorbox}
\begin{proof}
  If $X$ is equivalent to a finite $n$-dimensional CW complex iff $w(X)$ vanishes, take $\varphi: X\to X$ to be the identity. For the converse, repeat the construction in section $2$ on turning $\pi_n(\varphi)$ free, knowing we only have to attach finitely many $n-1$ cells based on the assumption that $\pi_n(\varphi)$ is stably free. 
\end{proof}

\end{document}