\documentclass{article}
\usepackage[utf8]{inputenc}
\usepackage{amsmath}
\usepackage{amsfonts}
\usepackage{amssymb}
\usepackage{tikz}
\usepackage{fullpage}
\usepackage{tikz-cd}
\usepackage{spectralsequences}
\usepackage{adjustbox}
\usepackage[backend=biber, style=alphabetic]{biblatex}
\usepackage{xfrac}
\usepackage{tcolorbox}
\usepackage{xcolor}
\usepackage{graphicx}
\graphicspath{ {D:/Chrome Downloads./} }
\usepackage[parfill]{parskip}
\usepackage{amsthm}
\addbibresource{sample.bib}
\theoremstyle{definition}
\newtheorem{theorem}{Theorem}[section]
\theoremstyle{definition}
\newtheorem{definition}{Definition}[theorem]
\theoremstyle{definition}
\newtheorem{remark}{Remark}[theorem]
\theoremstyle{definition}
\newtheorem{proposition}{Proposition}[theorem]
\theoremstyle{definition}
\newtheorem{lemma}[theorem]{Lemma}
\theoremstyle{definition}
\newtheorem{corollary}{Corollary}[theorem]
\theoremstyle{definition}
\newtheorem{example}{Example}[theorem]
\title{MATH 624 Algebraic Geometry}
\author{David Zhu}

\begin{document}
\maketitle

\section{Prevarieties and Varieties}

We will assume that $K|k$ a finite extension, $K$ is algebraically closed. We will use $\mathbb{A}^n(K)=K^n=\mathbb{A}^n_K$ to denote the underlying set, not the $n$-dimensional affine space. Given a point $a=(a_1,...,a_n)\in \mathbb{A}^n_k$, we will use $\varphi_a$ to denote the evaluation map $k[X]\to k$. Similarly, given $f\in k[x]$, we have the evalation map $\tilde{f}: \mathbb{A}_k\to k$. This gives a morphism of $k$-algebras $k[x]\to Maps_k(\mathbb{A}_k,k)$ given by $f\mapsto \tilde{f}$.


\begin{tcolorbox}[colback=purple!5!white,colframe=purple!75!black]
\begin{definition}
Given $\Sigma\subset k[x]$, define $V(\Sigma)=\{ a\in \mathbb{A}_k: f(a)=0 \textrm{for every} \ f\in \Sigma\}$. This is called the \underline{\textbf{affine k-algebraic set }} defined by $\Sigma$. If $\Sigma=\{f\}$, then $H_f:=V(\Sigma)=V(f)$ defines a \underline{\textbf{hyperplane}} in $\mathbb{A}_k$.
\end{definition}
\end{tcolorbox}



\begin{tcolorbox}[colback=yellow!5!white,colframe=yellow!30!white]
\begin{example}
Easy examples
    \begin{enumerate}
        \item $V((0))=\mathbb{A}_k$. 
        \item $V((1))=\emptyset$
        \item Let $k=\mathbb{C}$. Then, in $\mathbb{A}_k^1$, $V(x^2-1)=\{\pm 1\}$. In $\mathbb{A}_k^2$, $V(x^2-1)=\{(\pm 1,n): n\in k\}$

    \end{enumerate}

\end{example}
\end{tcolorbox}

\begin{tcolorbox}[colback=purple!5!white,colframe=purple!75!black]
\begin{definition}
Given $V\subset \mathbb{A_k}$, defined $I(V)=\{ f\in k[x]: f(V)=0 \}$. This is called the \underline{\textbf{ideal}} of $V$.  
\end{definition}
\end{tcolorbox}

\begin{tcolorbox}[colback=blue!5!white,colframe=blue!30!white]
\begin{proposition}
    \begin{enumerate}
        \item   Let $I_{\Sigma}\subset k[x]$ be the ideal generated by $\Sigma$. Then, $V(\Sigma)=V(I)$. 
        \item There exists a finite system $f_1,..,f_m$ such that $V(\Sigma)=V(f_1,...,f_m)$
        \item If $\Sigma_1\subset \Sigma_2$, then $V(\Sigma_1)\supset V(\Sigma_2)$
        \item Given $\mathfrak{a}$ an ideal, then $I(V(\mathfrak{a}))=\mathfrak{a}$ iff $\mathfrak{a}=\sqrt{\mathfrak{a}}$.
        \item Given ideals $\mathfrak{a},\mathfrak{b}$, then $V(\mathfrak{a})=V(\mathfrak{b})$ iff $\sqrt{\mathfrak{a}}=\sqrt{\mathfrak{b}}$.  
    \end{enumerate}
  
\end{proposition}
\end{tcolorbox}



\begin{tcolorbox}[colback=purple!5!white,colframe=purple!75!black]
\begin{definition}
Let $\mathcal{A}_K^n:=\{ V\subset \mathbb{A}_K^n: V \textrm{affine} \ k- \textrm{algebra} \}$. Given $V\in \mathcal{A}_K^n$, let $k[V]:=k[x]/I(V)$ be the \underline{\textbf{affine coordinate ring}} generated by $V$. 
\end{definition}
\end{tcolorbox}

Let $Id^{rd}(k[x])$ be the set of reduced ideals of $k[x]$. Let $R_n$ be the set of reduced $k$-algebras with $n$-generators.


\begin{tcolorbox}[colback=red!5!white,colframe=red!30!white]
\begin{theorem}
There is a canonical bijection between the set of reduced affine $k$-algebras and reduced ideals of $k[x]$, given by the maps

\[R_n\to Id^{re}(k[X])\to \mathcal{A}^k_K\]
\[k[\underline{x}]\mapsto \mathfrak{a}:=ker(k[x]\xrightarrow{f} k)\mapsto V(\mathfrak{a})\]
with $f$ given by $x\mapsto \underline{x}$. 
\end{theorem}
\end{tcolorbox}

\subsection{The Zariski Topology}
Given $V\in \mathcal{A}_K^n$, there is a canonical map $K[X]\to K[V]$ given by $f\mapsto f_V$. 
\begin{tcolorbox}[colback=blue!5!white,colframe=blue!30!white]
\begin{proposition}
Let $\Sigma_i\subset k[X]$, and $f\in k[X]$ be given. then
\begin{enumerate}
    \item $V(\cup_i \Sigma_i)=\cap_i V(\Sigma_i)$
    \item $V(\prod \Sigma_i)=\cup V(\Sigma_i)$
    \item $V((0))= \mathbb{A}_k^n$; $V((1))=\emptyset$
\end{enumerate}
\end{proposition}
\end{tcolorbox}

By the proposition above, we can define the Zariski topology on $\mathbb{A}_k^n$

\begin{tcolorbox}[colback=purple!5!white,colframe=purple!75!black]
\begin{definition}
The Zariski topology on $\mathbb{A}_K^n$ is given by the closed sets $V(\Sigma)$, with $\Sigma\in k[X]$. In particular, the sets $D_f:=\mathbb{A}_k^n-H_f$ is an open set and forms a basis for the topology. 
\end{definition}
\end{tcolorbox}
Note that the zariski topology on product spaces is not the product of zariski topologies. Moreover, the connectedness/irreducibility is dependent on $K|k$. A point is called a generic point of $V$ if its closure contains $V$. 


\begin{tcolorbox}[colback=yellow!5!white,colframe=yellow!30!white]
\begin{example}
If $K|k=\mathbb{C}|\mathbb{Q}$, then $V(x_1^2-2x_2^2)$ is connected and irreducible. If $K|k=\mathbb{C}|\mathbb{Q}[\sqrt{2}]$, then $V(x_1^2-2x_2^2)$ is connected but not irreducible.
\end{example}
\end{tcolorbox}

\begin{tcolorbox}[colback=green!5!white,colframe=green!30!white]
\begin{remark}
For a topological space, $X$, the following are equivalent:
\begin{enumerate}
    \item Every descending chain of closed subsets is stationary. 
    \item Every ascending chain of open subsets is stationary.
\end{enumerate}
A topological space satisfiying the above is called \underline{\textbf{Noetherian}}. For example, $Spec(R)$ is Noetherian if $R$ is Noetherian. Note that if $X$ is Noetherian, then it is automatically quasi-compact. Moreover, there are only finitely many irreducible components and connected components of $X$. 
\end{remark}
\end{tcolorbox}





\end{document}