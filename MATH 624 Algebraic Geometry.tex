\documentclass{article}
\usepackage[utf8]{inputenc}
\usepackage{amsmath}
\usepackage{amsfonts}
\usepackage{amssymb}
\usepackage{tikz}
\usepackage{fullpage}
\usepackage{tikz-cd}
\usepackage{spectralsequences}
\usepackage{adjustbox}
\usepackage[backend=biber, style=alphabetic]{biblatex}
\usepackage{xfrac}
\usepackage{tcolorbox}
\usepackage{xcolor}
\usepackage{graphicx}
\graphicspath{ {D:/Chrome Downloads./} }
\usepackage[parfill]{parskip}
\usepackage{amsthm}
\addbibresource{sample.bib}
\theoremstyle{definition}
\newtheorem{theorem}{Theorem}[section]
\theoremstyle{definition}
\newtheorem{definition}{Definition}[theorem]
\theoremstyle{definition}
\newtheorem{remark}{Remark}[theorem]
\theoremstyle{definition}
\newtheorem{proposition}{Proposition}[theorem]
\theoremstyle{definition}
\newtheorem{lemma}[theorem]{Lemma}
\theoremstyle{definition}
\newtheorem{corollary}{Corollary}[theorem]
\theoremstyle{definition}
\newtheorem{example}{Example}[theorem]
\title{MATH 624 Algebraic Geometry}
\author{David Zhu}

\begin{document}
\maketitle

\section{Prevarieties and Varieties}

We will assume that $K|k$ a finite extension, $K$ is algebraically closed. We will use $\mathbb{A}^n(K)=K^n=\mathbb{A}^n_K$ to denote the underlying set, not the $n$-dimensional affine space. Given a point $a=(a_1,...,a_n)\in \mathbb{A}^n_k$, we will use $\varphi_a$ to denote the evaluation map $k[X]\to k$. Similarly, given $f\in k[x]$, we have the evalation map $\tilde{f}: \mathbb{A}_k\to k$. This gives a morphism of $k$-algebras $k[x]\to Maps_k(\mathbb{A}_k,k)$ given by $f\mapsto \tilde{f}$.


\begin{tcolorbox}[colback=purple!5!white,colframe=purple!75!black]
\begin{definition}
Given $\Sigma\subset k[x]$, define $V(\Sigma)=\{ a\in \mathbb{A}_k: f(a)=0 \textrm{for every} \ f\in \Sigma\}$. This is called the \underline{\textbf{affine k-algebraic set }} defined by $\Sigma$. If $\Sigma=\{f\}$, then $H_f:=V(\Sigma)=V(f)$ defines a \underline{\textbf{hyperplane}} in $\mathbb{A}_k$.
\end{definition}
\end{tcolorbox}



\begin{tcolorbox}[colback=yellow!5!white,colframe=yellow!30!white]
\begin{example}
Easy examples
    \begin{enumerate}
        \item $V((0))=\mathbb{A}_k$. 
        \item $V((1))=\emptyset$
        \item Let $k=\mathbb{C}$. Then, in $\mathbb{A}_k^1$, $V(x^2-1)=\{\pm 1\}$. In $\mathbb{A}_k^2$, $V(x^2-1)=\{(\pm 1,n): n\in k\}$

    \end{enumerate}

\end{example}
\end{tcolorbox}

\begin{tcolorbox}[colback=purple!5!white,colframe=purple!75!black]
\begin{definition}
Given $V\subset \mathbb{A_k}$, defined $I(V)=\{ f\in k[x]: f(V)=0 \}$. This is called the \underline{\textbf{ideal}} of $V$.  
\end{definition}
\end{tcolorbox}

\begin{tcolorbox}[colback=blue!5!white,colframe=blue!30!white]
\begin{proposition}
    \begin{enumerate}
        \item   Let $I_{\Sigma}\subset k[x]$ be the ideal generated by $\Sigma$. Then, $V(\Sigma)=V(I)$. 
        \item There exists a finite system $f_1,..,f_m$ such that $V(\Sigma)=V(f_1,...,f_m)$
        \item If $\Sigma_1\subset \Sigma_2$, then $V(\Sigma_1)\supset V(\Sigma_2)$
        \item Given $\mathfrak{a}$ an ideal, then $I(V(\mathfrak{a}))=\mathfrak{a}$ iff $\mathfrak{a}=\sqrt{\mathfrak{a}}$.
        \item Given ideals $\mathfrak{a},\mathfrak{b}$, then $V(\mathfrak{a})=V(\mathfrak{b})$ iff $\sqrt{\mathfrak{a}}=\sqrt{\mathfrak{b}}$.  
    \end{enumerate}
  
\end{proposition}
\end{tcolorbox}



\begin{tcolorbox}[colback=purple!5!white,colframe=purple!75!black]
\begin{definition}
Let $\mathcal{A}_K^n:=\{ V\subset \mathbb{A}_K^n: V \ \textrm{affine} \ k- \textrm{algebraic sets}  \}$. Given $V\in \mathcal{A}_K^n$, let $k[V]:=k[x]/I(V)$ be the \underline{\textbf{affine coordinate ring}} generated by $V$. 
\end{definition}
\end{tcolorbox}

Let $Id^{rd}(k[x])$ be the set of reduced ideals of $k[x]$. Let $R_n$ be the set of reduced $k$-algebras with $n$-generators.


\begin{tcolorbox}[colback=red!5!white,colframe=red!30!white]
\begin{theorem}
There is a canonical bijection between the set of reduced affine $k$-algebras and reduced ideals of $k[x]$, given by the maps

\[R_n\to Id^{re}(k[X])\to \mathcal{A}^k_K\]
\[k[\underline{x}]\mapsto \mathfrak{a}:=ker(k[x]\xrightarrow{f} k)\mapsto V(\mathfrak{a})\]
with $f$ given by $x\mapsto \underline{x}$. 
\end{theorem}
\end{tcolorbox}

\subsection{The Zariski Topology}
Given $V\in \mathcal{A}_K^n$, there is a canonical map $K[X]\to K[V]$ given by $f\mapsto f_V$. 
\begin{tcolorbox}[colback=blue!5!white,colframe=blue!30!white]
\begin{proposition}
Let $\Sigma_i\subset k[X]$, and $f\in k[X]$ be given. then
\begin{enumerate}
    \item $V(\cup_i \Sigma_i)=\cap_i V(\Sigma_i)$
    \item $V(\prod \Sigma_i)=\cup V(\Sigma_i)$
    \item $V((0))= \mathbb{A}_k^n$; $V((1))=\emptyset$
\end{enumerate}
\end{proposition}
\end{tcolorbox}

By the proposition above, we can define the Zariski topology on $\mathbb{A}_k^n$

\begin{tcolorbox}[colback=purple!5!white,colframe=purple!75!black]
\begin{definition}
The Zariski topology on $\mathbb{A}_K^n$ is given by the closed sets $V(\Sigma)$, with $\Sigma\in k[X]$. In particular, the sets $D_f:=\mathbb{A}_k^n-H_f$ is an open set and forms a basis for the topology. 
\end{definition}
\end{tcolorbox}
Note that the zariski topology on product spaces is not the product of zariski topologies. Moreover, the connectedness/irreducibility is dependent on $K|k$. A point is called a generic point of $V$ if its closure contains $V$. 


\begin{tcolorbox}[colback=yellow!5!white,colframe=yellow!30!white]
\begin{example}
If $K|k=\mathbb{C}|\mathbb{Q}$, then $V(x_1^2-2x_2^2)$ is connected and irreducible. If $K|k=\mathbb{C}|\mathbb{Q}[\sqrt{2}]$, then $V(x_1^2-2x_2^2)$ is connected but not irreducible.
\end{example}
\end{tcolorbox}

\begin{tcolorbox}[colback=green!5!white,colframe=green!30!white]
\begin{remark}
For a topological space, $X$, the following are equivalent:
\begin{enumerate}
    \item Every descending chain of closed subsets is stationary. 
    \item Every ascending chain of open subsets is stationary.
\end{enumerate}
A topological space satisfiying the above is called \underline{\textbf{Noetherian}}. For example, $Spec(R)$ is Noetherian if $R$ is Noetherian. Note that if $X$ is Noetherian, then it is automatically quasi-compact. Moreover, there are only finitely many irreducible components and connected components of $X$. 
\end{remark}
\end{tcolorbox}


\begin{tcolorbox}[colback=blue!5!white,colframe=blue!30!white]
\begin{proposition}
The following hold:
\begin{enumerate}
    \item The Zariski topology is Noetherian on $\mathbb{A}_K$, therefore also on any $V\in \mathcal{A}_K^n$. 
    \item For every $V\in \mathcal{A}_K$, there are only finitely many irreducible components and connected components. 
    \item $V\in \mathcal{A}_K$ is irreducible iff $I(V)$ is a prime ideal. 
    \item Given $V_0\subset V$, $V_0$ is irreducible iff $I_V(V_0):=I(V_0)/I(V)\in Spec(k(V))$ is minimal. 
    \item The connected components in $V\in \mathcal{A}_K$ correspond bijectively to the indecomposable idempotents of $k[V]$. 
    \item  For $V\in \mathcal{A}_K$, $a\in V$ is a generic point iff the evaluation map $k[V]\to k[a]$ is an isomorphism of $k$-algebras. 
\end{enumerate}
\end{proposition}
\end{tcolorbox}



\begin{tcolorbox}[colback=purple!5!white,colframe=purple!75!black]
\begin{definition}
Let $T$ be a topological space, and let $V\subset T$.
\begin{enumerate}
    \item dim(V):=sup \{ \textrm{chain of irreducible components ending in }\ V: \}
    \item codim(V):=sup \{ \textrm{chain of irreducible components starting with $V$ and ending in }\ $T$: \}
\end{enumerate}
Note that if $V=\cup V_{\alpha}$, then $dim(V)=supdim(V_{\alpha})$, and similarly for codimensions. Moreover, $dim(V)=dim(\overline{V})$. 
\end{definition}
\end{tcolorbox}


\begin{tcolorbox}[colback=blue!5!white,colframe=blue!30!white]
\begin{proposition}
(Notions of dimension) Let $V\in \mathcal{A}_K$ be irreducible. Then, the dimension of $V$ is the same as the krull dimension of $K[V]$. 
\end{proposition}
\end{tcolorbox}


\begin{tcolorbox}[colback=blue!5!white,colframe=blue!30!white]
\begin{proposition}
Suppose irreducible $W\subset V\in \mathcal{A}_K$. Then, 
\[dim(W)+codim_V(W)=dim(V)\] 
\end{proposition}
\end{tcolorbox}


\begin{tcolorbox}[colback=blue!5!white,colframe=blue!30!white]
\begin{proposition}
$V\in \mathcal{A}_K$ has generic points $a$ iff $td(K|k)\geq dim(V)=td(k(V))$. 
\end{proposition}
\end{tcolorbox}

\subsection{ Base change and Rational Points}

\begin{tcolorbox}[colback=purple!5!white,colframe=purple!75!black]
\begin{definition}
Suppose there is an embedding 
\[\begin{tikzcd}
K\arrow[r]&L\\
k\arrow[u]\arrow[r]&l\arrow[u]
\end{tikzcd}\]
Then, there is a natural morphism $k[x]\to l[x]$, which induces a pushforward of ideals and a map $\mathcal{A}_K\to \mathcal{A}_L$. Take the vanishing locus of the pushforward of $I(V)$ gives the base change of $V$. 


\end{definition}
\end{tcolorbox}

\begin{tcolorbox}[colback=green!5!white,colframe=green!30!white]
\begin{remark}
Base change does not preserve connectedness or irreducibility.
\end{remark}
\end{tcolorbox}



\begin{tcolorbox}[colback=purple!5!white,colframe=purple!75!black]
\begin{definition}
$V\in\mathcal{A}_K$ is called \underline{\textbf{absolutely (geometrically) irreducible}} if $V_l$ is irreducible for all field extension $l|k$. It is \underline{\textbf{geometrically connected}} is $V_l$ is connected for all $l|k$.
\end{definition}
\end{tcolorbox}

\begin{tcolorbox}[colback=blue!5!white,colframe=blue!30!white]
\begin{proposition}
Let $V\in \mathcal{A}_K$ be affine $k$-algebraic set. Then the following are equivalent:
\begin{enumerate}
    \item $V$ is absolutely irreducible.
    \item $V_{k^s}$ is irreducible.
    \item $V_{\overline{k}}$ irreducible.
\end{enumerate}
\end{proposition}
\end{tcolorbox}
The key observation is that $K^s[x]\to \overline{k}[X]$ is an integral extensions of domains. Therefore, we have going up and going down, and it straightforward to show that $Spec(k^s[X])\to Spec(\overline{k}[X])$ is a homeomorphism. Thus, we have $(2)\implies (3)$. 

To $(3)\implies (1)$, apply the following: 

\begin{tcolorbox}
\begin{lemma}
For every $V\in \mathcal{A}_K$, one has $V(\overline{k})$ is zariski dense in $V$. Therefore, $V_{\overline{k}}$ irreducible implies $V$ irreducible. 
\end{lemma}
\end{tcolorbox}
The proof is exercise. The key point is that if there exists $f$ with $k$-coefficients such that $f$ vanishes on all of $A$



\begin{tcolorbox}[colback=blue!5!white,colframe=blue!30!white]
\begin{proposition}
    Let $V\in \mathcal{A}_K$ be affine $k$-algebraic set. Then the following are equivalent:
    \begin{enumerate}
        \item $V$ is geometrically connected. 
        \item $V_{K^s}$ is connected.
        \item $V_{\overline{k}}$ is connected.
    \end{enumerate}
\end{proposition}
\end{tcolorbox}

\section{The category of quasi-affine $k$-algebraic sets}

\begin{tcolorbox}[colback=purple!5!white,colframe=purple!75!black]
\begin{definition}
A \underline{\textbf{quasi-affine}} k-algebrac set is any zariski open subset $U\subset V$ for $V\in \mathcal{A}_K$. 
\end{definition}
\end{tcolorbox}
The complement of hyperplanes is a basis of quasi-affine $k$-algebraic sets.
Let $V\in \mathcal{A}_K$ be non-empty, $f\in K[V]$. Then, the evaluation map $f: V\to \mathcal{A}_K$ is continuous. Moreover, $\varphi=(f_1,...,f_n)$ is also continuous. 


\begin{tcolorbox}[colback=purple!5!white,colframe=purple!75!black]
\begin{definition}
Let $V\in \mathcal{A}_K$ and $\mathcal{V}\subset V$ be zariski dense. Then, a functions $\varphi: \mathcal{V}\to \mathcal{A}_K$ is called \underline{\textbf{regular}} at $x \in V$ if there exists $f_x,g_x\in k[x]$ and $\mathcal{x}\subset V$ such that $g_x\neq 0$ everywhere on $\mathcal{U}_x$ and $\varphi=\frac{f_x}{g_x}$. A fucntion $\varphi: \mathcal V\to \mathcal{A}_K$ is \underline{\textbf{regular}} if it is regular at every point in $V$. Let $\mathcal{O}_x:=\{ \varphi\in Maps(\mathcal{V}, K):\varphi \ \textrm{regular at } x \}$. Define an equivalence relation on $\mathcal{O}_x$ by equivalence on any open neiborhood around $x$. $\mathcal{O}(V)$ is the set of regular functions on $V$. 
\end{definition}
\end{tcolorbox}


\begin{tcolorbox}[colback=blue!5!white,colframe=blue!30!white]
\begin{proposition}
(rings of regular functions) We have the following:
\begin{enumerate}
    \item $k[V]\to \mathcal{\tilde{O}}(V)$ is an isomorphism of $k$-algebra.
    \item $k[V]_f\to \tilde{O}(U_f)$ is an isomorphism of $k$-algebra. 
\end{enumerate}
\end{proposition}
\end{tcolorbox}
It is helpful to remember that Zariski open sets are dense. Thus, it suffices to show that a function is zero on a basic open $U_f$ to deduce it is globally zero. 

\section{Presheaves and Sheaves}

\begin{tcolorbox}[colback=purple!5!white,colframe=purple!75!black]
\begin{definition}
Let $\mathcal{C}$ be a concrete category such as $\textbf{Top, Set, Ab}$. Let $X$ be a topological space with topology $\tau_X$. Then, $\tau_X$ is naturally poset category where morphisms are inclusions. A \underline{\textbf{presheaf}} is a contravariant functor $\mathcal{P}: \tau_X\to \mathcal{C}$. 
\end{definition}
\end{tcolorbox}
Explicitly, $\mathcal{P}$ is given by two data: $1. \mathcal{P}(U)\in Obj(\mathcal{C})$ for every $U\in \tau_X$. $2. \rho_{u',u''}: \mathcal{P}(U'')\to \mathcal{P}(U')$ for every $U'\subset U''$. The elements in the set $P(U)$ are called \underline{\textbf{sections}} above $U$. The image of a section under $\rho$ is called the \underline{\textbf{restriction}}. 


\begin{tcolorbox}[colback=purple!5!white,colframe=purple!75!black]
\begin{definition}
A presheaf is a \underline{\textbf{sheaf}} if it has the covering preperty: given an open cover of an open set $U=\cup_i U_i$, with $U+{i,j}:=U_i\cap U_j$ with $s_i\in \mathcal{P}(U_i)$ such that $\rho_{U_i,U_{i,j}}(s_i)=\rho_{U_j,U_{i,j}}(s_i)$, then there exists $s\in \mathcal{P}(U)$ such that $s_i\in \rho_{U,U_i}(s)$ for every $U_i$. 
\end{definition}
\end{tcolorbox}


\begin{tcolorbox}[colback=purple!5!white,colframe=purple!75!black]
\begin{definition}
Suppose that limits exists in $\mathcal{C}$. Then $\mathcal{P}_x:=\mathcal{P}(U_x)$ is called the \underline{\textbf{stalk}}
of $\mathcal{P}$ at $x$. 

\end{definition}
\end{tcolorbox}


\begin{tcolorbox}[colback=blue!5!white,colframe=blue!30!white]
\begin{proposition}
$\mathcal{P}$ is a sheaf iff for every $U\in \tau_X$, the map $\varphi_U: U\to \coprod_{x\in U}\mathcal{P}_x$ is injective. 
\end{proposition}
\end{tcolorbox}


\begin{tcolorbox}[colback=blue!5!white,colframe=blue!30!white]
\begin{proposition}
For every presheaf $\mathcal{P}$, there is a sheafification functor $\mathcal{P}\to \mathcal{F}$ that induces isomorphism on stalks. 
\end{proposition}
\end{tcolorbox}


\begin{tcolorbox}[colback=purple!5!white,colframe=purple!75!black]
\begin{definition}
Let $f:X\to Y$ be a continuous map of topological spaces. Then, 
\begin{enumerate}
    \item Given a (pre)sheaf $\mathcal{P}$ on $X$, then the \underline{\textbf{direct image}} (pre)sheaf $f_*\mathcal{P}$ on $Y$ is defined by $f_*\mathcal{V}:=\mathcal{P}(f^{-1}(V))$ for all $V\in \tau_Y$. In particular, the direct image sheaf is also a sheaf.
    \item Given a presheaf $\mathcal{P}$ on $Y$. There is an \underline{\textbf{inverse image}} sheaf $f^{-1}\mathcal{P}$ on $X$ defined by the limit:
    \[f^{-1}\mathcal{P}(U):=\varprojlim_{U\subset U'} \mathcal{P}(f(U')) \]
    where $U\subset U'$ and $f(U')$ is open. 
\end{enumerate}
\end{definition}
\end{tcolorbox}

\begin{tcolorbox}[colback=green!5!white,colframe=green!30!white]
\begin{remark}
Note that the preimage sheaf is always a preseeaf, but not necessarily a sheaf. 
\end{remark}
\end{tcolorbox}


\begin{tcolorbox}[colback=purple!5!white,colframe=purple!75!black]
\begin{definition}
A (locally) \underline{\textbf{ringed space}} is a pair $(X,\mathcal{F})$, where $X$ is a topological space and $\mathcal{F}$ a sheaf of rings on $X$ such that the stalks at each point is a local ring. 
\end{definition} 
\end{tcolorbox}


\begin{tcolorbox}[colback=purple!5!white,colframe=purple!75!black]
\begin{definition}
Given locally ringed spaces $(X,\mathcal{F})$, $(Y,\mathcal{G})$, a morphism of locally ringed space is a pair $(f,f^{\sharp})$ such that $f: X\to Y$ is continuous and $f^{\sharp}: \mathcal{G}\to f_*\mathcal{F}$ a morphism of sheaves. 
\end{definition}
\end{tcolorbox}

\section{Back to Varieties}


\begin{tcolorbox}[colback=blue!5!white,colframe=blue!30!white]
\begin{proposition}
    Let $V$ be an affine $k$-algebraic set, $U\subset V$ zariski open. 
    \begin{enumerate}
        \item The assignmenet $\tau_U$, $U'\mapsto \tilde{O}(U')$ defined a locally ringed space on $U$.
        \item A morphism of quasi-affine algebraic set $T\to U$ is any morphism of locally ringed spaces $(f,f^{\sharp}):(T,\mathcal{O}_T)\to (U,\mathcal{O}_U)$
    \end{enumerate}
\end{proposition}
\end{tcolorbox}
The checks are fullfilled by proposition $2.0.1$. 


\begin{tcolorbox}[colback=blue!5!white,colframe=blue!30!white]
\begin{proposition}
Let $(T, \mathcal{O}_T), (U,\mathcal{O}_U)$, and $\Phi: T\to U$ continuous. Then,
\begin{enumerate}
    \item $\Phi$ defined a morphism of locally ringed spaces iff $\mathcal{O}_U\circ \varphi\subset \mathcal{O}_T$, i.e for every $U$ and $T'$ open such that $\Phi(T')\subset U'$ and $\varphi\in \mathcal{O}_U(U')$, then $\varphi\circ \Phi\in \mathcal{O}_T(T')$. 
    \item Suppose $\Phi$ defines such a morphism, and let $U\subset \mathbb{A}_K$, $p: \mathbb{A}^n_K\to K$ the $i$th projection, then $p_i|_{U}\circ \Phi$ completely determines $\Phi$. 
\end{enumerate}
\end{proposition}
\end{tcolorbox}


\begin{tcolorbox}[colback=green!5!white,colframe=green!30!white]
\begin{remark}
Let $U_f:=\{ x\in V|f(x)\neq 0: \}$ be a basic open. Consider $W_f\subset \mathbb{A}^n_K$ defined by $W_f:=\{ (a,b)| a\in \mathbb{A}^n_K, b\in \mathbb{A}^1_K:f(a)b-1=0 \}$ is an algebraic set in $\mathbb{A}^{mn}_K$. Prove that $\Phi: W_f\to U_f$ given by $(a,b)\mapsto a$ is an isomorphism of quasi affine $k$-algebraic sets. Then inverse is given by $\psi: U_f\to W_f$ given by $a\mapsto (a,\frac{1}{f(a)})$.
\end{remark}
\end{tcolorbox}


\begin{tcolorbox}[colback=blue!5!white,colframe=blue!30!white]
\begin{proposition}
Every quasi-affine $k$-algebraic set contains a n zariski dense $k$-algebraic set. 
\end{proposition}
\end{tcolorbox}

\begin{tcolorbox}[colback=purple!5!white,colframe=purple!75!black]
\begin{definition}
A quasi-affine $k$-algebraic set is called \underline{\textbf{affine}} if it is isomorphic as a locally ringed space to an affine $k$-algebraic set. 

\end{definition}
\end{tcolorbox}


\begin{tcolorbox}[colback=red!5!white,colframe=red!30!white]
\begin{theorem}
The following hold:
\begin{enumerate}
    \item The catgeory of $K$-valued affine $k$-algebraic sets, $\mathcal{A}_k$, is anti-equivalent to the category of reduced $k$-algebras of finite type. In particular, a $k$-algebraic set $V\subset \mathcal{A}_K$ is mapped to $k[V]$. Note that the projection maps $V\to W\to \mathcal{A}_k$ defined a regular function on $V$, and by proposition $4.0.2$ determined the morphism of the algebraic set. There is a canonical map from the ring of regular functions on $V$ to the coordinate ring $k[V]$ by proposition $2.0.1$. 
    \item Let $U$ be a quasi-affine $k$-algebraic set, $W$ and affine $k$-algebraic set. Then, a morphism $\Phi: U\to W$ is determined by a map $\Phi^*: k[W]\to \tilde{O}(U)$. 
\end{enumerate}
\end{theorem}
\end{tcolorbox}


\begin{tcolorbox}[colback=purple!5!white,colframe=purple!75!black]
\begin{definition}
$\mathcal{A}^n_k:=(\mathcal{A}_K^n,\tilde{O}_{\mathcal{A}_K^n})$ is called the \underline{\textbf{n-dimensional affine sapce.}}
\end{definition}
\end{tcolorbox}


\begin{tcolorbox}[colback=purple!5!white,colframe=purple!75!black]
\begin{definition}
An \underline{\textbf{open immersion}} of quasi-affine $k$-algebriac setd $j: U\to T$ is any $k$-morphism which is a zariksi open immersion and $\tilde{O}_U=\tilde{O}_T\circ j$
\end{definition}
\end{tcolorbox}


\begin{tcolorbox}[colback=purple!5!white,colframe=purple!75!black]
\begin{definition}
A \underline{\textbf{closed immersion}} of quasi-affine $k$-algebraic sets $i: U\to T$ is a topological closed immersion and $i_*O_U$ is a factor sheaf of $\mathcal{O}_T$. In other words, the map $\Phi*: \tilde{\mathcal{O}}_T(T')\to \tilde{\mathcal{O}}_U(U')$ is surjective. 


\end{definition}
\end{tcolorbox}


\begin{tcolorbox}[colback=purple!5!white,colframe=purple!75!black]
\begin{definition}
A $k$-\underline{\textbf{prevariety}} is any quasi-compact locally ringed space $X$ that is locally isomorphic to $K$-valued affine $k$- algebraic sets. Locally isomorphic here means that there exists an finite open cover $X=\cup X_{\alpha}$ and isomorphism of locally ringed spaces $\varphi_{\alpha}: X_{\alpha}\to V_{\alpha}$, where $V_{\alpha}$ is affine $k$-algebraic set. Moreover, the transition maps are isomorphisms of quasi-affine $k$-algebraic sets. 
\end{definition}
\end{tcolorbox}


\begin{tcolorbox}[colback=green!5!white,colframe=green!30!white]
\begin{remark}
A $k$\underline{\textbf{-morphism}} of $k$-prevarieties is a morphism of locally ringed spaces, such that there exists $X=\cup X_{\alpha}, Y=\cup Y_{\alpha}$ and $f(X_{\alpha})\subset Y_{\alpha}$, and the structure maps induce a map of affine $k$-algebraic sets.
\end{remark}
\end{tcolorbox}


\begin{tcolorbox}[colback=purple!5!white,colframe=purple!75!black]
\begin{definition}
Let $f: X\to Y$ be a $k$-morphism of $k$-prevarieties. Then, 
\begin{enumerate}
    \item $f$ is an open immersion iff $f$ induced structure maps is an open immersions of affine $k$-algebraic sets.
    \item $f$ is an closed immersion iff $f$ induced structure maps is a closed immersions of affine $k$-algebraic sets.
    \item $X$ is called affine if it is isomorphic as a $k$-prevariety to an affine $k$-algebraic set. 
    \item $X$ is called quasi-affine if there is an open immersion into a affine $k$-prevariety. 
\end{enumerate}
\end{definition}
\end{tcolorbox}


\begin{tcolorbox}[colback=blue!5!white,colframe=blue!30!white]
\begin{proposition}
(Gluing datat for $k$-prevarieties and $k$-morphisms) 
\begin{enumerate}
    \item $(X_i)$ be a finite set of $k$-prevarieties. 
    \item $X_{ij}\subset X_i$ open for every $i,j$
    \item $\varphi_{ij}: X_{ij}\to X_{ji}$ a $k-$isomorphism such that $\varphi_{ii}=Id$, $\varphi_{ij}=\varphi_{ji}^{-1}$ and $\varphi_{ij}\circ \varphi_{jk}=\varphi_{ik}$. 
    \item A solution is $X=\cup X_i'$ and $k$-isomorphisms $X_i'\to X_i$
\end{enumerate}
\end{proposition}
\end{tcolorbox}

\begin{tcolorbox}[colback=green!5!white,colframe=green!30!white]
\begin{remark}
The solution is unique up to $k$-isomorphism. 
\end{remark}
\end{tcolorbox}

\begin{tcolorbox}[colback=blue!5!white,colframe=blue!30!white]
\begin{proposition}
(Gluing morphisms)Suppose $f_{\alpha}: X_{\alpha}\to Y_{\alpha}$ such that 
\[\begin{tikzcd}
	{X_{\alpha\beta}} & {Y_{\alpha\beta}} \\
	{X_{\beta\alpha}} & {Y_{\beta\alpha}}
	\arrow["{f_{\alpha}}", from=1-1, to=1-2]
	\arrow["{\varphi_{\alpha\beta}}"', from=1-1, to=2-1]
	\arrow["{\psi_{\beta\alpha}}", from=1-2, to=2-2]
	\arrow["{f_{\beta}}"', from=2-1, to=2-2]
\end{tikzcd}\]
The there exists a unique $k$-morphism $X\to Y$ compactible with the gluing data. 
\end{proposition}
\end{tcolorbox}
The idea of proof for $4.1.1$ and $4.1.2$ is to take the disjoint union of the topological spaces first and define an equivalence relation accordingly. Then, define the structure sheaf on the quotient as the unique sheaf of $k$-algebras such that $\mathcal{O}_X|_{i_{\alpha}(X_{\alpha})}=(i_{\alpha})_*\mathcal{O}_{X_{\alpha}}$. We can check that $\mathcal{O}_X$ is well-defined. The glued morphism is as expectedly induced by morphisms from the glued components. 


\begin{tcolorbox}[colback=yellow!5!white,colframe=yellow!30!white]
\begin{example}
(Line with two origins)Let $X_1,X_2=\mathbb{A}^1$ , and $U_{12}=U_{21}=\mathbb{A}^1-\{0\}$, and let $\varphi: U_{12}\to U_{21}$ be the identity. 
\end{example}
\end{tcolorbox}


\begin{tcolorbox}[colback=yellow!5!white,colframe=yellow!30!white]
\begin{example}
    (The projective line)Let $X_1,X_2=\mathbb{A}^1$ , and $U_{12}=U_{21}=\mathbb{A}^1-\{0\}$, and let $\varphi: U_{12}\to U_{21}$ be $\varphi(x)=\frac{1}{x}$. 
\end{example}
\end{tcolorbox}


\begin{tcolorbox}[colback=red!5!white,colframe=red!30!white]
\begin{theorem}
Let $X$ be a $k$-prevariety, and $V$ be an affine $k$-prevariety. Then, one has a canonical bijection
\[Hom_k(k[V], \mathcal{O}_X(X))\to Mor_K(X, V)\]
\end{theorem}
\end{tcolorbox}
Proof uses Theorem $4.1$ part 2. Break up the morphisms by $X=\cup X_{\alpha}$, where each $X_{\alpha}$ is affine. Then use the gluing theorems to glue back. 

\begin{tcolorbox}[colback=red!5!white,colframe=red!30!white]
\begin{theorem}
Finite products and coproducts exists in the catgeory of affine $k$-algebraic sets. The coproduct correspond to the product of the affine rings. The product correspond to the reduced tensor product of affine $k$-algebras.
\end{theorem}
\end{tcolorbox}

\begin{tcolorbox}[colback=green!5!white,colframe=green!30!white]
\begin{remark}
    Note that the $k$-tensor algebra of two reduced $k$-algebras might no longer be reduced. Therefore we need to take the quotient by the nilradical. 
\end{remark}
\end{tcolorbox}




\begin{tcolorbox}[colback=red!5!white,colframe=red!30!white]
    \begin{theorem}
    The category of $k$-prevarieties has finite products and coproducts.
    \end{theorem}
    \end{tcolorbox}
\begin{proof}
    Let $T$ be the category of locally ringed spaces in $k$-algebras. Let $T_0$ be a subcategory that is 
    \begin{enumerate}
        \item closed under open immersions.
        \item closed under finite products.
        \item Every object in $T$ can be glued from thatof $T_0$.
    \end{enumerate}
    Then, products exists in $T$. Take $T$ to be category of $k$-prevarieties and $T_0$ be subcategory quasi-affine $k$-prevarieties. The hard part is to show that $T_0$ has all fintie products. 
\end{proof}
    

\begin{tcolorbox}[colback=blue!5!white,colframe=blue!30!white]
\begin{proposition}
Let $f: X'\to X$. $g: Y'\to Y$ morphisms of $k$-prevarieties. Then, TFH
\begin{enumerate}
    \item There exists a canonical $k$-morphisms $f\times g: X'\times Y'\to X\times Y$
    \item $f,g$ are open/closed immersions iff $f\times g$ is.
    \item The diagonal morphism is a closed immersion of $k$-varities iff it is a topological closed immersion. 
\end{enumerate}
\end{proposition}
\end{tcolorbox}


\begin{tcolorbox}[colback=purple!5!white,colframe=purple!75!black]
\begin{definition}
Let $\mathcal{T}$ be a category of topological spaces in which finite products exist. Then, 
\begin{enumerate}
    \item an object $T$ is called \underline{\textbf{separated}} if the diagonal map is $T\to T\times T$ is a closed immersion.
    \item A $k$-prevariety is called \underline{\textbf{k-variety}} if it is separated.
\end{enumerate}
\end{definition}
\end{tcolorbox}


\begin{tcolorbox}[colback=blue!5!white,colframe=blue!30!white]
\begin{proposition}
Let $f: Y\to X$ be a morphism of $k$-prevarieties. Then,
\begin{enumerate}
    \item Suppose $X$ is a $k$-variety, and $f$ a closed/open immersion, then $Y$ is a $k$-variety.
    \item $X\times Y$ is a $k$-variety iff $X,Y$ are $k$-varieties. 
\end{enumerate}
\end{proposition}
\end{tcolorbox}


\begin{tcolorbox}[colback=red!5!white,colframe=red!30!white]
\begin{theorem}
    The following hold:
    \begin{enumerate}
        \item Affine $k$-prevarieties are actually $k$-varieties.
        \item Let $X$ be a $k$-prevariety such that for every $x,y\in X$, there is $V\subset X$ affine $k$-subprevariety such that $x,y\in V$. Then, $X$ is a $k$-variety.  
    \end{enumerate}
\end{theorem}
\end{tcolorbox}
\begin{proof}
    To $1$: $\mathbb{A}^n_K$ is separated. Then, $X$ is affine iff $\exists$ a closed immersion $X\to \mathbb{A}_K^n$. Deduce that the diagonal map is a closde immersion. 
\end{proof}


\begin{tcolorbox}[colback=yellow!5!white,colframe=yellow!30!white]
\begin{example}
The line with two origins is not separated. The diagonal morphism is not closed. 
\end{example}
\end{tcolorbox}


\begin{tcolorbox}[colback=yellow!5!white,colframe=yellow!30!white]
\begin{example}
The projective line is a $k$-variety. 
\end{example}
\end{tcolorbox}


\begin{tcolorbox}[colback=purple!5!white,colframe=purple!75!black]
\begin{definition}
Let $\mathcal{T}$ be a category of topological spaces in which finite product exists. Then,
\begin{enumerate}
    \item $T\in \mathcal{T}$ is called \underline{\textbf{universally closed}} if for every object $Y$ and $Y\times T$ if the projection of $T$ onto $Y$ is closed. 
    \item $X$ is called proper if separated and universally closed. 
\end{enumerate}
\end{definition}
\end{tcolorbox}


\begin{tcolorbox}[colback=blue!5!white,colframe=blue!30!white]
\begin{proposition}
If $X$ is universally closed/proper, $Y\to X$ a closed immersion. Then, $Y$ is universally closed/proper. 

$X\times Y$ is universally closed/proper if $X,Y$ are so.
\end{proposition}
\end{tcolorbox}


\begin{tcolorbox}[colback=purple!5!white,colframe=purple!75!black]
\begin{definition}
(Graded rings) $R=\oplus_{d\geq 0} R_d$ a ring such that 
\begin{enumerate}
    \item $R_d$ is a subgroup of $R$, and 
    \item $R_d\cdot R_q\subset R_{d+q}$.
\end{enumerate}
$R_d$ is called the \underline{\textbf{graded piece of degree d}}. 
\end{definition}
\end{tcolorbox}


\begin{tcolorbox}[colback=purple!5!white,colframe=purple!75!black]
\begin{definition}
    $I$ is a \underline{\textbf{graded ideal}} if $I=\oplus I_d$ and $I_d=I\cap R_d$. If so, $R/I=\oplus R_d/I_d$. 
\end{definition}
\end{tcolorbox}


\begin{tcolorbox}[colback=purple!5!white,colframe=purple!75!black]
\begin{definition}
\underline{\textbf{Proj(R)}}$:= \{ p\in spec(R) :p \ \textrm{graded}, p\neq \oplus_{d>0}R_d \} $
\end{definition}
\end{tcolorbox}


\begin{tcolorbox}[colback=blue!5!white,colframe=blue!30!white]
\begin{proposition}
$p\in Proj(R)$ iff for every $a,b$ homogeneous, $ab\in p$ implies $a$ or $b$ in $p$.
\end{proposition}
\end{tcolorbox}


\begin{tcolorbox}[colback=blue!5!white,colframe=blue!30!white]
\begin{proposition}
For $a\in R$ homogeneous, $deg(a)>0$, $D_a^{+}=\{g\in Proj(R)| a\not\in g\}$ define the open basis for Zariski topology on $Proj$. 
\end{proposition}
\end{tcolorbox}

Localization works the same way as in non-graded case. 

\begin{tcolorbox}[colback=purple!5!white,colframe=purple!75!black]
\begin{definition}
Let $\Sigma\subset R$ be a multiplicatively closed set of homogeneous elements. Define $\Sigma^{-1}R:=\{\frac{f}{g}: g\in \Sigma, deg(g)=deg(f)\}$
\end{definition}
\end{tcolorbox}


\begin{tcolorbox}[colback=purple!5!white,colframe=purple!75!black]
\begin{definition}
The \underline{\textbf{ith dehomogenization}} $D^i: \cup_i R_d\to k[\underline{y'}]$, where $y'=(\frac{x_1}{x_i},...\frac{\hat{x_i}}{x_i},...,\frac{x_n}{x_i})$.
\end{definition}
\end{tcolorbox}


\begin{tcolorbox}[colback=blue!5!white,colframe=blue!30!white]
\begin{proposition}
$D^i$ gives a bijection from $D^i_+ \to Spec(k[\underline{y'}]) $
\end{proposition}
\end{tcolorbox}


\begin{tcolorbox}[colback=purple!5!white,colframe=purple!75!black]
\begin{definition}
The \underline{\textbf{ith homogenization}} $H^i: k[y']\to k[x]$ given by $f(y')\mapsto x_i^{deg(f)}f(y')$
\end{definition}
\end{tcolorbox}

Note that both homogenization and dehomogenization are multiplicative, however they are not inverse to each other. In general $H^iD^i(f)=f_0x_i^n$. 


\begin{tcolorbox}[colback=purple!5!white,colframe=purple!75!black]
\begin{definition}
A \underline{\textbf{cone}} in $\mathbb{A}^{n+1}_K$ is any subset $T$ such that $x\in T$ implies $\lambda x\in T$ for all $\lambda \in K$. The \underline{\textbf{projectivization}} of $T$, denoted $\mathbb{P}(T)=T^*/\sim $, where the equivalence relstion is given by $x\sim y$ iff $x=\lambda y$ for some $\lambda$. Denote the projectivization of $\mathbb{A}^n_K$ as $\mathbb{P}^n(K)$, and the $K$-rational points of the $n$-th dimensional projective space. 
\end{definition}
\end{tcolorbox}



\begin{tcolorbox}[colback=purple!5!white,colframe=purple!75!black]
\begin{definition}
(Zariski topology on projective space)Given a homogeneous polynomial $f(x)\in k[X]$, define $D_f^+=\{ x\in \mathbb{P}^n_K:f(x)\neq 0 \}$. Then, $D_f^+$ is a basis for a topology on $\mathbb{P}^n_K$. 
\end{definition}
\end{tcolorbox}


\begin{tcolorbox}[colback=purple!5!white,colframe=purple!75!black]
\begin{definition}
The \underline{\textbf{standard open covering}} for $\mathbb{P}^n_K$ is the set $D_{f_i}^+$ where $f_i=x_i$. Note $\mathbb{A}^n_K\to D_{f_i}^+$ is a homeomorphism.
\end{definition}
\end{tcolorbox}


\begin{tcolorbox}[colback=green!5!white,colframe=green!30!white]
\begin{remark}
$\mathbb{P}^n_K$ admits the union of $n$-copies of $\mathbb{A}^n_K$ as the standard open covering. 
\end{remark}
\end{tcolorbox}


\begin{tcolorbox}[colback=green!5!white,colframe=green!30!white]
\begin{corollary}
The Zariski topology on $\mathbb{P}^n_K$ is Noetherian. In particular, one may define when $V$ is irreducible/connected, etc. 
\end{corollary}
\end{tcolorbox}


\begin{tcolorbox}[colback=purple!5!white,colframe=purple!75!black]
\begin{definition}[projective $k$-algebraic sets]
Let $\Sigma\subset k[X]$ be a set of homogeneous elements. Then $V(\Sigma)=\{ x\in \mathbb{P}^n(K): f(x)=0 \ \forall f\in \Sigma \}$ is called a \underline{\textbf{projective $k$-algebraic set}} in $\mathbb{P}^n$. 
\end{definition}
\end{tcolorbox}


\begin{tcolorbox}[colback=purple!5!white,colframe=purple!75!black]
\begin{definition}
Given $V'\in \mathbb{P}^n$, define $I(V):=\{ f\in k[X]:f(V)=0 \}$ is a homogeneous ideal in $k[X]$. 
\end{definition}
\end{tcolorbox}


\begin{tcolorbox}[colback=blue!5!white,colframe=blue!30!white]
\begin{proposition}
If $V$ is a projective $k$-algebraic set, then $V(I(V))=V$. Moreover, $I(V)$ is reduced. 
\end{proposition}
\end{tcolorbox}


\begin{tcolorbox}[colback=blue!5!white,colframe=blue!30!white]
\begin{proposition}
$k[V]:=k[X]/I(V)$ is canonically a graded reduced $k$-algebra. More precisely, $I(V)=\oplus_{d\geq 0}I(V)_d$, and $k[V]=\oplus_{d\geq 0}R_d/I(V)_d$
\end{proposition}
\end{tcolorbox}


\begin{tcolorbox}[colback=blue!5!white,colframe=blue!30!white]
\begin{proposition}
The projective $k$-algebraic subsets $V\subset \mathbb{P}^n$ are closed, and any closed subset is a projective $k$-algebraic subset. 
\end{proposition}
\end{tcolorbox}


\begin{tcolorbox}[colback=blue!5!white,colframe=blue!30!white]
\begin{proposition}
Given a projective $k$-algebraic set $V$, the standard covering of $\mathbb{P}^n$ induces a covering of $V$, and the parts $D_{x_i}^+\cap V_i$ is a closed affine $k$-algebraic set. To see this, consider the ith dehomogenization of $f$. 
\end{proposition}
\end{tcolorbox}



\begin{tcolorbox}[colback=red!5!white,colframe=red!30!white]
\begin{theorem}
In the above context, let $R_{n=1}^{gr, red}$ be the graded reduced algebra over $k$ generated by $n+1$ variables, and $P_K^n$ be the set of projective $k$-algebraic sets.. Then, one has bijections
\[R_{n+1}^{gr, red}\xleftarrow{g} Id^{gr, red}(k[X])\xleftarrow{f} P^n_K\]
where $f(V(\Sigma))=(\Sigma)$, and $g((\Sigma))=\frac{k[X]}{(\Sigma)}$

\end{theorem}
\end{tcolorbox}


\begin{tcolorbox}[colback=blue!5!white,colframe=blue!30!white]
\begin{proposition}
The following holds:
\begin{enumerate}
    \item $V$ is irreducible iff $I(V)$ is a prime ideal in $Proj(k[X])$ iff $R[V]:=k[X]/I(V)$ is a domain. 
    \item The irreducible components of $V$ is in bijections with the minimal projective prime ideals of $k[X]$. 
\end{enumerate}
\end{proposition}
\end{tcolorbox}


\begin{tcolorbox}[colback=purple!5!white,colframe=purple!75!black]
\begin{definition}[The ring of regular functions of projective algebraic sets]
let $V\subset \mathbb{P}^n_K$ be any non-empty subset. Then, 
\begin{enumerate}
    \item A function $\varphi: V\to K$ is called \underline{\textbf{regular at}}  $a\in V$, if there exists neighborhood $U$ of $a$, and $p,q\in k[X]$ homogeneous and of the same degree, $q$ non-vanishing on $U$, such that $\varphi=\frac{p}{q}$ on $U$.
    \item A function $\varphi: V\to K$ is called \underline{\textbf{regular}} if it is regular at all points of $V$.
    \item Let $\mathbb{O}_a:=\{ \varphi: V'\to K: \textrm{regular at } a \}$ modulo the relation of agreement on a neighborhood around $a$. 
\end{enumerate}
\end{definition}
\end{tcolorbox}


\begin{tcolorbox}[colback=blue!5!white,colframe=blue!30!white]
\begin{proposition}
In the above context, the set of regular functions $\varphi: V'\to K$ regular at $a$ is a $k$-subalgebra of $Maps(V',K)$. Hence, the set of regular functions on $V'$ is also a $k$-subalgebra.
\end{proposition}
\end{tcolorbox}


\begin{tcolorbox}[colback=blue!5!white,colframe=blue!30!white]
\begin{proposition}
If $V$ is projective $k$-algebraic set, then $U\mapsto \mathbb{O}_U$ defines a sheaf of $k$-algebras on $V$. Thus, $V$ is naturally a ringed space. Moreover, $\mathbb{O}'_x$ is the stalk of $\mathbb{O}'_{V'}$ if $V'$ is a projective $k$-algebraic set.
\end{proposition}
\end{tcolorbox}



\begin{tcolorbox}[colback=purple!5!white,colframe=purple!75!black]
\begin{definition}
Given a projective algebraic set $V$, an open subset $U\subset V$ is called a \underline{\textbf{quasi-projective}} set. In particular, $U$ is canonically a ringed space by restriction from $V$. 
\end{definition}
\end{tcolorbox}


\begin{tcolorbox}[colback=blue!5!white,colframe=blue!30!white]
\begin{proposition}
The inclusion map $i: U\to V$ from a quasi-projective algebraic set to a projective algebraic set induces an open immersion of ringed spaces. 
\end{proposition}
\end{tcolorbox}


\begin{tcolorbox}[colback=purple!5!white,colframe=purple!75!black]
    \begin{definition}
    A $k$-prevariety is called \underline{\textbf{projective}} $k$-variety if $X$ is isomorphic as $k$-prevarieties to $(V, \Tilde{O}_V)$ for some projective algebraic set $V$. 
    \end{definition}
    \end{tcolorbox}
    


\begin{tcolorbox}[colback=red!5!white,colframe=red!30!white]
\begin{theorem}
The following hold: 
\begin{enumerate}
    \item Every quasi-projective $k$-algebraic set $U\subset V$ is a $k$-variety. 
    \item Every quasi-projective $k$-algebraic set $V$ is a proper $k$-variety. 
\end{enumerate}
\end{theorem}
\end{tcolorbox}
\begin{proof}
    step 1: show that $\mathbb{P}^n_K$ endowed with the sheaf of regular functions is a $k$-prevariety. Moreover, the intersection of a projective $k$-algebraic set with any standard affine open is affine open, and the covering forms a $k$-subvariety. The situation is the same when we take $V$ a closed projective subset instead of $\mathbb{P}^n_K$. 

    Step 2: show that $\mathbb{P}^n_K$ is separated, and so are all quasi-projective sub-prevarieties since there is an open-immersion into $\mathbb{P}^n_K$. Recall that a $k$-prevariety $X$ is separated if for every $x,y$, there exists an open affinne set $U\subset X$ such that $x,y\in U$. A useful fact here is that $GL(k)$ defines an automorphism of $\mathbb{P}^n_K$ and takes (separated, affine)$k$-prevarieties to (separated, affine) $k$-prevarieties. Then, $x,y$ both live in the affine open $D_{a_i+a_j}$, where $a_i,a_j$ are the two non-zero coordinates of the two points. 

    Step 3: The reduction steps is assume $V$ is $\mathbb{P}^n$ by closed immersion reflects properness. To check universally closed property of $\mathbb{P}^n_K$, it suffices to show that the projection from $\mathbb{P}^n_K\times_k X\to X$ is closed for $X$ affine by choosing affine covers in the general case of $k$-prevarieties. The final reduction step reduces $X$ affine to $X=\mathbb{A}^n$, since $X\times_k \mathbb{P}^n\to \mathbb{A}^n\times_k \mathbb{P}^n$ is a closed immersion.

\end{proof}



\begin{tcolorbox}[colback=red!5!white,colframe=red!30!white]
\begin{theorem}[Fundamental Theorem of Elimination Theory]
Let $V$ be  $V(I)\subset \mathbb{A}^n_K\times_k \mathbb{P}^n_K$ a closed subset. Then the projection $pr_{\mathbb{A}^n_K}(V)=V(J)$, where $J$ is the set of all polynomials $J:=\{ b(\underline{y}):\exists N>0 \textrm{with } x_i^Nb(\underline{y})\in I \ \textrm{for every } i \}$
\end{theorem}
\end{tcolorbox}


\begin{tcolorbox}[colback=green!5!white,colframe=green!30!white]
\begin{remark}
Apparently this is equivalent to the statement in Model theory, which roughly states that an algebraically-closed field has elimination of quantifiers. 
\end{remark}
\end{tcolorbox}



\begin{tcolorbox}[colback=purple!5!white,colframe=purple!75!black]
\begin{definition}
Let $R=\oplus_d R_d$ and $S=\oplus_d S_d$ be graded $A$-algebras. Then, a \underline{\textbf{morphism of graded-A algebras of degree k}} is an $A$-algebra homomorphism $\phi: R\to S$ such that $\phi(R_d)\subset S_{kd}$. 
\end{definition}
\end{tcolorbox}


\begin{tcolorbox}[colback=yellow!5!white,colframe=yellow!30!white]
\begin{example}
Let $V\subset \mathbb{P}^n_K$ be a projective $k$-variety. Then, $k[V]=k[\underbar{X}]/I(V)$. Then, the projection $p: k[\underbar{X}]\to k[V]$ is a graded morphism of degree $1$. 
\end{example}
\end{tcolorbox}

Recall the category of affine $k$-varieties and $k$-morphisms is anti-equivalent to the category of reduced $k$-algebras of finite type and $k$-morphisms. We want a similar statement for projective $k$-varieties, but the answer is we do not really know what happens in general.

\begin{tcolorbox}[colback=green!5!white,colframe=green!30!white]
    \begin{remark}
    Starting with $\phi^{\sharp}: k[V]\to k[W]$ surjective, we have a map $\phi: W\to V$ a $k$-morphism of projective $k$-varieties. However, different $\phi^{\sharp}, \phi'^{\sharp}$ may induce the same map on projective $k$-varieties; if $\phi^{\sharp}$ is not surjective, then more may go wrong. 
    \end{remark}
    \end{tcolorbox}
    
    
    We do have the special case:



\begin{tcolorbox}[colback=blue!5!white,colframe=blue!30!white]
\begin{proposition}
Let $R$, $S$ be reduced graded $k$-algebras, and $\phi^{\sharp}: R\to S$ a graded morphisms of degree $>0$. Then, let $b=\phi^0(a)\neq 0$. Then, $\phi^{\sharp}$ gives rise to $\phi^{\sharp}_a: R_a^0\to S_b^0$, where $R_a$ and $S_b$ are dehomogenization are $R,S$ with respect to $a,b$. Let $I=ker(k[\underbar{X}]\to R)$ given by mapping $x_1,...,x_n$ to the generators of $R_1$ and  $J=ker(k[\underbar{Y}]\to R)$ given by mapping $y_1,...,y_m$ to the generators of $S_1$. Let $V=V(I)\subset \mathbb{P}^n_k$ and $W=V(J)$. Let $V_a^+:= V\cap D_a^+$ and $W_b^+=W\cap D_b^+$, then $\phi_a^{\sharp}: R_a^0\to S_b^0$ define $\phi_a: W^+_b\to V^+_a$ a $k$-morphism, hence $k[V^+_a]=R_a^0$ and $k[W^+_b]=S_b^0$. We may conlcude that if $\phi^{\sharp}: R\to S$ is a surjection, then for every $b_i\in S_1$, there exists $q_i\in R_1$ such that $\phi^{\sharp}(a_i)=b_i$, hence $\phi_{a_i}: W^+_{a_i}\to V^+_{a_i}$ can be glued to a map $\phi: W\to V$ given by $\phi^{\sharp}: k[V]\to k[W]$. 
\end{proposition}
\end{tcolorbox}


\begin{tcolorbox}[colback=green!5!white,colframe=green!30!white]
\begin{remark}
We will resolve this issue in later discussion on projective schemes. 
\end{remark}
\end{tcolorbox}
    

\subsection{Product of projective Varieties}

\begin{tcolorbox}[colback=red!5!white,colframe=red!30!white]
\begin{theorem}[Segre Embedding]
The product of projective $k$-varieties in the category of $k$-prevarieties is a projective $k$-variety. The product is called the \underline{\textbf{Segre Embedding.}}
\end{theorem}
\end{tcolorbox}
\begin{proof}
    Let $V\subset \mathbb{P}^m_K$ and $W\subset \mathbb{P}^n_K$ be projective varieties. Note $V\times_k W\to \mathbb{P}^m_K\times_k \mathbb{P}^n_K$ is a closed immersion. Thus, it suffices to show that $\mathbb{P}^m_K\times_k \mathbb{P}^n_K$ is a projective variety for all $m,n$. 

    The construction is as follows: let $N=(m+1)(n+1)-1=mn+m+n$. Then, define $\Phi: \mathbb{P}^m\times_k \mathbb{P}^n\to \mathbb{P}^N$ by $(a_0:...:a_m),(b_0:....b_n)\mapsto (a_ib_j)$ where $(a_ib_j)$ is ordered lexicographically. This is a topological embedding. Consider $\underbar{Z}=(Z_{ij})$ a set of projective variables. Then, $$im(\Phi)=V(Z_{ij}Z{kl}-Z_{il}Z_{kj})_{i,j,k,l}$$
\end{proof}



\begin{tcolorbox}[colback=green!5!white,colframe=green!30!white]
\begin{remark}
The embedding is an example to the following problem: given morphisms of projective varieties, there is no canonical unique morphisms of ring of regular functions, as we can embed the varieties to a projective space of higher dimension. 
\end{remark}
\end{tcolorbox}


\begin{tcolorbox}[colback=green!5!white,colframe=green!30!white]
\begin{remark}
Given $\mathbb{P}^n, \mathbb{P}^m$ with $m\leq n$. Let $x=(x_0:...x_n)$ and $y=(y_0:...y_n)$. Then, $k[\underbar{y}]\to k[\underbar{x}]$ given by $y_i\mapsto x_i$ for $i\leq m$ and $y_i\mapsto 0$ if $i>m$ defines a $k$-embedding $\mathbb{P}^m\to \mathbb{P}^n$ defines a $k$-embedding. The image is $V(y_{m+1},...,y_n)$. 
\end{remark}
\end{tcolorbox}



\begin{tcolorbox}[colback=red!5!white,colframe=red!30!white]
\begin{theorem}(Chow's Lemma)
Let $X$ be a proper $k$-variety. Then, there exists projective $k$-varieties $\tilde{X}$ together with a surjective morphism $\tilde{f}: \tilde{X}\to X$ satisfying $\tilde{f}$ is an isomorphism on an open affine subset $U\subset X$. 
\end{theorem}
\end{tcolorbox}
\begin{proof}
    If $X$ is reducible, decompose $X$ into irreducible components $\cup X_{\alpha}$. Let $U_{\alpha}\subset X_{\alpha}$ be affine open dense such that $U_{\alpha}\cap U_{\beta}=\emptyset$. (every open will be dense on irreducible susbet, and point-set topology argument on disjointness). Do the thing on each irreducible component, and then take the disjoint union. 

    Thus, we consider $X$ irreducible. Let $X=\cup U_{\beta}$ be an affine open covering. Pick closed immersions $U_{\beta}\to \mathbb{A}_{\beta}^n$, and take $U=\cup U_{\beta}$. 
\end{proof}








\begin{tcolorbox}[colback=red!5!white,colframe=red!30!white]
\begin{theorem}[Nagata's Theorem]
Let $X$ be a $k$-variety. Then, there is a proper $k$-variety $\widehat{X}$ and an open embedding $i: X\to \widehat{X}$ with $i(X)$ dense. 
\end{theorem}
\end{tcolorbox}


\section{Schemes and Varieties in Mordern Sense}

Recall that $Spec(A)$ has the zariski topology, and comes equipped with the structure sheaf 


\begin{tcolorbox}[colback=purple!5!white,colframe=purple!75!black]
\begin{definition}
The structure sheaf $\mathcal{O}_A$ associated to $Spec(A)$ is defined by the following: let $U\subset Spec(A)$, then $$\mathcal{O}_A(U)=\{f: U\to \coprod_{p\in U}A_p: \textrm{if } p\in U, \textrm{then } f(p)\in A_p \textrm{ and } f \textrm{is locally a constant fraction}\}$$.
One may check that this defines a sheaf on $Spec(A)$. 
\end{definition}
\end{tcolorbox}
Note that the global sections $\mathcal{O}_A(Spec(A))$ is canonically isomorphic to $A$.


\begin{tcolorbox}[colback=purple!5!white,colframe=purple!75!black]
\begin{definition}[Affine Schemes]
An \underline{\textbf{affine scheme}} is a locally ringed space isomorphic to $Spec(A)$ for some commutative ring $A$. 
\end{definition}
\end{tcolorbox}


\begin{tcolorbox}[colback=red!5!white,colframe=red!30!white]
\begin{theorem}
There is a canonical bijection:
$\textrm{Hom}_{\textrm{Aff Scheme}}(B,A)\cong \textrm{Hom}_{LRS}(Spec(A), Spec(B))$. 
\end{theorem}
\end{tcolorbox}


\begin{tcolorbox}[colback=purple!5!white,colframe=purple!75!black]
\begin{definition}
A \underline{\textbf{scheme}} is a locally ringed space locally isomorphism to an affine scheme. Alternatively, one may intepret a scheme as a glueing of affine schemes. 

A morphism of schemes $(f,f^{\sharp}):(X,\mathcal{O}_X)\to (Y,\mathcal{O}_Y)$ is any morphism of locally ringed spaces. 
\end{definition}
\end{tcolorbox}

Concretely, let $(X,\mathcal{O}_X)$ be the gluing of affine schemes $(X_{\alpha}, \mathcal{O}_{X_{\alpha}})\cong (Spec(A_{\alpha}), \mathcal{O}_{A_{\alpha}})$. Similarly, let $(Y,\mathcal{O}_Y)$ be the gluing of affine schemes $(Y_{\beta}, \mathcal{O}_{Y_{\beta}})\cong (Spec(B_{\beta}), \mathcal{O}_{B_{\beta}})$. Then, $f(X_{\alpha})\subset Y_{\beta}$ and $f^{\sharp}$ locally is the morphsim of affine schemes. Thus, a morphism is equivalent to gluing morphisms of affine schemes. 



\begin{tcolorbox}[colback=purple!5!white,colframe=purple!75!black]
\begin{definition}[Relative Morphisms]
Let $S$ be a fixed scheme. An \underline{\textbf{$S$-scheme}} is a morphism $\varphi: X\to S$. A morphism of $S$-schemes $f: X\to Y$ is a commutative triangle 
\[\begin{tikzcd}
	X && Y \\
	& S
	\arrow["f", from=1-1, to=1-3]
	\arrow["{\varphi_x}"', from=1-1, to=2-2]
	\arrow["{\varphi_y}", from=1-3, to=2-2]
\end{tikzcd}\]
Note that this is simply the slice catgeory $\textrm{Sch}/S$. 
\end{definition}
\end{tcolorbox}


\begin{tcolorbox}[colback=red!5!white,colframe=red!30!white]
\begin{theorem}
The product exists in the category of schemes over $S$. This is equivalent to saying fiber products exists in the category of schemes. 
\end{theorem}
\end{tcolorbox}
\begin{proof}
    The first step is to do it for affine schemes. Let $S$ be affine. Then, the product of affine $S$-schemes exists, and is realized as the spec of the tensor $S$-algebra. This is in fact the product in the catgeory of schemes, not only in the category of affine schemes. 

    Step $2$ is to show the product of quasi-affine schemes exists. 
    Step $3$ Break up general schemes $X,Y$ to affine charts, and use gluing schemes and morphism. 
    Step $4$ Break up $S$ into affine charts and glue. 
\end{proof}


\begin{tcolorbox}[colback=purple!5!white,colframe=purple!75!black]
\begin{definition}
$\mathbb{A}^n:=\mathbb{A}^n_{\mathbb{Z}}$ by definition is $Spec(\mathbb{Z}[x_1,...,x_n])$. In general, $\mathbb{A}^n_R$ for any commutative ring $R$ is $Spec(R[x_1,...,x_n])\cong Spec(R)\otimes_{\mathbb{Z}}\mathbb{A}^n_{\mathbb{Z}}$. For $S$ a scheme, we define $\mathbb{A}^n_S:= S\times_{\mathbb{Z}}\mathbb{A}^n$. 
\end{definition}
\end{tcolorbox}


\begin{tcolorbox}[colback=purple!5!white,colframe=purple!75!black]
\begin{definition}[Base change]
Let $T\to S$ be a morphism fixed schemes. Then, the we have the base change functor from $\textrm{Sch}_S$ to $\textrm{Sch}_T$ given by $X\mapsto X\times_S T$. 
\end{definition}
\end{tcolorbox}



\begin{tcolorbox}[colback=purple!5!white,colframe=purple!75!black]
\begin{definition}
Let $s\in S$ be a point. Then, $\mathcal{O}_s$ is a local ring, and let $\kappa(s)=\mathcal{O}_s/m_s$ be the residue field at $s$. Given a field $K$, $spec(K)$ has the data of the zero prime ideal, and the global section isomorphisc $K$. Given a morphism $Spec(K)\to S$ is precisely a point on $S$ and a field embedding from $\kappa(s)\to K$. 
\end{definition}
\end{tcolorbox}


\begin{tcolorbox}[colback=purple!5!white,colframe=purple!75!black]
\begin{definition}
Fiber of a $S$-scheme: $X$ $\varphi_X: X\to S$ at $s\in S$ is $X\times_S s$ is the fiber of $\varphi_X$ at $s$. 
\end{definition}
\end{tcolorbox}


\begin{tcolorbox}[colback=purple!5!white,colframe=purple!75!black]
\begin{definition}
A Scheme $X$ is irreducible if the underlying topological space is irreducible. A scheme is reduced if every stalk is reduced, equivalently covered by affine schemes corresponding to reduced rings.  A scheme is integral if $X$ is irreducible and reduced. 
\end{definition}
\end{tcolorbox}


\begin{tcolorbox}[colback=purple!5!white,colframe=purple!75!black]
\begin{definition}
A morphism of schemes $f: X\to Y$ is called \underline{\textbf{dominant}} is $f(X)$ is dense in $Y$. 
\end{definition}
\end{tcolorbox}


\begin{tcolorbox}[colback=purple!5!white,colframe=purple!75!black]
\begin{definition}
A scheme is called \underline{\textbf{normal}} if each stalk is integrally closed. 
\end{definition}
\end{tcolorbox}


\begin{tcolorbox}[colback=yellow!5!white,colframe=yellow!30!white]
\begin{example}
If $X=spec(R)$, then $X$ is normal iff $R$ is normal, i.e every localization at prime is integrally closed. 
\end{example}
\end{tcolorbox}






\begin{tcolorbox}[colback=red!5!white,colframe=red!30!white]
\begin{theorem}[Normalization and Generic Fiber ]
Given an integral scheme $X$, let $K=\kappa (X)$ be its function field and $L|K$ be an algebraic extension. Then, there exists a scheme $X_L$ such that $L:=\kappa(X_L)$ and there exists a dominant morphism 
$\varphi_L: X_L\to X$ satisfiying the universal property:

Given a integral scheme $Y$ and a dominant morphism $f: Y\to X$, there is a commutative diagram of fields
\[\begin{tikzcd}
	L && {K=\kappa(X)} \\
	& {\kappa(Y)}
	\arrow["{f_L}", dashed, from=1-1, to=2-2]
	\arrow["{\varphi_L*}"', from=1-3, to=1-1]
	\arrow["{f^{*}}"', from=1-3, to=2-2]
\end{tikzcd}\]
\end{theorem}
\end{tcolorbox}
and $f_L$ is the generic fiber. In the case of $L=K$, $X_L\xrightarrow{\varphi}X$ is called the normalization of $X$. 


\begin{tcolorbox}[colback=green!5!white,colframe=green!30!white]
\begin{corollary}
Let $X$ be a separated scheme of finite type over a field $k$. Then, the normalization of $X$ in any field extension $L|\kappa(X)$ is again a $k$-variety, and $\varphi; X_L\to X$  is a finite morphism. Moreover, of $X$ is a projective $k$-variety, then $X_L$ is projective as well. 
\end{corollary}
\end{tcolorbox}



\begin{tcolorbox}[colback=red!5!white,colframe=red!30!white]
\begin{theorem}
Let $X$ be a noetherian integral normal scheme, $L|\kappa(X)$ a finite field extension. Then, $X_L$ is Noetherian and $X_L\to X$ is a finite morphism. 
\end{theorem}
\end{tcolorbox}


\begin{tcolorbox}[colback=purple!5!white,colframe=purple!75!black]
\begin{definition}
Let $f: X\to Y$ be a morphism in $\textrm{Sch}_S$. Then,
\begin{enumerate}
    \item $f$ is \underline{\textbf{affine}} is it can be factored as 
    
    \[X\xrightarrow{\textrm{cl. imm}} \mathbb{A}^n_Y\to Y\]
    \item $f$ is \underline{\textbf{projective}} is it can be factored as 
    
    \[X\xrightarrow{\textrm{cl. imm}} \mathbb{P}^n_Y\to Y\]
    
\end{enumerate}
\end{definition}
\end{tcolorbox}


\begin{tcolorbox}[colback=blue!5!white,colframe=blue!30!white]
\begin{proposition}
Affine morphism are separated and affine schemes are separated. 
\end{proposition}
\end{tcolorbox}


\begin{tcolorbox}[colback=red!5!white,colframe=red!30!white]
\begin{theorem}
Projective schemes are proper and projective morphisms are proper. 
\end{theorem}
\end{tcolorbox}


\section{Valuative Criterion for Separatedness}

Let $X$ be an $S$-scheme. $x_1,x_0\in X$ such that $x_1\sim x_0$ a specialization. Equivalently, we have $x_1\in Spec(O_{x_0})$. By Chevalley's theorem, there exists a valuations rings $R\in Val(\kappa(x_1))$ such that $R$ dominates $O_{x_1x_0}:= O_{x_0}/x_1$. 


\begin{tcolorbox}[colback=blue!5!white,colframe=blue!30!white]
\begin{proposition}
Let $f: X\to Y$ be a quasi-compact morphism of $S$-schemes, and $Z\subset X$ a closed subset. Then the following hold:

\begin{enumerate}
    \item (Closedness vs specialization) $f(Z)\subset Y$ is closed iff $f(Z)\subset Y$ is closed under specialization. 
    \item (Closedness and valuations) $f(Z)\subset Y$ is closed for every $y_1$ specializing to $y_0$, and $f(x_1)=y_1$ for some $x_1\in Z$, there is a valuation ring $R\in Val(\kappa(x_0))$ such that $R$ dominates $O_{y_1y_0}$ where $\kappa(y_1)\hookrightarrow \kappa(x_1)$ and $R$ has a center $x_0\in Z$, which means that $R$ dominates $O_{x_1x_0}$. Equivalently, there exists some $\varphi_R: Spec(R)\to Z$ such that $\varphi_R$ maps the generic poitn to $x_1$ and the maximal ideal to $x_0$. 
\end{enumerate}
\end{proposition}
\end{tcolorbox}



\begin{tcolorbox}[colback=red!5!white,colframe=red!30!white]
\begin{theorem}[Valuative Criterion for Separatedness]
    Let $f: X\to Y$ be a quasi-compact morphism of $S$-schemes. Then, $f$ is separated iff for all $x_1\in X$, and $R\in Val(\kappa(x_1))$, and all diagrams
    \[\begin{tikzcd}
        X & Y \\
        {x_1} & {Spec(R)}
        \arrow[from=1-1, to=1-2]
        \arrow[tail, from=2-1, to=1-1]
        \arrow[from=2-1, to=2-2]
        \arrow[dashed, from=2-2, to=1-1]
        \arrow[from=2-2, to=1-2]
    \end{tikzcd}\]
    there exists at most $1$ dashed arrow to make the diagram commute. In fact, we only need $x_1\to X$ a generic point. 

\end{theorem}
\end{tcolorbox}


\begin{tcolorbox}[colback=red!5!white,colframe=red!30!white]
\begin{theorem}
Let $f: X\to Y$ be a quasi-compact morphism of $S$-schemes. Then, $f$ is universally closed iff for every $x_1\in X$ and $R\in Val(\kappa(x_1))$, and every diagram
\[\begin{tikzcd}
    X & Y \\
    {x_1} & {Spec(R)}
    \arrow[from=1-1, to=1-2]
    \arrow[tail, from=2-1, to=1-1]
    \arrow[from=2-1, to=2-2]
    \arrow[dashed, from=2-2, to=1-1]
    \arrow[from=2-2, to=1-2]
\end{tikzcd}\]

there exists a unique dashed arrow making the diagram commute.
\end{theorem}
\end{tcolorbox}


\begin{tcolorbox}[colback=green!5!white,colframe=green!30!white]
\begin{corollary}
The following hold:
\begin{enumerate}
    \item If $f$ is finite, then $f$ is proper.
    \item If $f$ is profinite, then $f$ is separated and universally closed.
    \item Integral $S$-scheme $X$ in field extension $L|K(X)$ are separated and universally closed. 
    \item iF $X$ is a $k$-prevariety, and $X_L\to X$ of $X$ in $L|K(X)$ finite is a $k$-prevariety. Then, $X$ affine/projective/proper implies $X_L$ affine/projective/proper. 
    \item Let $X=Proj R$ be a the projective scheme associated to the graded ring $R$. Then, $X$ is separated and $X$ quasi-compact implies $X$ is universally closed. 
\end{enumerate}
\end{corollary}
\end{tcolorbox}


\begin{tcolorbox}[colback=red!5!white,colframe=red!30!white]
\begin{theorem}[Kronecker-Weber Theorem for Curves]
The category of $k$-curves with dominant rational map to the catgeory of function fields of transcendence degree $1$ over $k$. 
\end{theorem}
\end{tcolorbox}


\begin{tcolorbox}[colback=purple!5!white,colframe=purple!75!black]
\begin{definition}
Let $X,T$ be $S$-schemes, and define $X(T):= Hom(T,X)$ as a set, called the $T$\underline{\textbf{-rational points}} of $X$. $X(T)$ is naturally a contravariant functor in $T$ for a fixed $X$, from the category of $S$-schemes to sets, denoted $h^T$. Similarly, it is also a covariant functor in $X$ for a fixed $T$, denoted $h_X$.
\end{definition}
\end{tcolorbox}


\begin{tcolorbox}[colback=blue!5!white,colframe=blue!30!white]
\begin{proposition}
The following hold: 
\begin{enumerate}
    \item The functor $h^T$ satisfies
    \[h^T(X\times_S Y)=h^T(X)\times h^T(Y)\]
and if $X_T:= X\times_ST$ is the base change of $X$ to $T$, then $X_S(T)=X_T(T)$.
    \item For $X,Y$ $S$-schemes, one has a canonical bijection 
    \[Hom_S(X,Y)\to Nat(h_X,h_Y)\]
\end{enumerate}
\end{proposition}
\end{tcolorbox}
\begin{proof}
    Basic abstract nonsense.
\end{proof}


\begin{tcolorbox}[colback=yellow!5!white,colframe=yellow!30!white]
\begin{example}[Representability]
Consider the functor $F: Sch_S\to Grp$ defined by $X\mapsto \Gamma(O_X,X,+)$. Then, $\mathbb{G}_a= Spec(S[t])$ satisfies $G_a(T)=F(T)$. \\

Consider the functor $F_m: Sch_S\to Grp$, defined by $F_m(T)=O_T(T)^{\times}$. Then, the functor is represented by $\mathbb{G}_m= Spec(S[t,t^{-1}])$. 
\\

These are called the additive and multiplicative $S$-\underline{\textbf{group schemes}}, respectively. 
\end{example}
\end{tcolorbox}



\begin{tcolorbox}[colback=blue!5!white,colframe=blue!30!white]
\begin{proposition}
Let $X$ be an $S$-scheme. Then, the follwoing hold:
\begin{enumerate}
    \item Let $K$ be an $S$-field. Then, giving a $k$-rational point, $\varphi\in X(K)$ is equivalent to gvinig $x\in X$ and an $S$-embedding $\kappa(x)\to K$, i.e an $S$-morphism $Spec(K)\to Spec(S)\to X$. 
    \item Let $(R,\mathfrak{m})$ be a local ring. Then, giving a $R$-rational point $\varphi\in X(R)$ is equivalent to giving $x_0\in X$ and a $S$-morphism $O_{X,x_0}\to R$ of local rings. 
\end{enumerate}
\end{proposition}
\end{tcolorbox}


\begin{tcolorbox}[colback=yellow!5!white,colframe=yellow!30!white]
\begin{example}
Consider $X:= \mathbb{A}^n_A$, and $k$ a field. Then, $X(K)== Hom(K, \mathbb{A}_A^n)\cong Hom_A(A[x_1,...,x_n], K)= K^n$, as a set. 
\end{example}
\end{tcolorbox}


\begin{tcolorbox}[colback=yellow!5!white,colframe=yellow!30!white]
\begin{example}
let $X:= \mathbb{P}^n_A=Proj(A[x_0,...,x_n])$. Then, 
\begin{enumerate}
    \item Let $K$ be an $A$-field, i.e there is structure ring homomorphism $\varphi_K: A\to K$. Then, $X(K)K^{n+1}/\sim$ as one would define the projective space. 
\end{enumerate}
\end{example}
\end{tcolorbox}


\begin{tcolorbox}[colback=yellow!5!white,colframe=yellow!30!white]
\begin{example}
The same situation as example $6.3.3$, but with $K$ replaced by an $A$-algebra $R$. Then, $\mathbb{P}^n_A(R)=\{(r_0,...r_n)\in R^n| (r_0,...,r_n)=1\}/ \textrm{units}$
\end{example}
\end{tcolorbox}


\begin{tcolorbox}[colback=red!5!white,colframe=red!30!white]
\begin{theorem}
Let $K|k$ be a field extension, with $K$ algebraically closed. Then, the functor $K$-points 
\[h^K: Sch_k^{red,f.t}\to Prevar_k\] 
defined by $X\mapsto X(K)$ is an equivalence of categories. The analogous statements hold if we add affine/projective/proper. 
\end{theorem}
\end{tcolorbox}
\begin{proof}
    \underline{\textbf{Step 1}}: Affine schemes of finite types is equivalent to the category of reduced $k$-algebras of finite type. 


    \underline{\textbf{Step 2}}: Let $U\to X$ be an open immersion into an affine subscheme/open subvariety. Then, $i$ is completely determined by $O_U(U)\hookleftarrow O_X(X)$.

    \underline{\textbf{Step 3}}: Any $X,Y$ in $Sch_k^{red,f.t}$ and $f: X\to Y$ is determined by the glueing data of affine and quasi-affine schemes. The same is true for varieties. 

    \underline{\textbf{Step 4}}: Let $C,C'$ be category of $k$-ringed spaces, and subcategories $C_0, C_0'$ such is cofinal (everything from $C,C'$ is glued from that of $C_0,C_0'$). 


\end{proof}



\begin{tcolorbox}[colback=blue!5!white,colframe=blue!30!white]
\begin{proposition}
If $\Sigma$ is a subset of a scheme $X$, then $\overline{\Sigma}$ carries a unique reduced scheme structure $X_{\Sigma}$. 
\end{proposition}
\end{tcolorbox}


\begin{tcolorbox}[colback=purple!5!white,colframe=purple!75!black]
\begin{definition}
Let $x\in \mathbb{P}^n(K)$ be given, say $x=(x_0:...:x_n)$. Let $\lambda\in K$ such that $v(\lambda)=min_iv(x_i)$. Then, take $y=(y_0:...:y_n)$ such that $y_i=x_i\lambda^{-1}$, and each $y_i$ is a unit. Define $\rho: \mathbb{P}^n_K(K)\to \mathbb{P}^n_k(k)$ by $x=\lambda y\mapsto (\overline{y_0}:...:\overline{y_n})$, wher $\overline{y_i}=(y_i mod \ \mathfrak{m})\in k$.
\end{definition}
\end{tcolorbox}


\begin{tcolorbox}[colback=purple!5!white,colframe=purple!75!black]
\begin{definition}[Reduction of Schemes]
Consider the composition 
\[\mathbb{P}^n_K\hookrightarrow \mathbb{P}^n_R\hookleftarrow \mathbb{P}^n_k\]
where the maps are the obvious base change, be the generic, respectively, special fibers. 

\end{definition}
\end{tcolorbox}


\begin{tcolorbox}[colback=red!5!white,colframe=red!30!white]
\begin{theorem}
Let $R$ be a valuation ring inside an algebraically closed field $K$, and $Z\subset \mathbb{P}^n_K$ be a closed projective $K$-variety. Then, then following hold: 
\begin{enumerate}
    \item there is a unique projective $k$-variety $Z_k\subset \mathbb{P}^n_k$ such that $Z_k(k)=\rho(Z(K))$. 
    \item The variety $Z_k$ is realized as the special fiber of the reduced scheme closure $Z_r:=i(Z_k)\subset \mathbb{P}^n_R$.
    \item $Z_R\subset \mathbb{P}^n_R$ is the unique closed $R$-subscheme containing $Z_k$ and satisfying $O_x$ is a $R$-torsion free module for all $x\in Z_K$.
\end{enumerate}
\end{theorem}
\end{tcolorbox}
The above theorem applies to the projective Noether normalization and projection from linear subspace. 







\end{document}